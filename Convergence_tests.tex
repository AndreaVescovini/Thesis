\documentclass[11pt, a4paper]{article}
\usepackage[british]{babel}
\usepackage[utf8]{inputenc}
\usepackage{amsmath}
\usepackage{amssymb}
\usepackage{amsthm}
\usepackage{booktabs}
\usepackage{subfig}
\usepackage{bbold}
\usepackage[hidelinks]{hyperref}
\usepackage{graphicx}
\usepackage{pgfplots}
\pgfplotsset{compat=newest}
%\usetikzlibrary{plotmarks}
%\usetikzlibrary{arrows.meta}
%\usepgfplotslibrary{patchplots}
%\usepackage{grffile}

\usepackage{tikz}
\usepackage{pgfgantt}
\usepackage{pdflscape}
\pgfplotsset{plot coordinates/math parser=false}

\newlength{\figwidth}
\setlength{\figwidth}{0.45\textwidth}

\graphicspath{{img/}}

\theoremstyle{definition}
\newtheorem{defn}{Definition}

\title{\textbf{Space convergence tests for TVD methods}}
\author{Andrea Vescovini}
\date{\today}

\begin{document}
	\maketitle

\section{Introduction}
We want to test the space convergence order of the $L^2$-error of the solution 
of the 
isothermal Navier-Stokes equations \eqref{eq:mass_bal}-\eqref{eq:mom_bal}, 
computed using TVD methods for the discretization of the convective term and to 
make a comparison with the results obtained employing the first order upwind 
approximation.
\begin{align}
\label{eq:mass_bal} \frac{\partial\rho}{\partial t} + \nabla \cdot (\rho 
\mathbf{v}) = 0&\\
\label{eq:mom_bal} \frac{\partial{(\rho \mathbf{v}})}{\partial t} + \nabla 
\cdot (\rho \mathbf{v} \mathbf{v^\mathrm{T}}) - \nabla \cdot (\mu (\nabla 
\mathbf{v} + \nabla \mathbf{v}^\mathrm{T})) + 
\nabla p = \mathbf{f}&
\end{align}

\begin{defn}
Let $h$ be the diameter of a cell of the grid and let $err_h$ be the 
$L^2$-error of the solution computed over a grid with cells of diameter $h$.\\
A method is said to have a convergence order of $p$ if
\begin{equation*}
\exists C > 0 \quad \text{such that} \quad err_h \leq C h^p \quad \forall h
\end{equation*}
\end{defn}

While discretizing equations \eqref{eq:mass_bal}-\eqref{eq:mom_bal} with finite 
volumes over a staggered grid, we use centred finite differences for both the 
diffusive term and the gradient of the pressure, with a second order accuracy.
For what concerns the convective term we can use the first order upwind or the 
TVD methods that are second order everywhere but at the points where we have 
local extrema.


\section{Tests}
We have chosen five test cases with a known analytical solution and we have 
solved the problem over sequences of uniform grids each time refined. Since we 
are interested in the space convergence, we have chosen four steady cases and 
only one unsteady case, for which we have computed the error at the second 
time-step. 
As TVD methods we have used the Van Leer and the MinMod. We have tried both 
low (\textasciitilde 1) and high (\textasciitilde 1000) Reynolds numbers, 
obtained modifying the viscosity.

\subsection{1D problem}
We have started with a one-dimensional problem in the domain $\Omega = [0,1]$:
\begin{align*}
\frac{ \partial (\rho v)}{\partial x} =& \; f_1 \quad \text{in} \quad \Omega\\
\frac{ \partial (\rho v^2)}{\partial x} - \frac{ \partial}{\partial x} \bigg (2 
\mu 
\frac{\partial v}{\partial x} \bigg) + \frac{\partial p}{\partial x} =& \; f_2 
\quad 
\text{in} \quad \Omega\\
v =& \; v_{ex} \quad \text{on} \quad \partial \Omega\\
p =& \; p_{ex} \quad \text{on} \quad \partial \Omega
\end{align*}
with the the source terms:
\begin{align*}
&f_1 = 6 x^2 \rho\\
&f_2 = 6 x^5 \rho - 24 \mu x - 2
\end{align*}
such that the exact solution is:
\begin{align*}
&v_{ex} = 2 x^3\\
&p_{ex} = 2 (1 - x)
\end{align*}
We have chosen $\rho = 1$ and $\mu = 1$ or $\mu = 10^{-3}$.
We have made seven refinements starting from a grid of 4 cells.

We have obtained the following convergence orders:
\begin{table}[h]
	\centering
	\[
	\begin{array}{c|ccc}
	\toprule
	& \text{Upwind} & \text{MinMod} & \text{VanLeer} \\ 
	\midrule
	p & 0.985 & 1.000 & 0.990\\
	v & 1.985 & 1.996 & 1.996\\
	\bottomrule
	\end{array}
	\]
	\caption{Convergence order with $Re = 1$ for the 1D problem.}
	\label{tab:1d_lre}

	\[
	\begin{array}{c|ccc}
	\toprule
	& \text{Upwind} & \text{MinMod} & \text{VanLeer} \\ 
	\midrule
	p & 0.852 & 1.501 & 2.172\\
	v & 1.318 & 2.274 & 2.356\\
	\bottomrule
	\end{array}
	\]
	\caption{Convergence order with $Re = 1000$ for the 1D problem.}
	\label{tab:1d_hre}
\end{table}

\begin{figure}[t]
	\centering
	\subfloat[][1D problem, Upwind, $Re = 1$]{
		% This file was created by matlab2tikz.
%
\definecolor{mycolor1}{rgb}{0.00000,0.44700,0.74100}%
\definecolor{mycolor2}{rgb}{0.85000,0.32500,0.09800}%
%
\begin{tikzpicture}

\begin{axis}[%
width=0.951\figwidth,
height=0.75\figwidth,
at={(0\figwidth,0\figwidth)},
scale only axis,
xmode=log,
xmin=4,
xmax=1000,
xminorticks=true,
ymode=log,
ymin=1e-06,
ymax=1,
yminorticks=true,
axis background/.style={fill=white},
legend style={at={(0.03,0.03)}, anchor=south west, legend cell align=left, align=left, draw=white!15!black}
]
\addplot [color=mycolor1, mark=x, mark options={solid, mycolor1}]
  table[row sep=crcr]{%
4	0.214116295\\
8	0.195589151\\
16	0.132868874\\
32	0.0777359686\\
64	0.0420904681\\
128	0.0219065573\\
256	0.011175991\\
512	0.00564462389\\
};
\addlegendentry{$p$}

\addplot [color=mycolor2, mark=x, mark options={solid, mycolor2}]
  table[row sep=crcr]{%
4	0.0216704604\\
8	0.0117239406\\
16	0.00417857985\\
32	0.00124123493\\
64	0.000337955248\\
128	8.81570481e-05\\
256	2.25117773e-05\\
512	5.68790627e-06\\
};
\addlegendentry{$u$}

\addplot [color=white!70!black, forget plot]
  table[row sep=crcr]{%
4	0.00025\\
1000	1e-06\\
};
\addplot [color=white!70!black, forget plot]
  table[row sep=crcr]
	\subfloat[][1D problem, Upwind, $Re = 1000$]{
		% This file was created by matlab2tikz.
%
\definecolor{mycolor1}{rgb}{0.00000,0.44700,0.74100}%
\definecolor{mycolor2}{rgb}{0.85000,0.32500,0.09800}%
%
\begin{tikzpicture}

\begin{axis}[%
width=0.951\figwidth,
height=0.75\figwidth,
at={(0\figwidth,0\figwidth)},
scale only axis,
xmode=log,
xmin=4,
xmax=1000,
xminorticks=true,
ymode=log,
ymin=0.001,
ymax=0.190700916,
yminorticks=true,
axis background/.style={fill=white},
legend style={at={(0.03,0.03)}, anchor=south west, legend cell align=left, align=left}
]
\addplot [color=mycolor1, mark=x, mark options={solid, mycolor1}]
  table[row sep=crcr]{%
4	0.0950934111\\
8	0.0956441228\\
16	0.0633456286\\
32	0.0359064412\\
64	0.0189587968\\
128	0.009733956\\
256	0.00507474159\\
512	0.00281243348\\
};
\addlegendentry{$p$}

\addplot [color=mycolor2, mark=x, mark options={solid, mycolor2}]
  table[row sep=crcr]{%
4	0.190700916\\
8	0.161981022\\
16	0.100050412\\
32	0.0542847205\\
64	0.0273944361\\
128	0.013011169\\
256	0.00574729458\\
512	0.00230500508\\
};
\addlegendentry{$u$}

\addplot [color=white!70!black, forget plot]
  table[row sep=crcr]{%
4	0.25\\
1000	0.001\\
};
\addplot [color=white!70!black, forget plot]
  table[row sep=crcr]\\
	\subfloat[][1D problem, MinMod, $Re = 1$]{
		% This file was created by matlab2tikz.
%
\definecolor{mycolor1}{rgb}{0.00000,0.44700,0.74100}%
\definecolor{mycolor2}{rgb}{0.85000,0.32500,0.09800}%
%
\begin{tikzpicture}

\begin{axis}[%
width=0.951\figwidth,
height=0.75\figwidth,
at={(0\figwidth,0\figwidth)},
scale only axis,
xmode=log,
xmin=4,
xmax=1000,
xminorticks=true,
ymode=log,
ymin=1e-07,
ymax=0.1,
yminorticks=true,
axis background/.style={fill=white},
legend style={at={(0.03,0.03)}, anchor=south west, legend cell align=left, align=left}
]
\addplot [color=mycolor1, mark=x, mark options={solid, mycolor1}]
  table[row sep=crcr]{%
4	0.0947400829\\
8	0.0579455977\\
16	0.0306653228\\
32	0.0155606681\\
64	0.00780733847\\
128	0.00390635907\\
256	0.0019533289\\
512	0.000976635645\\
};
\addlegendentry{$p$}

\addplot [color=mycolor2, mark=x, mark options={solid, mycolor2}]
  table[row sep=crcr]{%
4	0.00558067417\\
8	0.00185644832\\
16	0.000515125253\\
32	0.000134752887\\
64	3.44440641e-05\\
128	8.70859755e-06\\
256	2.18965445e-06\\
512	5.49000176e-07\\
};
\addlegendentry{$u$}

\addplot [color=white!70!black, forget plot]
  table[row sep=crcr]{%
4	2.5e-05\\
1000	1e-07\\
};
\addplot [color=white!70!black, forget plot]
  table[row sep=crcr]
	\subfloat[][1D problem, MinMod, $Re = 1000$]{
		\input{img/l2error_test_navierstokes_1d_mm_hre}}\\
	\subfloat[][1D problem, VanLeer, $Re = 1$]{
		% This file was created by matlab2tikz.
%
\definecolor{mycolor1}{rgb}{0.00000,0.44700,0.74100}%
\definecolor{mycolor2}{rgb}{0.85000,0.32500,0.09800}%
%
\begin{tikzpicture}

\begin{axis}[%
width=0.951\figwidth,
height=0.75\figwidth,
at={(0\figwidth,0\figwidth)},
scale only axis,
xmode=log,
xmin=4,
xmax=1000,
xminorticks=true,
ymode=log,
ymin=1e-07,
ymax=0.1,
yminorticks=true,
axis background/.style={fill=white},
legend style={at={(0.03,0.03)}, anchor=south west, legend cell align=left, align=left}
]
\addplot [color=mycolor1, mark=x, mark options={solid, mycolor1}]
  table[row sep=crcr]{%
4	0.058505198\\
8	0.0412346964\\
16	0.0251402529\\
32	0.0139748536\\
64	0.00738268213\\
128	0.00379649657\\
256	0.00192539001\\
512	0.000969591076\\
};
\addlegendentry{$p$}

\addplot [color=mycolor2, mark=x, mark options={solid, mycolor2}]
  table[row sep=crcr]{%
4	0.00804947871\\
8	0.0022466016\\
16	0.0005523443\\
32	0.000137569802\\
64	3.46360203e-05\\
128	8.7210858e-06\\
256	2.19045006e-06\\
512	5.49050369e-07\\
};
\addlegendentry{$u$}

\addplot [color=white!70!black, forget plot]
  table[row sep=crcr]{%
4	2.5e-05\\
1000	1e-07\\
};
\addplot [color=white!70!black, forget plot]
  table[row sep=crcr]
	\subfloat[][1D problem, VanLeer, $Re = 1000$]{
		% This file was created by matlab2tikz.
%
\definecolor{mycolor1}{rgb}{0.00000,0.44700,0.74100}%
\definecolor{mycolor2}{rgb}{0.85000,0.32500,0.09800}%
%
\begin{tikzpicture}

\begin{axis}[%
width=0.951\figwidth,
height=0.75\figwidth,
at={(0\figwidth,0\figwidth)},
scale only axis,
xmode=log,
xmin=4,
xmax=1000,
xminorticks=true,
ymode=log,
ymin=1e-06,
ymax=0.1,
yminorticks=true,
axis background/.style={fill=white},
legend style={at={(0.03,0.03)}, anchor=south west, legend cell align=left, align=left, draw=white!15!black}
]
\addplot [color=mycolor1, mark=x, mark options={solid, mycolor1}]
  table[row sep=crcr]{%
4	0.0161844304\\
8	0.00984258579\\
16	0.00351641157\\
32	0.00101607975\\
64	0.000263342669\\
128	6.34168067e-05\\
256	1.44388667e-05\\
512	3.20413694e-06\\
};
\addlegendentry{$p$}

\addplot [color=mycolor2, mark=x, mark options={solid, mycolor2}]
  table[row sep=crcr]{%
4	0.0562254093\\
8	0.020321909\\
16	0.00583980329\\
32	0.0015266502\\
64	0.000376324406\\
128	8.75670102e-05\\
256	1.88988619e-05\\
512	3.69275832e-06\\
};
\addlegendentry{$u$}

\addplot [color=white!70!black, forget plot]
  table[row sep=crcr]{%
4	0.00025\\
1000	1e-06\\
};
\addplot [color=white!70!black, forget plot]
  table[row sep=crcr]
    \caption{Errors for the 1D problem.}
    \label{fig:1d_err}
\end{figure}

\subsection{Steady channel problem}
The steady channel problem is solved over a domain $\Omega = [0,10] \times 
[-0.5, 
0.5]$:
\begin{align*}
\nabla \cdot (\rho \mathbf{v}) =& \; 0 \quad \text{in} \quad \Omega\\
\nabla \cdot (\rho \mathbf{v} \mathbf{v^\mathrm{T}}) - \nabla \cdot (\mu 
(\nabla 
\mathbf{v} + \nabla \mathbf{v}^\mathrm{T})) + 
\nabla p =& \; 0 \quad \text{in} \quad \Omega\\
\mathbf{v} =& \; \mathbf{v}_{ex} \quad \text{on} \quad \Gamma_{in} \cup 
\Gamma_w\\
(\nabla \mathbf{v}) \mathbf{n} =& [0, 0]^\mathrm{T} \quad 
\text{on} 
\quad \Gamma_{out}\\
p =& \; p_{ex} \quad \text{on} \quad \Gamma_{out}
\end{align*}
with $\Gamma_{in} = \{0\} \times [-0.5, 0.5]$ i.e. the inflow boundary, 
$\Gamma_w = [0, 10] \times 
\{-0.5, 
0.5\}$ i.e. the wall boundary, $\Gamma_{out} = \{10\} \times [-0.5, 0.5]$ i.e. 
the outflow boundary.
The exact solution is:
\begin{align*}
&v_{x,ex} = \frac{3}{2} V (1 - y^2)\\
&v_{y,ex} = 0\\
&p_{ex} = 3 \mu V (10 - x)
\end{align*}
We have chosen $\rho = 1$, $V = 1$ and $\mu = 1$ or $ \mu = 10^{-3}$.
We have made five refinements starting from a grid of $8 \times 4$ cells.
We have obtained the following convergence orders:
\begin{table}[h]
	\centering
	\[
	\begin{array}{c|ccc}
	\toprule
	& \text{Upwind} & \text{MinMod} & \text{VanLeer} \\ 
	\midrule
	p & 1.999 & 1.999 & 1.999\\
	v_x & 1.999 & 1.999 & 1.999\\
	v_y & 2.013 & 2.013 & 2.013\\
	\bottomrule
	\end{array}
	\]
	\caption{Convergence order with $Re = 1$ for the steady channel problem.}
	\label{tab:cha_lre}
	
	\[
	\begin{array}{c|ccc}
	\toprule
	& \text{Upwind} & \text{MinMod} & \text{VanLeer} \\ 
	\midrule
	p & 1.999 & 1.999 & 1.999\\
	v_x & 1.999 & 1.999 & 1.999\\
	v_y & 2.013 & 2.013 & 2.013\\
	\bottomrule
	\end{array}
	\]
	\caption{Convergence order with $Re = 1000$ for the steady channel 
	problem.}
	\label{tab:cha_hre}
\end{table}

\begin{figure}[t]
	\centering
	\subfloat[][Channel, Upwind, $Re = 1$]{
		\input{img/l2error_test_navierstokes_channel_steady}}
	\subfloat[][Channel, Upwind, $Re = 1000$]{
		\input{img/l2error_test_navierstokes_channel_steady_hre}}\\
	\subfloat[][Channel, MinMod, $Re = 1$]{
		\input{img/l2error_test_navierstokes_channel_steady_mm}}
	\subfloat[][Channel, MinMod, $Re = 1000$]{
		\input{img/l2error_test_navierstokes_channel_steady_mm_hre}}\\
	\subfloat[][Channel, VanLeer, $Re = 1$]{
		% This file was created by matlab2tikz.
%
\definecolor{mycolor1}{rgb}{0.00000,0.44700,0.74100}%
\definecolor{mycolor2}{rgb}{0.85000,0.32500,0.09800}%
\definecolor{mycolor3}{rgb}{0.92900,0.69400,0.12500}%
%
\begin{tikzpicture}

\begin{axis}[%
width=0.951\figwidth,
height=0.75\figwidth,
at={(0\figwidth,0\figwidth)},
scale only axis,
xmode=log,
xmin=5.65685424949238,
xmax=181.019335983756,
xminorticks=true,
ymode=log,
ymin=1e-08,
ymax=100,
yminorticks=true,
axis background/.style={fill=white},
legend style={legend cell align=left, align=left, draw=white!15!black}
]
\addplot [color=mycolor1, mark=x, mark options={solid, mycolor1}]
  table[row sep=crcr]{%
5.65685424949238	5.76306696\\
11.3137084989848	1.57416994\\
22.6274169979695	0.402855803\\
45.254833995939	0.101316682\\
90.5096679918781	0.0253689885\\
181.019335983756	0.00634520364\\
};
\addlegendentry{$p$}

\addplot [color=mycolor2, mark=x, mark options={solid, mycolor2}]
  table[row sep=crcr]{%
5.65685424949238	0.00755771065\\
11.3137084989848	0.0024004207\\
22.6274169979695	0.000637765901\\
45.254833995939	0.000162032481\\
90.5096679918781	4.06613238e-05\\
181.019335983756	1.0172044e-05\\
};
\addlegendentry{$u$}

\addplot [color=mycolor3, mark=x, mark options={solid, mycolor3}]
  table[row sep=crcr]{%
5.65685424949238	0.000194019117\\
11.3137084989848	7.70773454e-05\\
22.6274169979695	2.70614838e-05\\
45.254833995939	8.52850536e-06\\
90.5096679918781	2.25326552e-06\\
181.019335983756	5.58000331e-07\\
};
\addlegendentry{$v$}

\addplot [color=white!70!black, forget plot]
  table[row sep=crcr]{%
5.65685424949238	3.2e-07\\
181.019335983756	1e-08\\
};
\addplot [color=white!70!black, forget plot]
  table[row sep=crcr]
	\subfloat[][Channel, VanLeer, $Re = 1000$]{
		\input{img/l2error_test_navierstokes_channel_steady_vl_hre}}
    \caption{Errors for the steady channel problem.}
    \label{fig:cha_err}
\end{figure}

\subsection{Kovasznay problem}
The Kovasznay problem is solved over a domain $\Omega = [-0.5,2] \times [-0.5, 
1.5]$:
\begin{align*}
\nabla \cdot (\rho \mathbf{v}) =& \; 0 \quad \text{in} \quad \Omega\\
\nabla \cdot (\rho \mathbf{v} \mathbf{v^\mathrm{T}}) - \nabla \cdot (\mu 
(\nabla 
\mathbf{v} + \nabla \mathbf{v}^\mathrm{T})) + 
\nabla p =& \; 0 \quad \text{in} \quad \Omega\\
\mathbf{v} =& \; \mathbf{v}_{ex} \quad \text{on} \quad \partial \Omega\\
p =& \; p_{ex} \quad \text{on the lower let cell}
\end{align*}
The exact solution is:
\begin{align*}
&v_{x,ex} = 1 - e^{\lambda x} cos(2 \pi y)\\
&v_{y,ex} = \frac{\lambda}{2\pi} e^{\lambda x} sin(2 \pi y)\\
&p_{ex} = \frac{1}{2} (1 - e^{2 \lambda x})
\end{align*}
with $\lambda = \frac{1}{2 \mu} - \sqrt{\frac{1}{4\mu^2} + 4\pi^2}$. We have 
chosen $\rho = 1$ and $\mu = 2.5 \cdot 10^{-2}$ or $ \mu = 2.5 \cdot 10^{-4}$.
We have made five refinements starting from a grid of $4 \times 4$ cells.
We have obtained the following convergence orders:
\begin{table}[h]
	\centering
	\[
	\begin{array}{c|ccc}
	\toprule
	& \text{Upwind} & \text{MinMod} & \text{VanLeer} \\ 
	\midrule
	p & 0.823 & 0.890 & 0.902\\
	v_x & 0.928 & 2.436 & 2.379\\
	v_y & 0.940 & 2.422 & 2.565\\
	\bottomrule
	\end{array}
	\]
	\caption{Convergence order with $Re = 80$ for the Kovasznay 
	problem.}
	\label{tab:kov_lre}
	
	\[
	\begin{array}{c|ccc}
	\toprule
	& \text{Upwind} & \text{MinMod} & \text{VanLeer} \\ 
	\midrule
	p & 1.000 & 0.907 & 0.865\\
	v_x & 1.030 & 1.835 & 1.822\\
	v_y & 0.997 & 1.960 & 1.958\\
	\bottomrule
	\end{array}
	\]
	\caption{Convergence order with $Re = 8000$ for the Kovasznay 
	problem.}
	\label{tab:kov_hre}
\end{table}

\begin{figure}[t]
	\centering
	\subfloat[][Kovasznay, Upwind, $Re = 80$]{
		% This file was created by matlab2tikz.
%
\definecolor{mycolor1}{rgb}{0.00000,0.44700,0.74100}%
\definecolor{mycolor2}{rgb}{0.85000,0.32500,0.09800}%
\definecolor{mycolor3}{rgb}{0.92900,0.69400,0.12500}%
%
\begin{tikzpicture}

\begin{axis}[%
width=0.951\figwidth,
height=0.75\figwidth,
at={(0\figwidth,0\figwidth)},
scale only axis,
xmode=log,
xmin=4,
xmax=256,
xminorticks=true,
ymode=log,
ymin=0.000651880984,
ymax=1,
yminorticks=true,
axis background/.style={fill=white},
legend style={at={(0.03,0.03)}, anchor=south west, legend cell align=left, align=left, draw=white!15!black}
]
\addplot [color=mycolor1, mark=x, mark options={solid, mycolor1}]
  table[row sep=crcr]{%
4	0.463423878\\
8	0.143112573\\
16	0.0498339933\\
32	0.0274485113\\
64	0.0157273343\\
128	0.00905807802\\
256	0.00511952099\\
};
\addlegendentry{$p$}

\addplot [color=mycolor2, mark=x, mark options={solid, mycolor2}]
  table[row sep=crcr]{%
4	0.0365475504\\
8	0.0402359415\\
16	0.0284971904\\
32	0.016030142\\
64	0.00911637277\\
128	0.00494934791\\
256	0.00260057897\\
};
\addlegendentry{$u$}

\addplot [color=mycolor3, mark=x, mark options={solid, mycolor3}]
  table[row sep=crcr]{%
4	0.01620299\\
8	0.0203942699\\
16	0.00813891576\\
32	0.00425163801\\
64	0.00234145986\\
128	0.00125075128\\
256	0.000651880984\\
};
\addlegendentry{$v$}

\addplot [color=white!70!black, forget plot]
  table[row sep=crcr]{%
4	0.041720382976\\
256	0.000651880984\\
};
\addplot [color=white!70!black, forget plot]
  table[row sep=crcr]
	\subfloat[][Kovasznay, Upwind, $Re = 8000$]{
		% This file was created by matlab2tikz.
%
\definecolor{mycolor1}{rgb}{0.00000,0.44700,0.74100}%
\definecolor{mycolor2}{rgb}{0.85000,0.32500,0.09800}%
\definecolor{mycolor3}{rgb}{0.92900,0.69400,0.12500}%
%
\begin{tikzpicture}

\begin{axis}[%
width=0.951\figwidth,
height=0.75\figwidth,
at={(0\figwidth,0\figwidth)},
scale only axis,
xmode=log,
xmin=4,
xmax=256,
xminorticks=true,
ymode=log,
ymin=5.06127924e-05,
ymax=0.0108188379,
yminorticks=true,
axis background/.style={fill=white},
legend style={at={(0.03,0.03)}, anchor=south west, legend cell align=left, align=left}
]
\addplot [color=white!70!black, forget plot, line width=0.75pt]
  table[row sep=crcr]{%
4	0.0032392187136\\
256	5.06127924e-05\\
};
\addplot [color=white!70!black, forget plot, line width=0.75pt]
  table[row sep=crcr]{%
4	0.2073099976704\\
256	5.06127924e-05\\
};
\addplot [color=mycolor1, mark=x, mark options={solid, mycolor1}, line width=0.75pt]
  table[row sep=crcr]{%
4	0.0101205447\\
8	0.0108188379\\
16	0.00617459521\\
32	0.00142458715\\
64	0.00065817115\\
128	0.000326889512\\
256	0.000163405515\\
};
\addlegendentry{$p$}

\addplot [color=mycolor2, mark=x, mark options={solid, mycolor2}, line width=0.75pt]
  table[row sep=crcr]{%
4	0.000827330593\\
8	0.00859442106\\
16	0.00525853983\\
32	0.00227364257\\
64	0.00103524242\\
128	0.000497944856\\
256	0.000243793246\\
};
\addlegendentry{$u$}

\addplot [color=mycolor3, mark=x, mark options={solid, mycolor3}, line width=0.75pt]
  table[row sep=crcr]\\
	\subfloat[][Kovasznay, MinMod, $Re = 80$]{
		% This file was created by matlab2tikz.
%
\definecolor{mycolor1}{rgb}{0.00000,0.44700,0.74100}%
\definecolor{mycolor2}{rgb}{0.85000,0.32500,0.09800}%
\definecolor{mycolor3}{rgb}{0.92900,0.69400,0.12500}%
%
\begin{tikzpicture}

\begin{axis}[%
width=0.951\figwidth,
height=0.75\figwidth,
at={(0\figwidth,0\figwidth)},
scale only axis,
xmode=log,
xmin=4,
xmax=256,
xminorticks=true,
ymode=log,
ymin=1e-05,
ymax=1,
yminorticks=true,
axis background/.style={fill=white},
legend style={at={(0.03,0.03)}, anchor=south west, legend cell align=left, align=left}
]
\addplot [color=white!70!black, forget plot, line width=0.75pt]
  table[row sep=crcr]{%
4	0.00064\\
256	1e-05\\
};
\addplot [color=white!70!black, forget plot, line width=0.75pt]
  table[row sep=crcr]{%
4	0.04096\\
256	1e-05\\
};
\addplot [color=mycolor1, mark=x, mark options={solid, mycolor1}, line width=0.75pt]
  table[row sep=crcr]{%
4	0.465166938\\
8	0.211594301\\
16	0.136227298\\
32	0.104600388\\
64	0.0705134741\\
128	0.0412862288\\
256	0.0222713545\\
};
\addlegendentry{$p$}

\addplot [color=mycolor2, mark=x, mark options={solid, mycolor2}, line width=0.75pt]
  table[row sep=crcr]{%
4	0.0370158453\\
8	0.0494417118\\
16	0.0189924822\\
32	0.0070068795\\
64	0.0018286295\\
128	0.000384013264\\
256	7.098536e-05\\
};
\addlegendentry{$u$}

\addplot [color=mycolor3, mark=x, mark options={solid, mycolor3}, line width=0.75pt]
  table[row sep=crcr]
	\subfloat[][Kovasznay, MinMod, $Re = 8000$]{
		% This file was created by matlab2tikz.
%
\definecolor{mycolor1}{rgb}{0.00000,0.44700,0.74100}%
\definecolor{mycolor2}{rgb}{0.85000,0.32500,0.09800}%
\definecolor{mycolor3}{rgb}{0.92900,0.69400,0.12500}%
%
\begin{tikzpicture}

\begin{axis}[%
width=0.951\figwidth,
height=0.75\figwidth,
at={(0\figwidth,0\figwidth)},
scale only axis,
xmode=log,
xmin=4,
xmax=256,
xminorticks=true,
ymode=log,
ymin=1e-06,
ymax=0.010134474,
yminorticks=true,
axis background/.style={fill=white},
legend style={at={(0.03,0.03)}, anchor=south west, legend cell align=left, align=left}
]
\addplot [color=white!70!black, forget plot, line width=0.75pt]
  table[row sep=crcr]{%
4	6.4e-05\\
256	1e-06\\
};
\addplot [color=white!70!black, forget plot, line width=0.75pt]
  table[row sep=crcr]{%
4	0.004096\\
256	1e-06\\
};
\addplot [color=mycolor1, mark=x, mark options={solid, mycolor1}, line width=0.75pt]
  table[row sep=crcr]{%
4	0.010134474\\
8	0.00824868818\\
16	0.00116892786\\
32	0.000702043982\\
64	0.000420463449\\
128	0.000239681668\\
256	0.000127809753\\
};
\addlegendentry{$p$}

\addplot [color=mycolor2, mark=x, mark options={solid, mycolor2}, line width=0.75pt]
  table[row sep=crcr]{%
4	0.000766547494\\
8	0.00735792955\\
16	0.000706862636\\
32	0.00061606567\\
64	0.000254939342\\
128	9.30517021e-05\\
256	2.60788059e-05\\
};
\addlegendentry{$u$}

\addplot [color=mycolor3, mark=x, mark options={solid, mycolor3}, line width=0.75pt]
  table[row sep=crcr]\\
	\subfloat[][Kovasznay, VanLeer, $Re = 80$]{
		% This file was created by matlab2tikz.
%
\definecolor{mycolor1}{rgb}{0.00000,0.44700,0.74100}%
\definecolor{mycolor2}{rgb}{0.85000,0.32500,0.09800}%
\definecolor{mycolor3}{rgb}{0.92900,0.69400,0.12500}%
%
\begin{tikzpicture}

\begin{axis}[%
width=0.951\figwidth,
height=0.75\figwidth,
at={(0\figwidth,0\figwidth)},
scale only axis,
xmode=log,
xmin=4,
xmax=256,
xminorticks=true,
ymode=log,
ymin=1e-05,
ymax=1,
yminorticks=true,
axis background/.style={fill=white},
legend style={at={(0.03,0.03)}, anchor=south west, legend cell align=left, align=left, draw=white!15!black}
]
\addplot [color=mycolor1, mark=x, mark options={solid, mycolor1}]
  table[row sep=crcr]{%
4	0.465362151\\
8	0.217919739\\
16	0.139676226\\
32	0.108111944\\
64	0.0720073262\\
128	0.041719522\\
256	0.0223197113\\
};
\addlegendentry{$p$}

\addplot [color=mycolor2, mark=x, mark options={solid, mycolor2}]
  table[row sep=crcr]{%
4	0.0369687113\\
8	0.0508536301\\
16	0.0240504197\\
32	0.00891876699\\
64	0.00230043057\\
128	0.000492539951\\
256	9.46578832e-05\\
};
\addlegendentry{$u$}

\addplot [color=mycolor3, mark=x, mark options={solid, mycolor3}]
  table[row sep=crcr]{%
4	0.0170184173\\
8	0.0249081905\\
16	0.00786909699\\
32	0.00254735352\\
64	0.000629680162\\
128	0.000123018505\\
256	2.07816951e-05\\
};
\addlegendentry{$v$}

\addplot [color=white!70!black, forget plot]
  table[row sep=crcr]{%
4	0.00064\\
256	1e-05\\
};
\addplot [color=white!70!black, forget plot]
  table[row sep=crcr]
	\subfloat[][Kovasznay, VanLeer, $Re = 8000$]{
		% This file was created by matlab2tikz.
%
\definecolor{mycolor1}{rgb}{0.00000,0.44700,0.74100}%
\definecolor{mycolor2}{rgb}{0.85000,0.32500,0.09800}%
\definecolor{mycolor3}{rgb}{0.92900,0.69400,0.12500}%
%
\begin{tikzpicture}

\begin{axis}[%
width=0.951\figwidth,
height=0.75\figwidth,
at={(0\figwidth,0\figwidth)},
scale only axis,
xmode=log,
xmin=4,
xmax=256,
xminorticks=true,
ymode=log,
ymin=1e-06,
ymax=0.010132924,
yminorticks=true,
axis background/.style={fill=white},
legend style={at={(0.03,0.03)}, anchor=south west, legend cell align=left, align=left}
]
\addplot [color=mycolor1, mark=x, mark options={solid, mycolor1}]
  table[row sep=crcr]{%
4	0.010132924\\
8	0.00680316149\\
16	0.000677901356\\
32	0.000521130225\\
64	0.000372827168\\
128	0.000226965042\\
256	0.000124578854\\
};
\addlegendentry{$p$}

\addplot [color=mycolor2, mark=x, mark options={solid, mycolor2}]
  table[row sep=crcr]{%
4	0.000752792347\\
8	0.00792052753\\
16	0.000772150807\\
32	0.000461995506\\
64	0.000232426927\\
128	8.93024538e-05\\
256	2.52578072e-05\\
};
\addlegendentry{$u$}

\addplot [color=mycolor3, mark=x, mark options={solid, mycolor3}]
  table[row sep=crcr]{%
4	0.00040159829\\
8	0.00256437192\\
16	0.000197529591\\
32	0.000108825618\\
64	4.65481791e-05\\
128	1.4700347e-05\\
256	3.78297463e-06\\
};
\addlegendentry{$v$}

\addplot [color=white!70!black, forget plot]
  table[row sep=crcr]{%
4	6.4e-05\\
256	1e-06\\
};
\addplot [color=white!70!black, forget plot]
  table[row sep=crcr]
    \caption{Errors for the Kovasznay problem.}
    \label{fig:kov_err}
\end{figure}

\subsection{Sin-Cos problem}
The Sin-Cos problem is solved over a domain $\Omega = [0,1] \times [0,1]$:
\begin{align*}
\nabla \cdot (\rho \mathbf{v}) =& \; 0 \quad \text{in} \quad \Omega\\
\nabla \cdot (\rho \mathbf{v} \mathbf{v^\mathrm{T}}) - \nabla \cdot (\mu 
(\nabla 
\mathbf{v} + \nabla \mathbf{v}^\mathrm{T})) + 
\nabla p =& \; \mathbf{f} \quad \text{in} \quad \Omega\\
\mathbf{v} =& \; \mathbf{v}_{ex} \quad \text{on} \quad \partial \Omega\\
p =& \; p_{ex} \quad \text{on the lower let cell}
\end{align*}
with the the source term:
\begin{align*}
&f_1 = -2 \mu cos(x) sin(y)\\
&f_2 =  2 \mu cos(y) sin(x)
\end{align*}
such that the exact solution is:
\begin{align*}
&v_{x,ex} = -cos(x) sin(x)\\
&v_{y,ex} = sin(x) cos(y)\\
&p_{ex} = -\frac{1}{4} (cos(2x)+sin(2y)
\end{align*}
We have chosen $\rho = 1$ and $\mu = 1$ or $ \mu = 10^{-3}$.
We have made five refinements starting from a grid of $4 \times 4$ cells.
We have obtained the following convergence orders:
\begin{table}[h]
	\centering
	\[
	\begin{array}{c|ccc}
	\toprule
	& \text{Upwind} & \text{MinMod} & \text{VanLeer} \\ 
	\midrule
	p & 1.008 & 1.569 & 1.626\\
	v_x & 1.143 & 1.962 & 1.977\\
	v_y & 1.058 & 1.928 & 1.945\\
	\bottomrule
	\end{array}
	\]
	\caption{Convergence order with $Re = 1$ for the Sin-Cos problem.}
	\label{tab:sin_lre}
	
	\[
	\begin{array}{c|ccc}
	\toprule
	& \text{Upwind} & \text{MinMod} & \text{VanLeer} \\ 
	\midrule
	p & 1.148 & 1.659 & 1.058\\
	v_x & 1.071 & 1.441 & 1.437\\
	v_y & 1.068 & 1.533 & 1.560\\
	\bottomrule
	\end{array}
	\]
	\caption{Convergence order with $Re = 1000$ for the Sin-Cos problem.}
	\label{tab:sin_hre}
\end{table}

\begin{figure}[t]
	\centering
	\subfloat[][SinCos, Upwind, $Re = 1$]{
		% This file was created by matlab2tikz.
%
\definecolor{mycolor1}{rgb}{0.00000,0.44700,0.74100}%
\definecolor{mycolor2}{rgb}{0.85000,0.32500,0.09800}%
\definecolor{mycolor3}{rgb}{0.92900,0.69400,0.12500}%
%
\begin{tikzpicture}

\begin{axis}[%
width=0.951\figwidth,
height=0.75\figwidth,
at={(0\figwidth,0\figwidth)},
scale only axis,
xmode=log,
xmin=4,
xmax=100,
xminorticks=true,
ymode=log,
ymin=1e-05,
ymax=0.1,
yminorticks=true,
axis background/.style={fill=white},
legend style={at={(0.03,0.03)}, anchor=south west, legend cell align=left, align=left, draw=white!15!black}
]
\addplot [color=mycolor1, mark=x, mark options={solid, mycolor1}]
  table[row sep=crcr]{%
4	0.0377253281\\
8	0.016292839\\
16	0.00647505826\\
32	0.00277859939\\
64	0.00131783073\\
};
\addlegendentry{$u$}

\addplot [color=mycolor2, mark=x, mark options={solid, mycolor2}]
  table[row sep=crcr]{%
4	0.00100309389\\
8	0.000409725552\\
16	0.000134946447\\
32	5.06672786e-05\\
64	2.29407924e-05\\
};
\addlegendentry{$u$}

\addplot [color=mycolor3, mark=x, mark options={solid, mycolor3}]
  table[row sep=crcr]{%
4	0.000631316968\\
8	0.000278812074\\
16	0.00010907144\\
32	4.76177485e-05\\
64	2.28652417e-05\\
};
\addlegendentry{$v$}

\addplot [color=white!70!black, forget plot]
  table[row sep=crcr]{%
4	0.00025\\
100	1e-05\\
};
\addplot [color=white!70!black, forget plot]
  table[row sep=crcr]
	\subfloat[][SinCos, Upwind, $Re = 1000$]{
		\input{img/l2error_test_navierstokes_sincos_hre}}\\
	\subfloat[][SinCos, MinMod, $Re = 1$]{
		% This file was created by matlab2tikz.
%
\definecolor{mycolor1}{rgb}{0.00000,0.44700,0.74100}%
\definecolor{mycolor2}{rgb}{0.85000,0.32500,0.09800}%
\definecolor{mycolor3}{rgb}{0.92900,0.69400,0.12500}%
%
\begin{tikzpicture}

\begin{axis}[%
width=0.951\figwidth,
height=0.75\figwidth,
at={(0\figwidth,0\figwidth)},
scale only axis,
xmode=log,
xmin=4,
xmax=100,
xminorticks=true,
ymode=log,
ymin=1e-05,
ymax=0.1,
yminorticks=true,
axis background/.style={fill=white},
legend style={at={(0.97,0.97)}, anchor=north east, legend cell align=left, align=left}
]
\addplot [color=mycolor1, mark=x, mark options={solid, mycolor1}]
  table[row sep=crcr]{%
4	0.0201961443\\
8	0.00658170291\\
16	0.00181646151\\
32	0.000558641084\\
64	0.000188245776\\
};
\addlegendentry{$p$}

\addplot [color=mycolor2, mark=x, mark options={solid, mycolor2}]
  table[row sep=crcr]{%
4	0.00148598782\\
8	0.00053497283\\
16	0.000150385476\\
32	3.95894587e-05\\
64	1.01636468e-05\\
};
\addlegendentry{$u$}

\addplot [color=mycolor3, mark=x, mark options={solid, mycolor3}]
  table[row sep=crcr]{%
4	0.00119526142\\
8	0.00045470553\\
16	0.000138075359\\
32	3.80865907e-05\\
64	1.00057526e-05\\
};
\addlegendentry{$v$}

\addplot [color=white!70!black, forget plot]
  table[row sep=crcr]{%
4	0.00025\\
100	1e-05\\
};
\addplot [color=white!70!black, forget plot]
  table[row sep=crcr]
	\subfloat[][SinCos, MinMod, $Re = 1000$]{
		\input{img/l2error_test_navierstokes_sincos_mm_hre}}\\
	\subfloat[][SinCos, VanLeer, $Re = 1$]{
		% This file was created by matlab2tikz.
%
\definecolor{mycolor1}{rgb}{0.00000,0.44700,0.74100}%
\definecolor{mycolor2}{rgb}{0.85000,0.32500,0.09800}%
\definecolor{mycolor3}{rgb}{0.92900,0.69400,0.12500}%
%
\begin{tikzpicture}

\begin{axis}[%
width=0.951\figwidth,
height=0.75\figwidth,
at={(0\figwidth,0\figwidth)},
scale only axis,
xmode=log,
xmin=4,
xmax=100,
xminorticks=true,
ymode=log,
ymin=8.63199669e-06,
ymax=0.1,
yminorticks=true,
axis background/.style={fill=white},
legend style={at={(0.03,0.03)}, anchor=south west, legend cell align=left, align=left, draw=white!15!black}
]
\addplot [color=mycolor1, mark=x, mark options={solid, mycolor1}]
  table[row sep=crcr]{%
4	0.017335487\\
8	0.0062618753\\
16	0.00193371119\\
32	0.000605597659\\
64	0.000196206927\\
};
\addlegendentry{$p$}

\addplot [color=mycolor2, mark=x, mark options={solid, mycolor2}]
  table[row sep=crcr]{%
4	0.00136409678\\
8	0.000476275877\\
16	0.000131606902\\
32	3.41242821e-05\\
64	8.66824542e-06\\
};
\addlegendentry{$u$}

\addplot [color=mycolor3, mark=x, mark options={solid, mycolor3}]
  table[row sep=crcr]{%
4	0.00114596769\\
8	0.000417513208\\
16	0.000122988533\\
32	3.32359157e-05\\
64	8.63199669e-06\\
};
\addlegendentry{$v$}

\addplot [color=white!70!black, forget plot]
  table[row sep=crcr]{%
4	0.00021579991725\\
100	8.63199669e-06\\
};
\addplot [color=white!70!black, forget plot]
  table[row sep=crcr]
	\subfloat[][SinCos, VanLeer, $Re = 1000$]{
		% This file was created by matlab2tikz.
%
\definecolor{mycolor1}{rgb}{0.00000,0.44700,0.74100}%
\definecolor{mycolor2}{rgb}{0.85000,0.32500,0.09800}%
\definecolor{mycolor3}{rgb}{0.92900,0.69400,0.12500}%
%
\begin{tikzpicture}

\begin{axis}[%
width=0.951\figwidth,
height=0.75\figwidth,
at={(0\figwidth,0\figwidth)},
scale only axis,
xmode=log,
xmin=4,
xmax=100,
xminorticks=true,
ymode=log,
ymin=0.0001,
ymax=0.01454297,
yminorticks=true,
axis background/.style={fill=white},
legend style={at={(0.03,0.03)}, anchor=south west, legend cell align=left, align=left, draw=white!15!black}
]
\addplot [color=mycolor1, mark=x, mark options={solid, mycolor1}]
  table[row sep=crcr]{%
4	0.01454297\\
8	0.00653869074\\
16	0.00224775854\\
32	0.0005196908\\
64	0.000249600967\\
};
\addlegendentry{$p$}

\addplot [color=mycolor2, mark=x, mark options={solid, mycolor2}]
  table[row sep=crcr]{%
4	0.0107200353\\
8	0.00466675248\\
16	0.00177426299\\
32	0.000645881865\\
64	0.000238530337\\
};
\addlegendentry{$u$}

\addplot [color=mycolor3, mark=x, mark options={solid, mycolor3}]
  table[row sep=crcr]{%
4	0.0112112271\\
8	0.00583504088\\
16	0.00263953931\\
32	0.00102531826\\
64	0.000347747564\\
};
\addlegendentry{$v$}

\addplot [color=white!70!black, forget plot]
  table[row sep=crcr]{%
4	0.0025\\
100	0.0001\\
};
\addplot [color=white!70!black, forget plot]
  table[row sep=crcr]
	\caption{Errors for the Sin-Cos problem.}
	\label{fig:sin_err}
\end{figure}

\subsection{Angeli problem}
The Angeli problem is solved over a domain $\Omega = [0,1] \times [0,1]$ and in 
a time interval $[0, T] = [0, 2 \cdot 10^{-12}]$:
\begin{align*}
\frac{\partial \rho}{\partial t} + \nabla \cdot (\rho \mathbf{v}) 
=& \; 0 \quad \text{in} \quad \Omega \times (0, T)\\
\frac{\partial (\rho \mathbf{v})}{\partial t} + \nabla \cdot (\rho \mathbf{v} 
\mathbf{v^\mathrm{T}}) - \nabla \cdot (\mu 
(\nabla 
\mathbf{v} + \nabla \mathbf{v}^\mathrm{T})) + 
\nabla p =& \; 0 \quad \text{in} \quad \Omega \times (0, T)\\
\mathbf{v} =& \; \mathbf{v}_{ex} \quad \text{on} \quad \partial \Omega \times 
(0, T)\\
p =& \; p_{ex} \quad \text{on} \quad \partial \Omega \times (0, T)\\
\mathbf{v} =& \; \mathbf{v}_{ex} \quad \text{in} \quad \Omega, \quad 
\text{at} \quad t=0\\
p =& \; p_{ex} \quad \text{in} \quad \Omega, \quad 
\text{at} \quad t=0
\end{align*}
The exact solution is:
\begin{align*}
&v_{x,ex} = -2 \pi e^{-5 \mu \pi^2 t} cos(\pi x) sin(2 \pi y)\\
&v_{y,ex} = \pi e^{-5 \mu \pi^2 t} sin(\pi x) cos(2 \pi y)\\
&p_{ex} = -\frac{1}{4} e^{-10 \mu \pi^2 t} \pi^2 (4 cos(2 \pi x) + cos(4 \pi y)
\end{align*}
We have chosen $\rho = 1$ and $\mu = 1$ or $ \mu = 10^{-3}$.
We have made four refinements starting from a grid of $4 \times 4$ cells.
We have obtained the following convergence orders:
\begin{table}[h]
	\centering
	\[
	\begin{array}{c|ccc}
	\toprule
	& \text{Upwind} & \text{MinMod} & \text{VanLeer} \\ 
	\midrule
	p & 0.964 & 1.037 & 0.996\\
	v_x & 1.977 & 1.977 & 1.977\\
	v_y & 1.975 & 1.975 & 1.975\\
	\bottomrule
	\end{array}
	\]
	\caption{Convergence order with $Re=1$ for the Angeli problem.}
	\label{tab:ang_lre}
	
	\[
	\begin{array}{c|ccc}
	\toprule
	& \text{Upwind} & \text{MinMod} & \text{VanLeer} \\ 
	\midrule
	p & 1.001 & 1.931 & 1.936\\
	v_x & 1.977 & 1.977 & 1.977\\
	v_y & 1.975 & 1.975 & 1.975\\
	\bottomrule
	\end{array}
	\]
	\caption{Convergence order with $Re = 1000$ for the Angeli problem.}
	\label{tab:ang_hre}
\end{table}

\begin{figure}[t]
	\centering
	\subfloat[][Angeli, Upwind, $Re = 1$]{
		% This file was created by matlab2tikz.
%
\definecolor{mycolor1}{rgb}{0.00000,0.44700,0.74100}%
\definecolor{mycolor2}{rgb}{0.85000,0.32500,0.09800}%
\definecolor{mycolor3}{rgb}{0.92900,0.69400,0.12500}%
%
\begin{tikzpicture}

\begin{axis}[%
width=0.951\figwidth,
height=0.75\figwidth,
at={(0\figwidth,0\figwidth)},
scale only axis,
xmode=log,
xmin=4,
xmax=100,
xminorticks=true,
ymode=log,
ymin=0.0001,
ymax=1,
yminorticks=true,
axis background/.style={fill=white},
legend style={at={(0.03,0.03)}, anchor=south west, legend cell align=left, align=left, draw=white!15!black}
]
\addplot [color=mycolor1, mark=x, mark options={solid, mycolor1}]
  table[row sep=crcr]{%
4	0.858048474\\
8	0.471968875\\
16	0.266272176\\
32	0.139796737\\
64	0.0716425946\\
};
\addlegendentry{$p$}

\addplot [color=mycolor2, mark=x, mark options={solid, mycolor2}]
  table[row sep=crcr]{%
4	0.0298924376\\
8	0.0104567975\\
16	0.00283500097\\
32	0.000733232423\\
64	0.000186287549\\
};
\addlegendentry{$u$}

\addplot [color=mycolor3, mark=x, mark options={solid, mycolor3}]
  table[row sep=crcr]{%
4	0.0517752209\\
8	0.020358345\\
16	0.00563625816\\
32	0.0014643974\\
64	0.000372447694\\
};
\addlegendentry{$v$}

\addplot [color=white!70!black, forget plot]
  table[row sep=crcr]{%
4	0.0025\\
100	0.0001\\
};
\addplot [color=white!70!black, forget plot]
  table[row sep=crcr]
	\subfloat[][Angeli, Upwind, $Re = 1000$]{
		% This file was created by matlab2tikz.
%
\definecolor{mycolor1}{rgb}{0.00000,0.44700,0.74100}%
\definecolor{mycolor2}{rgb}{0.85000,0.32500,0.09800}%
\definecolor{mycolor3}{rgb}{0.92900,0.69400,0.12500}%
%
\begin{tikzpicture}

\begin{axis}[%
width=0.951\figwidth,
height=0.75\figwidth,
at={(0\figwidth,0\figwidth)},
scale only axis,
xmode=log,
xmin=4,
xmax=100,
xminorticks=true,
ymode=log,
ymin=0.0001,
ymax=1,
yminorticks=true,
axis background/.style={fill=white},
legend style={at={(0.03,0.03)}, anchor=south west, legend cell align=left, align=left, draw=white!15!black}
]
\addplot [color=mycolor1, mark=x, mark options={solid, mycolor1}]
  table[row sep=crcr]{%
4	0.613350181\\
8	0.44032744\\
16	0.224313492\\
32	0.110974562\\
64	0.0554588758\\
};
\addlegendentry{$p$}

\addplot [color=mycolor2, mark=x, mark options={solid, mycolor2}]
  table[row sep=crcr]{%
4	0.0298924377\\
8	0.0104567976\\
16	0.00283500112\\
32	0.000733232574\\
64	0.000186287701\\
};
\addlegendentry{$u$}

\addplot [color=mycolor3, mark=x, mark options={solid, mycolor3}]
  table[row sep=crcr]{%
4	0.0517752208\\
8	0.0203583449\\
16	0.00563625809\\
32	0.00146439732\\
64	0.000372447618\\
};
\addlegendentry{$v$}

\addplot [color=white!70!black, forget plot]
  table[row sep=crcr]{%
4	0.0025\\
100	0.0001\\
};
\addplot [color=white!70!black, forget plot]
  table[row sep=crcr]\\
	\subfloat[][Angeli, MinMod, $Re = 1$]{
		% This file was created by matlab2tikz.
%
\definecolor{mycolor1}{rgb}{0.00000,0.44700,0.74100}%
\definecolor{mycolor2}{rgb}{0.85000,0.32500,0.09800}%
\definecolor{mycolor3}{rgb}{0.92900,0.69400,0.12500}%
%
\begin{tikzpicture}

\begin{axis}[%
width=0.951\figwidth,
height=0.75\figwidth,
at={(0\figwidth,0\figwidth)},
scale only axis,
xmode=log,
xmin=4,
xmax=100,
xminorticks=true,
ymode=log,
ymin=0.0001,
ymax=1,
yminorticks=true,
axis background/.style={fill=white},
legend style={at={(0.03,0.03)}, anchor=south west, legend cell align=left, align=left}
]
\addplot [color=mycolor1, mark=x, mark options={solid, mycolor1}]
  table[row sep=crcr]{%
4	0.528626373\\
8	0.425133414\\
16	0.183714499\\
32	0.0842731543\\
64	0.0410645371\\
};
\addlegendentry{$p$}

\addplot [color=mycolor2, mark=x, mark options={solid, mycolor2}]
  table[row sep=crcr]{%
4	0.0298924376\\
8	0.0104567975\\
16	0.00283500096\\
32	0.000733232417\\
64	0.000186287546\\
};
\addlegendentry{$u$}

\addplot [color=mycolor3, mark=x, mark options={solid, mycolor3}]
  table[row sep=crcr]{%
4	0.0517752209\\
8	0.020358345\\
16	0.00563625817\\
32	0.0014643974\\
64	0.000372447696\\
};
\addlegendentry{$v$}

\addplot [color=white!70!black, forget plot]
  table[row sep=crcr]{%
4	0.0025\\
100	0.0001\\
};
\addplot [color=white!70!black, forget plot]
  table[row sep=crcr]
	\subfloat[][Angeli, MinMod, $Re = 1000$]{
		\input{img/l2error_test_navierstokes_angeli_mm_hre}}\\
	\subfloat[][Angeli, VanLeer, $Re = 1$]{
		\input{img/l2error_test_navierstokes_angeli_vl}}
	\subfloat[][Angeli, VanLeer, $Re = 1000$]{
		% This file was created by matlab2tikz.
%
\definecolor{mycolor1}{rgb}{0.00000,0.44700,0.74100}%
\definecolor{mycolor2}{rgb}{0.85000,0.32500,0.09800}%
\definecolor{mycolor3}{rgb}{0.92900,0.69400,0.12500}%
%
\begin{tikzpicture}

\begin{axis}[%
width=0.951\figwidth,
height=0.75\figwidth,
at={(0\figwidth,0\figwidth)},
scale only axis,
xmode=log,
xmin=4,
xmax=100,
xminorticks=true,
ymode=log,
ymin=0.0001,
ymax=1,
yminorticks=true,
axis background/.style={fill=white},
legend style={at={(0.03,0.03)}, anchor=south west, legend cell align=left, align=left, draw=white!15!black}
]
\addplot [color=mycolor1, mark=x, mark options={solid, mycolor1}]
  table[row sep=crcr]{%
4	0.338989838\\
8	0.230661031\\
16	0.0614189286\\
32	0.0161562754\\
64	0.00422232945\\
};
\addlegendentry{$p$}

\addplot [color=mycolor2, mark=x, mark options={solid, mycolor2}]
  table[row sep=crcr]{%
4	0.0298924377\\
8	0.0104567976\\
16	0.00283500111\\
32	0.000733232568\\
64	0.000186287698\\
};
\addlegendentry{$u$}

\addplot [color=mycolor3, mark=x, mark options={solid, mycolor3}]
  table[row sep=crcr]{%
4	0.0517752208\\
8	0.020358345\\
16	0.0056362581\\
32	0.00146439733\\
64	0.00037244762\\
};
\addlegendentry{$v$}

\addplot [color=white!70!black, forget plot]
  table[row sep=crcr]{%
4	0.0025\\
100	0.0001\\
};
\addplot [color=white!70!black, forget plot]
  table[row sep=crcr]
	\caption{Errors for the Angeli problem.}
	\label{fig:ang_err}
\end{figure}

\section{Conclusions}
In the 1D problem for low $Re$ (Table \ref{tab:1d_lre}) the results of the 
three 
methods are very similar, the convergence order for the pressure is always 1 
while the convergence order for the velocity is always 2. It could be that 
since $Re$ is low the diffusive term dominates over the convective term so that 
also the upwind method has an order of two for the velocity. At higher 
$Re$ we can see differences among the methods (Table \ref{tab:1d_hre}), with 
the TVD 
that behaves better then the upwind obtaining a  higher convergence order. In 
particular the VanLeer seems to be 
better then MinMod because it has a full second order for both the pressure and 
the velocity. Looking at Figure \ref{fig:1d_err} we can see the global trends 
of the errors and we can notice that the absolute values are much using the TVD 
methods at high $Re$, while they are similar at low $Re$.

The steady channel test shows identical results for all the methods (Tables 
\ref{tab:cha_lre} and \ref{tab:cha_hre}). Due to the 
fact that the velocity in the $x$ direction does not depend on $x$ and that the 
velocity in the $y$ direction is zero, we have that the result is independent 
of the method used because they are all exact. The second order convergence is 
determined by other terms of the equations.

In the Kovasznay test the MinMod and VanLeer produce almost the same result 
(Table \ref{tab:kov_lre}). Already at low $Re$ they have a second order for the 
velocities while the upwind has only a first order. For The pressure all the 
three methods have present a first order. At high $Re$ the conclusions are 
analogous.

In the Sin-Cos problem we observe a better behaviour of the TVD methods 
already both at low $Re$ and at high $Re$ (Tables \ref{tab:sin_lre} and 
\ref{tab:sin_hre}) with an order two for the 
velocities and between one and two for the pressure. The VanLeer at high $Re$ 
shows an order of one for the pressure, but, looking at Figure 
\ref{fig:sin_err}, we can see that the global trend is the same of the 
velocities and the same of the MinMod. Notice that the absolute values are 
lower the in the upwind case.

Eventually in the Angeli test we obtain always second order for the velocities 
but now we obtain clearly a second order also for the pressure with the TVD 
methods at high $Re$ (Table \ref{tab:ang_hre}).

In general if we employ a TVD method we obtain always higher or equal 
convergence rates than if we employ the upwind method. With the TVD we obtain 
always a second order for the velocities, while with the upwind sometimes we 
have it at low $Re$, but then it is lost at high $Re$ (only in the Angeli test 
it is maintained). For the pressure we have always a first order with the 
upwind method, while with the TVD we generally can obtain a higher order at 
least high $Re$, as it is happens in all the cases but in the Kovasznay's one.
Regarding the absolute values of the error, we have obtained lower ones using 
the TVD methods, since the error is reduced more as we refine the grid because 
of the higher convergence orders. Between the MinMod and the VanLeer it is hard 
so see if one is better than the other, the results are always very similar.
\end{document}