\documentclass[11pt, a4paper]{article}
\usepackage[british]{babel}
\usepackage[utf8]{inputenc}
\usepackage{amsmath}
\usepackage{amssymb}
\usepackage{amsthm}
\usepackage{booktabs}
\usepackage{subfig}
\usepackage{bbold}
\usepackage[hidelinks]{hyperref}
\usepackage{graphicx}
\usepackage{xspace}
\usepackage{pgfplots}
\pgfplotsset{compat=newest}
%\usetikzlibrary{plotmarks}
%\usetikzlibrary{arrows.meta}
%\usepgfplotslibrary{patchplots}
%\usepackage{grffile}
\usepackage[section]{placeins}
\usepackage{tikz}
\usepackage{pgfgantt}
\usepackage{pdflscape}
\usepackage[separate-uncertainty=true]{siunitx}
\pgfplotsset{plot coordinates/math parser=false}

\newlength{\figwidth}
\setlength{\figwidth}{0.45\textwidth}

\newcommand{\DUMUX}{DuMu$^\mathrm{x}$\xspace} %for convenience

\graphicspath{{img/}}

\theoremstyle{definition}
\newtheorem{defn}{Definition}

\title{\textbf{Backward facing step}}
\author{Andrea Vescovini}
\date{\today}

\begin{document}
	\maketitle
	
\section{Introduction}
The backward facing step is a widely-used test configuration in which a 
turbulent flow in a channel encounters separation because of a sudden 
enlargement; in Figure \ref{fig:bfs_glimphs} the velocity field is represented.
\begin{figure}[h]
	\centering
	\includegraphics[width=\textwidth]{bfs_glimphs}
	\caption{Velocity field after the backward facing step.}
	\label{fig:bfs_glimphs}
\end{figure}
We want to use this case to validate the results of our 
$k\text{-}\omega$ model, comparing them with experimental results from 
\cite{driver} and numerical results from \cite{sitonasa}.

\section{The test}
The domain is chosen as in Figure~\ref{fig:bfs_domain}, with a step height $H = 
\SI{1}{m}$. 
\begin{figure}[h]
	\centering
	\includegraphics[width=\textwidth]{bfs_domain}
	\caption{Domain of the backward facing step test.}
	\label{fig:bfs_domain}
\end{figure}
The following boundary conditions are imposed:
\begin{itemize}
	\item On the wall boundary $\Gamma_w$ we set no slip conditions for the 
	velocity 
	$\mathbf{v}$, a homogeneous Dirichlet condition for the turbulent kinetic 
	energy $k$ and a Dirichlet condition for the turbulent variable $\omega$ 
	such that:
	\begin{equation*}
	\omega = \frac{6 \nu}{\beta_\omega l_w^2}
	\end{equation*}
	where $\beta_\omega$ is a constant equal to 0.0708 and $l_w$ is the 
	distance from the wall. This formula implies that $\omega \rightarrow \inf$ 
	as $l_w \rightarrow 0$, so it is imposed at the centre of the cells that 
	touch the wall to avoid infinite values.
	%
	\item On the inflow boundary $\Gamma_{in}$ we set a Dirichlet condition for 
	the velocity, imposing a flat profile $\mathbf{v} = [U_{in}, 
	0]^\mathrm{T}$. For $k$ and $\omega$ we impose Dirichlet conditions such 
	that:
	\begin{equation} \label{bc:ko}
	k = \frac{3}{2} (U_{in}I)^2, \quad \omega = \frac{k^{1/2}}{C_{\mu}^{1/4}L}
	\end{equation}
	where $I = 0.16 Re^{-\frac{1}{4}}$ is the turbulence intensity, $C_\mu = 
	0.09$ and $L = \alpha H$ is a turbulence length scale, with 
	$\alpha = 0.07$.
	%
	\item On the outflow boundary $\Gamma_{out}$ we set a zero-gradient 
	condition for the velocity and a Dirichlet condition for the pressure $p = 
	p_{ext} = \SI{1.1e5}{Pa}$. For $k$ and $\omega$ we impose 
	Dirichlet conditions analogous to 
	\eqref{bc:ko}.
\end{itemize}

Notice that with respect to \cite{sitonasa} we do not use any symmetry 
condition near the inlet of the channel, but we start suddenly with a wall. For 
this reason we have adapted the length of the channel before the step in order 
to obtain at the position $x/H = -1$ a boundary layer thickness of $1.5H$, as 
it 
is in \cite{driver} and \cite{sitonasa}. In this way we have a correct velocity 
profile before the step.

In order to reproduce the results from \cite{driver} and \cite{sitonasa} we 
need 
a Reynolds number based on the step height of approximately 36000. Therefore we 
choose 
$U_{in} 
= \SI{44.2}{m/s}$ as in \cite{driver}, $\varrho = \SI{1}{kg/m^3}$ and $\mu 
= \SI{1.228e-3}{\pascal\second}$.

The results are adimensionalized using the step height $H$ as reference length 
and the velocity near the centre of the channel at $x/H = -4$ as a reference 
velocity $U_{ref}$.

For what concerns the time discretization we simulate the time interval $[0, 
T_{end}]$, with $T_{end} = \SI{30}{s}$. At approximately $t = \SI{20}{s}$ a 
stationary solution is reached, then we compare the results at $T_{end}$. We 
start the simulation with $\Delta t = \SI{e-5}{s}$, the the time 
step is adapted depending on the convergence of the Newton method at the 
previous time iteration.

\section{Grid analysis}
Several grids with different numbers of cells and different gradings have been 
tried. In order to compare the results we have compared the velocity profiles 
before and after the step, at $x/H = -4$ and $x/H = 1$, and the friction 
coefficient $C_f$ after the step. $C_f$ is a dimensionless form of the stress 
at the wall as is it defined by:
\begin{equation*}
C_f = \frac{\tau_w}{\frac{1}{2}\varrho U_{ref}^2}
\end{equation*}
In out case when $C_f < 0$ the flow is recirculating, so from its value we can 
predict the point where the flow \emph{reattaches}, as $C_f$ becomes positive; 
according to \cite{sitonasa} this is a relevant information about the 
reliability of the simulation.\\
The numerical reference predicts the reattachment near $x/H=6.8$; the 
experimental measure of the reattachment position was $x/H = \num{6.26+-0.10}$.

\begin{itemize}
	\item With coarse mesh the reattachment point was lower, a refinement of 
	the smallest cell in the $y$ direction was needed.
	\item Too stretched cells leaded to problems in the Newton convergence.
	\item I could coarsen cells in the $x$ direction in the downstream portion 
	of the channel.
	\item A refinement of the cells in the $y$ direction in the recirculation 
	area was useless.
\end{itemize}
\section{Results comparison}
The following results are obtained with a grid made of \num{121 x 70} cells, 
employing gradings up to 1.25 in order to better refine the grid near the 
walls. In general we have obtained a good agreement with the numerical solution 
from \cite{sitonasa}, in particular the Van Leer result is better than the 
Upwind one.
\begin{itemize}
	\item The position of main eddy is slightly anticipated, the profile at 
	$x/H = 1$ is that one that the reference has at $x/H = 1.2$ or something 
	like that, we can see it also from the $C_f$ plot.
	% 
	\item However the reattachment point is almost the same, but looking at the 
	$C_f$ plot is seems that out eddy is a bit bigger.
	%
	\item For the upwind this behaviour is enhanced, showing in the test plot 
	different results. (Reattachment point and small eddies in the $C_f$ plot)
	%
	\item Experimental results are sometimes different, in particular in the 
	velocity profiles near the wall at $x/H=6$ and $x/H=10$.
	%
	\item The value of $C_f$ going to the end of the channel.
\end{itemize}
\begin{figure}
	\centering
	\subfloat[Velocity profiles at $x/H = -4$]{% This file was created by matlab2tikz.
%
\definecolor{mycolor4}{rgb}{0.00000,0.44700,0.74100}%
\definecolor{mycolor3}{rgb}{0.85000,0.32500,0.09800}%
\definecolor{mycolor2}{rgb}{0.92900,0.69400,0.12500}%
\definecolor{mycolor1}{rgb}{0.49400,0.18400,0.55600}%
%
\begin{tikzpicture}

\begin{axis}[%
width=0.958\bfshalfwidth,
height=0.75\bfsheight,
at={(0\bfshalfwidth,0\bfsheight)},
scale only axis,
xmin=0,
xmax=1.2,
xlabel style={font=\color{white!15!black}},
xlabel={$u/U_\text{ref}$},
ymin=1,
ymax=4,
ylabel style={font=\color{white!15!black}, rotate=-90},
ylabel={$\dfrac{y}{H}$},
axis background/.style={fill=white},
legend style={at={(0.03,0.97)}, anchor=north west, legend cell align=left, align=left, draw=white!15!black, font=\scriptsize}
]
\addplot [color=mycolor1]
  table[row sep=crcr]{%
0.003994483035	1\\
0.008116972633	1.000150442\\
0.01663493924	1.000310659\\
0.0257113874	1.000481486\\
0.03538273647	1.000663638\\
0.04568762332	1.000857592\\
0.05666682497	1.00106442\\
0.06836281717	1.001284838\\
0.08081879467	1.001519799\\
0.09407651424	1.001770139\\
0.1081723496	1.002037048\\
0.123130694	1.002321362\\
0.1389538944	1.002624512\\
0.1556090117	1.00294745\\
0.1730137467	1.003291726\\
0.1910268217	1.003658652\\
0.2094493508	1.004049659\\
0.2280419767	1.004466295\\
0.2465548515	1.00491035\\
0.2647606432	1.005383611\\
0.282478869	1.005887985\\
0.2995863259	1.0064255\\
0.3160139918	1.006998301\\
0.331736654	1.007608771\\
0.3467606306	1.008259177\\
0.3611122668	1.008952498\\
0.3748295009	1.009691119\\
0.387955904	1.010478377\\
0.4005365372	1.011317253\\
0.4126155972	1.012211084\\
0.4242353439	1.013163567\\
0.4354348779	1.014178634\\
0.4462505281	1.0152601\\
0.4567151964	1.016412497\\
0.4668590128	1.017640352\\
0.4767090976	1.018948674\\
0.4862900376	1.020342588\\
0.4956241548	1.021827817\\
0.5047315359	1.023410201\\
0.5136304498	1.025096059\\
0.5223373771	1.026892185\\
0.5308673382	1.028805614\\
0.5392340422	1.030843973\\
0.5474497676	1.03301537\\
0.5555259585	1.035328388\\
0.5634730458	1.037792325\\
0.5713005066	1.040416718\\
0.5790169835	1.043212056\\
0.586630702	1.046189189\\
0.5941489339	1.049359918\\
0.6015787125	1.052736521\\
0.6089264154	1.05633235\\
0.6161980033	1.060161352\\
0.6233990192	1.06423831\\
0.6305348277	1.068579078\\
0.6376101971	1.073200345\\
0.6446298957	1.078119993\\
0.6515982151	1.083356857\\
0.6585193276	1.088930845\\
0.6653971076	1.094863057\\
0.6722354293	1.101176143\\
0.6790377498	1.107893705\\
0.6858073473	1.115041018\\
0.6925476193	1.122644544\\
0.6992616653	1.130732656\\
0.7059524059	1.139334917\\
0.7126229405	1.1484828\\
0.719276011	1.158209562\\
0.7259144783	1.168550134\\
0.7325410843	1.179541588\\
0.7391585112	1.191222668\\
0.7457693219	1.203634501\\
0.7523762584	1.216820002\\
0.7589818835	1.230824709\\
0.7655887604	1.245695949\\
0.7721994519	1.261483788\\
0.7788166404	1.278240323\\
0.7854428291	1.296020389\\
0.792080462	1.314881086\\
0.7987324595	1.33488214\\
0.8054010868	1.356085658\\
0.8120889664	1.378556371\\
0.8187987804	1.402361393\\
0.8255328536	1.427570581\\
0.8322937489	1.454256058\\
0.8390839696	1.48249197\\
0.8459054232	1.512355208\\
0.8527605534	1.543924451\\
0.8596510291	1.577280402\\
0.8665785193	1.612505317\\
0.8735439181	1.649683237\\
0.8805480599	1.688899159\\
0.8875908256	1.730238795\\
0.8946713805	1.773788691\\
0.9017875791	1.819635034\\
0.9089363217	1.867863655\\
0.916113019	1.918559432\\
0.9233118892	1.971805573\\
0.9305261374	2.02768302\\
0.9377582073	2.086269379\\
0.9451208711	2.147638798\\
0.9528451562	2.211860657\\
0.9610195756	2.278998852\\
0.9694645405	2.349110603\\
0.9778408408	2.422245979\\
0.9857334495	2.498446703\\
0.9926292896	2.577744484\\
0.9976097345	2.660161495\\
0.999725163	2.745707989\\
0.9999220371	2.834381342\\
0.999802053	2.926166773\\
0.9996836782	3.021034241\\
0.9995575547	3.118939638\\
0.9994264245	3.219822168\\
0.9992892146	3.323605776\\
0.9991465807	3.430197716\\
0.9989984632	3.539488077\\
0.998845458	3.65135026\\
0.9986877441	3.765640974\\
0.9985261559	3.882200718\\
0.9983609319	4.000854015\\
0.998192668	4.12141037\\
0.9980221391	4.243665218\\
0.997849822	4.3674016\\
0.9976763725	4.492391109\\
0.997502625	4.618394852\\
0.9973291755	4.745166302\\
0.9971565604	4.87245369\\
0.9969855547	5\\
0.9968168736	5.12754631\\
0.9966509342	5.254833698\\
0.9964883924	5.381605148\\
0.9963296652	5.507608891\\
0.9961752892	5.6325984\\
0.9960257411	5.756334782\\
0.9958810806	5.87858963\\
0.9957417846	5.999145985\\
0.9956079125	6.117799282\\
0.995480001	6.234358788\\
0.9953578115	6.348649502\\
0.9952415228	6.460511684\\
0.9951311946	6.569802284\\
0.9950268269	6.676393986\\
0.9949280024	6.78017807\\
0.9948350191	6.8810606\\
0.9947482347	6.978965759\\
0.9946722388	7.073832989\\
0.9946001172	7.16561842\\
0.9939549565	7.254292011\\
0.9911464453	7.339838505\\
0.9855794907	7.422255516\\
0.9782565236	7.501553535\\
0.9700331688	7.577754021\\
0.9614095688	7.650889397\\
0.9527845979	7.721001148\\
0.9444637299	7.788139343\\
0.9365931749	7.852361202\\
0.9290957451	7.913730621\\
0.9217606187	7.972317219\\
0.9144515395	8.028194427\\
0.9071487188	8.081440926\\
0.8998649716	8.132136345\\
0.8926087618	8.180364609\\
0.8853859305	8.226211548\\
0.8782002926	8.269761086\\
0.8710537553	8.31110096\\
0.8639472723	8.350317001\\
0.856880784	8.387495041\\
0.8498535156	8.422719955\\
0.8428639174	8.456075668\\
0.8359103203	8.487645149\\
0.8289906979	8.517508507\\
0.8221028447	8.545743942\\
0.8152442575	8.572429657\\
0.8084125519	8.59763813\\
0.8016052842	8.621443748\\
0.7948197126	8.643914223\\
0.7880535126	8.665118217\\
0.7813039422	8.685118675\\
0.7745687366	8.703979492\\
0.7678452134	8.721759796\\
0.7611309886	8.738515854\\
0.7544237971	8.754303932\\
0.747721076	8.769175529\\
0.7410206199	8.783180237\\
0.7343199849	8.796365738\\
0.7276168466	8.808777809\\
0.7209088206	8.820458412\\
0.7141935229	8.831449509\\
0.7074686885	8.841790199\\
0.7007317543	8.851517677\\
0.6939802766	8.860665321\\
0.6872118711	8.869267464\\
0.680423677	8.877355576\\
0.673613131	8.884959221\\
0.6667771935	8.892106056\\
0.659913063	8.898823738\\
0.6530176401	8.905137062\\
0.6460875869	8.911068916\\
0.6391192675	8.916643143\\
0.6321092248	8.921879768\\
0.6250534058	8.926799774\\
0.6179476976	8.93142128\\
0.6107875705	8.935761452\\
0.6035683155	8.939838409\\
0.5962848067	8.943667412\\
0.5889317393	8.947263718\\
0.5815031528	8.950639725\\
0.5739928484	8.953810692\\
0.5663939118	8.956788063\\
0.5586990118	8.959583282\\
0.5509002209	8.962207794\\
0.5429887772	8.964671135\\
0.5349552035	8.966984749\\
0.5267891884	8.969156265\\
0.5184792876	8.971194267\\
0.5100132227	8.973108292\\
0.5013773441	8.97490406\\
0.4925565124	8.976590157\\
0.4835344255	8.978172302\\
0.4742927849	8.979657173\\
0.4648114443	8.981051445\\
0.4550682902	8.982359886\\
0.4450387657	8.983587265\\
0.4346958101	8.984740257\\
0.4240095019	8.985821724\\
0.4129472077	8.986836433\\
0.4014729857	8.987789154\\
0.3895481527	8.988682747\\
0.3771312535	8.98952198\\
0.3641789854	8.990308762\\
0.3506478369	8.991047859\\
0.3364962935	8.99174118\\
0.3216893077	8.992391586\\
0.3062043786	8.993001938\\
0.290040642	8.993574142\\
0.273229748	8.994112015\\
0.2558479309	8.994616508\\
0.2380252481	8.995089531\\
0.2199462056	8.995533943\\
0.2018374801	8.995950699\\
0.1839422286	8.996341705\\
0.1664877385	8.996707916\\
0.1496582329	8.997052193\\
0.1335808486	8.997375488\\
0.1183264032	8.997678757\\
0.1039199308	8.997962952\\
0.09035436064	8.99822998\\
0.07760301977	8.998479843\\
0.06562872231	8.998715401\\
0.05438984558	8.998935699\\
0.04384382069	8.999142647\\
0.03394903243	8.999336243\\
0.02466563508	8.999518394\\
0.0159559641	8.999689102\\
0.007784577087	8.999849319\\
0.003827746492	9\\
};
\addlegendentry{CFL3D}

\addplot [color=mycolor2, draw=none, mark=x, mark options={solid, mycolor2}, only marks]
  table[row sep=crcr]{%
0.657	1.1\\
0.696	1.15\\
0.719	1.2\\
0.76	1.3\\
0.79	1.4\\
0.818	1.5\\
0.87	1.7\\
0.926	2\\
0.982	2.4\\
1.003	2.8\\
1.003	3.2\\
1	3.6\\
1.001	4\\
1.001	5\\
1	6\\
1.001	7.5\\
0.943	8.2\\
};
\addlegendentry{Experiment}

\addplot [color=mycolor3]
  table[row sep=crcr]{%
0	1\\
0.300948715366103	1.00800002\\
0.389110453441799	1.016\\
0.460850222124119	1.024\\
0.501908360138953	1.0319999\\
0.53336362136249	1.04\\
0.557386885997716	1.048\\
0.578956391667077	1.056\\
0.594561870085785	1.064\\
0.610165205483562	1.072\\
0.623619090887661	1.08\\
0.634439203567701	1.0880001\\
0.645259316247742	1.096\\
0.656081571948713	1.104\\
0.664220765444216	1.112\\
0.672059936009395	1.12\\
0.679896963553643	1.128\\
0.687733991097891	1.136\\
0.694660234746513	1.144\\
0.700519253971554	1.152\\
0.706376130175663	1.16\\
0.712235149400704	1.168\\
0.718092025604814	1.176\\
0.723948901808924	1.184\\
0.728558539831259	1.192\\
0.733046025660533	1.2\\
0.737533511489807	1.208\\
0.742020997319081	1.216\\
0.746506340127424	1.224\\
0.750993825956698	1.232\\
0.755481311785972	1.24\\
0.759015153301002	1.248\\
0.762523278564862	1.256\\
0.766031403828721	1.264\\
0.769541672113512	1.272\\
0.773049797377371	1.28\\
0.77655792264123	1.288\\
0.780068190926021	1.296\\
0.78357631618988	1.304\\
0.786865853318789	1.312\\
0.789658209591733	1.32\\
0.792452708885608	1.328\\
0.795245065158551	1.336\\
0.798037421431495	1.344\\
0.800829777704439	1.352\\
0.803624276998314	1.36\\
0.806416633271257	1.368\\
0.809208989544201	1.376\\
0.812001345817145	1.384\\
0.814795845111019	1.392\\
0.817337467935049	1.4\\
0.819596211996203	1.408\\
0.821854956057356	1.416\\
0.824113700118509	1.424\\
0.826372444179662	1.432\\
0.828631188240815	1.44\\
0.830889932301969	1.448\\
0.833148676363122	1.456\\
0.835407420424275	1.464\\
0.837668307506359	1.472\\
0.839927051567513	1.48\\
0.842185795628666	1.488\\
0.844444539689819	1.496\\
0.846703283750972	1.504\\
0.84858485612829	1.512\\
0.850436426212575	1.52\\
0.852285853275929	1.528\\
0.854137423360214	1.536\\
0.855988993444499	1.544\\
0.857838420507853	1.552\\
0.859689990592138	1.56\\
0.861539417655492	1.568\\
0.863390987739777	1.576\\
0.865242557824062	1.584\\
0.867091984887416	1.592\\
0.868943554971702	1.6\\
0.870792982035056	1.608\\
0.872644552119341	1.616\\
0.874496122203625	1.624\\
0.87634554926698	1.632\\
0.878197119351265	1.64\\
0.879729379316848	1.648\\
0.881257353240569	1.656\\
0.88278532716429	1.664\\
0.884313301088012	1.672\\
0.885841275011733	1.68\\
0.887369248935454	1.688\\
0.888897222859176	1.696\\
0.890425196782897	1.704\\
0.891953170706618	1.712\\
0.89348114463034	1.72\\
0.895009118554061	1.728\\
0.896537092477782	1.736\\
0.898065066401504	1.744\\
0.899593040325225	1.752\\
0.901121014248946	1.76\\
0.902648988172667	1.768\\
0.904176962096389	1.776\\
0.90570493602011	1.784\\
0.907232909943831	1.792\\
0.908760883867553	1.8\\
0.910288857791274	1.808\\
0.911602529621907	1.816\\
0.912871198012991	1.824\\
0.914137723383144	1.832\\
0.915404248753298	1.84\\
0.916670774123451	1.848\\
0.917939442514535	1.856\\
0.919205967884688	1.864\\
0.920472493254842	1.872\\
0.921739018624995	1.88\\
0.923007687016079	1.888\\
0.924274212386232	1.896\\
0.925540737756386	1.904\\
0.92680940614747	1.912\\
0.928075931517623	1.92\\
0.929342456887776	1.928\\
0.93060898225793	1.936\\
0.931877650649014	1.944\\
0.933144176019167	1.952\\
0.93441070138932	1.96\\
0.935677226759474	1.968\\
0.936945895150558	1.976\\
0.938212420520711	1.984\\
0.939478945890864	1.992\\
0.940747614281949	2\\
0.942014139652102	2.008\\
0.943280665022255	2.016\\
0.944444325387726	2.024\\
0.945427971995002	2.032\\
0.946409475581348	2.04\\
0.947390979167693	2.048\\
0.94837462577497	2.056\\
0.949356129361316	2.064\\
0.950339775968592	2.072\\
0.951321279554937	2.08\\
0.952302783141283	2.088\\
0.953286429748559	2.096\\
0.954267933334905	2.104\\
0.955251579942181	2.112\\
0.956233083528527	2.12\\
0.957214587114872	2.128\\
0.958198233722149	2.136\\
0.959179737308494	2.144\\
0.960163383915771	2.152\\
0.961144887502116	2.16\\
0.962126391088462	2.168\\
0.963110037695738	2.176\\
0.964091541282084	2.184\\
0.96507518788936	2.192\\
0.966056691475706	2.2\\
0.967038195062051	2.208\\
0.968021841669328	2.216\\
0.969003345255673	2.224\\
0.969986991862949	2.232\\
0.970968495449295	2.24\\
0.97194999903564	2.248\\
0.972933645642917	2.256\\
0.973915149229263	2.264\\
0.974898795836539	2.272\\
0.975880299422884	2.28\\
0.976724649669653	2.288\\
0.977228259588411	2.296\\
0.97773186950717	2.304\\
0.978235479425928	2.312\\
0.978741232365617	2.32\\
0.979244842284375	2.328\\
0.979748452203133	2.336\\
0.980252062121891	2.344\\
0.98075781506158	2.352\\
0.981261424980338	2.36\\
0.981765034899096	2.368\\
0.982268644817854	2.376\\
0.982774397757543	2.384\\
0.983278007676301	2.392\\
0.983781617595059	2.4\\
0.984285227513817	2.408\\
0.984790980453506	2.416\\
0.985294590372264	2.424\\
0.985798200291022	2.432\\
0.98630181020978	2.44\\
0.986807563149469	2.448\\
0.987311173068227	2.456\\
0.987814782986985	2.464\\
0.988318392905744	2.472\\
0.988824145845432	2.48\\
0.989327755764191	2.488\\
0.989831365682949	2.496\\
0.990334975601707	2.504\\
0.990840728541396	2.512\\
0.991344338460154	2.52\\
0.991847948378912	2.528\\
0.99235155829767	2.536\\
0.992857311237359	2.544\\
0.993360921156117	2.552\\
0.993864531074875	2.56\\
0.994368140993633	2.568\\
0.994871750912391	2.576\\
0.99537750385208	2.584\\
0.995881113770838	2.592\\
0.996384723689596	2.6\\
0.996888333608354	2.608\\
0.997394086548043	2.616\\
0.997520524782966	2.624\\
0.99762767582951	2.632\\
0.997734826876054	2.64\\
0.997841977922598	2.648\\
0.997946985948212	2.656\\
0.998054136994756	2.664\\
0.9981612880413	2.672\\
0.998268439087844	2.68\\
0.998375590134389	2.688\\
0.998482741180933	2.696\\
0.998587749206546	2.704\\
0.998694900253091	2.712\\
0.998802051299635	2.72\\
0.998909202346179	2.728\\
0.999016353392724	2.736\\
0.999121361418337	2.744\\
0.999228512464881	2.752\\
0.999335663511426	2.76\\
0.99944281455797	2.768\\
0.999549965604514	2.776\\
0.999654973630127	2.784\\
0.999762124676672	2.792\\
0.999869275723216	2.8\\
0.99997642676976	2.808\\
1.0000835778163	2.816\\
1.00019072886285	2.824\\
1.00029573688846	2.832\\
1.00040288793501	2.84\\
1.00051003898155	2.848\\
1.00061719002809	2.856\\
1.00072434107464	2.864\\
1.00082934910025	2.872\\
1.0009365001468	2.88\\
1.00104365119334	2.888\\
1.00115080223989	2.896\\
1.00125795328643	2.904\\
1.00136510433297	2.912\\
1.00147011235859	2.92\\
1.00157726340513	2.928\\
1.00168441445168	2.936\\
1.00179156549822	2.944\\
1.00189871654476	2.952\\
1.00200372457038	2.96\\
1.00211087561692	2.968\\
1.00221802666347	2.976\\
1.00232517771001	2.984\\
1.00243232875656	2.992\\
1.00253733678217	3\\
1.00264448782871	3.008\\
1.00275163887526	3.016\\
1.0028587899218	3.024\\
1.00293379565438	3.032\\
1.00292736659159	3.04\\
1.00291879450787	3.048\\
1.00291022242414	3.056\\
1.00290165034042	3.064\\
1.00289522127763	3.072\\
1.0028866491939	3.08\\
1.00287807711018	3.088\\
1.00287164804739	3.096\\
1.00286307596366	3.104\\
1.00285450387994	3.112\\
1.00284807481715	3.12\\
1.00283950273342	3.128\\
1.0028309306497	3.136\\
1.00282235856598	3.144\\
1.00281592950318	3.152\\
1.00280735741946	3.16\\
1.00279878533574	3.168\\
1.00279235627294	3.176\\
1.00278378418922	3.184\\
1.0027752121055	3.192\\
1.0027687830427	3.2\\
1.00276021095898	3.208\\
1.00275163887526	3.216\\
1.00274520981246	3.224\\
1.00273663772874	3.232\\
1.00272806564502	3.24\\
1.00271949356129	3.248\\
1.0027130644985	3.256\\
1.00270449241478	3.264\\
1.00269592033105	3.272\\
1.00268949126826	3.28\\
1.00268091918454	3.288\\
1.00267234710081	3.296\\
1.00266591803802	3.304\\
1.0026573459543	3.312\\
1.00264877387057	3.32\\
1.00264020178685	3.328\\
1.00263377272406	3.336\\
1.00262520064033	3.344\\
1.00261662855661	3.352\\
1.00261019949382	3.36\\
1.00260162741009	3.368\\
1.00259305532637	3.376\\
1.00258662626358	3.384\\
1.00257805417986	3.392\\
1.00256948209613	3.4\\
1.00256091001241	3.408\\
1.00255448094962	3.416\\
1.00254590886589	3.424\\
1.00253733678217	3.432\\
1.00253090771938	3.44\\
1.00252233563565	3.448\\
1.00251376355193	3.456\\
1.00250733448914	3.464\\
1.00249876240541	3.472\\
1.00249019032169	3.48\\
1.0024837612589	3.488\\
1.00247518917517	3.496\\
1.00246661709145	3.504\\
1.00245804500773	3.512\\
1.00245161594493	3.52\\
1.00244304386121	3.528\\
1.00243447177749	3.536\\
1.00242804271469	3.544\\
1.00241518458911	3.552\\
1.00239804042166	3.56\\
1.00238303927514	3.568\\
1.00236803812863	3.576\\
1.00235303698211	3.584\\
1.0023380358356	3.592\\
1.00232303468908	3.6\\
1.00230589052163	3.608\\
1.00229088937512	3.616\\
1.0022758882286	3.624\\
1.00226088708208	3.632\\
1.00224588593557	3.64\\
1.00222874176812	3.648\\
1.0022137406216	3.656\\
1.00219873947509	3.664\\
1.00218373832857	3.672\\
1.00216873718206	3.68\\
1.00215159301461	3.688\\
1.00213659186809	3.696\\
1.00212159072158	3.704\\
1.00210658957506	3.712\\
1.00209158842854	3.72\\
1.00207658728203	3.728\\
1.00205944311458	3.736\\
1.00204444196806	3.744\\
1.00202944082155	3.752\\
1.00201443967503	3.76\\
1.00199943852852	3.768\\
1.00198229436107	3.776\\
1.00196729321455	3.784\\
1.00195229206804	3.792\\
1.00193729092152	3.8\\
1.001922289775	3.808\\
1.00190514560756	3.816\\
1.00189014446104	3.824\\
1.00187514331452	3.832\\
1.00186014216801	3.84\\
1.00184514102149	3.848\\
1.00183013987498	3.856\\
1.00181299570753	3.864\\
1.00179799456101	3.872\\
1.0017829934145	3.88\\
1.00176799226798	3.888\\
1.00175299112146	3.896\\
1.00173584695402	3.904\\
1.0017208458075	3.912\\
1.00170584466098	3.92\\
1.00169084351447	3.928\\
1.00167584236795	3.936\\
1.00166084122144	3.944\\
1.00164369705399	3.952\\
1.00162869590747	3.96\\
1.00161369476096	3.968\\
1.00159869361444	3.976\\
1.00158369246792	3.984\\
1.00156654830048	3.992\\
1.00155154715396	4\\
1.00153654600744	4.008\\
1.00152154486093	4.016\\
1.00150654371441	4.024\\
1.00148939954697	4.032\\
1.00147439840045	4.04\\
1.00145939725393	4.048\\
1.00144439610742	4.056\\
1.0014293949609	4.064\\
1.00141439381438	4.072\\
1.00139724964694	4.08\\
1.00138224850042	4.088\\
1.0013672473539	4.096\\
1.00135224620739	4.104\\
1.00133724506087	4.112\\
1.00132010089343	4.12\\
1.00130509974691	4.128\\
1.00129009860039	4.136\\
1.00127509745388	4.144\\
1.00126009630736	4.152\\
1.00124295213991	4.16\\
1.0012279509934	4.168\\
1.00121294984688	4.176\\
1.00119794870037	4.184\\
1.00118294755385	4.192\\
1.00117008942826	4.2\\
1.00115937432361	4.208\\
1.00114651619802	4.216\\
1.00113580109337	4.224\\
1.00112294296778	4.232\\
1.00111222786313	4.24\\
1.00109936973754	4.248\\
1.00108865463289	4.256\\
1.0010757965073	4.264\\
1.00106508140265	4.272\\
1.00105222327706	4.28\\
1.00104150817241	4.288\\
1.00103079306776	4.296\\
1.00101793494217	4.304\\
1.00100721983752	4.312\\
1.00099436171193	4.32\\
1.00098364660728	4.328\\
1.00097078848169	4.336\\
1.00096007337704	4.344\\
1.00094721525145	4.352\\
1.0009365001468	4.36\\
1.00092364202121	4.368\\
1.00091292691656	4.376\\
1.00090006879097	4.384\\
1.00088935368632	4.392\\
1.00087863858166	4.4\\
1.00086578045608	4.408\\
1.00085506535142	4.416\\
1.00084220722584	4.424\\
1.00083149212118	4.432\\
1.0008186339956	4.44\\
1.00080791889094	4.448\\
1.00079506076536	4.456\\
1.0007843456607	4.464\\
1.00077148753512	4.472\\
1.00076077243046	4.48\\
1.00074791430488	4.488\\
1.00073719920022	4.496\\
1.00072648409557	4.504\\
1.00071362596998	4.512\\
1.00070291086533	4.52\\
1.00069005273975	4.528\\
1.00067933763509	4.536\\
1.00066647950951	4.544\\
1.00065576440485	4.552\\
1.00064290627927	4.56\\
1.00063219117461	4.568\\
1.00061933304903	4.576\\
1.00060861794437	4.584\\
1.00059575981879	4.592\\
1.00058504471413	4.6\\
1.00057218658855	4.608\\
1.00056147148389	4.616\\
1.00055075637924	4.624\\
1.00053789825365	4.632\\
1.000527183149	4.64\\
1.00051432502341	4.648\\
1.00050360991876	4.656\\
1.00049075179317	4.664\\
1.00048003668852	4.672\\
1.00046717856293	4.68\\
1.00045646345828	4.688\\
1.00044360533269	4.696\\
1.00043289022804	4.704\\
1.00042003210245	4.712\\
1.0004093169978	4.72\\
1.00039860189314	4.728\\
1.00038574376756	4.736\\
1.0003750286629	4.744\\
1.00036217053732	4.752\\
1.00035145543267	4.76\\
1.00033859730708	4.768\\
1.00032788220243	4.776\\
1.00031502407684	4.784\\
1.00030430897219	4.792\\
1.0002914508466	4.8\\
1.00028073574195	4.808\\
1.00026787761636	4.816\\
1.00025716251171	4.824\\
1.00024430438612	4.832\\
1.00023358928147	4.84\\
1.00022287417681	4.848\\
1.00021001605123	4.856\\
1.00019930094657	4.864\\
1.00018644282099	4.872\\
1.00017572771633	4.88\\
1.00016286959075	4.888\\
1.00015215448609	4.896\\
1.00013929636051	4.904\\
1.00012858125585	4.912\\
1.00011572313027	4.92\\
1.00010500802561	4.928\\
1.00009214990003	4.936\\
1.00008143479537	4.944\\
1.00007071969072	4.952\\
1.00005786156513	4.96\\
1.00004714646048	4.968\\
1.00003428833489	4.976\\
1.00002357323024	4.984\\
1.00001071510465	4.992\\
1	5\\
0.999991427916276	5.008\\
0.999982855832553	5.016\\
0.99997642676976	5.024\\
0.999967854686037	5.032\\
0.999959282602313	5.04\\
0.99995071051859	5.048\\
0.999944281455797	5.056\\
0.999935709372073	5.064\\
0.99992713728835	5.072\\
0.999918565204626	5.08\\
0.999912136141834	5.088\\
0.99990356405811	5.096\\
0.999894991974387	5.104\\
0.999886419890663	5.112\\
0.99987999082787	5.12\\
0.999871418744147	5.128\\
0.999862846660423	5.136\\
0.9998542745767	5.144\\
0.999847845513907	5.152\\
0.999839273430184	5.16\\
0.99983070134646	5.168\\
0.999824272283667	5.176\\
0.999815700199944	5.184\\
0.99980712811622	5.192\\
0.999798556032497	5.2\\
0.999792126969704	5.208\\
0.999783554885981	5.216\\
0.999774982802257	5.224\\
0.999766410718533	5.232\\
0.999759981655741	5.24\\
0.999751409572017	5.248\\
0.999742837488294	5.256\\
0.99973426540457	5.264\\
0.999727836341778	5.272\\
0.999719264258054	5.28\\
0.999710692174331	5.288\\
0.999702120090607	5.296\\
0.999695691027814	5.304\\
0.999687118944091	5.312\\
0.999678546860367	5.32\\
0.999672117797574	5.328\\
0.999663545713851	5.336\\
0.999654973630127	5.344\\
0.999646401546404	5.352\\
0.999639972483611	5.36\\
0.999631400399888	5.368\\
0.999622828316164	5.376\\
0.999614256232441	5.384\\
0.999607827169648	5.392\\
0.999599255085924	5.4\\
0.999590683002201	5.408\\
0.999582110918477	5.416\\
0.999575681855685	5.424\\
0.999567109771961	5.432\\
0.999558537688238	5.44\\
0.999549965604514	5.448\\
0.999543536541721	5.456\\
0.999534964457998	5.464\\
0.999526392374274	5.472\\
0.999517820290551	5.48\\
0.999511391227758	5.488\\
0.999502819144034	5.496\\
0.999494247060311	5.504\\
0.999487817997518	5.512\\
0.999479245913795	5.52\\
0.999470673830071	5.528\\
0.999462101746348	5.536\\
0.999455672683555	5.544\\
0.999447100599832	5.552\\
0.999438528516108	5.56\\
0.999429956432384	5.568\\
0.999423527369592	5.576\\
0.999414955285868	5.584\\
0.999406383202145	5.592\\
0.999397811118421	5.6\\
0.999391382055629	5.608\\
0.999382809971905	5.616\\
0.999374237888181	5.624\\
0.999365665804458	5.632\\
0.999359236741665	5.64\\
0.999350664657942	5.648\\
0.999342092574218	5.656\\
0.999335663511426	5.664\\
0.999327091427702	5.672\\
0.999318519343979	5.68\\
0.999309947260255	5.688\\
0.999303518197462	5.696\\
0.999294946113739	5.704\\
0.999286374030015	5.712\\
0.999277801946292	5.72\\
0.999271372883499	5.728\\
0.999262800799775	5.736\\
0.999254228716052	5.744\\
0.999245656632328	5.752\\
0.999239227569536	5.76\\
0.999230655485812	5.768\\
0.999222083402089	5.776\\
0.999213511318365	5.784\\
0.999207082255572	5.792\\
0.999198510171849	5.8\\
0.999189938088125	5.808\\
0.999185652046264	5.816\\
0.999181366004402	5.824\\
0.99917707996254	5.832\\
0.999170650899747	5.84\\
0.999166364857886	5.848\\
0.999162078816024	5.856\\
0.999157792774162	5.864\\
0.999151363711369	5.872\\
0.999147077669508	5.88\\
0.999142791627646	5.888\\
0.999138505585784	5.896\\
0.999132076522991	5.904\\
0.99912779048113	5.912\\
0.999123504439268	5.92\\
0.999119218397406	5.928\\
0.999112789334613	5.936\\
0.999108503292752	5.944\\
0.99910421725089	5.952\\
0.999099931209028	5.96\\
0.999093502146235	5.968\\
0.999089216104374	5.976\\
0.999084930062512	5.984\\
0.99908064402065	5.992\\
0.999074214957858	6\\
0.999069928915996	6.008\\
0.999065642874134	6.016\\
0.999061356832272	6.024\\
0.99905492776948	6.032\\
0.999050641727618	6.04\\
0.999046355685756	6.048\\
0.999042069643894	6.056\\
0.999037783602032	6.064\\
0.99903135453924	6.072\\
0.999027068497378	6.08\\
0.999022782455516	6.088\\
0.999018496413654	6.096\\
0.999012067350862	6.104\\
0.999007781309	6.112\\
0.999003495267138	6.12\\
0.998999209225277	6.128\\
0.998992780162484	6.136\\
0.998988494120622	6.144\\
0.99898420807876	6.152\\
0.998979922036898	6.16\\
0.998973492974106	6.168\\
0.998969206932244	6.176\\
0.998964920890382	6.184\\
0.998960634848521	6.192\\
0.998954205785728	6.2\\
0.998949919743866	6.208\\
0.998945633702004	6.216\\
0.998941347660143	6.224\\
0.99893491859735	6.232\\
0.998930632555488	6.24\\
0.998926346513626	6.248\\
0.998922060471765	6.256\\
0.998915631408972	6.264\\
0.99891134536711	6.272\\
0.998907059325248	6.28\\
0.998902773283387	6.288\\
0.998896344220594	6.296\\
0.998892058178732	6.304\\
0.99888777213687	6.312\\
0.998883486095009	6.32\\
0.998877057032216	6.328\\
0.998872770990354	6.336\\
0.998868484948493	6.344\\
0.998864198906631	6.352\\
0.998857769843838	6.36\\
0.998853483801976	6.368\\
0.998849197760114	6.376\\
0.998844911718253	6.384\\
0.99883848265546	6.392\\
0.998834196613598	6.4\\
0.998829910571737	6.408\\
0.998825624529875	6.416\\
0.998819195467082	6.424\\
0.99881490942522	6.432\\
0.998810623383359	6.44\\
0.998806337341497	6.448\\
0.998797765257773	6.456\\
0.998787050153119	6.464\\
0.998774192027533	6.472\\
0.998763476922879	6.48\\
0.998750618797294	6.488\\
0.998739903692639	6.496\\
0.998727045567054	6.504\\
0.9987163304624	6.512\\
0.998703472336814	6.52\\
0.99869275723216	6.528\\
0.998679899106575	6.536\\
0.99866918400192	6.544\\
0.998656325876335	6.552\\
0.99864561077168	6.56\\
0.998632752646095	6.568\\
0.998622037541441	6.576\\
0.998609179415855	6.584\\
0.998598464311201	6.592\\
0.998585606185616	6.6\\
0.998574891080961	6.608\\
0.998562032955376	6.616\\
0.998551317850721	6.624\\
0.998538459725136	6.632\\
0.998527744620482	6.64\\
0.998514886494896	6.648\\
0.998504171390242	6.656\\
0.998491313264657	6.664\\
0.998480598160002	6.672\\
0.998467740034417	6.68\\
0.998457024929762	6.688\\
0.998444166804177	6.696\\
0.998433451699523	6.704\\
0.998420593573937	6.712\\
0.998409878469283	6.72\\
0.998397020343698	6.728\\
0.998386305239043	6.736\\
0.998373447113458	6.744\\
0.998362732008804	6.752\\
0.998349873883218	6.76\\
0.998339158778564	6.768\\
0.998326300652978	6.776\\
0.998313442527393	6.784\\
0.998302727422739	6.792\\
0.998289869297153	6.8\\
0.998279154192499	6.808\\
0.998266296066914	6.816\\
0.998255580962259	6.824\\
0.998242722836674	6.832\\
0.998232007732019	6.84\\
0.998219149606434	6.848\\
0.99820843450178	6.856\\
0.998195576376194	6.864\\
0.99818486127154	6.872\\
0.998172003145955	6.88\\
0.9981612880413	6.888\\
0.998148429915715	6.896\\
0.998137714811061	6.904\\
0.998124856685475	6.912\\
0.998114141580821	6.92\\
0.998101283455236	6.928\\
0.998090568350581	6.936\\
0.998077710224996	6.944\\
0.998066995120341	6.952\\
0.998054136994756	6.96\\
0.998043421890102	6.968\\
0.99794269990635	6.976\\
0.997809832608635	6.984\\
0.997674822289989	6.992\\
0.997541954992274	7\\
0.997409087694559	7.008\\
0.997276220396845	7.016\\
0.99714335309913	7.024\\
0.997008342780484	7.032\\
0.996875475482769	7.04\\
0.996742608185054	7.048\\
0.996609740887339	7.056\\
0.996474730568694	7.064\\
0.996341863270979	7.072\\
0.996208995973264	7.08\\
0.996076128675549	7.088\\
0.995941118356903	7.096\\
0.995808251059188	7.104\\
0.995675383761473	7.112\\
0.995542516463758	7.12\\
0.995409649166043	7.128\\
0.995274638847398	7.136\\
0.995141771549683	7.144\\
0.995008904251968	7.152\\
0.994876036954253	7.16\\
0.994741026635607	7.168\\
0.994608159337892	7.176\\
0.994475292040177	7.184\\
0.994342424742462	7.192\\
0.994207414423817	7.2\\
0.994074547126102	7.208\\
0.993941679828387	7.216\\
0.993808812530672	7.224\\
0.993675945232957	7.232\\
0.993540934914311	7.24\\
0.993408067616596	7.248\\
0.993275200318882	7.256\\
0.993142333021167	7.264\\
0.993007322702521	7.272\\
0.992874455404806	7.28\\
0.992741588107091	7.288\\
0.992608720809376	7.296\\
0.992475853511661	7.304\\
0.992340843193016	7.312\\
0.992207975895301	7.32\\
0.992075108597586	7.328\\
0.991942241299871	7.336\\
0.991807230981225	7.344\\
0.99167436368351	7.352\\
0.991541496385795	7.36\\
0.99140862908808	7.368\\
0.991273618769434	7.376\\
0.991119321262411	7.384\\
0.990581423008759	7.392\\
0.990041381734175	7.4\\
0.989501340459592	7.408\\
0.98896344220594	7.416\\
0.988423400931357	7.424\\
0.987883359656774	7.432\\
0.987345461403122	7.44\\
0.986805420128538	7.448\\
0.986265378853955	7.456\\
0.985727480600303	7.464\\
0.98518743932572	7.472\\
0.984647398051137	7.48\\
0.984109499797485	7.488\\
0.983569458522901	7.496\\
0.983029417248318	7.504\\
0.982491518994666	7.512\\
0.981951477720083	7.52\\
0.9814114364455	7.528\\
0.980873538191848	7.536\\
0.980333496917264	7.544\\
0.979795598663612	7.552\\
0.979255557389029	7.56\\
0.978715516114446	7.568\\
0.978177617860794	7.576\\
0.977637576586211	7.584\\
0.977097535311627	7.592\\
0.976559637057975	7.6\\
0.976019595783392	7.608\\
0.975479554508809	7.616\\
0.974941656255157	7.624\\
0.974401614980574	7.632\\
0.97386157370599	7.64\\
0.973323675452338	7.648\\
0.972783634177755	7.656\\
0.972243592903172	7.664\\
0.97170569464952	7.672\\
0.971165653374936	7.68\\
0.970625612100353	7.688\\
0.970087713846701	7.696\\
0.969547672572118	7.704\\
0.969007631297535	7.712\\
0.968133278757734	7.72\\
0.967119629857425	7.728\\
0.966105980957116	7.736\\
0.965094475077738	7.744\\
0.964080826177429	7.752\\
0.963069320298051	7.76\\
0.962055671397743	7.768\\
0.961042022497434	7.776\\
0.960030516618056	7.784\\
0.959016867717747	7.792\\
0.958005361838369	7.8\\
0.95699171293806	7.808\\
0.955978064037751	7.816\\
0.954966558158374	7.824\\
0.953952909258065	7.832\\
0.952941403378687	7.84\\
0.951927754478378	7.848\\
0.950916248599	7.856\\
0.949902599698691	7.864\\
0.948888950798383	7.872\\
0.947877444919004	7.88\\
0.946863796018696	7.888\\
0.945852290139318	7.896\\
0.944838641239009	7.904\\
0.9438249923387	7.912\\
0.942813486459322	7.92\\
0.941799837559013	7.928\\
0.940788331679636	7.936\\
0.939774682779327	7.944\\
0.938761033879018	7.952\\
0.93774952799964	7.96\\
0.936735879099331	7.968\\
0.935724373219953	7.976\\
0.934534996603312	7.984\\
0.933244898002919	7.992\\
0.931956942423457	8\\
0.930666843823064	8.008\\
0.929378888243601	8.016\\
0.928088789643208	8.024\\
0.926800834063746	8.032\\
0.925512878484284	8.04\\
0.924222779883891	8.048\\
0.922934824304429	8.056\\
0.921644725704036	8.064\\
0.920356770124574	8.072\\
0.919066671524181	8.08\\
0.917778715944719	8.088\\
0.916488617344326	8.096\\
0.915200661764863	8.104\\
0.91391056316447	8.112\\
0.912622607585008	8.12\\
0.911332508984615	8.128\\
0.910044553405153	8.136\\
0.908756597825691	8.144\\
0.907466499225298	8.152\\
0.906178543645836	8.16\\
0.904888445045443	8.168\\
0.903600489465981	8.176\\
0.902310390865588	8.184\\
0.900975288825646	8.192\\
0.899425884692616	8.2\\
0.897878623580516	8.208\\
0.896329219447486	8.216\\
0.894781958335387	8.224\\
0.893232554202357	8.232\\
0.891683150069327	8.24\\
0.890135888957227	8.248\\
0.888586484824197	8.256\\
0.887037080691167	8.264\\
0.885489819579068	8.272\\
0.883940415446038	8.28\\
0.882391011313007	8.288\\
0.880843750200908	8.296\\
0.879294346067878	8.304\\
0.877744941934848	8.312\\
0.876197680822749	8.32\\
0.874648276689718	8.328\\
0.873101015577619	8.336\\
0.871551611444589	8.344\\
0.870002207311559	8.352\\
0.868450660157598	8.36\\
0.866575516843073	8.368\\
0.864700373528548	8.376\\
0.862825230214023	8.384\\
0.860950086899499	8.392\\
0.859074943584974	8.4\\
0.857199800270449	8.408\\
0.855324656955925	8.416\\
0.853451656662331	8.424\\
0.851576513347806	8.432\\
0.849701370033281	8.44\\
0.847826226718756	8.448\\
0.845951083404232	8.456\\
0.844075940089707	8.464\\
0.842200796775182	8.472\\
0.840325653460657	8.48\\
0.838452653167064	8.488\\
0.836543221517644	8.496\\
0.83425661818439	8.504\\
0.831967871830204	8.512\\
0.829681268496949	8.52\\
0.827394665163695	8.528\\
0.825105918809509	8.536\\
0.822819315476254	8.544\\
0.820530569122069	8.552\\
0.818243965788814	8.56\\
0.815955219434628	8.568\\
0.813668616101373	8.576\\
0.811379869747188	8.584\\
0.809093266413933	8.592\\
0.806804520059747	8.6\\
0.804230751921754	8.608\\
0.801401964292985	8.616\\
0.798575319685147	8.624\\
0.79574867507731	8.632\\
0.792919887448541	8.64\\
0.790093242840703	8.648\\
0.787264455211934	8.656\\
0.784437810604096	8.664\\
0.781611165996258	8.672\\
0.77878237836749	8.68\\
0.775955733759652	8.688\\
0.772627622253987	8.696\\
0.76907877959244	8.704\\
0.765529936930894	8.712\\
0.761981094269348	8.72\\
0.758434394628732	8.728\\
0.754885551967186	8.736\\
0.75133670930564	8.744\\
0.747787866644094	8.752\\
0.744215450752308	8.76\\
0.739685104504416	8.768\\
0.735156901277455	8.776\\
0.730626555029563	8.784\\
0.726098351802602	8.792\\
0.72156800555471	8.8\\
0.717037659306819	8.808\\
0.712387303886797	8.816\\
0.706491853305931	8.824\\
0.700596402725065	8.832\\
0.6947009521442	8.84\\
0.688805501563334	8.848\\
0.682910050982468	8.856\\
0.675955948061745	8.864\\
0.66810177635005	8.872\\
0.660245461617424	8.88\\
0.652389146884798	8.888\\
0.644234952242779	8.896\\
0.63344269883484	8.904\\
0.622650445426901	8.912\\
0.611856048998031	8.92\\
0.598481455368375	8.928\\
0.582998129142727	8.936\\
0.56751480291708	8.944\\
0.546189601633839	8.952\\
0.522464216908006	8.96\\
0.491435416849716	8.968\\
0.450979467716461	8.976\\
0.380364785022855	8.984\\
0.293626012845267	8.992\\
0	9\\
};
\addlegendentry{Van Leer}

\addplot [color=mycolor4]
  table[row sep=crcr]{%
0	1\\
0.302170953182457	1.00800002\\
0.390621482251788	1.016\\
0.462602785252527	1.024\\
0.503811228920313	1.0319999\\
0.535381712533636	1.04\\
0.559492693804475	1.048\\
0.581144333544175	1.056\\
0.596803070424649	1.064\\
0.612463951456415	1.072\\
0.625961383835243	1.08\\
0.636817221823172	1.0880001\\
0.6476730598111	1.096\\
0.658528897799029	1.104\\
0.666693825916357	1.112\\
0.674554284550318	1.12\\
0.682414743184279	1.128\\
0.69027520181824	1.136\\
0.69722010785081	1.144\\
0.703095082389013	1.152\\
0.708967912775925	1.16\\
0.714842887314129	1.168\\
0.720715717701041	1.176\\
0.726590692239244	1.184\\
0.731211338272029	1.192\\
0.735711911832499	1.2\\
0.740212485392969	1.208\\
0.74471305895344	1.216\\
0.749211488362619	1.224\\
0.753712061923089	1.232\\
0.75821263548356	1.24\\
0.761756917568104	1.248\\
0.765275469837152	1.256\\
0.768796166257491	1.264\\
0.772314718526539	1.272\\
0.775835414946879	1.28\\
0.779356111367218	1.288\\
0.782874663636266	1.296\\
0.786395360056605	1.304\\
0.78969520889394	1.312\\
0.792497614631688	1.32\\
0.795300020369437	1.328\\
0.798102426107186	1.336\\
0.800906975996226	1.344\\
0.803709381733975	1.352\\
0.806511787471724	1.36\\
0.809314193209473	1.368\\
0.812116598947222	1.376\\
0.814919004684971	1.384\\
0.81772141042272	1.392\\
0.820272950459384	1.4\\
0.822539318374304	1.408\\
0.824805686289225	1.416\\
0.827072054204145	1.424\\
0.829338422119065	1.432\\
0.831606934185276	1.44\\
0.833873302100196	1.448\\
0.836139670015116	1.456\\
0.838406037930036	1.464\\
0.840672405844957	1.472\\
0.842938773759876	1.48\\
0.845205141674797	1.488\\
0.847471509589717	1.496\\
0.849737877504637	1.504\\
0.851624730640994	1.512\\
0.853481565659273	1.52\\
0.85533625652626	1.528\\
0.857190947393248	1.536\\
0.859047782411527	1.544\\
0.860902473278515	1.552\\
0.862757164145502	1.56\\
0.864613999163781	1.568\\
0.866468690030769	1.576\\
0.868323380897756	1.584\\
0.870180215916035	1.592\\
0.872034906783023	1.6\\
0.87388959765001	1.608\\
0.875746432668289	1.616\\
0.877601123535277	1.624\\
0.879455814402264	1.632\\
0.881312649420543	1.64\\
0.882843573442542	1.648\\
0.88437235331325	1.656\\
0.885898989032666	1.664\\
0.887425624752082	1.672\\
0.88895440462279	1.68\\
0.890481040342207	1.688\\
0.892009820212914	1.696\\
0.893536455932331	1.704\\
0.895065235803038	1.712\\
0.896591871522455	1.72\\
0.898120651393162	1.728\\
0.899647287112579	1.736\\
0.901176066983286	1.744\\
0.902702702702703	1.752\\
0.90423148257341	1.76\\
0.905758118292827	1.768\\
0.907286898163534	1.776\\
0.908813533882951	1.784\\
0.910342313753658	1.792\\
0.911868949473075	1.8\\
0.913397729343782	1.808\\
0.914701373328902	1.816\\
0.915955701834321	1.824\\
0.917210030339741	1.832\\
0.918466502996451	1.84\\
0.919720831501871	1.848\\
0.92097516000729	1.856\\
0.922231632664001	1.864\\
0.92348596116942	1.872\\
0.924740289674839	1.88\\
0.92599676233155	1.888\\
0.92725109083697	1.896\\
0.928505419342389	1.904\\
0.929759747847808	1.912\\
0.931016220504519	1.92\\
0.932270549009938	1.928\\
0.933524877515358	1.936\\
0.934781350172068	1.944\\
0.936035678677487	1.952\\
0.937290007182907	1.96\\
0.938546479839617	1.968\\
0.939800808345037	1.976\\
0.941055136850456	1.984\\
0.942311609507167	1.992\\
0.943565938012586	2\\
0.944820266518005	2.008\\
0.946076739174716	2.016\\
0.947219571812987	2.024\\
0.94816943083504	2.032\\
0.949119289857092	2.04\\
0.950067004727854	2.048\\
0.951016863749906	2.056\\
0.951966722771959	2.064\\
0.952916581794011	2.072\\
0.953864296664773	2.08\\
0.954814155686825	2.088\\
0.955764014708878	2.096\\
0.956711729579639	2.104\\
0.957661588601692	2.112\\
0.958611447623744	2.12\\
0.959561306645797	2.128\\
0.960509021516558	2.136\\
0.961458880538611	2.144\\
0.962408739560663	2.152\\
0.963358598582716	2.16\\
0.964306313453477	2.168\\
0.96525617247553	2.176\\
0.966206031497582	2.184\\
0.967153746368344	2.192\\
0.968103605390396	2.2\\
0.969053464412449	2.208\\
0.970003323434502	2.216\\
0.970951038305263	2.224\\
0.971900897327315	2.232\\
0.972850756349368	2.24\\
0.973800615371421	2.248\\
0.974748330242182	2.256\\
0.975698189264234	2.264\\
0.976648048286287	2.272\\
0.977595763157048	2.28\\
0.978410540647748	2.288\\
0.978890830537003	2.296\\
0.979373264577548	2.304\\
0.979853554466803	2.312\\
0.980333844356058	2.32\\
0.980816278396604	2.328\\
0.981296568285858	2.336\\
0.981776858175113	2.344\\
0.982259292215659	2.352\\
0.982739582104913	2.36\\
0.983219871994168	2.368\\
0.983702306034714	2.376\\
0.984182595923968	2.384\\
0.984662885813223	2.392\\
0.985145319853769	2.4\\
0.985625609743023	2.408\\
0.986108043783569	2.416\\
0.986588333672824	2.424\\
0.987068623562079	2.432\\
0.987551057602624	2.44\\
0.988031347491879	2.448\\
0.988511637381134	2.456\\
0.988994071421679	2.464\\
0.989474361310934	2.472\\
0.989954651200189	2.48\\
0.990437085240735	2.488\\
0.990917375129989	2.496\\
0.991397665019244	2.504\\
0.99188009905979	2.512\\
0.992360388949044	2.52\\
0.99284282298959	2.528\\
0.993323112878845	2.536\\
0.993803402768099	2.544\\
0.994285836808645	2.552\\
0.9947661266979	2.56\\
0.995246416587154	2.568\\
0.9957288506277	2.576\\
0.996209140516955	2.584\\
0.996689430406209	2.592\\
0.997171864446755	2.6\\
0.99765215433601	2.608\\
0.998132444225265	2.616\\
0.998258949151452	2.624\\
0.998361868413435	2.632\\
0.99846693182671	2.64\\
0.998571995239984	2.648\\
0.998677058653259	2.656\\
0.998782122066533	2.664\\
0.998887185479807	2.672\\
0.998992248893082	2.68\\
0.999095168155065	2.688\\
0.999200231568339	2.696\\
0.999305294981614	2.704\\
0.999410358394888	2.712\\
0.999515421808163	2.72\\
0.999620485221437	2.728\\
0.999725548634712	2.736\\
0.999830612047986	2.744\\
0.999933531309969	2.752\\
1.00003859472324	2.76\\
1.00014365813652	2.768\\
1.00024872154979	2.776\\
1.00035378496307	2.784\\
1.00045884837634	2.792\\
1.00056391178962	2.8\\
1.0006668310516	2.808\\
1.00077189446487	2.816\\
1.00087695787815	2.824\\
1.00098202129142	2.832\\
1.0010870847047	2.84\\
1.00119214811797	2.848\\
1.00129721153125	2.856\\
1.00140013079323	2.864\\
1.0015051942065	2.872\\
1.00161025761978	2.88\\
1.00171532103305	2.888\\
1.00182038444633	2.896\\
1.0019254478596	2.904\\
1.00203051127288	2.912\\
1.00213343053486	2.92\\
1.00223849394813	2.928\\
1.00234355736141	2.936\\
1.00244862077468	2.944\\
1.00255368418796	2.952\\
1.00265874760123	2.96\\
1.00276381101451	2.968\\
1.00286673027649	2.976\\
1.00297179368976	2.984\\
1.00307685710304	2.992\\
1.00318192051631	3\\
1.00328698392959	3.008\\
1.00339204734286	3.016\\
1.00349711075614	3.024\\
1.00357001190004	3.032\\
1.00356143529487	3.04\\
1.00355285868971	3.048\\
1.00354213793325	3.056\\
1.00353356132809	3.064\\
1.00352498472292	3.072\\
1.00351640811776	3.08\\
1.0035056873613	3.088\\
1.00349711075614	3.096\\
1.00348853415097	3.104\\
1.0034799575458	3.112\\
1.00346923678935	3.12\\
1.00346066018418	3.128\\
1.00345208357902	3.136\\
1.00344350697385	3.144\\
1.0034327862174	3.152\\
1.00342420961223	3.16\\
1.00341563300706	3.168\\
1.0034070564019	3.176\\
1.00339633564544	3.184\\
1.00338775904028	3.192\\
1.00337918243511	3.2\\
1.00337060582995	3.208\\
1.00335988507349	3.216\\
1.00335130846833	3.224\\
1.00334273186316	3.232\\
1.00333415525799	3.24\\
1.00332343450154	3.248\\
1.00331485789637	3.256\\
1.00330628129121	3.264\\
1.00329770468604	3.272\\
1.00328698392959	3.28\\
1.00327840732442	3.288\\
1.00326983071926	3.296\\
1.00326125411409	3.304\\
1.00325053335763	3.312\\
1.00324195675247	3.32\\
1.0032333801473	3.328\\
1.00322480354214	3.336\\
1.00321622693697	3.344\\
1.00320550618052	3.352\\
1.00319692957535	3.36\\
1.00318835297019	3.368\\
1.00317977636502	3.376\\
1.00316905560856	3.384\\
1.0031604790034	3.392\\
1.00315190239823	3.4\\
1.00314332579307	3.408\\
1.00313260503661	3.416\\
1.00312402843145	3.424\\
1.00311545182628	3.432\\
1.00310687522112	3.44\\
1.00309615446466	3.448\\
1.00308757785949	3.456\\
1.00307900125433	3.464\\
1.00307042464916	3.472\\
1.00305970389271	3.48\\
1.00305112728754	3.488\\
1.00304255068238	3.496\\
1.00303397407721	3.504\\
1.00302325332075	3.512\\
1.00301467671559	3.52\\
1.00300610011042	3.528\\
1.00299752350526	3.536\\
1.0029868027488	3.544\\
1.00297179368976	3.552\\
1.00295249632814	3.56\\
1.00293534311781	3.568\\
1.00291604575619	3.576\\
1.00289674839457	3.584\\
1.00287745103294	3.592\\
1.00286029782261	3.6\\
1.00284100046099	3.608\\
1.00282170309937	3.616\\
1.00280240573775	3.624\\
1.00278525252742	3.632\\
1.0027659551658	3.64\\
1.00274665780417	3.648\\
1.00272736044255	3.656\\
1.00271020723222	3.664\\
1.0026909098706	3.672\\
1.00267161250898	3.68\\
1.00265231514736	3.688\\
1.00263516193703	3.696\\
1.0026158645754	3.704\\
1.00259656721378	3.712\\
1.00257726985216	3.72\\
1.00256011664183	3.728\\
1.00254081928021	3.736\\
1.00252152191859	3.744\\
1.00250222455696	3.752\\
1.00248292719534	3.76\\
1.00246577398501	3.768\\
1.00244647662339	3.776\\
1.00242717926177	3.784\\
1.00240788190015	3.792\\
1.00239072868982	3.8\\
1.00237143132819	3.808\\
1.00235213396657	3.816\\
1.00233283660495	3.824\\
1.00231568339462	3.832\\
1.002296386033	3.84\\
1.00227708867138	3.848\\
1.00225779130975	3.856\\
1.00224063809942	3.864\\
1.0022213407378	3.872\\
1.00220204337618	3.88\\
1.00218274601456	3.888\\
1.00216559280423	3.896\\
1.00214629544261	3.904\\
1.00212699808098	3.912\\
1.00210770071936	3.92\\
1.00209054750903	3.928\\
1.00207125014741	3.936\\
1.00205195278579	3.944\\
1.00203265542417	3.952\\
1.00201550221384	3.96\\
1.00199620485221	3.968\\
1.00197690749059	3.976\\
1.00195761012897	3.984\\
1.00194045691864	3.992\\
1.00192115955702	4\\
1.0019018621954	4.008\\
1.00188256483377	4.016\\
1.00186541162344	4.024\\
1.00184611426182	4.032\\
1.0018268169002	4.04\\
1.00180751953858	4.048\\
1.00179036632825	4.056\\
1.00177106896663	4.064\\
1.001751771605	4.072\\
1.00173247424338	4.08\\
1.00171532103305	4.088\\
1.00169602367143	4.096\\
1.00167672630981	4.104\\
1.00165742894819	4.112\\
1.00164027573786	4.12\\
1.00162097837623	4.128\\
1.00160168101461	4.136\\
1.00158238365299	4.144\\
1.00156523044266	4.152\\
1.00154593308104	4.16\\
1.00152663571942	4.168\\
1.00150733835779	4.176\\
1.00149018514746	4.184\\
1.00147088778584	4.192\\
1.0014558787268	4.2\\
1.00144086966776	4.208\\
1.00142586060872	4.216\\
1.00141299570098	4.224\\
1.00139798664194	4.232\\
1.0013829775829	4.24\\
1.00136796852386	4.248\\
1.00135295946482	4.256\\
1.00134009455707	4.264\\
1.00132508549803	4.272\\
1.00131007643899	4.28\\
1.00129506737995	4.288\\
1.00128005832092	4.296\\
1.00126719341317	4.304\\
1.00125218435413	4.312\\
1.00123717529509	4.32\\
1.00122216623605	4.328\\
1.00120715717701	4.336\\
1.00119429226926	4.344\\
1.00117928321022	4.352\\
1.00116427415118	4.36\\
1.00114926509214	4.368\\
1.00113425603311	4.376\\
1.00112139112536	4.384\\
1.00110638206632	4.392\\
1.00109137300728	4.4\\
1.00107636394824	4.408\\
1.00106349904049	4.416\\
1.00104848998145	4.424\\
1.00103348092241	4.432\\
1.00101847186337	4.44\\
1.00100346280434	4.448\\
1.00099059789659	4.456\\
1.00097558883755	4.464\\
1.00096057977851	4.472\\
1.00094557071947	4.48\\
1.00093056166043	4.488\\
1.00091769675268	4.496\\
1.00090268769364	4.504\\
1.0008876786346	4.512\\
1.00087266957557	4.52\\
1.00085766051653	4.528\\
1.00084479560878	4.536\\
1.00082978654974	4.544\\
1.0008147774907	4.552\\
1.00079976843166	4.56\\
1.00078475937262	4.568\\
1.00077189446487	4.576\\
1.00075688540583	4.584\\
1.0007418763468	4.592\\
1.00072686728776	4.6\\
1.00071400238001	4.608\\
1.00069899332097	4.616\\
1.00068398426193	4.624\\
1.00066897520289	4.632\\
1.00065396614385	4.64\\
1.0006411012361	4.648\\
1.00062609217706	4.656\\
1.00061108311802	4.664\\
1.00059607405899	4.672\\
1.00058106499995	4.68\\
1.0005682000922	4.688\\
1.00055319103316	4.696\\
1.00053818197412	4.704\\
1.00052317291508	4.712\\
1.00050816385604	4.72\\
1.00049529894829	4.728\\
1.00048028988925	4.736\\
1.00046528083022	4.744\\
1.00045027177118	4.752\\
1.00043526271214	4.76\\
1.00042239780439	4.768\\
1.00040738874535	4.776\\
1.00039237968631	4.784\\
1.00037737062727	4.792\\
1.00036450571952	4.8\\
1.00034949666048	4.808\\
1.00033448760145	4.816\\
1.00031947854241	4.824\\
1.00030446948337	4.832\\
1.00029160457562	4.84\\
1.00027659551658	4.848\\
1.00026158645754	4.856\\
1.0002465773985	4.864\\
1.00023156833946	4.872\\
1.00021870343171	4.88\\
1.00020369437268	4.888\\
1.00018868531364	4.896\\
1.0001736762546	4.904\\
1.00015866719556	4.912\\
1.00014580228781	4.92\\
1.00013079322877	4.928\\
1.00011578416973	4.936\\
1.00010077511069	4.944\\
1.00008576605165	4.952\\
1.0000729011439	4.96\\
1.00005789208487	4.968\\
1.00004288302583	4.976\\
1.00002787396679	4.984\\
1.00001500905904	4.992\\
1	5\\
0.999989279243543	5.008\\
0.999978558487087	5.016\\
0.999969981881922	5.024\\
0.999959261125465	5.032\\
0.999948540369008	5.04\\
0.999939963763843	5.048\\
0.999929243007387	5.056\\
0.99991852225093	5.064\\
0.999909945645765	5.072\\
0.999899224889308	5.08\\
0.999888504132852	5.088\\
0.999879927527686	5.096\\
0.99986920677123	5.104\\
0.999858486014773	5.112\\
0.999849909409608	5.12\\
0.999839188653151	5.128\\
0.999828467896695	5.136\\
0.999819891291529	5.144\\
0.999809170535073	5.152\\
0.999798449778616	5.16\\
0.999789873173451	5.168\\
0.999779152416995	5.176\\
0.999768431660538	5.184\\
0.999759855055373	5.192\\
0.999749134298916	5.2\\
0.99973841354246	5.208\\
0.999729836937294	5.216\\
0.999719116180838	5.224\\
0.999708395424381	5.232\\
0.999699818819216	5.24\\
0.999689098062759	5.248\\
0.999678377306303	5.256\\
0.999669800701138	5.264\\
0.999659079944681	5.272\\
0.999648359188224	5.28\\
0.999639782583059	5.288\\
0.999629061826602	5.296\\
0.999618341070146	5.304\\
0.999609764464981	5.312\\
0.999599043708524	5.32\\
0.999588322952068	5.328\\
0.999579746346902	5.336\\
0.999569025590446	5.344\\
0.999558304833989	5.352\\
0.999549728228824	5.36\\
0.999539007472367	5.368\\
0.999528286715911	5.376\\
0.999519710110745	5.384\\
0.999508989354289	5.392\\
0.999498268597832	5.4\\
0.999489691992667	5.408\\
0.99947897123621	5.416\\
0.999468250479754	5.424\\
0.999459673874589	5.432\\
0.999448953118132	5.44\\
0.999438232361675	5.448\\
0.99942965575651	5.456\\
0.999418935000054	5.464\\
0.999408214243597	5.472\\
0.999399637638432	5.48\\
0.999388916881975	5.488\\
0.999378196125519	5.496\\
0.999369619520353	5.504\\
0.999358898763897	5.512\\
0.99934817800744	5.52\\
0.999339601402275	5.528\\
0.999328880645818	5.536\\
0.999318159889362	5.544\\
0.999309583284197	5.552\\
0.99929886252774	5.56\\
0.999288141771283	5.568\\
0.999279565166118	5.576\\
0.999268844409661	5.584\\
0.999258123653205	5.592\\
0.99924954704804	5.6\\
0.999238826291583	5.608\\
0.999228105535127	5.616\\
0.999219528929961	5.624\\
0.999208808173505	5.632\\
0.999198087417048	5.64\\
0.999189510811883	5.648\\
0.999178790055426	5.656\\
0.99916806929897	5.664\\
0.999159492693804	5.672\\
0.999148771937348	5.68\\
0.999138051180891	5.688\\
0.999129474575726	5.696\\
0.99911875381927	5.704\\
0.999108033062813	5.712\\
0.999099456457648	5.72\\
0.999088735701191	5.728\\
0.999078014944734	5.736\\
0.999069438339569	5.744\\
0.999058717583113	5.752\\
0.999047996826656	5.76\\
0.999039420221491	5.768\\
0.999028699465034	5.776\\
0.999017978708578	5.784\\
0.999007257952121	5.792\\
0.998998681346956	5.8\\
0.998987960590499	5.808\\
0.998981528136625	5.816\\
0.998975095682751	5.824\\
0.998968663228877	5.832\\
0.998962230775003	5.84\\
0.998955798321129	5.848\\
0.998949365867256	5.856\\
0.998942933413382	5.864\\
0.998936500959508	5.872\\
0.998930068505634	5.88\\
0.99892363605176	5.888\\
0.998917203597886	5.896\\
0.998910771144012	5.904\\
0.998902194538847	5.912\\
0.998895762084973	5.92\\
0.998889329631099	5.928\\
0.998882897177225	5.936\\
0.998876464723351	5.944\\
0.998870032269477	5.952\\
0.998863599815603	5.96\\
0.998857167361729	5.968\\
0.998850734907855	5.976\\
0.998844302453981	5.984\\
0.998837870000107	5.992\\
0.998831437546233	6\\
0.998825005092359	6.008\\
0.998818572638485	6.016\\
0.998812140184612	6.024\\
0.998805707730737	6.032\\
0.998797131125572	6.04\\
0.998790698671698	6.048\\
0.998784266217824	6.056\\
0.99877783376395	6.064\\
0.998771401310076	6.072\\
0.998764968856202	6.08\\
0.998758536402328	6.088\\
0.998752103948455	6.096\\
0.998745671494581	6.104\\
0.998739239040707	6.112\\
0.998732806586833	6.12\\
0.998726374132959	6.128\\
0.998719941679085	6.136\\
0.998713509225211	6.144\\
0.998707076771337	6.152\\
0.998700644317463	6.16\\
0.998692067712298	6.168\\
0.998685635258424	6.176\\
0.99867920280455	6.184\\
0.998672770350676	6.192\\
0.998666337896802	6.2\\
0.998659905442928	6.208\\
0.998653472989054	6.216\\
0.99864704053518	6.224\\
0.998640608081306	6.232\\
0.998634175627432	6.24\\
0.998627743173558	6.248\\
0.998621310719684	6.256\\
0.99861487826581	6.264\\
0.998608445811936	6.272\\
0.998602013358063	6.28\\
0.998595580904189	6.288\\
0.998589148450315	6.296\\
0.998580571845149	6.304\\
0.998574139391275	6.312\\
0.998567706937401	6.32\\
0.998561274483527	6.328\\
0.998554842029654	6.336\\
0.99854840957578	6.344\\
0.998541977121906	6.352\\
0.998535544668032	6.36\\
0.998529112214158	6.368\\
0.998522679760284	6.376\\
0.99851624730641	6.384\\
0.998509814852536	6.392\\
0.998503382398662	6.4\\
0.998496949944788	6.408\\
0.998490517490914	6.416\\
0.99848408503704	6.424\\
0.998475508431875	6.432\\
0.998469075978001	6.44\\
0.998462643524127	6.448\\
0.998454066918962	6.456\\
0.99843476955734	6.464\\
0.998417616347009	6.472\\
0.998400463136679	6.48\\
0.998381165775057	6.488\\
0.998364012564727	6.496\\
0.998346859354396	6.504\\
0.998327561992774	6.512\\
0.998310408782444	6.52\\
0.998293255572113	6.528\\
0.998273958210491	6.536\\
0.998256805000161	6.544\\
0.99823965178983	6.552\\
0.998220354428208	6.56\\
0.998203201217878	6.568\\
0.998186048007547	6.576\\
0.998166750645926	6.584\\
0.998149597435595	6.592\\
0.998132444225265	6.6\\
0.998113146863643	6.608\\
0.998095993653312	6.616\\
0.998078840442982	6.624\\
0.99805954308136	6.632\\
0.998042389871029	6.64\\
0.998025236660699	6.648\\
0.998005939299077	6.656\\
0.997988786088746	6.664\\
0.997971632878416	6.672\\
0.997952335516794	6.68\\
0.997935182306464	6.688\\
0.997918029096133	6.696\\
0.997898731734511	6.704\\
0.997881578524181	6.712\\
0.99786442531385	6.72\\
0.997845127952228	6.728\\
0.997827974741898	6.736\\
0.997810821531567	6.744\\
0.997791524169945	6.752\\
0.997774370959615	6.76\\
0.997757217749284	6.768\\
0.997737920387663	6.776\\
0.997720767177332	6.784\\
0.997703613967001	6.792\\
0.99768431660538	6.8\\
0.997667163395049	6.808\\
0.997650010184719	6.816\\
0.997630712823097	6.824\\
0.997613559612766	6.832\\
0.997596406402436	6.84\\
0.997577109040814	6.848\\
0.997559955830483	6.856\\
0.997542802620153	6.864\\
0.997523505258531	6.872\\
0.9975063520482	6.88\\
0.99748919883787	6.888\\
0.997469901476248	6.896\\
0.997452748265918	6.904\\
0.997435595055587	6.912\\
0.997416297693965	6.92\\
0.997399144483635	6.928\\
0.997381991273304	6.936\\
0.997362693911682	6.944\\
0.997345540701352	6.952\\
0.997328387491021	6.96\\
0.9973090901294	6.968\\
0.997201882564834	6.976\\
0.997062512730898	6.984\\
0.996920998745672	6.992\\
0.996781628911736	7\\
0.996640114926509	7.008\\
0.996500745092574	7.016\\
0.996359231107347	7.024\\
0.996219861273412	7.032\\
0.996078347288185	7.04\\
0.995938977454249	7.048\\
0.995797463469022	7.056\\
0.995658093635087	7.064\\
0.99551657964986	7.072\\
0.995377209815925	7.08\\
0.995235695830698	7.088\\
0.995094181845471	7.096\\
0.994954812011535	7.104\\
0.994813298026309	7.112\\
0.994673928192373	7.12\\
0.994532414207146	7.128\\
0.994393044373211	7.136\\
0.994251530387984	7.144\\
0.994112160554049	7.152\\
0.993970646568822	7.16\\
0.993831276734886	7.168\\
0.99368976274966	7.176\\
0.993550392915724	7.184\\
0.993408878930497	7.192\\
0.993269509096562	7.2\\
0.993127995111335	7.208\\
0.9929886252774	7.216\\
0.992847111292173	7.224\\
0.992707741458237	7.232\\
0.99256622747301	7.24\\
0.992426857639075	7.248\\
0.992285343653848	7.256\\
0.992145973819913	7.264\\
0.992004459834686	7.272\\
0.99186509000075	7.28\\
0.991723576015524	7.288\\
0.991584206181588	7.296\\
0.991442692196361	7.304\\
0.991301178211135	7.312\\
0.991161808377199	7.32\\
0.991020294391972	7.328\\
0.990880924558037	7.336\\
0.99073941057281	7.344\\
0.990600040738875	7.352\\
0.990458526753648	7.36\\
0.990319156919712	7.368\\
0.990177642934485	7.376\\
0.990018975738928	7.384\\
0.989495802823847	7.392\\
0.988974774060058	7.4\\
0.988453745296268	7.408\\
0.987930572381187	7.416\\
0.987409543617398	7.424\\
0.986888514853608	7.432\\
0.986367486089818	7.44\\
0.985844313174738	7.448\\
0.985323284410948	7.456\\
0.984802255647158	7.464\\
0.984281226883369	7.472\\
0.983758053968288	7.48\\
0.983237025204498	7.488\\
0.982715996440709	7.496\\
0.982194967676919	7.504\\
0.981671794761838	7.512\\
0.981150765998049	7.52\\
0.980629737234259	7.528\\
0.98010870847047	7.536\\
0.979585535555389	7.544\\
0.979064506791599	7.552\\
0.97854347802781	7.56\\
0.97802244926402	7.568\\
0.977499276348939	7.576\\
0.97697824758515	7.584\\
0.97645721882136	7.592\\
0.97593619005757	7.6\\
0.97541301714249	7.608\\
0.9748919883787	7.616\\
0.97437095961491	7.624\\
0.973849930851121	7.632\\
0.97332675793604	7.64\\
0.97280572917225	7.648\\
0.972284700408461	7.656\\
0.971763671644671	7.664\\
0.97124049872959	7.672\\
0.970719469965801	7.68\\
0.970198441202011	7.688\\
0.969677412438222	7.696\\
0.969154239523141	7.704\\
0.968633210759351	7.712\\
0.96778412684799	7.72\\
0.966799961405277	7.728\\
0.965815795962563	7.736\\
0.964833774671141	7.744\\
0.963849609228427	7.752\\
0.962865443785713	7.76\\
0.961881278343	7.768\\
0.960899257051578	7.776\\
0.959915091608864	7.784\\
0.95893092616615	7.792\\
0.957948904874728	7.8\\
0.956964739432014	7.808\\
0.955980573989301	7.816\\
0.954998552697878	7.824\\
0.954014387255165	7.832\\
0.953030221812451	7.84\\
0.952048200521029	7.848\\
0.951064035078315	7.856\\
0.950079869635601	7.864\\
0.949097848344179	7.872\\
0.948113682901466	7.88\\
0.947129517458752	7.888\\
0.94614749616733	7.896\\
0.945163330724616	7.904\\
0.944179165281902	7.912\\
0.94319714399048	7.92\\
0.942212978547766	7.928\\
0.941228813105053	7.936\\
0.94024679181363	7.944\\
0.939262626370917	7.952\\
0.938278460928203	7.96\\
0.937296439636781	7.968\\
0.936312274194067	7.976\\
0.935137279286426	7.984\\
0.93385507681422	7.992\\
0.932570730190722	8\\
0.931288527718516	8.008\\
0.930004181095018	8.016\\
0.928721978622812	8.024\\
0.927437631999314	8.032\\
0.926155429527107	8.04\\
0.924873227054901	8.048\\
0.923588880431403	8.056\\
0.922306677959197	8.064\\
0.921022331335699	8.072\\
0.919740128863493	8.08\\
0.918455782239995	8.088\\
0.917173579767788	8.096\\
0.915889233144291	8.104\\
0.914607030672084	8.112\\
0.913322684048586	8.12\\
0.91204048157638	8.128\\
0.910758279104174	8.136\\
0.909473932480676	8.144\\
0.908191730008469	8.152\\
0.906907383384972	8.16\\
0.905625180912765	8.168\\
0.904340834289267	8.176\\
0.903058631817061	8.184\\
0.901727113865154	8.192\\
0.900172604178951	8.2\\
0.898618094492747	8.208\\
0.897063584806544	8.216\\
0.895509075120341	8.224\\
0.893954565434137	8.232\\
0.892400055747933	8.24\\
0.89084554606173	8.248\\
0.889291036375527	8.256\\
0.887736526689323	8.264\\
0.88618201700312	8.272\\
0.884627507316916	8.28\\
0.883072997630713	8.288\\
0.881518487944509	8.296\\
0.879963978258306	8.304\\
0.878409468572102	8.312\\
0.876854958885899	8.32\\
0.875300449199696	8.328\\
0.873745939513492	8.336\\
0.872191429827289	8.344\\
0.870636920141085	8.352\\
0.869078122152299	8.36\\
0.867191269015942	8.368\\
0.865304415879584	8.376\\
0.863417562743227	8.384\\
0.86153070960687	8.392\\
0.859646000621804	8.4\\
0.857759147485447	8.408\\
0.855872294349089	8.416\\
0.853985441212732	8.424\\
0.852098588076375	8.432\\
0.850213879091309	8.44\\
0.848327025954951	8.448\\
0.846440172818594	8.456\\
0.844553319682237	8.464\\
0.842666466545879	8.472\\
0.840779613409522	8.48\\
0.838894904424456	8.488\\
0.836973744867438	8.496\\
0.834670926380565	8.504\\
0.832368107893693	8.512\\
0.830065289406821	8.52\\
0.827764615071239	8.528\\
0.825461796584367	8.536\\
0.823158978097495	8.544\\
0.820856159610622	8.552\\
0.81855334112375	8.56\\
0.816250522636877	8.568\\
0.813947704150005	8.576\\
0.811644885663132	8.584\\
0.80934206717626	8.592\\
0.807039248689387	8.6\\
0.804446969778188	8.608\\
0.801603825165904	8.616\\
0.798758536402329	8.624\\
0.795913247638753	8.632\\
0.79307010302647	8.64\\
0.790224814262894	8.648\\
0.787379525499319	8.656\\
0.784536380887035	8.664\\
0.78169109212346	8.672\\
0.778845803359885	8.68\\
0.776002658747601	8.688\\
0.772655638581858	8.696\\
0.769089914984401	8.704\\
0.765522047235653	8.712\\
0.761956323638196	8.72\\
0.758390600040739	8.728\\
0.75482273229199	8.736\\
0.751257008694533	8.744\\
0.747689140945785	8.752\\
0.744097687532832	8.76\\
0.739552086795244	8.768\\
0.735004341906365	8.776\\
0.730456597017486	8.784\\
0.725908852128606	8.792\\
0.721363251391018	8.8\\
0.716815506502139	8.808\\
0.712147689140946	8.816\\
0.706234119879499	8.824\\
0.700324838920634	8.832\\
0.694411269659187	8.84\\
0.688499844549031	8.848\\
0.682588419438876	8.856\\
0.675619927742101	8.864\\
0.667750892502975	8.872\\
0.659881857263849	8.88\\
0.652012822024722	8.888\\
0.643845749756103	8.896\\
0.633043515550457	8.904\\
0.622241281344812	8.912\\
0.611439047139166	8.92\\
0.59805954308136	8.928\\
0.582574482455482	8.936\\
0.567089421829604	8.944\\
0.545772269691349	8.952\\
0.522053668106822	8.96\\
0.491042807980531	8.968\\
0.450610547080202	8.976\\
0.380037951477856	8.984\\
0.293354203072569	8.992\\
0	9\\
};
\addlegendentry{Upwind}

\end{axis}
\end{tikzpicture}%}
	\subfloat[Velocity profiles at $x/H = 1$]{\input{img/bfs_fin_vel1}}\\
	\subfloat[Velocity profiles at $x/H = 4$]{% This file was created by matlab2tikz.
%
\definecolor{mycolor4}{rgb}{0.00000,0.44700,0.74100}%
\definecolor{mycolor3}{rgb}{0.85000,0.32500,0.09800}%
\definecolor{mycolor2}{rgb}{0.92900,0.69400,0.12500}%
\definecolor{mycolor1}{rgb}{0.49400,0.18400,0.55600}%
%
\begin{tikzpicture}

\begin{axis}[%
width=0.958\bfshalfwidth,
height=0.75\bfsheight,
at={(0\bfshalfwidth,0\bfsheight)},
scale only axis,
xmin=-0.2,
xmax=1,
xlabel={$u/U_\text{ref}$},
ymin=0,
ymax=3,
ylabel style={rotate=-90},
ylabel={$\dfrac{y}{H}$},
axis background/.style={fill=white},
legend style={at={(0.03,0.97)}, anchor=north west, legend cell align=left, align=left, font=\scriptsize}
]
\addplot [color=mycolor1]
  table[row sep=crcr]{%
0	0\\
-0.003276591888	0.0001491991279\\
-0.006617190316	0.0003034094116\\
-0.01005433034	0.0004630724143\\
-0.01359560434	0.0006286503049\\
-0.01724872552	0.0008006272255\\
-0.0210214965	0.0009795111837\\
-0.0249218028	0.00116583507\\
-0.02895756625	0.001360159134\\
-0.03313669562	0.001563072088\\
-0.03746701032	0.001775193494\\
-0.04195610806	0.001997175394\\
-0.04661120474	0.002229704522\\
-0.05143888295	0.002473504748\\
-0.05644467846	0.002729339059\\
-0.06163259968	0.002998012351\\
-0.06700434536	0.003280373057\\
-0.07255829871	0.003577317111\\
-0.07828819752	0.003889790038\\
-0.08418145031	0.004218789749\\
-0.09021719545	0.004565369803\\
-0.09636429697	0.004930642899\\
-0.1025795937	0.005315783899\\
-0.1088069826	0.00572203286\\
-0.1149781272	0.006150700618\\
-0.1210152507	0.006603169255\\
-0.1268361509	0.007080899552\\
-0.1323611736	0.007585432846\\
-0.1375207752	0.008118395694\\
-0.1422620416	0.008681505919\\
-0.1465528905	0.009276572615\\
-0.1503828168	0.009905505925\\
-0.1537608504	0.01057032123\\
-0.1567111313	0.01127313823\\
-0.1592675447	0.01201619487\\
-0.1614689082	0.01280184276\\
-0.1633549333	0.01363256015\\
-0.1649634838	0.01451095287\\
-0.1663289517	0.01543976087\\
-0.1674817055	0.01642186195\\
-0.1684478074	0.01746027544\\
-0.1692494303	0.01855817065\\
-0.1699051261	0.01971887052\\
-0.1704303771	0.02094585076\\
-0.1708379984	0.02224275284\\
-0.171138525	0.02361337468\\
-0.171340704	0.02506168745\\
-0.1714515984	0.02659182623\\
-0.1714770049	0.02820809931\\
-0.1714215875	0.02991498262\\
-0.1712890863	0.03171712533\\
-0.1710823774	0.03361934423\\
-0.1708037108	0.03562662005\\
-0.1704547703	0.03774409369\\
-0.1700366288	0.03997706622\\
-0.1695500016	0.04233097658\\
-0.1689951718	0.04481140897\\
-0.1683720499	0.04742406681\\
-0.1676803082	0.05017476901\\
-0.1669192016	0.05306942016\\
-0.1660878509	0.05611399934\\
-0.165185079	0.05931453779\\
-0.1642094851	0.0626770854\\
-0.1631594747	0.06620769203\\
-0.1620332897	0.06991236657\\
-0.160828948	0.07379703224\\
-0.1595443338	0.07786751539\\
-0.158177197	0.082129471\\
-0.1567250937	0.08658834547\\
-0.1551854759	0.09124935418\\
-0.1535556912	0.09611737728\\
-0.1518329978	0.1011969596\\
-0.1500145346	0.1064922065\\
-0.148097381	0.1120067686\\
-0.1460785419	0.1177437529\\
-0.1439550221	0.1237056628\\
-0.141723752	0.1298943758\\
-0.1393817216	0.1363110244\\
-0.1369259357	0.1429560333\\
-0.1343533844	0.1498289704\\
-0.1316612512	0.1569285691\\
-0.128846705	0.1642526984\\
-0.1259071231	0.171798259\\
-0.1228400171	0.1795612723\\
-0.1196430847	0.1875367612\\
-0.1163142622	0.1957188249\\
-0.1128517389	0.2041006386\\
-0.109253943	0.2126744539\\
-0.1055196375	0.2214316726\\
-0.1016478837	0.2303628474\\
-0.09763808548	0.2394578308\\
-0.09348998964	0.2487057447\\
-0.08920376748	0.258095175\\
-0.08477989584	0.2676141858\\
-0.08021926135	0.2772505283\\
-0.07552310079	0.286991626\\
-0.07069294155	0.2968248427\\
-0.0657306239	0.3067375124\\
-0.06063821167	0.3167170882\\
-0.05541799217	0.3267513216\\
-0.05007235706	0.3368283808\\
-0.04460378364	0.3469369113\\
-0.03901472688	0.3570661843\\
-0.03330756724	0.3672063053\\
-0.02748451196	0.3773481846\\
-0.02154751867	0.3874836862\\
-0.01549820136	0.397605747\\
-0.009337735362	0.4077083766\\
-0.003066778183	0.4177867472\\
0.003314632922	0.4278372228\\
0.009807162918	0.4378574789\\
0.01641227491	0.4478463233\\
0.02313230559	0.4578039348\\
0.02997056209	0.4677316844\\
0.03693140671	0.4776322544\\
0.04402032867	0.4875094593\\
0.05124402791	0.4973684549\\
0.05861051008	0.5072153807\\
0.06612915546	0.5170577168\\
0.07381080836	0.5269038081\\
0.08166785538	0.5367631316\\
0.08971433342	0.546646297\\
0.09796600789	0.5565645695\\
0.1064405069	0.5665303469\\
0.1151573732	0.5765568614\\
0.1241382062	0.5866580606\\
0.1334068775	0.5968486667\\
0.1429891884	0.6071442962\\
0.1528553665	0.6175611019\\
0.1628791541	0.6278738976\\
0.1729457527	0.6379622221\\
0.183022052	0.6478335261\\
0.1931058913	0.6574950814\\
0.2031952441	0.6669541597\\
0.2132879347	0.6762180924\\
0.2233819962	0.6852939129\\
0.2334752679	0.6941888332\\
0.2435657084	0.7029097676\\
0.2536511123	0.7114635706\\
0.2637292445	0.7198569179\\
0.2737977207	0.7280963063\\
0.2838541269	0.7361881137\\
0.2938958108	0.7441385388\\
0.3039200902	0.7519535422\\
0.3139240742	0.7596390247\\
0.3239048123	0.7672005892\\
0.3338591158	0.7746436596\\
0.3437838256	0.7819736004\\
0.3536755145	0.7891953588\\
0.363530755	0.7963139415\\
0.3733481169	0.8033339977\\
0.3831413984	0.8102601171\\
0.3929426968	0.8170966506\\
0.4027783871	0.8238478303\\
0.4126566052	0.8305176497\\
0.4225740433	0.8371100426\\
0.4325222969	0.8436287045\\
0.4424921274	0.8500772119\\
0.4524760246	0.856458962\\
0.4624685645	0.8627773523\\
0.472466886	0.8690354228\\
0.4824691713	0.8752363324\\
0.4924746752	0.8813829422\\
0.5024820566	0.8874780536\\
0.5124890804	0.8935243487\\
0.5224920511	0.8995244503\\
0.5324850678	0.9054808617\\
0.5424596071	0.9113958478\\
0.5524048209	0.9172718525\\
0.562306881	0.9231110215\\
0.5721493363	0.9289155006\\
0.58191365	0.9346873164\\
0.5915787816	0.9404284954\\
0.6011229157	0.9461408854\\
0.6105228066	0.9518263936\\
0.6197552681	0.9574867487\\
0.6287974119	0.9631237388\\
0.6376273632	0.9687389135\\
0.6462245584	0.9743340611\\
0.6545704603	0.9799106121\\
0.6626486182	0.9854701161\\
0.6704452634	0.9910140634\\
0.677949369	0.9965439439\\
0.6851526499	1.002061009\\
0.6920498013	1.00756681\\
0.6986383796	1.013062596\\
0.704918921	1.018549681\\
0.7108945251	1.024029255\\
0.7165703177	1.02950263\\
0.7219541669	1.034971118\\
0.7270553708	1.040435791\\
0.7318848372	1.045897961\\
0.7364556193	1.051358581\\
0.7407830954	1.056823611\\
0.7448831797	1.062298656\\
0.7487700582	1.067785144\\
0.7524570823	1.07328403\\
0.7559578419	1.078796625\\
0.759286046	1.084324241\\
0.7624551654	1.089868069\\
0.7654784918	1.095429659\\
0.7683702111	1.101010084\\
0.7711449862	1.106611013\\
0.7738209367	1.112233639\\
0.7764152288	1.117879629\\
0.7789364457	1.123550415\\
0.7813873291	1.129247665\\
0.7837705016	1.13497293\\
0.786088407	1.140727997\\
0.7883442044	1.146514416\\
0.790540874	1.152334332\\
0.7926816344	1.158189297\\
0.7947695851	1.164081573\\
0.7968082428	1.170012951\\
0.798800528	1.175985575\\
0.8007497787	1.182001829\\
0.8026589751	1.18806386\\
0.8045310378	1.194174051\\
0.8063690066	1.200334907\\
0.8081756234	1.206549168\\
0.8099533916	1.212819338\\
0.8117051721	1.219148517\\
0.8134332299	1.225539446\\
0.8151399493	1.231995344\\
0.8168275952	1.238519669\\
0.8184982538	1.245115519\\
0.8201540112	1.251786709\\
0.8217968941	1.258536816\\
0.8234285712	1.265370131\\
0.8250510097	1.272290468\\
0.8266658187	1.279302359\\
0.8282747865	1.286410451\\
0.8298793435	1.293619394\\
0.8314810991	1.300934434\\
0.8330815434	1.308360815\\
0.8346819282	1.31590426\\
0.8362839222	1.323570609\\
0.8378887773	1.331366181\\
0.8394978046	1.339297533\\
0.8411123753	1.347371459\\
0.8427336812	1.355595469\\
0.8443630934	1.363977194\\
0.8460017443	1.372524619\\
0.847651124	1.381246448\\
0.8493122458	1.390151739\\
0.8509865999	1.399249911\\
0.8526751399	1.408550978\\
0.8543794155	1.418065548\\
0.8561006188	1.427804947\\
0.8578398824	1.437780738\\
0.8595987558	1.448005438\\
0.8613783717	1.45849216\\
0.8631801009	1.469254732\\
0.8650054336	1.480307698\\
0.8668557405	1.491666317\\
0.8687322736	1.503346801\\
0.8706384301	1.515366316\\
0.8725886941	1.527776718\\
0.8746096492	1.540907621\\
0.8767162561	1.554798007\\
0.8789104223	1.569488525\\
0.8811938167	1.585021496\\
0.8835683465	1.601441145\\
0.8860354424	1.618793368\\
0.888596952	1.637126088\\
0.8912541866	1.656488895\\
0.8940086961	1.676933408\\
0.8968619108	1.698512673\\
0.8998149037	1.721282005\\
0.9028691649	1.745298147\\
0.9060260057	1.77061975\\
0.9092865586	1.797306657\\
0.9126526713	1.825420737\\
0.9161256552	1.855024934\\
0.9197074175	1.8861835\\
0.9234000444	1.918962002\\
0.9272053242	1.95342648\\
0.9311249852	1.989644051\\
0.9351583123	2.027682066\\
0.9393029213	2.067608118\\
0.9435614944	2.109489679\\
0.9479507804	2.153393745\\
0.9524921179	2.19938612\\
0.9571847916	2.247531652\\
0.9619756341	2.297893524\\
0.9667423964	2.350532293\\
0.9713013768	2.405506372\\
0.9754461646	2.462870359\\
0.9790047407	2.522675276\\
0.9818444252	2.584967613\\
0.9838433862	2.649788618\\
0.9849327207	2.717174053\\
0.9851805568	2.787153482\\
0.9848105311	2.859748602\\
0.9841305017	2.934974194\\
0.9834123254	3.012836218\\
0.982798934	3.093332291\\
0.9822963476	3.176449299\\
0.9818434715	3.26216507\\
0.9813891053	3.350445986\\
0.9809225798	3.441247702\\
0.9804570079	3.53451395\\
0.9800021648	3.630176783\\
0.9795580506	3.728156328\\
0.9791222215	3.828359842\\
0.9786953926	3.930683374\\
0.9782793522	4.035009861\\
0.9778739214	4.14121151\\
0.9774799943	4.249147415\\
0.9770978689	4.358667374\\
0.9767278433	4.469609261\\
0.9763702154	4.581803322\\
0.9760252237	4.695068836\\
0.9756929874	4.80921936\\
0.9753736854	4.924060345\\
0.9750673771	5.039393425\\
0.9747739434	5.155014515\\
0.9744933844	5.270719051\\
0.9742255807	5.386298656\\
0.973970294	5.501547813\\
0.9737274051	5.616261005\\
0.9734965563	5.73023653\\
0.973277688	5.843276978\\
0.9730707407	5.955190659\\
0.9728753567	6.065793991\\
0.9726900458	6.174911022\\
0.9725122452	6.282374859\\
0.9723392725	6.388030052\\
0.9721755981	6.491731167\\
0.9720413685	6.593345642\\
0.9719694257	6.692752838\\
0.9719656706	6.78984499\\
0.9719174504	6.884526253\\
0.9715050459	6.976716042\\
0.9702282548	7.066344261\\
0.9675689936	7.153355122\\
0.9631528258	7.237704754\\
0.9569205642	7.319362164\\
0.9492759705	7.398306847\\
0.9409108162	7.47453022\\
0.9324241281	7.548032761\\
0.9241579175	7.618826866\\
0.9162654281	7.68693161\\
0.9087853432	7.752375603\\
0.9016263485	7.81519413\\
0.8946051002	7.875430584\\
0.8875908256	7.933132648\\
0.8805577755	7.988354206\\
0.8735169172	8.041152954\\
0.866476953	8.091591835\\
0.859444499	8.139735222\\
0.8524245024	8.185649872\\
0.8454203606	8.22940731\\
0.838434279	8.271077156\\
0.8314675689	8.310731888\\
0.8245209455	8.348443985\\
0.817594409	8.38428688\\
0.810687542	8.418331146\\
0.8037998676	8.450650215\\
0.7969304919	8.481315613\\
0.7900784016	8.51039505\\
0.7832427025	8.537959099\\
0.7764223814	8.564073563\\
0.769616425	8.588805199\\
0.7628239989	8.612216949\\
0.7560439706	8.634370804\\
0.7492758036	8.65532589\\
0.7425185442	8.675142288\\
0.7357717752	8.693873405\\
0.7290347219	8.711575508\\
0.7223069668	8.728298187\\
0.715587914	8.744092941\\
0.7088772655	8.7590065\\
0.7021746635	8.773085594\\
0.6954795122	8.786373138\\
0.6887913942	8.798910141\\
0.6821098328	8.810738564\\
0.6754341722	8.821894646\\
0.6687637568	8.832415581\\
0.6620978117	8.842334747\\
0.6554353833	8.851687431\\
0.6487755179	8.860502243\\
0.6421166658	8.8688097\\
0.6354573965	8.876639366\\
0.6287962794	8.884016037\\
0.6221312284	8.890965462\\
0.615460217	8.897512436\\
0.6087809205	8.903678894\\
0.6020907164	8.909486771\\
0.5953868032	8.914956093\\
0.588666141	8.920106888\\
0.5819255114	8.924956322\\
0.5751610994	8.929522514\\
0.5683692098	8.933821678\\
0.5615454316	8.937869072\\
0.5546854138	8.941678047\\
0.5477839112	8.945264816\\
0.5408358574	8.94863987\\
0.5338354111	8.951816559\\
0.5267763138	8.954807281\\
0.5196519494	8.957620621\\
0.5124549866	8.960268974\\
0.5051777959	8.962760925\\
0.4978118539	8.96510601\\
0.4903479815	8.967312813\\
0.482776314	8.969389915\\
0.4750860333	8.971343994\\
0.4672654271	8.973181725\\
0.4593017399	8.97491169\\
0.4511811435	8.976539612\\
0.4428882599	8.978070259\\
0.4344065189	8.979511261\\
0.4257176518	8.980866432\\
0.4168016613	8.982141495\\
0.4076366127	8.983341217\\
0.3981984556	8.984470367\\
0.3884610534	8.985531807\\
0.3783957362	8.986530304\\
0.3679716289	8.987470627\\
0.3571555316	8.988354683\\
0.3459125161	8.989186287\\
0.3342064321	8.9899683\\
0.3220015764	8.990704536\\
0.3092646003	8.991396904\\
0.2959683836	8.992048264\\
0.282096833	8.992661476\\
0.2676522136	8.993237495\\
0.252663672	8.993780136\\
0.2371961325	8.994289398\\
0.2213572413	8.99477005\\
0.2052977532	8.995221138\\
0.1892025769	8.995645523\\
0.1732715815	8.996045113\\
0.1576952487	8.99642086\\
0.1426326782	8.996774673\\
0.1281988919	8.997106552\\
0.1144631132	8.997420311\\
0.1014553159	8.997714043\\
0.08917656541	8.997990608\\
0.07760902494	8.998251915\\
0.06672399491	8.998496056\\
0.05648751184	8.998726845\\
0.04686375707	8.998943329\\
0.03781713545	8.999147415\\
0.02931324393	8.999340057\\
0.02131936327	8.999520302\\
0.01380462479	8.999690056\\
0.006739998702	8.999849319\\
0	9\\
};
\addlegendentry{CFL3D}

\addplot [color=mycolor2, draw=none, mark=x, mark options={solid, mycolor2}, only marks]
  table[row sep=crcr]{%
-0.171	0.1\\
-0.154	0.15\\
-0.134	0.2\\
-0.057	0.3\\
0.052	0.4\\
0.169	0.5\\
0.279	0.6\\
0.381	0.7\\
0.473	0.8\\
0.568	0.9\\
0.656	1\\
0.73	1.1\\
0.754	1.15\\
0.778	1.2\\
0.811	1.3\\
0.835	1.4\\
0.856	1.5\\
0.895	1.7\\
0.94	2\\
0.978	2.4\\
0.986	2.8\\
0.984	3.2\\
0.985	3.6\\
0.985	4\\
0.983	5\\
0.98	6\\
0.98	7.5\\
0.892	8.2\\
};
\addlegendentry{Experiment}

\addplot [color=mycolor3]
  table[row sep=crcr]{%
0	0\\
-0.128323879039327	0.009\\
-0.160634419916379	0.018\\
-0.16483581245138	0.027\\
-0.164580792960605	0.036\\
-0.162535493784168	0.045\\
-0.159679918393763	0.054\\
-0.156788125949626	0.063\\
-0.153644314244017	0.072\\
-0.150500288236315	0.081\\
-0.147264112328585	0.09\\
-0.143984433095958	0.099\\
-0.140704968165424	0.108\\
-0.137309351500436	0.117\\
-0.133898090782653	0.126\\
-0.130486615762776	0.135\\
-0.126985991072175	0.144\\
-0.123402002867362	0.153\\
-0.119817800360456	0.162\\
-0.116233812155643	0.171\\
-0.11253324361219	0.18\\
-0.108706451135908	0.189\\
-0.104879658659626	0.198\\
-0.101052866183344	0.207\\
-0.097226288009155	0.216\\
-0.0931813360021087	0.225\\
-0.0890125173852573	0.234\\
-0.0848436987684059	0.243\\
-0.0806746658494614	0.252\\
-0.0765058472326099	0.261\\
-0.0723370286157585	0.27\\
-0.0677736798455311	0.279\\
-0.0631278247694645	0.288\\
-0.0584817553913049	0.297\\
-0.0538359003152384	0.306\\
-0.0491898309370788	0.315\\
-0.0445439758610122	0.324\\
-0.0398979064828526	0.333\\
-0.0346901513187079	0.342\\
-0.0293861745147665	0.351\\
-0.024081983408732	0.36\\
-0.0187778565933253	0.369\\
-0.0134737297779187	0.378\\
-0.00816960296251213	0.387\\
-0.00286549757731484	0.396\\
0.00243862923809177	0.405\\
0.00802396326004916	0.414\\
0.0139069843195159	0.423\\
0.0197899839487732	0.432\\
0.0256729621478213	0.441\\
0.031555983207288	0.45\\
0.0374390042667547	0.459\\
0.0433220253262214	0.468\\
0.0492050463856881	0.477\\
0.0550880674451547	0.486\\
0.0609710885046214	0.495\\
0.0673125017412045	0.504\\
0.0742269587747064	0.513\\
0.0811414158082082	0.522\\
0.08805587284171	0.531\\
0.0949703298752119	0.54\\
0.101884786908714	0.549\\
0.108799243942216	0.558\\
0.115713700975717	0.567\\
0.122628158009219	0.576\\
0.129542615042721	0.585\\
0.137080048260831	0.594\\
0.145540266291781	0.603\\
0.15400048432273	0.612\\
0.16246070235368	0.621\\
0.170920920384629	0.63\\
0.179381352717672	0.639\\
0.187841570748621	0.648\\
0.196301788779571	0.657\\
0.204804010020766	0.666\\
0.214456390595567	0.675\\
0.224108556868275	0.684\\
0.233760723140983	0.693\\
0.243412889413691	0.702\\
0.253065055686399	0.711\\
0.262717221959107	0.72\\
0.272442250943465	0.729\\
0.283277364770022	0.738\\
0.294110335575648	0.747\\
0.304943306381273	0.756\\
0.315776277186899	0.765\\
0.326611391013456	0.774\\
0.33772938360289	0.783\\
0.349696012480954	0.792\\
0.361664784379949	0.801\\
0.373631413258013	0.81\\
0.385600185157008	0.819\\
0.397965415928217	0.828\\
0.410909262350765	0.837\\
0.423850965752382	0.846\\
0.436792669154	0.855\\
0.450002250171977	0.864\\
0.463657579543579	0.873\\
0.477315051936112	0.882\\
0.490974667349576	0.891\\
0.505058600907355	0.9\\
0.519140391444203	0.909\\
0.533254327295015	0.918\\
0.547492558359817	0.927\\
0.56173078942462	0.936\\
0.575904729861497	0.945\\
0.590067955193718	0.954\\
0.603993305202612	0.963\\
0.61778364489286	0.972\\
0.631303963945816	0.981\\
0.644378534645148	0.99\\
0.657620260977089	0.999\\
0.669784047780795	1.008\\
0.680754171925997	1.017\\
0.690957094577943	1.026\\
0.700699267729748	1.035\\
0.70929706770446	1.044\\
0.717828434030315	1.053\\
0.725136135404635	1.062\\
0.732343114795202	1.071\\
0.739020768015841	1.08\\
0.744903360471122	1.089\\
0.750785952926402	1.098\\
0.756383523597875	1.107\\
0.761089597562099	1.116\\
0.765795671526324	1.125\\
0.770501745490548	1.134\\
0.774927083712827	1.143\\
0.778694514509323	1.152\\
0.78246194530582	1.161\\
0.786227233081386	1.17\\
0.789994663877882	1.179\\
0.79349850309988	1.188\\
0.796543735842668	1.197\\
0.799591111606387	1.206\\
0.802636344349175	1.215\\
0.805681577091963	1.224\\
0.808728952855683	1.233\\
0.811664891530996	1.242\\
0.814161510915477	1.251\\
0.81666027332089	1.26\\
0.819156892705371	1.269\\
0.821655655110784	1.278\\
0.824154417516196	1.287\\
0.826651036900677	1.296\\
0.82914979930609	1.305\\
0.83144926076493	1.314\\
0.833545135235336	1.323\\
0.835641009705742	1.332\\
0.837739027197079	1.341\\
0.839834901667485	1.35\\
0.841930776137891	1.359\\
0.844026650608297	1.368\\
0.846124668099633	1.377\\
0.848220542570039	1.386\\
0.850316417040445	1.395\\
0.852159415041007	1.404\\
0.853961695643881	1.413\\
0.855763976246756	1.422\\
0.857566256849631	1.431\\
0.859368537452505	1.44\\
0.86117081805538	1.449\\
0.862973098658255	1.458\\
0.864775379261129	1.467\\
0.866577659864004	1.476\\
0.868379940466878	1.485\\
0.870180078048822	1.494\\
0.871982358651697	1.503\\
0.873596053412654	1.512\\
0.875164744734062	1.521\\
0.876735579076401	1.53\\
0.87830641341874	1.539\\
0.879877247761079	1.548\\
0.881445939082487	1.557\\
0.883016773424826	1.566\\
0.884587607767165	1.575\\
0.886158442109504	1.584\\
0.887729276451843	1.593\\
0.889297967773251	1.602\\
0.89086880211559	1.611\\
0.892439636457929	1.62\\
0.894010470800268	1.629\\
0.895579162121676	1.638\\
0.896993555936061	1.647\\
0.898360803289966	1.656\\
0.899728050643871	1.665\\
0.901095297997776	1.674\\
0.90246040233075	1.683\\
0.903827649684655	1.692\\
0.905194897038559	1.701\\
0.906562144392464	1.71\\
0.907927248725438	1.719\\
0.909294496079343	1.728\\
0.910661743433248	1.737\\
0.912028990787153	1.746\\
0.913394095120127	1.755\\
0.914761342474032	1.764\\
0.916128589827937	1.773\\
0.917493694160911	1.782\\
0.918860941514816	1.791\\
0.920228188868721	1.8\\
0.921595436222626	1.809\\
0.922739809399718	1.818\\
0.923871324451226	1.827\\
0.925004982523664	1.836\\
0.926136497575172	1.845\\
0.92727015564761	1.854\\
0.928403813720049	1.863\\
0.929535328771556	1.872\\
0.930668986843995	1.881\\
0.931802644916433	1.89\\
0.93293415996794	1.899\\
0.934067818040379	1.908\\
0.935201476112817	1.917\\
0.936332991164325	1.926\\
0.937466649236763	1.935\\
0.938598164288271	1.944\\
0.939731822360709	1.953\\
0.940865480433147	1.962\\
0.941996995484655	1.971\\
0.943130653557093	1.98\\
0.944264311629532	1.989\\
0.945395826681039	1.998\\
0.946529484753478	2.007\\
0.947660999804985	2.016\\
0.948642503391331	2.025\\
0.949422563010173	2.034\\
0.950202622629015	2.043\\
0.950982682247857	2.052\\
0.9517627418667	2.061\\
0.952542801485542	2.07\\
0.953322861104384	2.079\\
0.954102920723227	2.088\\
0.954882980342069	2.097\\
0.955663039960911	2.106\\
0.956443099579754	2.115\\
0.957223159198596	2.124\\
0.958003218817438	2.133\\
0.958783278436281	2.142\\
0.959563338055123	2.151\\
0.960343397673965	2.16\\
0.961123457292807	2.169\\
0.96190351691165	2.178\\
0.962683576530492	2.187\\
0.963463636149334	2.196\\
0.964243695768176	2.205\\
0.965023755387019	2.214\\
0.965803815005861	2.223\\
0.966583874624703	2.232\\
0.967366077264477	2.241\\
0.968146136883319	2.25\\
0.968926196502161	2.259\\
0.969706256121003	2.268\\
0.970486315739846	2.277\\
0.971251374212172	2.286\\
0.971600686623906	2.295\\
0.97194999903564	2.304\\
0.972299311447375	2.313\\
0.972648623859109	2.322\\
0.972997936270844	2.331\\
0.973347248682578	2.34\\
0.973696561094312	2.349\\
0.974045873506047	2.358\\
0.974395185917781	2.367\\
0.974744498329515	2.376\\
0.97509381074125	2.385\\
0.975443123152984	2.394\\
0.975792435564718	2.403\\
0.976141747976452	2.412\\
0.976491060388187	2.421\\
0.976840372799921	2.43\\
0.977189685211656	2.439\\
0.97753899762339	2.448\\
0.977888310035124	2.457\\
0.978237622446858	2.466\\
0.978586934858593	2.475\\
0.978936247270327	2.484\\
0.979285559682061	2.493\\
0.979634872093796	2.502\\
0.97998418450553	2.511\\
0.980333496917264	2.52\\
0.980682809328999	2.529\\
0.981032121740733	2.538\\
0.981381434152467	2.547\\
0.981730746564202	2.556\\
0.982080058975936	2.565\\
0.98242937138767	2.574\\
0.982778683799405	2.583\\
0.983127996211139	2.592\\
0.983477308622873	2.601\\
0.983826621034608	2.61\\
0.984090212609106	2.619\\
0.984137359069586	2.628\\
0.984186648550996	2.637\\
0.984235938032407	2.646\\
0.984283084492886	2.655\\
0.984332373974297	2.664\\
0.984381663455707	2.673\\
0.984428809916187	2.682\\
0.984478099397597	2.691\\
0.984527388879007	2.7\\
0.984574535339487	2.709\\
0.984623824820897	2.718\\
0.984670971281376	2.727\\
0.984720260762787	2.736\\
0.984769550244197	2.745\\
0.984816696704677	2.754\\
0.984865986186087	2.763\\
0.984915275667497	2.772\\
0.984962422127977	2.781\\
0.985011711609387	2.79\\
0.985061001090798	2.799\\
0.985108147551277	2.808\\
0.985157437032688	2.817\\
0.985204583493167	2.826\\
0.985253872974577	2.835\\
0.985303162455988	2.844\\
0.985350308916467	2.853\\
0.985399598397877	2.862\\
0.985448887879288	2.871\\
0.985496034339767	2.88\\
0.985545323821178	2.889\\
0.985594613302588	2.898\\
0.985641759763068	2.907\\
0.985691049244478	2.916\\
0.985738195704957	2.925\\
0.985787485186368	2.934\\
0.985836774667778	2.943\\
0.985883921128258	2.952\\
0.985933210609668	2.961\\
0.985982500091078	2.97\\
0.986029646551558	2.979\\
0.986078936032968	2.988\\
0.986128225514379	2.997\\
0.986175371974858	3.006\\
0.986224661456268	3.015\\
0.986271807916748	3.024\\
0.986288952084195	3.033\\
0.986246091665577	3.042\\
0.98620323124696	3.051\\
0.986160370828342	3.06\\
0.986117510409724	3.069\\
0.986074649991106	3.078\\
0.986031789572489	3.087\\
0.985988929153871	3.096\\
0.985948211756184	3.105\\
0.985905351337567	3.114\\
0.985862490918949	3.123\\
0.985819630500331	3.132\\
0.985776770081713	3.141\\
0.985733909663096	3.15\\
0.985691049244478	3.159\\
0.98564818882586	3.168\\
0.985605328407243	3.177\\
0.985562467988625	3.186\\
0.985519607570007	3.195\\
0.985476747151389	3.204\\
0.985433886732772	3.213\\
0.985391026314154	3.222\\
0.985348165895536	3.231\\
0.985305305476919	3.24\\
0.985262445058301	3.249\\
0.985221727660614	3.258\\
0.985178867241996	3.267\\
0.985136006823379	3.276\\
0.985093146404761	3.285\\
0.985050285986143	3.294\\
0.985007425567526	3.303\\
0.984964565148908	3.312\\
0.98492170473029	3.321\\
0.984878844311672	3.33\\
0.984835983893055	3.339\\
0.984793123474437	3.348\\
0.984750263055819	3.357\\
0.984707402637202	3.366\\
0.984664542218584	3.375\\
0.984621681799966	3.384\\
0.984578821381348	3.393\\
0.984538103983662	3.402\\
0.984495243565044	3.411\\
0.984452383146426	3.42\\
0.984409522727808	3.429\\
0.984366662309191	3.438\\
0.984323801890573	3.447\\
0.984280941471955	3.456\\
0.984238081053338	3.465\\
0.98419522063472	3.474\\
0.984152360216102	3.483\\
0.984109499797485	3.492\\
0.984066639378867	3.501\\
0.984023778960249	3.51\\
0.983980918541631	3.519\\
0.983938058123014	3.528\\
0.983895197704396	3.537\\
0.983852337285778	3.546\\
0.983813762909022	3.555\\
0.983775188532266	3.564\\
0.98373447113458	3.573\\
0.983695896757824	3.582\\
0.983655179360137	3.591\\
0.983616604983381	3.6\\
0.983575887585694	3.609\\
0.983537313208938	3.618\\
0.983498738832182	3.627\\
0.983458021434495	3.636\\
0.983419447057739	3.645\\
0.983378729660053	3.654\\
0.983340155283297	3.663\\
0.98329943788561	3.672\\
0.983260863508854	3.681\\
0.983222289132098	3.69\\
0.983181571734411	3.699\\
0.983142997357655	3.708\\
0.983102279959968	3.717\\
0.983063705583212	3.726\\
0.983025131206457	3.735\\
0.98298441380877	3.744\\
0.982945839432014	3.753\\
0.982905122034327	3.762\\
0.982866547657571	3.771\\
0.982825830259884	3.78\\
0.982787255883128	3.789\\
0.982748681506372	3.798\\
0.982707964108685	3.807\\
0.98266938973193	3.816\\
0.982628672334243	3.825\\
0.982590097957487	3.834\\
0.982551523580731	3.843\\
0.982510806183044	3.852\\
0.982472231806288	3.861\\
0.982431514408601	3.87\\
0.982392940031845	3.879\\
0.982352222634158	3.888\\
0.982313648257403	3.897\\
0.982275073880647	3.906\\
0.98223435648296	3.915\\
0.982195782106204	3.924\\
0.982155064708517	3.933\\
0.982116490331761	3.942\\
0.982075772934074	3.951\\
0.982037198557318	3.96\\
0.981998624180562	3.969\\
0.981957906782876	3.978\\
0.98191933240612	3.987\\
0.981878615008433	3.996\\
0.981840040631677	4.005\\
0.981801466254921	4.014\\
0.981760748857234	4.023\\
0.981722174480478	4.032\\
0.981681457082791	4.041\\
0.981642882706035	4.05\\
0.981602165308349	4.059\\
0.981563590931593	4.068\\
0.981525016554837	4.077\\
0.98148429915715	4.086\\
0.981445724780394	4.095\\
0.981405007382707	4.104\\
0.981366433005951	4.113\\
0.981327858629195	4.122\\
0.981287141231509	4.131\\
0.981248566854752	4.14\\
0.981207849457066	4.149\\
0.98116927508031	4.158\\
0.981128557682623	4.167\\
0.981089983305867	4.176\\
0.981051408929111	4.185\\
0.981012834552355	4.194\\
0.980987118301184	4.203\\
0.980961402050014	4.212\\
0.980935685798843	4.221\\
0.980912112568603	4.23\\
0.980886396317433	4.239\\
0.980860680066262	4.248\\
0.980834963815092	4.257\\
0.980809247563921	4.266\\
0.98078353131275	4.275\\
0.98075781506158	4.284\\
0.980732098810409	4.293\\
0.980706382559238	4.302\\
0.980680666308068	4.311\\
0.980654950056897	4.32\\
0.980629233805727	4.329\\
0.980605660575487	4.338\\
0.980579944324316	4.347\\
0.980554228073146	4.356\\
0.980528511821975	4.365\\
0.980502795570804	4.374\\
0.980477079319634	4.383\\
0.980451363068463	4.392\\
0.980425646817292	4.401\\
0.980399930566122	4.41\\
0.980374214314951	4.419\\
0.980348498063781	4.428\\
0.98032278181261	4.437\\
0.98029920858237	4.446\\
0.9802734923312	4.455\\
0.980247776080029	4.464\\
0.980222059828858	4.473\\
0.980196343577688	4.482\\
0.980170627326517	4.491\\
0.980144911075347	4.5\\
0.980119194824176	4.509\\
0.980093478573005	4.518\\
0.980067762321835	4.527\\
0.980042046070664	4.536\\
0.980016329819493	4.545\\
0.979992756589254	4.554\\
0.979967040338083	4.563\\
0.979941324086912	4.572\\
0.979915607835742	4.581\\
0.979889891584571	4.59\\
0.979864175333401	4.599\\
0.97983845908223	4.608\\
0.979812742831059	4.617\\
0.979787026579889	4.626\\
0.979761310328718	4.635\\
0.979735594077547	4.644\\
0.979712020847308	4.653\\
0.979686304596137	4.662\\
0.979660588344966	4.671\\
0.979634872093796	4.68\\
0.979609155842625	4.689\\
0.979583439591454	4.698\\
0.979557723340284	4.707\\
0.979532007089113	4.716\\
0.979506290837943	4.725\\
0.979480574586772	4.734\\
0.979454858335601	4.743\\
0.979429142084431	4.752\\
0.97940342583326	4.761\\
0.97937985260302	4.77\\
0.97935413635185	4.779\\
0.979328420100679	4.788\\
0.979302703849508	4.797\\
0.979276987598338	4.806\\
0.979251271347167	4.815\\
0.979225555095997	4.824\\
0.979199838844826	4.833\\
0.979174122593655	4.842\\
0.979148406342485	4.851\\
0.979122690091314	4.86\\
0.979099116861074	4.869\\
0.979073400609904	4.878\\
0.979047684358733	4.887\\
0.979021968107562	4.896\\
0.978996251856392	4.905\\
0.978970535605221	4.914\\
0.978944819354051	4.923\\
0.97891910310288	4.932\\
0.978893386851709	4.941\\
0.978867670600539	4.95\\
0.978841954349368	4.959\\
0.978816238098198	4.968\\
0.978792664867958	4.977\\
0.978766948616787	4.986\\
0.978741232365617	4.995\\
0.978719802156308	5.004\\
0.978702657988861	5.013\\
0.978687656842344	5.022\\
0.978670512674897	5.031\\
0.978655511528381	5.04\\
0.978638367360934	5.049\\
0.978623366214418	5.058\\
0.978606222046971	5.067\\
0.978591220900454	5.076\\
0.978574076733007	5.085\\
0.978559075586491	5.094\\
0.978541931419044	5.103\\
0.978526930272528	5.112\\
0.978509786105081	5.121\\
0.978494784958565	5.13\\
0.978477640791118	5.139\\
0.978462639644601	5.148\\
0.978445495477154	5.157\\
0.978430494330638	5.166\\
0.978413350163191	5.175\\
0.978398349016675	5.184\\
0.978381204849228	5.193\\
0.978366203702712	5.202\\
0.978349059535264	5.211\\
0.978334058388748	5.22\\
0.978316914221301	5.229\\
0.978301913074785	5.238\\
0.978284768907338	5.247\\
0.978269767760822	5.256\\
0.978252623593375	5.265\\
0.978237622446858	5.274\\
0.978222621300342	5.283\\
0.978205477132895	5.292\\
0.978190475986379	5.301\\
0.978173331818932	5.31\\
0.978158330672416	5.319\\
0.978141186504969	5.328\\
0.978126185358452	5.337\\
0.978109041191005	5.346\\
0.978094040044489	5.355\\
0.978076895877042	5.364\\
0.978061894730526	5.373\\
0.978044750563079	5.382\\
0.978029749416563	5.391\\
0.978012605249115	5.4\\
0.977997604102599	5.409\\
0.977980459935152	5.418\\
0.977965458788636	5.427\\
0.977948314621189	5.436\\
0.977933313474673	5.445\\
0.977916169307226	5.454\\
0.977901168160709	5.463\\
0.977884023993262	5.472\\
0.977869022846746	5.481\\
0.977851878679299	5.49\\
0.977836877532783	5.499\\
0.977819733365336	5.508\\
0.97780473221882	5.517\\
0.977787588051373	5.526\\
0.977772586904856	5.535\\
0.977755442737409	5.544\\
0.977740441590893	5.553\\
0.977723297423446	5.562\\
0.97770829627693	5.571\\
0.977691152109483	5.58\\
0.977676150962966	5.589\\
0.977659006795519	5.598\\
0.977644005649003	5.607\\
0.977626861481556	5.616\\
0.97761186033504	5.625\\
0.977594716167593	5.634\\
0.977579715021077	5.643\\
0.97756257085363	5.652\\
0.977547569707113	5.661\\
0.977530425539666	5.67\\
0.97751542439315	5.679\\
0.977498280225703	5.688\\
0.977483279079187	5.697\\
0.97746613491174	5.706\\
0.977451133765224	5.715\\
0.977433989597776	5.724\\
0.97741898845126	5.733\\
0.977401844283813	5.742\\
0.977386843137297	5.751\\
0.97736969896985	5.76\\
0.977354697823334	5.769\\
0.977337553655887	5.778\\
0.97732255250937	5.787\\
0.977305408341923	5.796\\
0.977290407195407	5.805\\
0.977277549069822	5.814\\
0.977266833965167	5.823\\
0.977253975839582	5.832\\
0.977243260734928	5.841\\
0.977232545630273	5.85\\
0.977219687504688	5.859\\
0.977208972400033	5.868\\
0.977196114274448	5.877\\
0.977185399169794	5.886\\
0.977174684065139	5.895\\
0.977161825939554	5.904\\
0.977151110834899	5.913\\
0.977138252709314	5.922\\
0.97712753760466	5.931\\
0.977116822500005	5.94\\
0.97710396437442	5.949\\
0.977093249269766	5.958\\
0.977082534165111	5.967\\
0.977069676039526	5.976\\
0.977058960934871	5.985\\
0.977046102809286	5.994\\
0.977035387704632	6.003\\
0.977024672599977	6.012\\
0.977011814474392	6.021\\
0.977001099369738	6.03\\
0.976988241244152	6.039\\
0.976977526139498	6.048\\
0.976966811034843	6.057\\
0.976953952909258	6.066\\
0.976943237804604	6.075\\
0.976932522699949	6.084\\
0.976919664574364	6.093\\
0.976908949469709	6.102\\
0.976896091344124	6.111\\
0.97688537623947	6.12\\
0.976874661134815	6.129\\
0.97686180300923	6.138\\
0.976851087904576	6.147\\
0.976840372799921	6.156\\
0.976827514674336	6.165\\
0.976816799569681	6.174\\
0.976803941444096	6.183\\
0.976793226339442	6.192\\
0.976782511234787	6.201\\
0.976769653109202	6.21\\
0.976758938004547	6.219\\
0.976746079878962	6.228\\
0.976735364774308	6.237\\
0.976724649669653	6.246\\
0.976711791544068	6.255\\
0.976701076439414	6.264\\
0.976690361334759	6.273\\
0.976677503209174	6.282\\
0.976666788104519	6.291\\
0.976653929978934	6.3\\
0.97664321487428	6.309\\
0.976632499769625	6.318\\
0.97661964164404	6.327\\
0.976608926539385	6.336\\
0.976598211434731	6.345\\
0.976585353309146	6.354\\
0.976574638204491	6.363\\
0.976561780078906	6.372\\
0.976551064974252	6.381\\
0.976540349869597	6.39\\
0.976527491744012	6.399\\
0.976516776639358	6.408\\
0.976503918513772	6.417\\
0.976493203409118	6.426\\
0.976482488304463	6.435\\
0.976469630178878	6.444\\
0.976458915074224	6.453\\
0.976413911634675	6.462\\
0.976366765174195	6.471\\
0.976319618713716	6.48\\
0.976272472253237	6.489\\
0.976225325792757	6.498\\
0.976178179332277	6.507\\
0.976133175892729	6.516\\
0.976086029432249	6.525\\
0.97603888297177	6.534\\
0.97599173651129	6.543\\
0.975944590050811	6.552\\
0.975897443590332	6.561\\
0.975850297129852	6.57\\
0.975803150669373	6.579\\
0.975758147229824	6.588\\
0.975711000769344	6.597\\
0.975663854308865	6.606\\
0.975616707848386	6.615\\
0.975569561387906	6.624\\
0.975522414927427	6.633\\
0.975475268466947	6.642\\
0.975428122006468	6.651\\
0.975383118566919	6.66\\
0.97533597210644	6.669\\
0.97528882564596	6.678\\
0.975241679185481	6.687\\
0.975194532725001	6.696\\
0.975147386264522	6.705\\
0.975100239804042	6.714\\
0.975053093343563	6.723\\
0.975008089904014	6.732\\
0.974960943443535	6.741\\
0.974913796983055	6.75\\
0.974866650522576	6.759\\
0.974819504062096	6.768\\
0.974772357601617	6.777\\
0.974725211141137	6.786\\
0.974678064680658	6.795\\
0.974633061241109	6.804\\
0.97458591478063	6.813\\
0.97453876832015	6.822\\
0.974491621859671	6.831\\
0.974444475399191	6.84\\
0.974397328938712	6.849\\
0.974350182478232	6.858\\
0.974303036017753	6.867\\
0.974258032578204	6.876\\
0.974210886117725	6.885\\
0.974163739657245	6.894\\
0.974116593196766	6.903\\
0.974069446736286	6.912\\
0.974022300275807	6.921\\
0.973975153815327	6.93\\
0.973928007354848	6.939\\
0.973883003915299	6.948\\
0.97383585745482	6.957\\
0.97378871099434	6.966\\
0.973602268173353	6.975\\
0.973297959201167	6.984\\
0.972991507208051	6.993\\
0.972687198235865	7.002\\
0.972380746242749	7.011\\
0.972076437270563	7.02\\
0.971769985277446	7.029\\
0.97146567630526	7.038\\
0.971159224312144	7.047\\
0.970854915339958	7.056\\
0.970548463346842	7.065\\
0.970244154374656	7.074\\
0.969937702381539	7.083\\
0.969633393409353	7.092\\
0.969326941416237	7.101\\
0.969022632444051	7.11\\
0.968716180450934	7.119\\
0.968411871478749	7.128\\
0.968105419485632	7.137\\
0.967801110513446	7.146\\
0.96749465852033	7.155\\
0.967190349548144	7.164\\
0.966883897555027	7.173\\
0.966579588582842	7.182\\
0.966273136589725	7.191\\
0.965968827617539	7.2\\
0.965662375624423	7.209\\
0.965358066652237	7.218\\
0.96505161465912	7.227\\
0.964747305686935	7.236\\
0.964440853693818	7.245\\
0.964136544721632	7.254\\
0.963830092728516	7.263\\
0.96352578375633	7.272\\
0.963219331763213	7.281\\
0.962915022791028	7.29\\
0.962608570797911	7.299\\
0.962304261825725	7.308\\
0.961997809832609	7.317\\
0.961693500860423	7.326\\
0.961387048867306	7.335\\
0.96108273989512	7.344\\
0.960776287902004	7.353\\
0.960471978929818	7.362\\
0.960165526936702	7.371\\
0.959861217964516	7.38\\
0.959244027936421	7.389\\
0.958421107898961	7.398\\
0.957598187861501	7.407\\
0.956775267824041	7.416\\
0.955952347786581	7.425\\
0.955129427749121	7.434\\
0.95430436469073	7.443\\
0.95348144465327	7.452\\
0.95265852461581	7.461\\
0.95183560457835	7.47\\
0.95101268454089	7.479\\
0.95018976450343	7.488\\
0.94936684446597	7.497\\
0.94854392442851	7.506\\
0.94772100439105	7.515\\
0.94689808435359	7.524\\
0.94607516431613	7.533\\
0.94525224427867	7.542\\
0.94442932424121	7.551\\
0.94360640420375	7.56\\
0.94278348416629	7.569\\
0.941958421107899	7.578\\
0.941135501070439	7.587\\
0.940312581032979	7.596\\
0.939489660995519	7.605\\
0.938666740958059	7.614\\
0.937843820920599	7.623\\
0.937020900883139	7.632\\
0.936197980845679	7.641\\
0.935375060808219	7.65\\
0.934552140770759	7.659\\
0.933729220733299	7.668\\
0.932906300695839	7.677\\
0.932083380658379	7.686\\
0.931260460620919	7.695\\
0.930435397562528	7.704\\
0.929612477525068	7.713\\
0.928435959034012	7.722\\
0.92719729293596	7.731\\
0.925958626837908	7.74\\
0.924719960739857	7.749\\
0.923481294641805	7.758\\
0.922240485522822	7.767\\
0.92100181942477	7.776\\
0.919763153326718	7.785\\
0.918524487228667	7.794\\
0.917285821130615	7.803\\
0.916047155032563	7.812\\
0.914808488934511	7.821\\
0.91356982283646	7.83\\
0.912331156738408	7.839\\
0.911092490640356	7.848\\
0.909853824542304	7.857\\
0.908615158444253	7.866\\
0.907376492346201	7.875\\
0.906137826248149	7.884\\
0.904899160150097	7.893\\
0.903660494052045	7.902\\
0.902421827953994	7.911\\
0.901183161855942	7.92\\
0.89994449575789	7.929\\
0.898705829659838	7.938\\
0.897467163561786	7.947\\
0.896228497463735	7.956\\
0.894989831365683	7.965\\
0.893751165267631	7.974\\
0.892403205102104	7.983\\
0.890926663680724	7.992\\
0.889450122259344	8.001\\
0.887973580837964	8.01\\
0.886499182437515	8.019\\
0.885022641016135	8.028\\
0.883546099594755	8.037\\
0.882069558173375	8.046\\
0.880593016751995	8.055\\
0.879116475330615	8.064\\
0.877639933909235	8.073\\
0.876163392487854	8.082\\
0.874686851066474	8.091\\
0.873210309645094	8.1\\
0.871733768223714	8.109\\
0.870257226802334	8.118\\
0.868782828401885	8.127\\
0.867306286980505	8.136\\
0.865829745559125	8.145\\
0.864353204137745	8.154\\
0.862876662716365	8.163\\
0.861400121294985	8.172\\
0.859923579873605	8.181\\
0.858447038452225	8.19\\
0.856700476393553	8.199\\
0.854936770167434	8.208\\
0.853170920920385	8.217\\
0.851407214694266	8.226\\
0.849643508468147	8.235\\
0.847877659221098	8.244\\
0.846113952994979	8.253\\
0.844348103747929	8.262\\
0.842584397521811	8.271\\
0.840818548274761	8.28\\
0.839054842048642	8.289\\
0.837288992801593	8.298\\
0.835525286575474	8.307\\
0.833761580349355	8.316\\
0.831995731102306	8.325\\
0.830232024876187	8.334\\
0.828466175629137	8.343\\
0.826702469403019	8.352\\
0.82489161671642	8.361\\
0.822748595785535	8.37\\
0.82060771787558	8.379\\
0.818464696944695	8.388\\
0.81632167601381	8.397\\
0.814180798103855	8.406\\
0.81203777717297	8.415\\
0.809896899263015	8.424\\
0.80775387833213	8.433\\
0.805610857401244	8.442\\
0.80346997949129	8.451\\
0.801326958560404	8.46\\
0.79918608065045	8.469\\
0.797043059719564	8.478\\
0.794900038788679	8.487\\
0.79272487254383	8.496\\
0.790101814924426	8.505\\
0.787478757305023	8.514\\
0.784855699685619	8.523\\
0.782232642066215	8.532\\
0.779609584446811	8.541\\
0.776986526827408	8.55\\
0.774363469208004	8.559\\
0.7717404115886	8.568\\
0.769117353969196	8.577\\
0.766494296349792	8.586\\
0.763871238730389	8.595\\
0.761231036943538	8.604\\
0.757984360233246	8.613\\
0.754739826543886	8.622\\
0.751495292854525	8.631\\
0.748248616144234	8.64\\
0.745004082454873	8.649\\
0.741759548765513	8.658\\
0.738512872055221	8.667\\
0.735268338365861	8.676\\
0.732021661655569	8.685\\
0.728455674826576	8.694\\
0.724394650162548	8.703\\
0.72033362549852	8.712\\
0.716274743855423	8.721\\
0.712213719191395	8.73\\
0.708152694527368	8.739\\
0.704093812884271	8.748\\
0.700032788220243	8.757\\
0.695228135293197	8.766\\
0.690084885059072	8.775\\
0.684941634824947	8.784\\
0.679798384590822	8.793\\
0.674657277377628	8.802\\
0.669514027143503	8.811\\
0.66360143239519	8.82\\
0.656988069802478	8.829\\
0.650374707209765	8.838\\
0.643761344617053	8.847\\
0.63714798202434	8.856\\
0.629310954480092	8.865\\
0.620627433668145	8.874\\
0.611941769835266	8.883\\
0.603258249023318	8.892\\
0.592596719892163	8.901\\
0.58085082216998	8.91\\
0.569104924447797	8.919\\
0.5549781304714	8.928\\
0.538365432215176	8.937\\
0.521752733958952	8.946\\
0.496861545846718	8.955\\
0.471653190636713	8.964\\
0.429675696642529	8.973\\
0.369673253598668	8.982\\
0.275341758262953	8.991\\
0	9\\
};
\addlegendentry{Van Leer}

\addplot [color=mycolor4]
  table[row sep=crcr]{%
0	0\\
-0.143853254285622	0.009\\
-0.172158838727661	0.018\\
-0.174433354417488	0.027\\
-0.172845610386269	0.036\\
-0.16978890830537	0.045\\
-0.166103755480987	0.054\\
-0.162390514274687	0.063\\
-0.158473793110842	0.072\\
-0.154557071946997	0.081\\
-0.150542791899396	0.09\\
-0.14648305584442	0.099\\
-0.142423319789444	0.108\\
-0.138208990426364	0.117\\
-0.133973219550371	0.126\\
-0.129737234259249	0.135\\
-0.125373028720907	0.144\\
-0.120887893049734	0.153\\
-0.11640297179369	0.162\\
-0.111917836122517	0.171\\
-0.107264384574976	0.18\\
-0.102428465752544	0.189\\
-0.0975925469301114	0.198\\
-0.0927566281076793	0.207\\
-0.0879207092852472	0.216\\
-0.0827783912432861	0.225\\
-0.0774619681164703	0.234\\
-0.0721455449896545	0.243\\
-0.0668291218628386	0.252\\
-0.0615126987360228	0.261\\
-0.056196275609207	0.27\\
-0.050375762513803	0.279\\
-0.0444495427597371	0.288\\
-0.0385235374208004	0.297\\
-0.0325975320818637	0.306\\
-0.0266713123277979	0.315\\
-0.0207453069888611	0.324\\
-0.0148192373253857	0.333\\
-0.00830121037340395	0.342\\
-0.00168161711890391	0.351\\
0.00493798042389871	0.36\\
0.0115575329395242	0.369\\
0.0181771283381755	0.378\\
0.024796680853801	0.387\\
0.0314163191354782	0.396\\
0.0380359574171554	0.405\\
0.0448685099220601	0.414\\
0.0519261983125529	0.423\\
0.0589841011181749	0.432\\
0.0660420039237969	0.441\\
0.0730999067294188	0.45\\
0.0801578095350408	0.459\\
0.0872157123406628	0.468\\
0.0942736151462847	0.477\\
0.101331517951907	0.486\\
0.108389420757529	0.495\\
0.115865218649828	0.504\\
0.123863117381562	0.513\\
0.131861230528426	0.522\\
0.13985934367529	0.531\\
0.147857242407024	0.54\\
0.155855355553888	0.549\\
0.163853468700752	0.558\\
0.171851367432486	0.567\\
0.17984948057935	0.576\\
0.187847379311084	0.585\\
0.196455503500327	0.594\\
0.205966958628601	0.603\\
0.215478628172004	0.612\\
0.224990083300278	0.621\\
0.234501538428552	0.63\\
0.244012993556825	0.639\\
0.253524448685099	0.648\\
0.263035903813373	0.657\\
0.272583809513599	0.666\\
0.283167340287531	0.675\\
0.293750871061462	0.684\\
0.304332257684102	0.693\\
0.314915788458034	0.702\\
0.325499319231965	0.711\\
0.336080705854605	0.72\\
0.346726417015985	0.729\\
0.358300545686504	0.738\\
0.369874674357023	0.747\\
0.381448803027542	0.756\\
0.393020787546769	0.765\\
0.404594916217288	0.774\\
0.416374883411774	0.783\\
0.428765933724284	0.792\\
0.441159128188085	0.801\\
0.453550178500595	0.81\\
0.465941228813105	0.819\\
0.478559559162494	0.828\\
0.491501656356872	0.837\\
0.504445897702542	0.846\\
0.51738799489692	0.855\\
0.530424434748116	0.864\\
0.543615253492286	0.873\\
0.556806072236457	0.882\\
0.569996890980628	0.891\\
0.583138394245098	0.9\\
0.596279897509568	0.909\\
0.609357076235299	0.918\\
0.62218982171382	0.927\\
0.635022567192341	0.936\\
0.647400752597103	0.945\\
0.659710325160543	0.954\\
0.671400238000793	0.963\\
0.682809267021881	0.972\\
0.693688690674014	0.981\\
0.703751192684156	0.99\\
0.713326972351169	0.999\\
0.721320368365192	1.008\\
0.72805085926863	1.017\\
0.734159546297587	1.026\\
0.739929457422516	1.035\\
0.745077564672963	1.044\\
0.75019136550275	1.053\\
0.754741254542921	1.062\\
0.759246116405974	1.071\\
0.763525842383438	1.08\\
0.767464648305584	1.089\\
0.771401310076439	1.098\\
0.775211466921106	1.107\\
0.778616379171714	1.116\\
0.782021291422323	1.125\\
0.785428347824222	1.134\\
0.788687457787021	1.143\\
0.791607791845793	1.152\\
0.794528125904564	1.161\\
0.797446315812044	1.17\\
0.800366649870815	1.179\\
0.803134749187903	1.188\\
0.805636973744867	1.197\\
0.808139198301832	1.206\\
0.810643567010088	1.215\\
0.813145791567053	1.224\\
0.815650160275309	1.233\\
0.818083771990952	1.242\\
0.820247220643889	1.251\\
0.822408525145534	1.26\\
0.824571973798471	1.269\\
0.826735422451408	1.278\\
0.828896726953054	1.287\\
0.831060175605991	1.296\\
0.833221480107636	1.305\\
0.835247703077929	1.314\\
0.837130267911704	1.323\\
0.83901497689677	1.332\\
0.840899685881836	1.341\\
0.842782250715611	1.35\\
0.844666959700676	1.359\\
0.846551668685742	1.368\\
0.848434233519517	1.377\\
0.850318942504583	1.386\\
0.852203651489649	1.395\\
0.853882521950749	1.404\\
0.85553137429377	1.413\\
0.8571780824855	1.422\\
0.85882479067723	1.431\\
0.860473643020252	1.44\\
0.862120351211981	1.449\\
0.863767059403711	1.458\\
0.865415911746733	1.467\\
0.867062619938463	1.476\\
0.868709328130193	1.485\\
0.870356036321923	1.494\\
0.872004888664944	1.503\\
0.873475776450786	1.512\\
0.874905925362093	1.521\\
0.876338218424692	1.53\\
0.87777051148729	1.539\\
0.879200660398598	1.548\\
0.880632953461196	1.557\\
0.882065246523795	1.566\\
0.883497539586393	1.575\\
0.8849276884977	1.584\\
0.886359981560299	1.593\\
0.887792274622897	1.602\\
0.889222423534205	1.611\\
0.890654716596803	1.62\\
0.892087009659402	1.629\\
0.893519302722	1.638\\
0.894786496135167	1.647\\
0.896006518219926	1.656\\
0.897224396153393	1.665\\
0.898444418238151	1.674\\
0.899662296171618	1.683\\
0.900882318256376	1.692\\
0.902100196189843	1.701\\
0.903320218274601	1.71\\
0.904538096208068	1.719\\
0.905758118292827	1.728\\
0.906975996226294	1.737\\
0.908196018311052	1.746\\
0.909413896244519	1.755\\
0.910633918329277	1.764\\
0.911853940414036	1.773\\
0.913071818347503	1.782\\
0.914291840432261	1.791\\
0.915509718365728	1.8\\
0.916729740450486	1.809\\
0.917726770800948	1.818\\
0.918715224546244	1.827\\
0.919701534140249	1.836\\
0.920689987885545	1.845\\
0.92167629747955	1.854\\
0.922664751224846	1.863\\
0.923651060818851	1.872\\
0.924639514564148	1.881\\
0.925625824158153	1.89\\
0.926614277903449	1.899\\
0.927600587497454	1.908\\
0.92858904124275	1.917\\
0.929575350836755	1.926\\
0.930563804582051	1.935\\
0.931550114176056	1.944\\
0.932538567921352	1.953\\
0.933524877515358	1.962\\
0.934513331260654	1.971\\
0.935499640854659	1.98\\
0.936488094599955	1.989\\
0.93747440419396	1.998\\
0.938462857939256	2.007\\
0.939449167533261	2.016\\
0.940315404654952	2.025\\
0.941018686278504	2.034\\
0.941724112053346	2.043\\
0.942429537828189	2.052\\
0.943134963603032	2.061\\
0.943838245226583	2.07\\
0.944543671001426	2.079\\
0.945249096776269	2.088\\
0.94595237839982	2.097\\
0.946657804174663	2.106\\
0.947363229949505	2.115\\
0.948066511573057	2.124\\
0.948771937347899	2.133\\
0.949477363122742	2.142\\
0.950182788897585	2.151\\
0.950886070521136	2.16\\
0.951591496295979	2.169\\
0.952296922070821	2.178\\
0.953000203694373	2.187\\
0.953705629469215	2.196\\
0.954411055244058	2.205\\
0.955116481018901	2.214\\
0.955819762642452	2.223\\
0.956525188417295	2.232\\
0.957230614192137	2.241\\
0.957933895815689	2.25\\
0.958639321590531	2.259\\
0.959344747365374	2.268\\
0.960048028988925	2.277\\
0.96074058985602	2.286\\
0.961096518970379	2.295\\
0.961452448084737	2.304\\
0.961808377199095	2.313\\
0.962164306313453	2.322\\
0.962520235427812	2.331\\
0.96287616454217	2.34\\
0.963232093656528	2.349\\
0.963588022770887	2.358\\
0.963941807733954	2.367\\
0.964297736848312	2.376\\
0.96465366596267	2.385\\
0.965009595077029	2.394\\
0.965365524191387	2.403\\
0.965721453305745	2.412\\
0.966077382420103	2.421\\
0.966433311534462	2.43\\
0.96678924064882	2.439\\
0.967145169763179	2.448\\
0.967498954726245	2.457\\
0.967854883840604	2.466\\
0.968210812954962	2.475\\
0.96856674206932	2.484\\
0.968922671183679	2.493\\
0.969278600298037	2.502\\
0.969634529412395	2.511\\
0.969990458526754	2.52\\
0.970346387641112	2.529\\
0.970700172604179	2.538\\
0.971056101718537	2.547\\
0.971412030832896	2.556\\
0.971767959947254	2.565\\
0.972123889061612	2.574\\
0.97247981817597	2.583\\
0.972835747290329	2.592\\
0.973191676404687	2.601\\
0.973547605519045	2.61\\
0.973824201035625	2.619\\
0.973903534633404	2.628\\
0.973982868231182	2.637\\
0.974062201828961	2.646\\
0.97414153542674	2.655\\
0.974220869024518	2.664\\
0.974300202622297	2.673\\
0.974379536220076	2.682\\
0.974458869817854	2.691\\
0.974538203415633	2.7\\
0.974617537013412	2.709\\
0.97469687061119	2.718\\
0.974776204208969	2.727\\
0.974855537806748	2.736\\
0.974937015555818	2.745\\
0.975016349153596	2.754\\
0.975095682751375	2.763\\
0.975175016349154	2.772\\
0.975254349946932	2.781\\
0.975333683544711	2.79\\
0.97541301714249	2.799\\
0.975492350740268	2.808\\
0.975571684338047	2.817\\
0.975651017935825	2.826\\
0.975730351533604	2.835\\
0.975809685131383	2.844\\
0.975889018729162	2.853\\
0.97596835232694	2.862\\
0.976047685924719	2.871\\
0.976129163673789	2.88\\
0.976208497271567	2.889\\
0.976287830869346	2.898\\
0.976367164467125	2.907\\
0.976446498064903	2.916\\
0.976525831662682	2.925\\
0.976605165260461	2.934\\
0.976684498858239	2.943\\
0.976763832456018	2.952\\
0.976843166053797	2.961\\
0.976922499651575	2.97\\
0.977001833249354	2.979\\
0.977081166847133	2.988\\
0.977160500444911	2.997\\
0.97723983404269	3.006\\
0.977319167640469	3.015\\
0.977400645389539	3.024\\
0.977443528415365	3.033\\
0.977426375205035	3.042\\
0.977407077843413	3.051\\
0.977387780481791	3.06\\
0.97737062727146	3.069\\
0.977351329909838	3.078\\
0.977332032548217	3.087\\
0.977312735186595	3.096\\
0.977295581976264	3.105\\
0.977276284614642	3.114\\
0.977256987253021	3.123\\
0.977237689891399	3.132\\
0.977220536681068	3.141\\
0.977201239319446	3.15\\
0.977181941957825	3.159\\
0.977164788747494	3.168\\
0.977145491385872	3.177\\
0.97712619402425	3.186\\
0.977106896662629	3.195\\
0.977089743452298	3.204\\
0.977070446090676	3.213\\
0.977051148729054	3.222\\
0.977031851367433	3.231\\
0.977014698157102	3.24\\
0.97699540079548	3.249\\
0.976976103433858	3.258\\
0.976958950223528	3.267\\
0.976939652861906	3.276\\
0.976920355500284	3.285\\
0.976901058138662	3.294\\
0.976883904928332	3.303\\
0.97686460756671	3.312\\
0.976845310205088	3.321\\
0.976826012843466	3.33\\
0.976808859633136	3.339\\
0.976789562271514	3.348\\
0.976770264909892	3.357\\
0.97675096754827	3.366\\
0.97673381433794	3.375\\
0.976714516976318	3.384\\
0.976695219614696	3.393\\
0.976678066404365	3.402\\
0.976658769042744	3.411\\
0.976639471681122	3.42\\
0.9766201743195	3.429\\
0.976603021109169	3.438\\
0.976583723747548	3.447\\
0.976564426385926	3.456\\
0.976545129024304	3.465\\
0.976527975813973	3.474\\
0.976508678452352	3.483\\
0.97648938109073	3.492\\
0.976472227880399	3.501\\
0.976452930518777	3.51\\
0.976433633157156	3.519\\
0.976414335795534	3.528\\
0.976397182585203	3.537\\
0.976377885223581	3.546\\
0.976354299559377	3.555\\
0.976328569743881	3.564\\
0.976304984079677	3.573\\
0.976281398415472	3.582\\
0.976257812751268	3.591\\
0.976232082935772	3.6\\
0.976208497271567	3.609\\
0.976184911607363	3.618\\
0.976159181791867	3.627\\
0.976135596127663	3.636\\
0.976112010463458	3.645\\
0.976088424799254	3.654\\
0.976062694983758	3.663\\
0.976039109319554	3.672\\
0.976015523655349	3.681\\
0.975989793839853	3.69\\
0.975966208175649	3.699\\
0.975942622511444	3.708\\
0.97591903684724	3.717\\
0.975893307031744	3.726\\
0.97586972136754	3.735\\
0.975846135703335	3.744\\
0.975820405887839	3.753\\
0.975796820223635	3.762\\
0.975773234559431	3.771\\
0.975747504743935	3.78\\
0.97572391907973	3.789\\
0.975700333415526	3.798\\
0.975676747751321	3.807\\
0.975651017935825	3.816\\
0.975627432271621	3.825\\
0.975603846607417	3.834\\
0.975578116791921	3.843\\
0.975554531127716	3.852\\
0.975530945463512	3.861\\
0.975507359799307	3.87\\
0.975481629983812	3.879\\
0.975458044319607	3.888\\
0.975434458655403	3.897\\
0.975408728839907	3.906\\
0.975385143175702	3.915\\
0.975361557511498	3.924\\
0.975337971847294	3.933\\
0.975312242031798	3.942\\
0.975288656367593	3.951\\
0.975265070703389	3.96\\
0.975239340887893	3.969\\
0.975215755223689	3.978\\
0.975192169559484	3.987\\
0.975166439743988	3.996\\
0.975142854079784	4.005\\
0.975119268415579	4.014\\
0.975095682751375	4.023\\
0.975069952935879	4.032\\
0.975046367271675	4.041\\
0.97502278160747	4.05\\
0.974997051791974	4.059\\
0.97497346612777	4.068\\
0.974949880463565	4.077\\
0.974926294799361	4.086\\
0.974900564983865	4.095\\
0.974876979319661	4.104\\
0.974853393655456	4.113\\
0.974827663839961	4.122\\
0.974804078175756	4.131\\
0.974780492511552	4.14\\
0.974756906847347	4.149\\
0.974731177031851	4.158\\
0.974707591367647	4.167\\
0.974684005703442	4.176\\
0.974658275887947	4.185\\
0.974636834375033	4.194\\
0.974619681164703	4.203\\
0.974604672105664	4.212\\
0.974587518895333	4.221\\
0.974572509836294	4.23\\
0.974555356625963	4.239\\
0.974540347566924	4.248\\
0.974523194356594	4.257\\
0.974508185297555	4.266\\
0.974491032087224	4.275\\
0.974476023028185	4.284\\
0.974458869817854	4.293\\
0.974443860758815	4.302\\
0.974426707548485	4.311\\
0.974411698489445	4.32\\
0.974396689430406	4.329\\
0.974379536220076	4.338\\
0.974364527161036	4.347\\
0.974347373950706	4.356\\
0.974332364891667	4.365\\
0.974315211681336	4.374\\
0.974300202622297	4.383\\
0.974283049411967	4.392\\
0.974268040352927	4.401\\
0.974250887142597	4.41\\
0.974235878083558	4.419\\
0.974218724873227	4.428\\
0.974203715814188	4.437\\
0.974186562603857	4.446\\
0.974171553544818	4.455\\
0.974154400334488	4.464\\
0.974139391275448	4.473\\
0.974122238065118	4.482\\
0.974107229006079	4.491\\
0.974090075795748	4.5\\
0.974075066736709	4.509\\
0.974057913526378	4.518\\
0.974042904467339	4.527\\
0.9740278954083	4.536\\
0.974010742197969	4.545\\
0.97399573313893	4.554\\
0.9739785799286	4.563\\
0.973963570869561	4.572\\
0.97394641765923	4.581\\
0.973931408600191	4.59\\
0.97391425538986	4.599\\
0.973899246330821	4.608\\
0.973882093120491	4.617\\
0.973867084061451	4.626\\
0.973849930851121	4.635\\
0.973834921792082	4.644\\
0.973817768581751	4.653\\
0.973802759522712	4.662\\
0.973785606312381	4.671\\
0.973770597253342	4.68\\
0.973753444043012	4.689\\
0.973738434983972	4.698\\
0.973721281773642	4.707\\
0.973706272714603	4.716\\
0.973689119504272	4.725\\
0.973674110445233	4.734\\
0.973659101386194	4.743\\
0.973641948175863	4.752\\
0.973626939116824	4.761\\
0.973609785906493	4.77\\
0.973594776847454	4.779\\
0.973577623637124	4.788\\
0.973562614578085	4.797\\
0.973545461367754	4.806\\
0.973530452308715	4.815\\
0.973513299098384	4.824\\
0.973498290039345	4.833\\
0.973481136829015	4.842\\
0.973466127769976	4.851\\
0.973448974559645	4.86\\
0.973433965500606	4.869\\
0.973416812290275	4.878\\
0.973401803231236	4.887\\
0.973384650020905	4.896\\
0.973369640961866	4.905\\
0.973352487751536	4.914\\
0.973337478692497	4.923\\
0.973320325482166	4.932\\
0.973305316423127	4.941\\
0.973288163212796	4.95\\
0.973273154153757	4.959\\
0.973258145094718	4.968\\
0.973240991884387	4.977\\
0.973225982825348	4.986\\
0.973208829615018	4.995\\
0.97319596470727	5.004\\
0.973185243950813	5.013\\
0.973176667345648	5.022\\
0.973165946589191	5.031\\
0.973157369984026	5.04\\
0.973146649227569	5.049\\
0.973135928471113	5.058\\
0.973127351865948	5.067\\
0.973116631109491	5.076\\
0.973108054504326	5.085\\
0.973097333747869	5.094\\
0.973088757142704	5.103\\
0.973078036386247	5.112\\
0.973067315629791	5.121\\
0.973058739024625	5.13\\
0.973048018268169	5.139\\
0.973039441663004	5.148\\
0.973028720906547	5.157\\
0.973018000150091	5.166\\
0.973009423544925	5.175\\
0.972998702788469	5.184\\
0.972990126183303	5.193\\
0.972979405426847	5.202\\
0.97296868467039	5.211\\
0.972960108065225	5.22\\
0.972949387308768	5.229\\
0.972940810703603	5.238\\
0.972930089947147	5.247\\
0.972921513341981	5.256\\
0.972910792585525	5.265\\
0.972900071829068	5.274\\
0.972891495223903	5.283\\
0.972880774467446	5.292\\
0.972872197862281	5.301\\
0.972861477105824	5.31\\
0.972850756349368	5.319\\
0.972842179744203	5.328\\
0.972831458987746	5.337\\
0.972822882382581	5.346\\
0.972812161626124	5.355\\
0.972803585020959	5.364\\
0.972792864264503	5.373\\
0.972782143508046	5.382\\
0.972773566902881	5.391\\
0.972762846146424	5.4\\
0.972754269541259	5.409\\
0.972743548784802	5.418\\
0.972732828028346	5.427\\
0.97272425142318	5.436\\
0.972713530666724	5.445\\
0.972704954061559	5.454\\
0.972694233305102	5.463\\
0.972685656699937	5.472\\
0.97267493594348	5.481\\
0.972664215187024	5.49\\
0.972655638581858	5.499\\
0.972644917825402	5.508\\
0.972636341220236	5.517\\
0.97262562046378	5.526\\
0.972614899707323	5.535\\
0.972606323102158	5.544\\
0.972595602345701	5.553\\
0.972587025740536	5.562\\
0.97257630498408	5.571\\
0.972567728378914	5.58\\
0.972557007622458	5.589\\
0.972546286866001	5.598\\
0.972537710260836	5.607\\
0.972526989504379	5.616\\
0.972518412899214	5.625\\
0.972507692142758	5.634\\
0.972496971386301	5.643\\
0.972488394781136	5.652\\
0.972477674024679	5.661\\
0.972469097419514	5.67\\
0.972458376663057	5.679\\
0.972449800057892	5.688\\
0.972439079301435	5.697\\
0.972428358544979	5.706\\
0.972419781939814	5.715\\
0.972409061183357	5.724\\
0.972400484578192	5.733\\
0.972389763821735	5.742\\
0.972379043065279	5.751\\
0.972370466460113	5.76\\
0.972359745703657	5.769\\
0.972351169098492	5.778\\
0.972340448342035	5.787\\
0.97233187173687	5.796\\
0.972321150980413	5.805\\
0.972310430223957	5.814\\
0.9722997094675	5.823\\
0.972288988711043	5.832\\
0.972278267954587	5.841\\
0.97226754719813	5.85\\
0.972256826441674	5.859\\
0.972246105685217	5.868\\
0.972233240777469	5.877\\
0.972222520021013	5.886\\
0.972211799264556	5.895\\
0.972201078508099	5.904\\
0.972190357751643	5.913\\
0.972179636995186	5.922\\
0.97216891623873	5.931\\
0.972158195482273	5.94\\
0.972147474725817	5.949\\
0.97213675396936	5.958\\
0.972126033212904	5.967\\
0.972115312456447	5.976\\
0.97210459169999	5.985\\
0.972093870943534	5.994\\
0.972081006035786	6.003\\
0.972070285279329	6.012\\
0.972059564522873	6.021\\
0.972048843766416	6.03\\
0.97203812300996	6.039\\
0.972027402253503	6.048\\
0.972016681497046	6.057\\
0.97200596074059	6.066\\
0.971995239984133	6.075\\
0.971984519227677	6.084\\
0.97197379847122	6.093\\
0.971963077714764	6.102\\
0.971952356958307	6.111\\
0.971939492050559	6.12\\
0.971928771294103	6.129\\
0.971918050537646	6.138\\
0.971907329781189	6.147\\
0.971896609024733	6.156\\
0.971885888268276	6.165\\
0.97187516751182	6.174\\
0.971864446755363	6.183\\
0.971853725998907	6.192\\
0.97184300524245	6.201\\
0.971832284485993	6.21\\
0.971821563729537	6.219\\
0.97181084297308	6.228\\
0.971800122216624	6.237\\
0.971787257308876	6.246\\
0.971776536552419	6.255\\
0.971765815795963	6.264\\
0.971755095039506	6.273\\
0.971744374283049	6.282\\
0.971733653526593	6.291\\
0.971722932770136	6.3\\
0.97171221201368	6.309\\
0.971701491257223	6.318\\
0.971690770500766	6.327\\
0.97168004974431	6.336\\
0.971669328987853	6.345\\
0.971658608231397	6.354\\
0.971645743323649	6.363\\
0.971635022567192	6.372\\
0.971624301810736	6.381\\
0.971613581054279	6.39\\
0.971602860297823	6.399\\
0.971592139541366	6.408\\
0.971581418784909	6.417\\
0.971570698028453	6.426\\
0.971559977271996	6.435\\
0.97154925651554	6.444\\
0.971538535759083	6.453\\
0.971478499522926	6.462\\
0.971416319135478	6.471\\
0.97135413874803	6.48\\
0.971291958360582	6.489\\
0.971231922124425	6.498\\
0.971169741736977	6.507\\
0.971107561349529	6.516\\
0.971045380962081	6.525\\
0.970983200574633	6.534\\
0.970921020187184	6.543\\
0.970860983951028	6.552\\
0.97079880356358	6.561\\
0.970736623176131	6.57\\
0.970674442788683	6.579\\
0.970612262401235	6.588\\
0.970550082013787	6.597\\
0.970487901626339	6.606\\
0.970427865390182	6.615\\
0.970365685002734	6.624\\
0.970303504615286	6.633\\
0.970241324227837	6.642\\
0.970179143840389	6.651\\
0.970116963452941	6.66\\
0.970056927216784	6.669\\
0.969994746829336	6.678\\
0.969932566441888	6.687\\
0.96987038605444	6.696\\
0.969808205666992	6.705\\
0.969746025279544	6.714\\
0.969685989043387	6.723\\
0.969623808655939	6.732\\
0.969561628268491	6.741\\
0.969499447881042	6.75\\
0.969437267493594	6.759\\
0.969375087106146	6.768\\
0.969312906718698	6.777\\
0.969252870482541	6.786\\
0.969190690095093	6.795\\
0.969128509707645	6.804\\
0.969066329320197	6.813\\
0.969004148932749	6.822\\
0.9689419685453	6.831\\
0.968881932309144	6.84\\
0.968819751921696	6.849\\
0.968757571534247	6.858\\
0.968695391146799	6.867\\
0.968633210759351	6.876\\
0.968571030371903	6.885\\
0.968508849984455	6.894\\
0.968448813748298	6.903\\
0.96838663336085	6.912\\
0.968324452973402	6.921\\
0.968262272585954	6.93\\
0.968200092198506	6.939\\
0.968137911811057	6.948\\
0.968077875574901	6.957\\
0.968015695187452	6.966\\
0.967822721571234	6.975\\
0.967518252087867	6.984\\
0.967213782604501	6.993\\
0.966909313121134	7.002\\
0.966606987789058	7.011\\
0.966302518305692	7.02\\
0.965998048822325	7.029\\
0.965693579338958	7.038\\
0.965389109855591	7.047\\
0.965084640372225	7.056\\
0.964780170888858	7.065\\
0.964475701405491	7.074\\
0.964171231922124	7.083\\
0.963866762438758	7.092\\
0.963562292955391	7.101\\
0.963259967623315	7.11\\
0.962955498139949	7.119\\
0.962651028656582	7.128\\
0.962346559173215	7.137\\
0.962042089689849	7.146\\
0.961737620206482	7.155\\
0.961433150723115	7.164\\
0.961128681239748	7.173\\
0.960824211756382	7.182\\
0.960519742273015	7.191\\
0.960215272789648	7.2\\
0.959912947457573	7.209\\
0.959608477974206	7.218\\
0.959304008490839	7.227\\
0.958999539007472	7.236\\
0.958695069524106	7.245\\
0.958390600040739	7.254\\
0.958086130557372	7.263\\
0.957781661074005	7.272\\
0.957477191590639	7.281\\
0.957172722107272	7.29\\
0.956868252623905	7.299\\
0.95656592729183	7.308\\
0.956261457808463	7.317\\
0.955956988325096	7.326\\
0.955652518841729	7.335\\
0.955348049358363	7.344\\
0.955043579874996	7.353\\
0.954739110391629	7.362\\
0.954434640908262	7.371\\
0.954130171424896	7.38\\
0.953542673971075	7.389\\
0.952766491203619	7.398\\
0.951992452587455	7.407\\
0.951216269819999	7.416\\
0.950440087052542	7.425\\
0.949666048436378	7.434\\
0.948889865668922	7.443\\
0.948113682901466	7.452\\
0.947339644285301	7.461\\
0.946563461517845	7.47\\
0.945787278750389	7.479\\
0.945011095982933	7.488\\
0.944237057366768	7.497\\
0.943460874599312	7.506\\
0.942684691831856	7.515\\
0.941910653215691	7.524\\
0.941134470448235	7.533\\
0.940358287680779	7.542\\
0.939584249064614	7.551\\
0.938808066297158	7.56\\
0.938031883529702	7.569\\
0.937257844913537	7.578\\
0.936481662146081	7.587\\
0.935705479378625	7.596\\
0.93493144076246	7.605\\
0.934155257995004	7.614\\
0.933379075227548	7.623\\
0.932605036611383	7.632\\
0.931828853843927	7.641\\
0.931052671076471	7.65\\
0.930278632460306	7.659\\
0.92950244969285	7.668\\
0.928726266925394	7.677\\
0.927950084157938	7.686\\
0.927176045541773	7.695\\
0.926399862774317	7.704\\
0.925623680006861	7.713\\
0.924485135671173	7.722\\
0.923284410948036	7.731\\
0.922081542073609	7.74\\
0.920880817350472	7.749\\
0.919677948476044	7.758\\
0.918477223752908	7.767\\
0.91727435487848	7.776\\
0.916073630155344	7.785\\
0.914870761280916	7.794\\
0.91367003655778	7.803\\
0.912467167683352	7.812\\
0.911266442960215	7.821\\
0.910065718237079	7.83\\
0.908862849362651	7.839\\
0.907662124639515	7.848\\
0.906459255765087	7.857\\
0.90525853104195	7.866\\
0.904055662167523	7.875\\
0.902854937444386	7.884\\
0.901652068569958	7.893\\
0.900451343846822	7.902\\
0.899250619123685	7.911\\
0.898047750249258	7.92\\
0.896847025526121	7.929\\
0.895644156651693	7.938\\
0.894443431928557	7.947\\
0.893240563054129	7.956\\
0.892039838330993	7.965\\
0.890836969456565	7.974\\
0.889509739807241	7.983\\
0.888032419567525	7.992\\
0.886555099327809	8.001\\
0.885077779088092	8.01\\
0.883600458848376	8.019\\
0.882120994457369	8.028\\
0.880643674217653	8.037\\
0.879166353977937	8.046\\
0.877689033738221	8.055\\
0.876211713498504	8.064\\
0.874732249107497	8.073\\
0.873254928867781	8.082\\
0.871777608628065	8.091\\
0.870300288388349	8.1\\
0.868822968148633	8.109\\
0.867345647908917	8.118\\
0.865866183517909	8.127\\
0.864388863278193	8.136\\
0.862911543038477	8.145\\
0.861434222798761	8.154\\
0.859956902559045	8.163\\
0.858477438168037	8.172\\
0.857000117928321	8.181\\
0.855522797688605	8.19\\
0.853762449478435	8.199\\
0.851982803906644	8.208\\
0.850201014183561	8.217\\
0.848421368611769	8.226\\
0.846641723039978	8.235\\
0.844862077468186	8.244\\
0.843082431896395	8.253\\
0.841302786324603	8.262\\
0.839523140752812	8.271\\
0.837741351029729	8.28\\
0.835961705457937	8.289\\
0.834182059886146	8.298\\
0.832402414314354	8.307\\
0.830622768742562	8.316\\
0.828843123170771	8.325\\
0.827061333447688	8.334\\
0.825281687875897	8.343\\
0.823502042304105	8.352\\
0.821675225403904	8.361\\
0.819513920902259	8.37\\
0.817350472249322	8.379\\
0.815189167747676	8.388\\
0.813025719094739	8.397\\
0.810864414593094	8.406\\
0.808700965940157	8.415\\
0.806539661438511	8.424\\
0.804376212785574	8.433\\
0.802212764132637	8.442\\
0.800051459630992	8.451\\
0.797888010978055	8.46\\
0.795726706476409	8.469\\
0.793565401974763	8.478\\
0.791401953321826	8.487\\
0.78920634239952	8.496\\
0.786560459706037	8.505\\
0.783914577012554	8.514\\
0.781268694319071	8.523\\
0.778622811625588	8.532\\
0.775976928932105	8.541\\
0.773331046238623	8.55\\
0.77068516354514	8.559\\
0.768039280851657	8.568\\
0.765393398158174	8.577\\
0.762747515464691	8.586\\
0.7601037769225	8.595\\
0.757438596867395	8.604\\
0.754173054450722	8.613\\
0.750905367882758	8.622\\
0.747637681314793	8.631\\
0.744369994746829	8.64\\
0.741104452330156	8.649\\
0.737836765762192	8.658\\
0.734569079194228	8.667\\
0.731301392626264	8.676\\
0.728035850209591	8.685\\
0.724448685099221	8.694\\
0.720370509343139	8.703\\
0.716294477738349	8.712\\
0.712218446133559	8.721\\
0.708142414528769	8.73\\
0.704064238772688	8.739\\
0.699988207167898	8.748\\
0.695912175563108	8.757\\
0.691098555914105	8.766\\
0.685950448663658	8.775\\
0.680804485564501	8.784\\
0.675656378314054	8.793\\
0.670510415214898	8.802\\
0.66536230796445	8.811\\
0.659455171156877	8.82\\
0.652855473482209	8.829\\
0.646255775807541	8.838\\
0.639656078132873	8.847\\
0.633058524609496	8.856\\
0.625249525606527	8.865\\
0.616606451751236	8.874\\
0.607963377895944	8.883\\
0.599320304040653	8.892\\
0.588723908358974	8.901\\
0.577053292880346	8.91\\
0.565382677401717	8.919\\
0.551355639653934	8.928\\
0.534867116223721	8.937\\
0.518376448642216	8.946\\
0.493669393312392	8.955\\
0.468647147742745	8.964\\
0.4269348285215	8.973\\
0.367265242235492	8.982\\
0.273394298701716	8.991\\
0	9\\
};
\addlegendentry{Upwind}

\end{axis}
\end{tikzpicture}%}
	\subfloat[Velocity profiles at $x/H = 6$]{% This file was created by matlab2tikz.
%
\definecolor{mycolor4}{rgb}{0.00000,0.44700,0.74100}%
\definecolor{mycolor3}{rgb}{0.85000,0.32500,0.09800}%
\definecolor{mycolor2}{rgb}{0.92900,0.69400,0.12500}%
\definecolor{mycolor1}{rgb}{0.49400,0.18400,0.55600}%
%
\begin{tikzpicture}

\begin{axis}[%
width=0.958\bfshalfwidth,
height=0.75\bfsheight,
at={(0\bfshalfwidth,0\bfsheight)},
scale only axis,
xmin=-0.2,
xmax=1,
xlabel style={font=\color{white!15!black}},
xlabel={$u/U_\text{ref}$},
ymin=0,
ymax=5,
ylabel style={font=\color{white!15!black}, rotate=-90},
ylabel={$\dfrac{y}{H}$},
axis background/.style={fill=white},
legend style={at={(0.03,0.97)}, anchor=north west, legend cell align=left, align=left, draw=white!15!black, font=\scriptsize}
]
\addplot [color=mycolor1]
  table[row sep=crcr]{%
0	0\\
-0.001509253751	0.0001486152614\\
-0.003016794799	0.0002998616837\\
-0.004530657083	0.0004540858499\\
-0.006052900571	0.0006116450531\\
-0.007585445419	0.0007729082718\\
-0.00913005136	0.0009382571443\\
-0.01068830676	0.001108087017\\
-0.01226159837	0.001282808022\\
-0.01385108102	0.001462846179\\
-0.01545764599	0.001648644567\\
-0.01708186418	0.001840664423\\
-0.01872392185	0.002039386658\\
-0.02038354613	0.002245313255\\
-0.02205987647	0.0024589682\\
-0.0237513613	0.00268089911\\
-0.02545555308	0.002911679214\\
-0.02716889605	0.003151908284\\
-0.02888647839	0.003402214963\\
-0.03060168773	0.003663257696\\
-0.03230588511	0.003935726825\\
-0.03398806974	0.004220347852\\
-0.0356346108	0.004517879803\\
-0.03722923622	0.004829122685\\
-0.03875335306	0.005154913291\\
-0.0401869677	0.005496133585\\
-0.04151019454	0.005853707902\\
-0.04270537198	0.006228608545\\
-0.04375920445	0.006621857174\\
-0.04466451332	0.007034527604\\
-0.04542086273	0.007467746735\\
-0.04603401944	0.007922700606\\
-0.04651438072	0.008400633931\\
-0.04687502608	0.008902854286\\
-0.04712991044	0.009430736303\\
-0.04729250818	0.00998572167\\
-0.04737500846	0.01056932472\\
-0.04738797992	0.01118313242\\
-0.04734027758	0.01182881277\\
-0.04723918065	0.01250811014\\
-0.04709056765	0.01322285552\\
-0.04689915106	0.01397496555\\
-0.04666863754	0.01476644445\\
-0.04640194774	0.01559939049\\
-0.04610133171	0.01647599414\\
-0.04576849192	0.01739854366\\
-0.0454046838	0.01836942695\\
-0.04501081258	0.01939112879\\
-0.04458745196	0.02046624199\\
-0.04413494468	0.02159745246\\
-0.04365339503	0.02278756164\\
-0.04314273596	0.02403946407\\
-0.04260271788	0.02535616048\\
-0.04203298315	0.02674075589\\
-0.04143300653	0.02819645032\\
-0.04080218822	0.02972654253\\
-0.0401398018	0.0313344188\\
-0.03944502771	0.03302355856\\
-0.03871696815	0.03479751945\\
-0.03795465454	0.0366599299\\
-0.03715700284	0.03861448541\\
-0.03632289171	0.04066493362\\
-0.03545111045	0.04281506687\\
-0.03454039246	0.04506869614\\
-0.03358940408	0.0474296622\\
-0.0325967446	0.04990179092\\
-0.03156095371	0.05248888955\\
-0.03048052453	0.05519472063\\
-0.0293538738	0.05802299082\\
-0.02817939408	0.06097731367\\
-0.0269554127	0.06406119466\\
-0.02568020672	0.06727801263\\
-0.02435201965	0.07063096017\\
-0.02296905406	0.07412306219\\
-0.0215294715	0.07775711268\\
-0.02003140934	0.08153568208\\
-0.01847297139	0.08546103537\\
-0.0168522466	0.08953517675\\
-0.01516730618	0.09375976026\\
-0.01341620367	0.098136127\\
-0.01159699541	0.1026652232\\
-0.009707730263	0.1073476374\\
-0.007746464573	0.1121835709\\
-0.005711266771	0.1171728224\\
-0.00360021391	0.1223147884\\
-0.001411403879	0.1276084781\\
0.0008570466307	0.1330525279\\
0.003206999274	0.1386451721\\
0.005640295334	0.1443843395\\
0.00815875642	0.1502676159\\
0.01076419093	0.1562923193\\
0.01345839631	0.162455529\\
0.01624317653	0.1687541306\\
0.01912034489	0.1751849204\\
0.02209175006	0.1817445606\\
0.02515928075	0.188429758\\
0.02832490765	0.1952372491\\
0.0315906927	0.2021639198\\
0.03495882824	0.2092068344\\
0.03843166679	0.2163633555\\
0.04201176018	0.2236311883\\
0.04570187628	0.2310084403\\
0.04950505495	0.2384937108\\
0.05342467874	0.2460861355\\
0.05746445432	0.2537854314\\
0.06162854657	0.2615920603\\
0.06592155248	0.2695070505\\
0.07034865022	0.2775323093\\
0.07491555065	0.2856704593\\
0.07962863892	0.2939249277\\
0.08449500054	0.3023000062\\
0.08952248096	0.3108008802\\
0.09471975267	0.3194335103\\
0.1000963748	0.3282048404\\
0.10566286	0.337122649\\
0.1114307567	0.3461955786\\
0.1174126491	0.3554331958\\
0.1236223131	0.3648459315\\
0.1300747097	0.3744450808\\
0.1367860883	0.3842427731\\
0.1437740326	0.3942520916\\
0.1510575563	0.404486835\\
0.1586571485	0.4149617255\\
0.1665948182	0.4256922901\\
0.1748941988	0.4366948307\\
0.1835805625	0.4479865432\\
0.1926808953	0.4595854282\\
0.2022233903	0.4715102017\\
0.2121638954	0.4837805629\\
0.2223348469	0.4960536659\\
0.2325842977	0.5081439018\\
0.2428663224	0.5200572014\\
0.2531748712	0.5317991376\\
0.2635040581	0.5433754325\\
0.2738480568	0.5547917485\\
0.2842011452	0.5660534501\\
0.2945576608	0.5771659613\\
0.3049119413	0.5881346464\\
0.3152583241	0.5989645123\\
0.3255913258	0.6096607447\\
0.3359053135	0.620228231\\
0.3461947441	0.6306717396\\
0.3564543426	0.6409959197\\
0.3667142689	0.6512053609\\
0.3770779669	0.6613044739\\
0.3876620829	0.6712974906\\
0.3985123038	0.681188643\\
0.4096008539	0.6909818649\\
0.4208640158	0.70068115\\
0.4322321713	0.7102902532\\
0.4436478019	0.71981287\\
0.4550714791	0.7292524576\\
0.4664795697	0.7386125326\\
0.4778588712	0.7478964329\\
0.4891995192	0.7571073174\\
0.5004912019	0.7662483454\\
0.5117201209	0.7753225565\\
0.5228685737	0.7843328118\\
0.5339156389	0.793282032\\
0.5448380709	0.8021728396\\
0.5556115508	0.8110080361\\
0.5662115812	0.8197900653\\
0.5766142607	0.8285215497\\
0.5867964625	0.837204814\\
0.5967365503	0.8458423018\\
0.6064140797	0.8544362187\\
0.6158105135	0.8629888892\\
0.6249090433	0.8715023994\\
0.6336951852	0.8799789548\\
0.6421567202	0.8884204626\\
0.6502844691	0.8968290687\\
0.6580714583	0.9052066207\\
0.6655137539	0.9135550261\\
0.6726103425	0.9218762517\\
0.6793627739	0.9301720262\\
0.6857753396	0.9384441972\\
0.6918550134	0.9466944337\\
0.6976104975	0.9549245238\\
0.703053236	0.9631360769\\
0.7081959844	0.9713308215\\
0.7130528688	0.9795103073\\
0.7176394463	0.9876762033\\
0.7219718099	0.9958299994\\
0.7260665894	1.003973365\\
0.729940474	1.01210773\\
0.7336102128	1.020234585\\
0.7370919585	1.028355479\\
0.7404016852	1.036471963\\
0.7435539961	1.044585347\\
0.746563375	1.052697182\\
0.7494427562	1.060809016\\
0.7522044182	1.068922043\\
0.7548597455	1.07703793\\
0.7574192882	1.085160136\\
0.7598930001	1.093292713\\
0.7622891068	1.101436853\\
0.7646151185	1.109594226\\
0.766877532	1.117766142\\
0.7690823078	1.125954151\\
0.7712346911	1.134159684\\
0.7733396292	1.142384529\\
0.7754009366	1.150630116\\
0.7774227262	1.158897996\\
0.7794082165	1.167189717\\
0.7813604474	1.175507188\\
0.7832819819	1.183851838\\
0.7851756215	1.192225575\\
0.7870432138	1.200630069\\
0.7888869643	1.209067106\\
0.7907087803	1.217538595\\
0.7925100327	1.226046205\\
0.7942925692	1.234592199\\
0.7960576415	1.243178248\\
0.7978066802	1.251806617\\
0.7995409966	1.260479212\\
0.8012614846	1.269198298\\
0.8029695749	1.277966142\\
0.8046659827	1.286784768\\
0.8063520789	1.2956568\\
0.8080285192	1.304584503\\
0.8096962571	1.31357038\\
0.8113561869	1.322617173\\
0.8130093217	1.331727386\\
0.8146563172	1.340903759\\
0.8162979484	1.350149393\\
0.8179351687	1.35946703\\
0.8195685148	1.368859768\\
0.8211989999	1.378330946\\
0.8228270411	1.387883782\\
0.8244535923	1.397521734\\
0.8260792494	1.407248497\\
0.8277047276	1.417067528\\
0.8293305635	1.426982999\\
0.8309575319	1.436998844\\
0.8325862885	1.447119355\\
0.8342175484	1.457348704\\
0.8358517289	1.467691779\\
0.8374898434	1.47815311\\
0.8391322494	1.488737941\\
0.8407797813	1.49945128\\
0.8424330354	1.51029861\\
0.8440927863	1.521285772\\
0.845759809	1.532418728\\
0.8474345803	1.543703556\\
0.8491178751	1.555146933\\
0.8508105278	1.566755652\\
0.8525131345	1.578536868\\
0.8542265892	1.590498209\\
0.8559516072	1.602647424\\
0.8576888442	1.614992738\\
0.8594392538	1.627542853\\
0.8612034917	1.640306711\\
0.8629825711	1.653294086\\
0.8647772074	1.666514635\\
0.8665886521	1.679978967\\
0.8684177399	1.693697929\\
0.8702672124	1.707683206\\
0.872153759	1.721997499\\
0.8741109967	1.737120032\\
0.8761597872	1.753092289\\
0.8783028126	1.769957781\\
0.8805415034	1.787761569\\
0.8828784823	1.806550384\\
0.8853167892	1.826372862\\
0.8878597617	1.847279191\\
0.8905110359	1.869321108\\
0.8932746053	1.892552137\\
0.8961548805	1.917027235\\
0.8991571665	1.942802906\\
0.902287662	1.969937086\\
0.9055518508	1.998489022\\
0.9089553356	2.028518915\\
0.9125006199	2.060088396\\
0.9161850214	2.093259335\\
0.9199967384	2.128094673\\
0.9239119291	2.164657593\\
0.9278899431	2.203011751\\
0.9318727255	2.243220091\\
0.9357857108	2.285345554\\
0.9395421743	2.329449892\\
0.943051517	2.375594378\\
0.9462277889	2.4238379\\
0.9489985704	2.474237442\\
0.9513119459	2.526847839\\
0.9531416297	2.581720829\\
0.9544817805	2.638904333\\
0.9553439617	2.698442221\\
0.9557833076	2.760373831\\
0.955919385	2.824733257\\
0.9559096694	2.891548634\\
0.9558912516	2.960841417\\
0.9559394717	3.032626629\\
0.9560654759	3.106910706\\
0.9562397599	3.183692455\\
0.9564246535	3.262961864\\
0.9565964341	3.344699144\\
0.9567505121	3.428875208\\
0.9568932056	3.515450716\\
0.9570314288	3.604375362\\
0.957167387	3.69558835\\
0.9572995305	3.789017677\\
0.9574261904	3.884580135\\
0.9575468302	3.982181787\\
0.9576621056	4.081717014\\
0.9577719569	4.183070183\\
0.9578765035	4.286113739\\
0.9579754472	4.390712261\\
0.9580690265	4.49671793\\
0.9581574202	4.603977203\\
0.9582406878	4.712325573\\
0.9583188891	4.821593761\\
0.9583922625	4.931604385\\
0.9584608674	5.042176247\\
0.9585249424	5.153123856\\
0.9585846663	5.264257908\\
0.9586401582	5.375389099\\
0.9586917162	5.486326694\\
0.9587395787	5.596880913\\
0.9587841034	5.706864834\\
0.9588258266	5.816094875\\
0.9588645101	5.924391747\\
0.9588991404	6.031581879\\
0.9589274526	6.137498856\\
0.9589477181	6.241983414\\
0.958964169	6.34488678\\
0.9589931369	6.446067333\\
0.9590632319	6.545394897\\
0.959192276	6.642749786\\
0.9593337178	6.738023758\\
0.9593110681	6.831119061\\
0.9587980509	6.921949863\\
0.9573966861	7.010442257\\
0.9547620416	7.096533775\\
0.9506445527	7.180172443\\
0.9449290037	7.26131916\\
0.9378022552	7.339943409\\
0.9297752976	7.416027069\\
0.921440661	7.489560604\\
0.9132467508	7.560543537\\
0.9054341316	7.628985882\\
0.898065269	7.694903374\\
0.8910431862	7.758320808\\
0.8841731548	7.819269657\\
0.8773137331	7.87778616\\
0.8704314828	7.933914661\\
0.8635334373	7.987701416\\
0.8566265106	8.039198875\\
0.8497165442	8.088461876\\
0.842808187	8.135549545\\
0.8359050751	8.180522919\\
0.829009831	8.223443985\\
0.8221244216	8.264378548\\
0.8152501583	8.303390503\\
0.8083879352	8.340547562\\
0.8015381098	8.375915527\\
0.794700861	8.409560204\\
0.7878760099	8.441549301\\
0.7810636759	8.471946716\\
0.7742635012	8.500817299\\
0.7674753666	8.528224945\\
0.7606990933	8.554232597\\
0.7539348006	8.578900337\\
0.74718225	8.602288246\\
0.7404416203	8.624453545\\
0.7337132692	8.645453453\\
0.7269972563	8.665341377\\
0.7202940583	8.684169769\\
0.7136041522	8.701991081\\
0.7069279552	8.718853951\\
0.7002658844	8.734804153\\
0.6936184168	8.74988842\\
0.6869861484	8.764149666\\
0.6803691983	8.777630806\\
0.6737679839	8.790369987\\
0.6671825647	8.802407265\\
0.6606130004	8.813777924\\
0.6540591121	8.82451725\\
0.6475206017	8.834658623\\
0.640996635	8.844234467\\
0.6344867945	8.853273392\\
0.6279897094	8.861805916\\
0.6215043068	8.869857788\\
0.6150289178	8.877456665\\
0.6085618138	8.884626389\\
0.6021010876	8.891389847\\
0.595644474	8.897770882\\
0.589189291	8.903788567\\
0.5827330351	8.909464836\\
0.5762726665	8.91481781\\
0.5698048472	8.919865608\\
0.5633261204	8.924625397\\
0.5568326116	8.929113388\\
0.5503202677	8.933343887\\
0.543784678	8.937333107\\
0.5372210741	8.941092491\\
0.5306245685	8.944637299\\
0.5239896178	8.94797802\\
0.5173105001	8.951126099\\
0.5105811357	8.954093933\\
0.5037949085	8.956890106\\
0.4969446957	8.959526062\\
0.4900230467	8.96200943\\
0.4830217957	8.9643507\\
0.4759323001	8.966555595\\
0.4687450826	8.968633652\\
0.4614500701	8.970592499\\
0.4540362954	8.972436905\\
0.4464918673	8.974175453\\
0.4388040304	8.975813866\\
0.4309586287	8.977356911\\
0.4229405224	8.97881031\\
0.4147331119	8.98018074\\
0.4063181877	8.981471062\\
0.3976759017	8.982686996\\
0.388784796	8.983832359\\
0.3796210587	8.984910965\\
0.3701592684	8.985927582\\
0.3603715599	8.986885071\\
0.3502283692	8.987787247\\
0.3396980762	8.988636971\\
0.3287480175	8.989438057\\
0.3173452914	8.99019146\\
0.3054583669	8.990901947\\
0.2930596769	8.991571426\\
0.2801297903	8.992201805\\
0.2666623592	8.992795944\\
0.2526715696	8.993355751\\
0.2381997705	8.993882179\\
0.2233248353	8.994379044\\
0.2081638724	8.994846344\\
0.1928699315	8.995286942\\
0.1776200086	8.99570179\\
0.1625955999	8.996092796\\
0.1479613781	8.996460915\\
0.1338484138	8.996807098\\
0.1203468964	8.997134209\\
0.1075073928	8.997442245\\
0.09534819424	8.997732162\\
0.08386452496	8.998004913\\
0.07303687185	8.998262405\\
0.0628374964	8.998504639\\
0.05323480442	8.998732567\\
0.04419604316	8.998948097\\
0.03568885103	8.999150276\\
0.02768205665	8.999341011\\
0.02014602721	8.999521255\\
0.01305279229	8.999690056\\
0.006376028992	8.999849319\\
0	9\\
};
\addlegendentry{Ref data}

\addplot [color=mycolor2, draw=none, mark=x, mark options={solid, mycolor2}, only marks]
  table[row sep=crcr]{%
0.147	0.1\\
0.169	0.15\\
0.188	0.2\\
0.248	0.3\\
0.306	0.4\\
0.369	0.5\\
0.435	0.6\\
0.501	0.7\\
0.569	0.8\\
0.632	0.9\\
0.686	1\\
0.73	1.1\\
0.747	1.15\\
0.763	1.2\\
0.794	1.3\\
0.815	1.4\\
0.834	1.5\\
0.869	1.7\\
0.915	2\\
0.949	2.4\\
0.954	2.8\\
0.956	3.2\\
0.959	3.6\\
0.961	4\\
0.963	5\\
0.965	6\\
0.969	7.5\\
0.87	8.2\\
};
\addlegendentry{Experiment}

\addplot [color=mycolor3]
  table[row sep=crcr]{%
0	0\\
-0.0346061448982172	0.009\\
-0.0325964198692329	0.018\\
-0.0274857435532573	0.027\\
-0.0226039418727003	0.036\\
-0.0180382143492395	0.045\\
-0.0139491161110171	0.054\\
-0.0098828624759178	0.063\\
-0.0060655421521502	0.072\\
-0.0022482218283826	0.081\\
0.00150301630196022	0.09\\
0.0052232706356843	0.099\\
0.0089435335414921	0.108\\
0.0126838979836316	0.117\\
0.0164270483529813	0.126\\
0.0201702201525402	0.135\\
0.0239819043312596	0.144\\
0.027857771986859	0.153\\
0.0317336396424584	0.162\\
0.0356097216001509	0.171\\
0.0396049555216006	0.18\\
0.0437289850009965	0.189\\
0.0478532287824855	0.198\\
0.0519774725639745	0.207\\
0.0561017163454635	0.216\\
0.0604679071900495	0.225\\
0.0649714656763053	0.234\\
0.0694752384646541	0.243\\
0.0739787969509098	0.252\\
0.0784825697392586	0.261\\
0.0829861282255144	0.27\\
0.0879279344921362	0.279\\
0.092961676356693	0.288\\
0.0979952039191567	0.297\\
0.103028945783713	0.306\\
0.10806268764827	0.315\\
0.113096215210734	0.324\\
0.118129957075291	0.333\\
0.123751100977003	0.342\\
0.12947318116456	0.351\\
0.135195261352118	0.36\\
0.140917341539675	0.369\\
0.146639207425139	0.378\\
0.152361287612696	0.387\\
0.158083367800253	0.396\\
0.16380544798781	0.405\\
0.169769903842651	0.414\\
0.175990665000825	0.423\\
0.182211640461092	0.432\\
0.18843261592136	0.441\\
0.194653377079534	0.45\\
0.200874352539801	0.459\\
0.207095113697976	0.468\\
0.213316089158243	0.477\\
0.219537493222696	0.486\\
0.225758682985057	0.495\\
0.232372045577769	0.504\\
0.239476159963654	0.513\\
0.24658241737047	0.522\\
0.253686531756356	0.531\\
0.260792789163172	0.54\\
0.267896903549057	0.549\\
0.275003160955873	0.558\\
0.282107275341758	0.567\\
0.289213532748574	0.576\\
0.29631764713446	0.585\\
0.303983233004237	0.594\\
0.312476024953336	0.603\\
0.320968816902435	0.612\\
0.329461608851534	0.621\\
0.337954400800633	0.63\\
0.346447192749732	0.639\\
0.354939984698831	0.648\\
0.36343491966886	0.657\\
0.371959856931923	0.666\\
0.381399864132473	0.675\\
0.390839871333023	0.684\\
0.400279878533574	0.693\\
0.409719885734124	0.702\\
0.419157749913743	0.711\\
0.428597757114294	0.72\\
0.438074195670669	0.729\\
0.448105676648144	0.738\\
0.458137157625619	0.747\\
0.468168638603093	0.756\\
0.478197976559637	0.765\\
0.488229457537112	0.774\\
0.498290940807619	0.783\\
0.508436001894431	0.792\\
0.518581062981242	0.801\\
0.528728267088985	0.81\\
0.538873328175796	0.819\\
0.548909095195133	0.828\\
0.558784135644653	0.837\\
0.568657033073242	0.846\\
0.578532073522762	0.855\\
0.588218528130364	0.864\\
0.597587815640195	0.873\\
0.606954960129096	0.882\\
0.616315675555203	0.891\\
0.625072059078801	0.9\\
0.633828442602399	0.909\\
0.642454101849213	0.918\\
0.650584723260992	0.927\\
0.658715344672771	0.936\\
0.666333784082069	0.945\\
0.673877217758786	0.954\\
0.680934185684192	0.963\\
0.687817568914196	0.972\\
0.69438807108829	0.981\\
0.700592116683204	0.99\\
0.706798305299048	0.999\\
0.712481596807756	1.008\\
0.717648420272121	1.017\\
0.722521649868954	1.026\\
0.727219151749455	1.035\\
0.73151590871588	1.044\\
0.735789092452066	1.053\\
0.739650816169522	1.062\\
0.743478251552083	1.071\\
0.747129959218312	1.08\\
0.750518075310042	1.089\\
0.75390404838084	1.098\\
0.757191442488819	1.107\\
0.760168098561819	1.116\\
0.763142611613888	1.125\\
0.766119267686887	1.134\\
0.768986629692412	1.143\\
0.771601115228092	1.152\\
0.774215600763773	1.161\\
0.776830086299453	1.17\\
0.779444571835133	1.179\\
0.781949763303338	1.188\\
0.784262082887764	1.197\\
0.78657654549312	1.206\\
0.788888865077545	1.215\\
0.791201184661971	1.224\\
0.793513504246396	1.233\\
0.795776534349411	1.242\\
0.797840263505854	1.251\\
0.799906135683227	1.26\\
0.80196986483967	1.269\\
0.804033593996113	1.278\\
0.806097323152555	1.287\\
0.808163195329929	1.296\\
0.810226924486371	1.305\\
0.812189931659062	1.314\\
0.814052216848002	1.323\\
0.815912359016011	1.332\\
0.81777464420495	1.341\\
0.819634786372958	1.35\\
0.821494928540967	1.359\\
0.823357213729907	1.368\\
0.825217355897915	1.377\\
0.827079641086854	1.386\\
0.828939783254863	1.395\\
0.830654199999571	1.404\\
0.83234504351404	1.413\\
0.834033744007578	1.422\\
0.835724587522046	1.431\\
0.837415431036515	1.44\\
0.839104131530053	1.449\\
0.840794975044521	1.458\\
0.84248581855899	1.467\\
0.844174519052528	1.476\\
0.845865362566996	1.485\\
0.847556206081465	1.494\\
0.849247049595933	1.503\\
0.850805025812687	1.512\\
0.852332999736408	1.521\\
0.853863116681061	1.53\\
0.855391090604782	1.539\\
0.856921207549434	1.548\\
0.858451324494086	1.557\\
0.859979298417808	1.566\\
0.86150941536246	1.575\\
0.863037389286181	1.584\\
0.864567506230833	1.593\\
0.866095480154555	1.602\\
0.867625597099207	1.611\\
0.869155714043859	1.62\\
0.87068368796758	1.629\\
0.872213804912233	1.638\\
0.873596053412654	1.647\\
0.874933298473526	1.656\\
0.87627268655533	1.665\\
0.877609931616202	1.674\\
0.878947176677075	1.683\\
0.880286564758878	1.692\\
0.881623809819751	1.701\\
0.882961054880623	1.71\\
0.884300442962426	1.719\\
0.885637688023299	1.728\\
0.886974933084172	1.737\\
0.888314321165975	1.746\\
0.889651566226847	1.755\\
0.890990954308651	1.764\\
0.892328199369523	1.773\\
0.893665444430396	1.782\\
0.895004832512199	1.791\\
0.896342077573072	1.8\\
0.897679322633944	1.809\\
0.89875726216218	1.818\\
0.89982234356483	1.827\\
0.900885281946549	1.836\\
0.901950363349199	1.845\\
0.903013301730918	1.854\\
0.904078383133568	1.863\\
0.905141321515287	1.872\\
0.906206402917937	1.881\\
0.907271484320587	1.89\\
0.908334422702307	1.899\\
0.909399504104957	1.908\\
0.910462442486676	1.917\\
0.911527523889326	1.926\\
0.912590462271045	1.935\\
0.913655543673695	1.944\\
0.914718482055414	1.953\\
0.915783563458064	1.962\\
0.916848644860714	1.971\\
0.917911583242433	1.98\\
0.918976664645084	1.989\\
0.920039603026803	1.998\\
0.921104684429453	2.007\\
0.922167622811172	2.016\\
0.923071977644006	2.025\\
0.923768459446543	2.034\\
0.924464941249081	2.043\\
0.925159280030688	2.052\\
0.925855761833226	2.061\\
0.926552243635763	2.07\\
0.92724658241737	2.079\\
0.927943064219908	2.088\\
0.928639546022446	2.097\\
0.929333884804053	2.106\\
0.930030366606591	2.115\\
0.930726848409128	2.124\\
0.931421187190735	2.133\\
0.932117668993273	2.142\\
0.932814150795811	2.151\\
0.933508489577418	2.16\\
0.934204971379956	2.169\\
0.934901453182493	2.178\\
0.9355957919641	2.187\\
0.936292273766638	2.196\\
0.936988755569176	2.205\\
0.937683094350783	2.214\\
0.93837957615332	2.223\\
0.939076057955858	2.232\\
0.939770396737465	2.241\\
0.940466878540003	2.25\\
0.94116121732161	2.259\\
0.941857699124147	2.268\\
0.942554180926685	2.277\\
0.943235661582707	2.286\\
0.943561400764201	2.295\\
0.943887139945696	2.304\\
0.94421287912719	2.313\\
0.944538618308685	2.322\\
0.94486435749018	2.331\\
0.945190096671674	2.34\\
0.945515835853169	2.349\\
0.945841575034663	2.358\\
0.946167314216158	2.367\\
0.946493053397653	2.376\\
0.946818792579147	2.385\\
0.947144531760642	2.394\\
0.947470270942136	2.403\\
0.947796010123631	2.412\\
0.948121749305125	2.421\\
0.94844748848662	2.43\\
0.948773227668115	2.439\\
0.949098966849609	2.448\\
0.949424706031104	2.457\\
0.949750445212598	2.466\\
0.950076184394093	2.475\\
0.950401923575588	2.484\\
0.950727662757082	2.493\\
0.951053401938577	2.502\\
0.951379141120071	2.511\\
0.951704880301566	2.52\\
0.952030619483061	2.529\\
0.952356358664555	2.538\\
0.95268209784605	2.547\\
0.953007837027544	2.556\\
0.953333576209039	2.565\\
0.953659315390534	2.574\\
0.953985054572028	2.583\\
0.954310793753523	2.592\\
0.954636532935017	2.601\\
0.954962272116512	2.61\\
0.95522372067008	2.619\\
0.955318013591039	2.628\\
0.955414449532929	2.637\\
0.955510885474818	2.646\\
0.955605178395777	2.655\\
0.955701614337667	2.664\\
0.955798050279557	2.673\\
0.955892343200516	2.682\\
0.955988779142406	2.691\\
0.956085215084296	2.7\\
0.956179508005255	2.709\\
0.956275943947144	2.718\\
0.956372379889034	2.727\\
0.956466672809993	2.736\\
0.956563108751883	2.745\\
0.956659544693773	2.754\\
0.956753837614732	2.763\\
0.956850273556622	2.772\\
0.956946709498512	2.781\\
0.957041002419471	2.79\\
0.95713743836136	2.799\\
0.957231731282319	2.808\\
0.957328167224209	2.817\\
0.957424603166099	2.826\\
0.957521039107989	2.835\\
0.957615332028948	2.844\\
0.957711767970838	2.853\\
0.957806060891797	2.862\\
0.957902496833687	2.871\\
0.957998932775576	2.88\\
0.958093225696535	2.889\\
0.958189661638425	2.898\\
0.958286097580315	2.907\\
0.958380390501274	2.916\\
0.958476826443164	2.925\\
0.958573262385054	2.934\\
0.958667555306013	2.943\\
0.958763991247903	2.952\\
0.958860427189792	2.961\\
0.958954720110751	2.97\\
0.959051156052641	2.979\\
0.959147591994531	2.988\\
0.95924188491549	2.997\\
0.95933832085738	3.006\\
0.95943475679927	3.015\\
0.959529049720229	3.024\\
0.959599769410948	3.033\\
0.959619056599326	3.042\\
0.959640486808635	3.051\\
0.959659773997013	3.06\\
0.959681204206322	3.069\\
0.959700491394699	3.078\\
0.959721921604008	3.087\\
0.959741208792386	3.096\\
0.959762639001695	3.105\\
0.959781926190073	3.114\\
0.959803356399382	3.123\\
0.95982264358776	3.132\\
0.959844073797069	3.141\\
0.959863360985447	3.15\\
0.959884791194756	3.159\\
0.959904078383134	3.168\\
0.959925508592443	3.177\\
0.95994479578082	3.186\\
0.959966225990129	3.195\\
0.959985513178507	3.204\\
0.960006943387816	3.213\\
0.960026230576194	3.222\\
0.960047660785503	3.231\\
0.960066947973881	3.24\\
0.96008837818319	3.249\\
0.960107665371568	3.258\\
0.960129095580876	3.267\\
0.960148382769255	3.276\\
0.960169812978563	3.285\\
0.960189100166941	3.294\\
0.96021053037625	3.303\\
0.960229817564628	3.312\\
0.960251247773937	3.321\\
0.960270534962315	3.33\\
0.960291965171624	3.339\\
0.960311252360002	3.348\\
0.960332682569311	3.357\\
0.960351969757689	3.366\\
0.960373399966997	3.375\\
0.960392687155376	3.384\\
0.960414117364684	3.393\\
0.960433404553062	3.402\\
0.960454834762371	3.411\\
0.960474121950749	3.42\\
0.960495552160058	3.429\\
0.960514839348436	3.438\\
0.960536269557745	3.447\\
0.960555556746123	3.456\\
0.960576986955432	3.465\\
0.96059627414381	3.474\\
0.960617704353118	3.483\\
0.960639134562427	3.492\\
0.960658421750805	3.501\\
0.960679851960114	3.51\\
0.960699139148492	3.519\\
0.960720569357801	3.528\\
0.960739856546179	3.537\\
0.960761286755488	3.546\\
0.960769858839211	3.555\\
0.960780573943866	3.564\\
0.960789146027589	3.573\\
0.960799861132244	3.582\\
0.960810576236898	3.591\\
0.960819148320622	3.6\\
0.960829863425276	3.609\\
0.960838435509	3.618\\
0.960849150613654	3.627\\
0.960857722697378	3.636\\
0.960868437802032	3.645\\
0.960877009885756	3.654\\
0.96088772499041	3.663\\
0.960896297074133	3.672\\
0.960907012178788	3.681\\
0.960915584262512	3.69\\
0.960926299367166	3.699\\
0.96093487145089	3.708\\
0.960945586555544	3.717\\
0.960956301660198	3.726\\
0.960964873743922	3.735\\
0.960975588848576	3.744\\
0.9609841609323	3.753\\
0.960994876036954	3.762\\
0.961003448120678	3.771\\
0.961014163225332	3.78\\
0.961022735309056	3.789\\
0.96103345041371	3.798\\
0.961042022497434	3.807\\
0.961052737602088	3.816\\
0.961061309685812	3.825\\
0.961072024790466	3.834\\
0.96108059687419	3.843\\
0.961091311978844	3.852\\
0.961102027083498	3.861\\
0.961110599167222	3.87\\
0.961121314271877	3.879\\
0.9611298863556	3.888\\
0.961140601460254	3.897\\
0.961149173543978	3.906\\
0.961159888648633	3.915\\
0.961168460732356	3.924\\
0.96117917583701	3.933\\
0.961187747920734	3.942\\
0.961198463025388	3.951\\
0.961207035109112	3.96\\
0.961217750213766	3.969\\
0.961228465318421	3.978\\
0.961237037402144	3.987\\
0.961247752506799	3.996\\
0.961256324590522	4.005\\
0.961267039695177	4.014\\
0.9612756117789	4.023\\
0.961286326883555	4.032\\
0.961294898967278	4.041\\
0.961305614071933	4.05\\
0.961314186155656	4.059\\
0.961324901260311	4.068\\
0.961333473344034	4.077\\
0.961344188448689	4.086\\
0.961352760532412	4.095\\
0.961363475637066	4.104\\
0.961374190741721	4.113\\
0.961382762825445	4.122\\
0.961393477930099	4.131\\
0.961402050013823	4.14\\
0.961412765118477	4.149\\
0.961421337202201	4.158\\
0.961432052306855	4.167\\
0.961440624390578	4.176\\
0.961451339495233	4.185\\
0.961459911578956	4.194\\
0.96146848366268	4.203\\
0.961477055746403	4.212\\
0.961485627830127	4.221\\
0.96149205689292	4.23\\
0.961500628976643	4.239\\
0.961509201060367	4.248\\
0.96151777314409	4.257\\
0.961526345227814	4.266\\
0.961532774290606	4.275\\
0.96154134637433	4.284\\
0.961549918458053	4.293\\
0.961558490541777	4.302\\
0.961567062625501	4.311\\
0.961575634709224	4.32\\
0.961582063772017	4.329\\
0.96159063585574	4.338\\
0.961599207939464	4.347\\
0.961607780023187	4.356\\
0.961616352106911	4.365\\
0.961622781169704	4.374\\
0.961631353253427	4.383\\
0.961639925337151	4.392\\
0.961648497420874	4.401\\
0.961657069504598	4.41\\
0.961663498567391	4.419\\
0.961672070651114	4.428\\
0.961680642734838	4.437\\
0.961689214818561	4.446\\
0.961697786902285	4.455\\
0.961704215965077	4.464\\
0.961712788048801	4.473\\
0.961721360132524	4.482\\
0.961729932216248	4.491\\
0.961738504299971	4.5\\
0.961744933362764	4.509\\
0.961753505446488	4.518\\
0.961762077530211	4.527\\
0.961770649613935	4.536\\
0.961779221697658	4.545\\
0.961785650760451	4.554\\
0.961794222844174	4.563\\
0.961802794927898	4.572\\
0.961811367011622	4.581\\
0.961819939095345	4.59\\
0.961826368158138	4.599\\
0.961834940241861	4.608\\
0.961843512325585	4.617\\
0.961852084409308	4.626\\
0.961860656493032	4.635\\
0.961867085555825	4.644\\
0.961875657639548	4.653\\
0.961884229723272	4.662\\
0.961892801806995	4.671\\
0.961901373890719	4.68\\
0.961907802953511	4.689\\
0.961916375037235	4.698\\
0.961924947120959	4.707\\
0.961933519204682	4.716\\
0.961942091288406	4.725\\
0.961950663372129	4.734\\
0.961957092434922	4.743\\
0.961965664518645	4.752\\
0.961974236602369	4.761\\
0.961982808686092	4.77\\
0.961991380769816	4.779\\
0.961997809832609	4.788\\
0.962006381916332	4.797\\
0.962014954000056	4.806\\
0.962023526083779	4.815\\
0.962032098167503	4.824\\
0.962038527230295	4.833\\
0.962047099314019	4.842\\
0.962055671397743	4.851\\
0.962064243481466	4.86\\
0.96207281556519	4.869\\
0.962079244627982	4.878\\
0.962087816711706	4.887\\
0.962096388795429	4.896\\
0.962104960879153	4.905\\
0.962113532962876	4.914\\
0.962119962025669	4.923\\
0.962128534109393	4.932\\
0.962137106193116	4.941\\
0.96214567827684	4.95\\
0.962154250360563	4.959\\
0.962160679423356	4.968\\
0.962169251507079	4.977\\
0.962177823590803	4.986\\
0.962186395674527	4.995\\
0.96219496775825	5.004\\
0.962201396821043	5.013\\
0.962209968904766	5.022\\
0.96221854098849	5.031\\
0.962224970051282	5.04\\
0.962233542135006	5.049\\
0.962239971197799	5.058\\
0.962248543281522	5.067\\
0.962257115365246	5.076\\
0.962263544428038	5.085\\
0.962272116511762	5.094\\
0.962280688595485	5.103\\
0.962287117658278	5.112\\
0.962295689742002	5.121\\
0.962304261825725	5.13\\
0.962310690888518	5.139\\
0.962319262972241	5.148\\
0.962327835055965	5.157\\
0.962334264118758	5.166\\
0.962342836202481	5.175\\
0.962351408286205	5.184\\
0.962357837348997	5.193\\
0.962366409432721	5.202\\
0.962374981516445	5.211\\
0.962381410579237	5.22\\
0.962389982662961	5.229\\
0.962398554746684	5.238\\
0.962404983809477	5.247\\
0.9624135558932	5.256\\
0.962422127976924	5.265\\
0.962428557039717	5.274\\
0.96243712912344	5.283\\
0.962445701207164	5.292\\
0.962452130269956	5.301\\
0.96246070235368	5.31\\
0.962469274437403	5.319\\
0.962475703500196	5.328\\
0.96248427558392	5.337\\
0.962490704646712	5.346\\
0.962499276730436	5.355\\
0.962507848814159	5.364\\
0.962514277876952	5.373\\
0.962522849960675	5.382\\
0.962531422044399	5.391\\
0.962537851107192	5.4\\
0.962546423190915	5.409\\
0.962554995274639	5.418\\
0.962561424337432	5.427\\
0.962569996421155	5.436\\
0.962578568504879	5.445\\
0.962584997567671	5.454\\
0.962593569651395	5.463\\
0.962602141735118	5.472\\
0.962608570797911	5.481\\
0.962617142881635	5.49\\
0.962625714965358	5.499\\
0.962632144028151	5.508\\
0.962640716111874	5.517\\
0.962649288195598	5.526\\
0.96265571725839	5.535\\
0.962664289342114	5.544\\
0.962672861425838	5.553\\
0.96267929048863	5.562\\
0.962687862572354	5.571\\
0.962696434656077	5.58\\
0.96270286371887	5.589\\
0.962711435802593	5.598\\
0.962720007886317	5.607\\
0.96272643694911	5.616\\
0.962735009032833	5.625\\
0.962741438095626	5.634\\
0.962750010179349	5.643\\
0.962758582263073	5.652\\
0.962765011325866	5.661\\
0.962773583409589	5.67\\
0.962782155493313	5.679\\
0.962788584556105	5.688\\
0.962797156639829	5.697\\
0.962805728723553	5.706\\
0.962812157786345	5.715\\
0.962820729870069	5.724\\
0.962829301953792	5.733\\
0.962835731016585	5.742\\
0.962844303100308	5.751\\
0.962852875184032	5.76\\
0.962859304246825	5.769\\
0.962867876330548	5.778\\
0.962876448414272	5.787\\
0.962882877477064	5.796\\
0.962891449560788	5.805\\
0.962897878623581	5.814\\
0.962904307686373	5.823\\
0.962908593728235	5.832\\
0.962915022791028	5.841\\
0.96292145185382	5.85\\
0.962925737895682	5.859\\
0.962932166958475	5.868\\
0.962938596021267	5.877\\
0.96294502508406	5.886\\
0.962949311125922	5.895\\
0.962955740188714	5.904\\
0.962962169251507	5.913\\
0.962966455293369	5.922\\
0.962972884356161	5.931\\
0.962979313418954	5.94\\
0.962985742481747	5.949\\
0.962990028523609	5.958\\
0.962996457586401	5.967\\
0.963002886649194	5.976\\
0.963007172691056	5.985\\
0.963013601753848	5.994\\
0.963020030816641	6.003\\
0.963024316858503	6.012\\
0.963030745921295	6.021\\
0.963037174984088	6.03\\
0.963043604046881	6.039\\
0.963047890088742	6.048\\
0.963054319151535	6.057\\
0.963060748214328	6.066\\
0.96306503425619	6.075\\
0.963071463318982	6.084\\
0.963077892381775	6.093\\
0.963084321444568	6.102\\
0.963088607486429	6.111\\
0.963095036549222	6.12\\
0.963101465612015	6.129\\
0.963105751653876	6.138\\
0.963112180716669	6.147\\
0.963118609779462	6.156\\
0.963122895821324	6.165\\
0.963129324884116	6.174\\
0.963135753946909	6.183\\
0.963142183009701	6.192\\
0.963146469051563	6.201\\
0.963152898114356	6.21\\
0.963159327177149	6.219\\
0.96316361321901	6.228\\
0.963170042281803	6.237\\
0.963176471344596	6.246\\
0.963182900407388	6.255\\
0.96318718644925	6.264\\
0.963193615512043	6.273\\
0.963200044574835	6.282\\
0.963204330616697	6.291\\
0.96321075967949	6.3\\
0.963217188742282	6.309\\
0.963223617805075	6.318\\
0.963227903846937	6.327\\
0.963234332909729	6.336\\
0.963240761972522	6.345\\
0.963245048014384	6.354\\
0.963251477077177	6.363\\
0.963257906139969	6.372\\
0.963262192181831	6.381\\
0.963268621244624	6.39\\
0.963275050307416	6.399\\
0.963281479370209	6.408\\
0.963285765412071	6.417\\
0.963292194474863	6.426\\
0.963298623537656	6.435\\
0.963302909579518	6.444\\
0.96330933864231	6.453\\
0.963262192181831	6.462\\
0.96321075967949	6.471\\
0.963161470198079	6.48\\
0.963112180716669	6.489\\
0.963060748214328	6.498\\
0.963011458732917	6.507\\
0.962960026230576	6.516\\
0.962910736749166	6.525\\
0.962859304246825	6.534\\
0.962810014765414	6.543\\
0.962758582263073	6.552\\
0.962709292781663	6.561\\
0.962660003300252	6.57\\
0.962608570797911	6.579\\
0.962559281316501	6.588\\
0.962507848814159	6.597\\
0.962458559332749	6.606\\
0.962407126830408	6.615\\
0.962357837348997	6.624\\
0.962306404846656	6.633\\
0.962257115365246	6.642\\
0.962207825883835	6.651\\
0.962156393381494	6.66\\
0.962107103900084	6.669\\
0.962055671397743	6.678\\
0.962006381916332	6.687\\
0.961954949413991	6.696\\
0.961905659932581	6.705\\
0.961854227430239	6.714\\
0.961804937948829	6.723\\
0.961753505446488	6.732\\
0.961704215965077	6.741\\
0.961654926483667	6.75\\
0.961603493981326	6.759\\
0.961554204499915	6.768\\
0.961502771997574	6.777\\
0.961453482516164	6.786\\
0.961402050013823	6.795\\
0.961352760532412	6.804\\
0.961301328030071	6.813\\
0.96125203854866	6.822\\
0.96120274906725	6.831\\
0.961151316564909	6.84\\
0.961102027083498	6.849\\
0.961050594581157	6.858\\
0.961001305099747	6.867\\
0.960949872597406	6.876\\
0.960900583115995	6.885\\
0.960849150613654	6.894\\
0.960799861132244	6.903\\
0.960750571650833	6.912\\
0.960699139148492	6.921\\
0.960649849667082	6.93\\
0.96059841716474	6.939\\
0.96054912768333	6.948\\
0.960497695180989	6.957\\
0.960448405699579	6.966\\
0.96023410360649	6.975\\
0.959880505152894	6.984\\
0.959529049720229	6.993\\
0.959175451266633	7.002\\
0.958821852813036	7.011\\
0.95846825435944	7.02\\
0.958114655905844	7.029\\
0.957761057452248	7.038\\
0.957407458998652	7.047\\
0.957053860545056	7.056\\
0.95670026209146	7.065\\
0.956346663637864	7.074\\
0.955993065184268	7.083\\
0.955639466730672	7.092\\
0.955285868277075	7.101\\
0.954932269823479	7.11\\
0.954578671369883	7.119\\
0.954225072916287	7.128\\
0.953871474462691	7.137\\
0.953517876009095	7.146\\
0.953164277555499	7.155\\
0.952810679101903	7.164\\
0.952457080648307	7.173\\
0.952103482194711	7.182\\
0.951749883741114	7.191\\
0.951396285287518	7.2\\
0.951042686833922	7.209\\
0.950689088380326	7.218\\
0.95033548992673	7.227\\
0.949981891473134	7.236\\
0.949628293019538	7.245\\
0.949274694565942	7.254\\
0.948921096112346	7.263\\
0.94856749765875	7.272\\
0.948213899205154	7.281\\
0.947860300751557	7.29\\
0.947506702297961	7.299\\
0.947153103844365	7.308\\
0.946799505390769	7.317\\
0.946445906937173	7.326\\
0.946092308483577	7.335\\
0.945738710029981	7.344\\
0.945385111576385	7.353\\
0.945031513122789	7.362\\
0.944677914669193	7.371\\
0.944324316215597	7.38\\
0.94365569368516	7.389\\
0.942779198124428	7.398\\
0.941904845584627	7.407\\
0.941028350023895	7.416\\
0.940151854463163	7.425\\
0.93927535890243	7.434\\
0.938398863341698	7.443\\
0.937524510801897	7.452\\
0.936648015241165	7.461\\
0.935771519680433	7.47\\
0.934895024119701	7.479\\
0.934018528558968	7.488\\
0.933144176019167	7.497\\
0.932267680458435	7.506\\
0.931391184897703	7.515\\
0.930514689336971	7.524\\
0.929638193776239	7.533\\
0.928763841236437	7.542\\
0.927887345675705	7.551\\
0.927010850114973	7.56\\
0.926134354554241	7.569\\
0.925257858993509	7.578\\
0.924383506453707	7.587\\
0.923507010892975	7.596\\
0.922630515332243	7.605\\
0.921754019771511	7.614\\
0.920877524210779	7.623\\
0.920003171670978	7.632\\
0.919126676110246	7.641\\
0.918250180549513	7.65\\
0.917373684988781	7.659\\
0.916497189428049	7.668\\
0.915622836888248	7.677\\
0.914746341327516	7.686\\
0.913869845766784	7.695\\
0.912993350206051	7.704\\
0.912116854645319	7.713\\
0.910918905944954	7.722\\
0.909665238700386	7.731\\
0.908411571455818	7.74\\
0.907157904211251	7.749\\
0.905904236966682	7.758\\
0.904650569722114	7.767\\
0.903396902477546	7.776\\
0.902143235232979	7.785\\
0.900889567988411	7.794\\
0.899635900743842	7.803\\
0.898382233499275	7.812\\
0.897128566254707	7.821\\
0.895874899010139	7.83\\
0.894621231765571	7.839\\
0.893367564521003	7.848\\
0.892113897276435	7.857\\
0.890860230031867	7.866\\
0.889606562787299	7.875\\
0.888352895542731	7.884\\
0.887097085277232	7.893\\
0.885843418032664	7.902\\
0.884589750788096	7.911\\
0.883336083543528	7.92\\
0.88208241629896	7.929\\
0.880828749054392	7.938\\
0.879575081809824	7.947\\
0.878321414565256	7.956\\
0.877067747320688	7.965\\
0.87581408007612	7.974\\
0.874453261785008	7.983\\
0.872966005258973	7.992\\
0.871478748732939	8.001\\
0.869991492206904	8.01\\
0.86850423568087	8.019\\
0.867016979154835	8.028\\
0.865529722628801	8.037\\
0.864040323081836	8.046\\
0.862553066555801	8.055\\
0.861065810029767	8.064\\
0.859578553503732	8.073\\
0.858091296977698	8.082\\
0.856604040451663	8.091\\
0.855114640904698	8.1\\
0.853627384378663	8.109\\
0.852140127852629	8.118\\
0.850652871326594	8.127\\
0.84916561480056	8.136\\
0.847676215253594	8.145\\
0.84618895872756	8.154\\
0.844701702201526	8.163\\
0.843214445675491	8.172\\
0.841727189149456	8.181\\
0.840239932623422	8.19\\
0.838474083376372	8.199\\
0.836693232982807	8.208\\
0.83491023956831	8.217\\
0.833127246153813	8.226\\
0.831344252739317	8.235\\
0.82956125932482	8.244\\
0.827778265910323	8.253\\
0.825995272495826	8.262\\
0.82421227908133	8.271\\
0.822429285666833	8.28\\
0.820646292252336	8.289\\
0.81886329883784	8.298\\
0.817080305423343	8.307\\
0.815297312008846	8.316\\
0.813516461615281	8.325\\
0.811733468200784	8.334\\
0.809950474786287	8.343\\
0.808167481371791	8.352\\
0.806337341496814	8.361\\
0.80417289035662	8.37\\
0.802006296195495	8.379\\
0.799841845055301	8.388\\
0.797677393915106	8.397\\
0.795510799753981	8.406\\
0.793346348613787	8.415\\
0.791181897473593	8.424\\
0.789015303312467	8.433\\
0.786850852172273	8.442\\
0.784684258011148	8.451\\
0.782519806870954	8.46\\
0.780355355730759	8.469\\
0.778188761569634	8.478\\
0.77602431042944	8.487\\
0.773825570954351	8.496\\
0.771174654062846	8.505\\
0.768525880192272	8.514\\
0.765874963300767	8.523\\
0.763224046409261	8.532\\
0.760575272538687	8.541\\
0.757924355647182	8.55\\
0.755273438755676	8.559\\
0.752624664885102	8.568\\
0.749973747993597	8.577\\
0.747322831102091	8.586\\
0.744674057231517	8.595\\
0.742005996172565	8.604\\
0.738735746232033	8.613\\
0.735465496291502	8.622\\
0.732195246350971	8.631\\
0.72892499641044	8.64\\
0.725652603448978	8.649\\
0.722382353508447	8.658\\
0.719112103567915	8.667\\
0.715841853627384	8.676\\
0.712571603686853	8.685\\
0.708984186648551	8.694\\
0.704912446879869	8.703\\
0.700840707111186	8.712\\
0.696768967342504	8.721\\
0.692697227573822	8.73\\
0.688625487805139	8.739\\
0.684553748036457	8.748\\
0.680482008267775	8.757\\
0.675685927424453	8.766\\
0.670564107399637	8.775\\
0.665442287374821	8.784\\
0.660318324329074	8.793\\
0.655196504304258	8.802\\
0.650074684279441	8.811\\
0.644211379012539	8.82\\
0.637670879131476	8.829\\
0.631132522271345	8.838\\
0.624594165411213	8.847\\
0.618055808551082	8.856\\
0.610343076220825	8.865\\
0.601818138957763	8.874\\
0.593293201694701	8.883\\
0.58477040745257	8.892\\
0.574351039686605	8.901\\
0.562890163748229	8.91\\
0.551429287809854	8.919\\
0.537675379475431	8.928\\
0.52152771676121	8.937\\
0.505382197067919	8.946\\
0.4812067779466	8.955\\
0.456724906832165	8.964\\
0.415808208198769	8.973\\
0.357200871780915	8.982\\
0.264905246329541	8.991\\
0	9\\
};
\addlegendentry{Van Leer}

\addplot [color=mycolor4]
  table[row sep=crcr]{%
0	0\\
0.00978191837216034	0.009\\
0.0251011503371678	0.018\\
0.0341591174673285	0.027\\
0.0412112310644639	0.036\\
0.0471421679513707	0.045\\
0.0520368365191848	0.054\\
0.056882618437557	0.063\\
0.0612406059371549	0.072\\
0.0655985934367529	0.081\\
0.0698082056669919	0.09\\
0.0739483473953922	0.099\\
0.0780882747086634	0.108\\
0.0821881063927871	0.117\\
0.0862821488684242	0.126\\
0.0903761913440612	0.135\\
0.0945161186573324	0.144\\
0.0986991434115591	0.153\\
0.102882168165786	0.162\\
0.107065192920012	0.171\\
0.111351994596739	0.18\\
0.115750935386001	0.189\\
0.120149876175263	0.198\\
0.124548816964525	0.207\\
0.128947757753787	0.216\\
0.133561113672181	0.225\\
0.13829647179905	0.234\\
0.14303182992592	0.243\\
0.147767188052789	0.252\\
0.152502546179658	0.261\\
0.157237689891399	0.27\\
0.162339483473954	0.279\\
0.167518037672738	0.288\\
0.172696591871522	0.297\\
0.177875146070307	0.306\\
0.183053700269091	0.315\\
0.188232254467875	0.324\\
0.19341059425153	0.333\\
0.199030200370938	0.342\\
0.2047252806158	0.351\\
0.210420360860662	0.36\\
0.216115441105524	0.369\\
0.221810306935257	0.378\\
0.22750517276499	0.387\\
0.233200038594723	0.396\\
0.238894904424456	0.405\\
0.244733428390707	0.414\\
0.25072204294735	0.423\\
0.256710657503993	0.432\\
0.262699272060637	0.441\\
0.26868788661728	0.45\\
0.274676501173923	0.459\\
0.280665115730566	0.468\\
0.286653730287209	0.477\\
0.292642344843852	0.486\\
0.298628815249204	0.495\\
0.304921899289214	0.504\\
0.311590209805204	0.513\\
0.318260664472485	0.522\\
0.324931119139766	0.531\\
0.331599429655757	0.54\\
0.338269884323038	0.549\\
0.344938194839028	0.558\\
0.351608649506309	0.567\\
0.35827910417359	0.576\\
0.364947414689581	0.585\\
0.372074573581912	0.594\\
0.379874995979716	0.603\\
0.387675418377521	0.612\\
0.395475840775325	0.621\\
0.403278407324421	0.63\\
0.411078829722225	0.639\\
0.41887925212003	0.648\\
0.426679674517834	0.657\\
0.434507970882425	0.666\\
0.443045981324442	0.675\\
0.45158613591775	0.684\\
0.460126290511058	0.693\\
0.468664300953075	0.702\\
0.477204455546383	0.711\\
0.4857424659884	0.72\\
0.494312638699787	0.729\\
0.503330939031058	0.738\\
0.512351383513621	0.747\\
0.521371827996183	0.756\\
0.530390128327455	0.765\\
0.539410572810017	0.774\\
0.548446026351619	0.783\\
0.557532939524213	0.792\\
0.566619852696806	0.801\\
0.5757067658694	0.81\\
0.584793679041993	0.819\\
0.593762663893564	0.828\\
0.602562260793122	0.837\\
0.61136400184397	0.846\\
0.620163598743527	0.855\\
0.628770222026866	0.864\\
0.637050934313925	0.873\\
0.645331646600984	0.882\\
0.653608070585461	0.891\\
0.66125411409029	0.9\\
0.668902301746411	0.909\\
0.676411119568597	0.918\\
0.683401052778284	0.927\\
0.690390985987971	0.936\\
0.696840593072247	0.945\\
0.703213010710036	0.954\\
0.709077264491783	0.963\\
0.714757121262476	0.972\\
0.720108922885599	0.981\\
0.725046903309498	0.99\\
0.729798342571052	0.999\\
0.733880806629716	1.008\\
0.737433665319425	1.017\\
0.740744234913215	1.026\\
0.743924011278236	1.035\\
0.746882940060251	1.044\\
0.749831148085809	1.053\\
0.752582094192566	1.062\\
0.755315887088993	1.071\\
0.757970346387641	1.08\\
0.760500444911393	1.089\\
0.763028399283853	1.098\\
0.765509182327905	1.107\\
0.767839874781564	1.116\\
0.770170567235224	1.125\\
0.772499115537592	1.134\\
0.774774060057678	1.143\\
0.776916067197701	1.152\\
0.779055930186434	1.161\\
0.781197937326458	1.17\\
0.783339944466482	1.179\\
0.785417627067766	1.188\\
0.787383813801902	1.197\\
0.789350000536038	1.206\\
0.791316187270174	1.215\\
0.793284518155601	1.224\\
0.795250704889737	1.233\\
0.797184729354503	1.242\\
0.798987960590499	1.251\\
0.800791191826495	1.26\\
0.802596567213783	1.269\\
0.804399798449779	1.278\\
0.806203029685775	1.287\\
0.808006260921771	1.296\\
0.809811636309058	1.305\\
0.811537678098567	1.314\\
0.813186530441588	1.323\\
0.814833238633318	1.332\\
0.816479946825048	1.341\\
0.818128799168069	1.35\\
0.819775507359799	1.359\\
0.821424359702821	1.368\\
0.823071067894551	1.377\\
0.824719920237572	1.386\\
0.826366628429302	1.395\\
0.82788039924097	1.404\\
0.829372728539726	1.413\\
0.830865057838481	1.422\\
0.832355242985945	1.431\\
0.8338475722847	1.44\\
0.835339901583456	1.449\\
0.83683008673092	1.458\\
0.838322416029675	1.467\\
0.83981474532843	1.476\\
0.841307074627186	1.485\\
0.842799403925941	1.494\\
0.844289589073405	1.503\\
0.845646836840808	1.512\\
0.846974066490132	1.521\\
0.848301296139456	1.53\\
0.84962852578878	1.539\\
0.850955755438104	1.548\\
0.852282985087428	1.557\\
0.853610214736752	1.566\\
0.854939588537367	1.575\\
0.856266818186691	1.584\\
0.857594047836015	1.593\\
0.858921277485339	1.602\\
0.860248507134663	1.611\\
0.861575736783987	1.62\\
0.862902966433312	1.629\\
0.864230196082636	1.638\\
0.865415911746733	1.647\\
0.866560888536295	1.656\\
0.867703721174566	1.665\\
0.868848697964128	1.674\\
0.869991530602399	1.683\\
0.871136507391962	1.692\\
0.872279340030233	1.701\\
0.873424316819795	1.71\\
0.874567149458066	1.719\\
0.875709982096337	1.728\\
0.876854958885899	1.737\\
0.87799779152417	1.746\\
0.879142768313732	1.755\\
0.880285600952003	1.764\\
0.881430577741566	1.773\\
0.882573410379836	1.782\\
0.883718387169399	1.791\\
0.88486121980767	1.8\\
0.886006196597232	1.809\\
0.886943190711536	1.818\\
0.887871608220676	1.827\\
0.888797881578524	1.836\\
0.889726299087664	1.845\\
0.890654716596803	1.854\\
0.891580989954651	1.863\\
0.892509407463791	1.872\\
0.89343782497293	1.881\\
0.894364098330778	1.89\\
0.895292515839918	1.899\\
0.896218789197766	1.908\\
0.897147206706905	1.917\\
0.898075624216045	1.926\\
0.899001897573893	1.935\\
0.899930315083032	1.944\\
0.900858732592172	1.953\\
0.90178500595002	1.962\\
0.902713423459159	1.971\\
0.903641840968299	1.98\\
0.904568114326147	1.989\\
0.905496531835286	1.998\\
0.906422805193134	2.007\\
0.907351222702274	2.016\\
0.908166000192974	2.025\\
0.908832831244573	2.034\\
0.909499662296172	2.043\\
0.910166493347771	2.052\\
0.91083332439937	2.061\\
0.911500155450969	2.07\\
0.912166986502568	2.079\\
0.912833817554167	2.088\\
0.913500648605766	2.097\\
0.914167479657365	2.106\\
0.914834310708964	2.115\\
0.915501141760563	2.124\\
0.916167972812162	2.133\\
0.916834803863761	2.142\\
0.91750163491536	2.151\\
0.918168465966959	2.16\\
0.918835297018558	2.169\\
0.919502128070157	2.178\\
0.920168959121756	2.187\\
0.920835790173355	2.196\\
0.921502621224954	2.205\\
0.922169452276553	2.214\\
0.922836283328152	2.223\\
0.923503114379751	2.232\\
0.92416994543135	2.241\\
0.924836776482949	2.25\\
0.925503607534548	2.259\\
0.926170438586147	2.268\\
0.926837269637746	2.277\\
0.927495524084179	2.286\\
0.927868606408868	2.295\\
0.928243832884848	2.304\\
0.928619059360828	2.313\\
0.928992141685517	2.322\\
0.929367368161497	2.331\\
0.929742594637478	2.34\\
0.930115676962166	2.349\\
0.930490903438147	2.358\\
0.930866129914127	2.367\\
0.931241356390107	2.376\\
0.931614438714796	2.385\\
0.931989665190776	2.394\\
0.932364891666756	2.403\\
0.932737973991445	2.412\\
0.933113200467425	2.421\\
0.933488426943405	2.43\\
0.933861509268094	2.439\\
0.934236735744074	2.448\\
0.934611962220054	2.457\\
0.934985044544743	2.466\\
0.935360271020723	2.475\\
0.935735497496703	2.484\\
0.936110723972683	2.493\\
0.936483806297372	2.502\\
0.936859032773352	2.511\\
0.937234259249333	2.52\\
0.937607341574021	2.529\\
0.937982568050002	2.538\\
0.938357794525982	2.547\\
0.938730876850671	2.556\\
0.939106103326651	2.565\\
0.939481329802631	2.574\\
0.939856556278611	2.583\\
0.9402296386033	2.592\\
0.94060486507928	2.601\\
0.94098009155526	2.61\\
0.941288849341209	2.619\\
0.94143679578031	2.628\\
0.941584742219411	2.637\\
0.941732688658512	2.646\\
0.941880635097613	2.655\\
0.942028581536713	2.664\\
0.942176527975814	2.673\\
0.942324474414915	2.682\\
0.942472420854016	2.691\\
0.942620367293116	2.7\\
0.942768313732217	2.709\\
0.942916260171318	2.718\\
0.943064206610418	2.727\\
0.943212153049519	2.736\\
0.94336009948862	2.745\\
0.943508045927721	2.754\\
0.943655992366821	2.763\\
0.943803938805922	2.772\\
0.943951885245023	2.781\\
0.944097687532832	2.79\\
0.944245633971933	2.799\\
0.944393580411034	2.808\\
0.944541526850134	2.817\\
0.944689473289235	2.826\\
0.944837419728336	2.835\\
0.944985366167437	2.844\\
0.945133312606538	2.853\\
0.945281259045638	2.862\\
0.945429205484739	2.871\\
0.94557715192384	2.88\\
0.94572509836294	2.889\\
0.945873044802041	2.898\\
0.946020991241142	2.907\\
0.946168937680243	2.916\\
0.946316884119343	2.925\\
0.946464830558444	2.934\\
0.946612776997545	2.943\\
0.946760723436646	2.952\\
0.946908669875746	2.961\\
0.947056616314847	2.97\\
0.947204562753948	2.979\\
0.947352509193049	2.988\\
0.947500455632149	2.997\\
0.94764840207125	3.006\\
0.947796348510351	3.015\\
0.947944294949452	3.024\\
0.948057934967891	3.033\\
0.948111538750174	3.042\\
0.948165142532457	3.051\\
0.94821874631474	3.06\\
0.948270205945732	3.069\\
0.948323809728014	3.078\\
0.948377413510297	3.087\\
0.94843101729258	3.096\\
0.948484621074863	3.105\\
0.948536080705855	3.114\\
0.948589684488138	3.123\\
0.94864328827042	3.132\\
0.948696892052703	3.141\\
0.948750495834986	3.15\\
0.948801955465978	3.159\\
0.948855559248261	3.168\\
0.948909163030543	3.177\\
0.948962766812826	3.186\\
0.949016370595109	3.195\\
0.949067830226101	3.204\\
0.949121434008384	3.213\\
0.949175037790666	3.222\\
0.949228641572949	3.231\\
0.949282245355232	3.24\\
0.949335849137515	3.249\\
0.949387308768507	3.258\\
0.94944091255079	3.267\\
0.949494516333072	3.276\\
0.949548120115355	3.285\\
0.949601723897638	3.294\\
0.94965318352863	3.303\\
0.949706787310913	3.312\\
0.949760391093196	3.321\\
0.949813994875478	3.33\\
0.949867598657761	3.339\\
0.949919058288753	3.348\\
0.949972662071036	3.357\\
0.950026265853319	3.366\\
0.950079869635601	3.375\\
0.950133473417884	3.384\\
0.950184933048876	3.393\\
0.950238536831159	3.402\\
0.950292140613442	3.411\\
0.950345744395725	3.42\\
0.950399348178008	3.429\\
0.950450807808999	3.438\\
0.950504411591282	3.447\\
0.950558015373565	3.456\\
0.950611619155848	3.465\\
0.95066522293813	3.474\\
0.950716682569122	3.483\\
0.950770286351405	3.492\\
0.950823890133688	3.501\\
0.950877493915971	3.51\\
0.950931097698254	3.519\\
0.950982557329245	3.528\\
0.951036161111528	3.537\\
0.951089764893811	3.546\\
0.951121927163181	3.555\\
0.95115408943255	3.564\\
0.95118625170192	3.573\\
0.951216269819999	3.582\\
0.951248432089368	3.591\\
0.951280594358738	3.6\\
0.951312756628108	3.609\\
0.951342774746186	3.618\\
0.951374937015556	3.627\\
0.951407099284926	3.636\\
0.951437117403004	3.645\\
0.951469279672374	3.654\\
0.951501441941743	3.663\\
0.951531460059822	3.672\\
0.951563622329192	3.681\\
0.951595784598561	3.69\\
0.95162580271664	3.699\\
0.951657964986009	3.708\\
0.951690127255379	3.717\\
0.951720145373458	3.726\\
0.951752307642827	3.735\\
0.951784469912197	3.744\\
0.951814488030275	3.753\\
0.951846650299645	3.762\\
0.951878812569015	3.771\\
0.951908830687093	3.78\\
0.951940992956463	3.789\\
0.951973155225833	3.798\\
0.952005317495202	3.807\\
0.952035335613281	3.816\\
0.952067497882651	3.825\\
0.95209966015202	3.834\\
0.952129678270099	3.843\\
0.952161840539469	3.852\\
0.952194002808838	3.861\\
0.952224020926917	3.87\\
0.952256183196286	3.879\\
0.952288345465656	3.888\\
0.952318363583735	3.897\\
0.952350525853104	3.906\\
0.952382688122474	3.915\\
0.952412706240552	3.924\\
0.952444868509922	3.933\\
0.952477030779292	3.942\\
0.95250704889737	3.951\\
0.95253921116674	3.96\\
0.95257137343611	3.969\\
0.952601391554188	3.978\\
0.952633553823558	3.987\\
0.952665716092927	3.996\\
0.952697878362297	4.005\\
0.952727896480376	4.014\\
0.952760058749745	4.023\\
0.952792221019115	4.032\\
0.952822239137194	4.041\\
0.952854401406563	4.05\\
0.952886563675933	4.059\\
0.952916581794011	4.068\\
0.952948744063381	4.077\\
0.952980906332751	4.086\\
0.953010924450829	4.095\\
0.953043086720199	4.104\\
0.953075248989569	4.113\\
0.953105267107647	4.122\\
0.953137429377017	4.131\\
0.953169591646387	4.14\\
0.953199609764465	4.149\\
0.953231772033835	4.158\\
0.953263934303204	4.167\\
0.953293952421283	4.176\\
0.953326114690653	4.185\\
0.953356132808731	4.194\\
0.953377574321644	4.203\\
0.953401159985849	4.212\\
0.953422601498762	4.221\\
0.953444043011675	4.23\\
0.953465484524588	4.239\\
0.953486926037501	4.248\\
0.953510511701706	4.257\\
0.953531953214619	4.266\\
0.953553394727532	4.275\\
0.953574836240445	4.284\\
0.95359842190465	4.293\\
0.953619863417563	4.302\\
0.953641304930476	4.311\\
0.953662746443389	4.32\\
0.953684187956302	4.329\\
0.953707773620507	4.338\\
0.95372921513342	4.347\\
0.953750656646333	4.356\\
0.953772098159246	4.365\\
0.953795683823451	4.374\\
0.953817125336364	4.383\\
0.953838566849277	4.392\\
0.95386000836219	4.401\\
0.953883594026394	4.41\\
0.953905035539308	4.419\\
0.953926477052221	4.428\\
0.953947918565134	4.437\\
0.953969360078047	4.446\\
0.953992945742252	4.455\\
0.954014387255165	4.464\\
0.954035828768078	4.473\\
0.954057270280991	4.482\\
0.954080855945195	4.491\\
0.954102297458109	4.5\\
0.954123738971022	4.509\\
0.954145180483935	4.518\\
0.954166621996848	4.527\\
0.954190207661053	4.536\\
0.954211649173966	4.545\\
0.954233090686879	4.554\\
0.954254532199792	4.563\\
0.954278117863997	4.572\\
0.95429955937691	4.581\\
0.954321000889823	4.59\\
0.954342442402736	4.599\\
0.954363883915649	4.608\\
0.954387469579854	4.617\\
0.954408911092767	4.626\\
0.95443035260568	4.635\\
0.954451794118593	4.644\\
0.954475379782798	4.653\\
0.954496821295711	4.662\\
0.954518262808624	4.671\\
0.954539704321537	4.68\\
0.954563289985741	4.689\\
0.954584731498655	4.698\\
0.954606173011568	4.707\\
0.954627614524481	4.716\\
0.954649056037394	4.725\\
0.954672641701598	4.734\\
0.954694083214512	4.743\\
0.954715524727425	4.752\\
0.954736966240338	4.761\\
0.954760551904542	4.77\\
0.954781993417456	4.779\\
0.954803434930369	4.788\\
0.954824876443282	4.797\\
0.954846317956195	4.806\\
0.954869903620399	4.815\\
0.954891345133313	4.824\\
0.954912786646226	4.833\\
0.954934228159139	4.842\\
0.954957813823343	4.851\\
0.954979255336257	4.86\\
0.95500069684917	4.869\\
0.955022138362083	4.878\\
0.955043579874996	4.887\\
0.9550671655392	4.896\\
0.955088607052114	4.905\\
0.955110048565027	4.914\\
0.95513149007794	4.923\\
0.955155075742144	4.932\\
0.955176517255058	4.941\\
0.955197958767971	4.95\\
0.955219400280884	4.959\\
0.955242985945088	4.968\\
0.955264427458001	4.977\\
0.955285868970915	4.986\\
0.955307310483828	4.995\\
0.95532660784545	5.004\\
0.95534376105578	5.013\\
0.955360914266111	5.022\\
0.955378067476441	5.031\\
0.955395220686772	5.04\\
0.955412373897102	5.049\\
0.955429527107433	5.058\\
0.955444536166472	5.067\\
0.955461689376802	5.076\\
0.955478842587133	5.085\\
0.955495995797463	5.094\\
0.955513149007794	5.103\\
0.955530302218125	5.112\\
0.955547455428455	5.121\\
0.955562464487494	5.13\\
0.955579617697825	5.139\\
0.955596770908155	5.148\\
0.955613924118486	5.157\\
0.955631077328816	5.166\\
0.955648230539147	5.175\\
0.955665383749477	5.184\\
0.955682536959808	5.193\\
0.955697546018847	5.202\\
0.955714699229178	5.211\\
0.955731852439508	5.22\\
0.955749005649839	5.229\\
0.955766158860169	5.238\\
0.9557833120705	5.247\\
0.95580046528083	5.256\\
0.955815474339869	5.265\\
0.9558326275502	5.274\\
0.95584978076053	5.283\\
0.955866933970861	5.292\\
0.955884087181192	5.301\\
0.955901240391522	5.31\\
0.955918393601852	5.319\\
0.955933402660892	5.328\\
0.955950555871222	5.337\\
0.955967709081553	5.346\\
0.955984862291883	5.355\\
0.956002015502214	5.364\\
0.956019168712544	5.373\\
0.956036321922875	5.382\\
0.956053475133205	5.391\\
0.956068484192245	5.4\\
0.956085637402575	5.409\\
0.956102790612906	5.418\\
0.956119943823236	5.427\\
0.956137097033567	5.436\\
0.956154250243897	5.445\\
0.956171403454228	5.454\\
0.956186412513267	5.463\\
0.956203565723597	5.472\\
0.956220718933928	5.481\\
0.956237872144258	5.49\\
0.956255025354589	5.499\\
0.95627217856492	5.508\\
0.95628933177525	5.517\\
0.956306484985581	5.526\\
0.95632149404462	5.535\\
0.95633864725495	5.544\\
0.956355800465281	5.553\\
0.956372953675611	5.562\\
0.956390106885942	5.571\\
0.956407260096272	5.58\\
0.956424413306603	5.589\\
0.956439422365642	5.598\\
0.956456575575973	5.607\\
0.956473728786303	5.616\\
0.956490881996634	5.625\\
0.956508035206964	5.634\\
0.956525188417295	5.643\\
0.956542341627625	5.652\\
0.956557350686664	5.661\\
0.956574503896995	5.67\\
0.956591657107325	5.679\\
0.956608810317656	5.688\\
0.956625963527987	5.697\\
0.956643116738317	5.706\\
0.956660269948648	5.715\\
0.956677423158978	5.724\\
0.956692432218017	5.733\\
0.956709585428348	5.742\\
0.956726738638678	5.751\\
0.956743891849009	5.76\\
0.956761045059339	5.769\\
0.95677819826967	5.778\\
0.95679535148	5.787\\
0.95681036053904	5.796\\
0.95682751374937	5.805\\
0.956836090354535	5.814\\
0.956842522808409	5.823\\
0.956848955262283	5.832\\
0.956853243564866	5.841\\
0.95685967601874	5.85\\
0.956863964321322	5.859\\
0.956870396775196	5.868\\
0.95687682922907	5.877\\
0.956881117531653	5.886\\
0.956887549985527	5.895\\
0.956893982439401	5.904\\
0.956898270741984	5.913\\
0.956904703195857	5.922\\
0.95690899149844	5.931\\
0.956915423952314	5.94\\
0.956921856406188	5.949\\
0.956926144708771	5.958\\
0.956932577162645	5.967\\
0.956939009616519	5.976\\
0.956943297919101	5.985\\
0.956949730372975	5.994\\
0.956954018675558	6.003\\
0.956960451129432	6.012\\
0.956966883583306	6.021\\
0.956971171885888	6.03\\
0.956977604339762	6.039\\
0.956984036793636	6.048\\
0.956988325096219	6.057\\
0.956994757550093	6.066\\
0.956999045852675	6.075\\
0.957005478306549	6.084\\
0.957011910760423	6.093\\
0.957016199063006	6.102\\
0.95702263151688	6.111\\
0.957026919819463	6.12\\
0.957033352273336	6.129\\
0.95703978472721	6.138\\
0.957044073029793	6.147\\
0.957050505483667	6.156\\
0.957056937937541	6.165\\
0.957061226240123	6.174\\
0.957067658693997	6.183\\
0.95707194699658	6.192\\
0.957078379450454	6.201\\
0.957084811904328	6.21\\
0.957089100206911	6.219\\
0.957095532660785	6.228\\
0.957101965114659	6.237\\
0.957106253417241	6.246\\
0.957112685871115	6.255\\
0.957116974173698	6.264\\
0.957123406627572	6.273\\
0.957129839081446	6.282\\
0.957134127384028	6.291\\
0.957140559837902	6.3\\
0.957146992291776	6.309\\
0.957151280594359	6.318\\
0.957157713048233	6.327\\
0.957162001350815	6.336\\
0.957168433804689	6.345\\
0.957174866258563	6.354\\
0.957179154561146	6.363\\
0.95718558701502	6.372\\
0.957192019468894	6.381\\
0.957196307771476	6.39\\
0.95720274022535	6.399\\
0.957207028527933	6.408\\
0.957213460981807	6.417\\
0.957219893435681	6.426\\
0.957224181738264	6.435\\
0.957230614192137	6.444\\
0.957237046646011	6.453\\
0.95717057795598	6.462\\
0.957101965114659	6.471\\
0.957031208122045	6.48\\
0.956962595280723	6.489\\
0.956893982439401	6.498\\
0.956823225446788	6.507\\
0.956754612605465	6.516\\
0.956685999764143	6.525\\
0.95661524277153	6.534\\
0.956546629930208	6.543\\
0.956478017088886	6.552\\
0.956409404247564	6.561\\
0.95633864725495	6.57\\
0.956270034413628	6.579\\
0.956201421572306	6.588\\
0.956130664579693	6.597\\
0.956062051738371	6.606\\
0.955993438897049	6.615\\
0.955924826055726	6.624\\
0.955854069063113	6.633\\
0.955785456221791	6.642\\
0.955716843380469	6.651\\
0.955646086387855	6.66\\
0.955577473546533	6.669\\
0.955508860705211	6.678\\
0.955440247863889	6.687\\
0.955369490871276	6.696\\
0.955300878029954	6.705\\
0.955232265188632	6.714\\
0.955161508196018	6.723\\
0.955092895354696	6.732\\
0.955024282513374	6.741\\
0.954955669672052	6.75\\
0.954884912679439	6.759\\
0.954816299838117	6.768\\
0.954747686996794	6.777\\
0.954676930004181	6.786\\
0.954608317162859	6.795\\
0.954539704321537	6.804\\
0.954468947328924	6.813\\
0.954400334487601	6.822\\
0.954331721646279	6.831\\
0.954263108804957	6.84\\
0.954192351812344	6.849\\
0.954123738971022	6.858\\
0.9540551261297	6.867\\
0.953984369137086	6.876\\
0.953915756295764	6.885\\
0.953847143454442	6.894\\
0.95377853061312	6.903\\
0.953707773620507	6.912\\
0.953639160779185	6.921\\
0.953570547937863	6.93\\
0.953499790945249	6.939\\
0.953431178103927	6.948\\
0.953362565262605	6.957\\
0.953293952421283	6.966\\
0.953075248989569	6.975\\
0.952727896480376	6.984\\
0.952380543971183	6.993\\
0.952035335613281	7.002\\
0.951687983104088	7.011\\
0.951340630594895	7.02\\
0.950995422236993	7.029\\
0.9506480697278	7.038\\
0.950300717218607	7.047\\
0.949955508860705	7.056\\
0.949608156351512	7.065\\
0.949260803842319	7.074\\
0.948915595484417	7.083\\
0.948568242975224	7.092\\
0.948220890466031	7.101\\
0.94787568210813	7.11\\
0.947528329598936	7.119\\
0.947180977089743	7.128\\
0.946835768731842	7.137\\
0.946488416222649	7.146\\
0.946141063713456	7.155\\
0.945795855355554	7.164\\
0.945448502846361	7.173\\
0.945101150337168	7.182\\
0.944753797827975	7.191\\
0.944408589470073	7.2\\
0.94406123696088	7.209\\
0.943713884451687	7.218\\
0.943368676093785	7.227\\
0.943021323584592	7.236\\
0.942673971075399	7.245\\
0.942328762717497	7.254\\
0.941981410208304	7.263\\
0.941634057699111	7.272\\
0.941288849341209	7.281\\
0.940941496832016	7.29\\
0.940594144322823	7.299\\
0.940248935964922	7.308\\
0.939901583455729	7.317\\
0.939554230946536	7.326\\
0.939209022588634	7.335\\
0.938861670079441	7.344\\
0.938514317570248	7.353\\
0.938169109212346	7.362\\
0.937821756703153	7.371\\
0.93747440419396	7.38\\
0.936844023714313	7.389\\
0.936027102072322	7.398\\
0.935210180430331	7.407\\
0.93439325878834	7.416\\
0.933576337146349	7.425\\
0.932759415504358	7.434\\
0.931940349711076	7.443\\
0.931123428069085	7.452\\
0.930306506427094	7.461\\
0.929489584785102	7.47\\
0.928672663143111	7.479\\
0.92785574150112	7.488\\
0.927036675707838	7.497\\
0.926219754065847	7.506\\
0.925402832423856	7.515\\
0.924585910781865	7.524\\
0.923768989139874	7.533\\
0.922952067497883	7.542\\
0.922135145855892	7.551\\
0.921316080062609	7.56\\
0.920499158420618	7.569\\
0.919682236778627	7.578\\
0.918865315136636	7.587\\
0.918048393494645	7.596\\
0.917231471852654	7.605\\
0.916414550210663	7.614\\
0.91559548441738	7.623\\
0.914778562775389	7.632\\
0.913961641133398	7.641\\
0.913144719491407	7.65\\
0.912327797849416	7.659\\
0.911510876207425	7.668\\
0.910693954565434	7.677\\
0.909874888772152	7.686\\
0.909057967130161	7.695\\
0.90824104548817	7.704\\
0.907424123846179	7.713\\
0.906266282148868	7.722\\
0.905052692517984	7.731\\
0.903836958735808	7.74\\
0.902621224953633	7.749\\
0.901407635322748	7.758\\
0.900191901540573	7.767\\
0.898978311909688	7.776\\
0.897762578127513	7.785\\
0.896546844345337	7.794\\
0.895333254714453	7.803\\
0.894117520932277	7.812\\
0.892903931301393	7.821\\
0.891688197519217	7.83\\
0.890472463737041	7.839\\
0.889258874106157	7.848\\
0.888043140323981	7.857\\
0.886829550693097	7.866\\
0.885613816910921	7.875\\
0.884398083128746	7.884\\
0.883184493497861	7.893\\
0.881968759715685	7.902\\
0.880755170084801	7.911\\
0.879539436302626	7.92\\
0.87832370252045	7.929\\
0.877110112889565	7.938\\
0.87589437910739	7.947\\
0.874680789476505	7.956\\
0.87346505569433	7.965\\
0.872249321912154	7.974\\
0.870909227355082	7.983\\
0.869419042207618	7.992\\
0.867931001211446	8.001\\
0.866440816063981	8.01\\
0.864950630916518	8.019\\
0.863460445769053	8.028\\
0.861970260621589	8.037\\
0.860482219625417	8.046\\
0.858992034477953	8.055\\
0.857501849330489	8.064\\
0.856011664183025	8.073\\
0.854521479035561	8.082\\
0.853033438039388	8.091\\
0.851543252891924	8.1\\
0.85005306774446	8.109\\
0.848562882596996	8.118\\
0.847072697449532	8.127\\
0.845584656453359	8.136\\
0.844094471305895	8.145\\
0.842604286158431	8.154\\
0.841114101010967	8.163\\
0.839623915863503	8.172\\
0.838135874867331	8.181\\
0.836645689719867	8.19\\
0.834866044148075	8.199\\
0.833067101214662	8.208\\
0.831268158281248	8.217\\
0.829471359499126	8.226\\
0.827672416565713	8.235\\
0.825873473632299	8.244\\
0.824074530698886	8.253\\
0.822275587765473	8.262\\
0.820476644832059	8.271\\
0.818679846049937	8.28\\
0.816880903116524	8.289\\
0.815081960183111	8.298\\
0.813283017249697	8.307\\
0.811484074316284	8.316\\
0.80968513138287	8.325\\
0.807888332600748	8.334\\
0.806089389667335	8.343\\
0.804290446733921	8.352\\
0.802444332472099	8.361\\
0.80026158645754	8.37\\
0.798078840442982	8.379\\
0.795893950277132	8.388\\
0.793711204262573	8.397\\
0.791528458248014	8.406\\
0.789345712233455	8.415\\
0.787160822067605	8.424\\
0.784978076053046	8.433\\
0.782795330038488	8.442\\
0.780612584023929	8.451\\
0.778427693858079	8.46\\
0.77624494784352	8.469\\
0.774062201828961	8.478\\
0.771879455814402	8.487\\
0.769662403379182	8.496\\
0.766995079172787	8.505\\
0.76432775496639	8.514\\
0.761660430759994	8.523\\
0.758993106553598	8.532\\
0.756325782347202	8.541\\
0.753660602292098	8.55\\
0.750993278085702	8.559\\
0.748325953879306	8.568\\
0.74565862967291	8.577\\
0.742991305466514	8.586\\
0.740326125411409	8.595\\
0.737641647994683	8.604\\
0.734358952367679	8.613\\
0.731078400891967	8.622\\
0.727797849416255	8.631\\
0.724515153789251	8.64\\
0.721234602313539	8.649\\
0.717954050837827	8.658\\
0.714673499362115	8.667\\
0.711390803735111	8.676\\
0.708110252259399	8.685\\
0.704516654695155	8.694\\
0.700447055544239	8.703\\
0.696375312242032	8.712\\
0.692305713091116	8.721\\
0.688233969788908	8.73\\
0.684164370637992	8.739\\
0.680092627335785	8.748\\
0.676023028184869	8.757\\
0.671241570805236	8.766\\
0.666136346580615	8.775\\
0.661033266507285	8.784\\
0.655930186433955	8.793\\
0.650827106360625	8.802\\
0.645721882136004	8.811\\
0.639889790623626	8.82\\
0.633393012210942	8.829\\
0.626896233798257	8.838\\
0.620399455385572	8.847\\
0.613902676972887	8.856\\
0.606250201014184	8.865\\
0.597797956623819	8.874\\
0.589343568082164	8.883\\
0.5808913236918	8.892\\
0.570567235224117	8.901\\
0.559211809985313	8.91\\
0.547858528897799	8.919\\
0.534232447441491	8.928\\
0.518234934656989	8.937\\
0.502237421872487	8.946\\
0.47825937798171	8.955\\
0.453979008758858	8.964\\
0.413321611972941	8.973\\
0.355026426664665	8.982\\
0.263136678924065	8.991\\
0	9\\
};
\addlegendentry{Upwind}

\end{axis}
\end{tikzpicture}%}\\
	\subfloat[Velocity profiles at $x/H = 10$]{\input{img/bfs_fin_vel4}}
	\subfloat[Friction coefficient at the bottom wall]{% This file was created by matlab2tikz.
%
\definecolor{mycolor4}{rgb}{0.00000,0.44700,0.74100}%
\definecolor{mycolor3}{rgb}{0.85000,0.32500,0.09800}%
\definecolor{mycolor2}{rgb}{0.92900,0.69400,0.12500}%
\definecolor{mycolor1}{rgb}{0.49400,0.18400,0.55600}%
%
\begin{tikzpicture}

\begin{axis}[%
width=0.951\bfswidth,
height=0.75\bfsheight,
at={(0\bfswidth,0\bfsheight)},
scale only axis,
xmin=0,
xmax=30,
xlabel style={font=\color{white!15!black}},
xlabel={$x/H$},
ymin=-0.0015,
ymax=0.003,
ylabel style={font=\color{white!15!black}, rotate=-90},
ylabel={$C_f$},
axis background/.style={fill=white},
legend style={at={(0.97,0.03)}, anchor=south east, legend cell align=left, align=left, draw=white!15!black, font=\scriptsize}
]
\addplot [color=mycolor1]
  table[row sep=crcr]{%
-126.659821	0\\
-123.869415	0\\
-121.536697	0\\
-119.584694	0\\
-117.948982	0\\
-116.575592	0\\
-115.419197	0\\
-114.441628	0\\
-113.610641	0\\
-112.898804	0\\
-112.282623	0\\
-111.741768	0\\
-111.258392	0\\
-110.816551	0\\
-110.401649	0\\
-110	0.00812491402\\
-109.598709	0.0107412264\\
-109.192223	0.00526966201\\
-108.78054	0.00532762799\\
-108.363647	0.00526078884\\
-107.941513	0.00510590943\\
-107.514137	0.00494101224\\
-107.081512	0.00479498133\\
-106.643623	0.00467133615\\
-106.20047	0.00456604315\\
-105.752045	0.00447485177\\
-105.298347	0.00439458713\\
-104.839378	0.00432301546\\
-104.375137	0.00425852556\\
-103.90564	0.00419991836\\
-103.430878	0.00414627325\\
-102.950867	0.00409686565\\
-102.465614	0.00405111257\\
-101.975151	0.00400854275\\
-101.479469	0.00396876922\\
-100.978607	0.00393147068\\
-100.47258	0.00389637472\\
-99.9614029	0.00386324921\\
-99.4451218	0.00383189856\\
-98.9237442	0.00380215677\\
-98.3973236	0.00377388159\\
-97.8658752	0.0037469489\\
-97.3294449	0.00372125208\\
-96.7880783	0.00369669567\\
-96.2418137	0.00367319607\\
-95.6906891	0.00365067925\\
-95.1347656	0.00362907816\\
-94.5740891	0.00360833248\\
-94.0087128	0.00358838751\\
-93.4386978	0.00356919412\\
-92.8640976	0.00355070736\\
-92.2849731	0.00353288511\\
-91.7014008	0.00351568731\\
-91.1134415	0.00349907787\\
-90.5211716	0.00348302512\\
-89.9246597	0.00346749928\\
-89.3239822	0.00345247402\\
-88.719223	0.0034379235\\
-88.110466	0.00342382537\\
-87.4977875	0.00341015938\\
-86.881279	0.00339690736\\
-86.2610397	0.00338405324\\
-85.637146	0.00337158027\\
-85.0097122	0.00335947261\\
-84.3788223	0.00334771629\\
-83.7445831	0.00333629688\\
-83.1070938	0.00332519948\\
-82.4664688	0.00331441243\\
-81.8228073	0.00330392155\\
-81.1762238	0.00329371286\\
-80.526825	0.00328377704\\
-79.8747253	0.00327410363\\
-79.2200546	0.00326468237\\
-78.5629196	0.00325550698\\
-77.9034424	0.00324657047\\
-77.241745	0.003237864\\
-76.5779495	0.00322938082\\
-75.9121857	0.0032211158\\
-75.2445755	0.00321306172\\
-74.5752563	0.00320521113\\
-73.9043503	0.00319755822\\
-73.2319946	0.00319009693\\
-72.5583191	0.00318281981\\
-71.8834457	0.00317571987\\
-71.2075272	0.00316879246\\
-70.5306931	0.00316203269\\
-69.8530731	0.00315543427\\
-69.1748047	0.00314899208\\
-68.4960327	0.00314270402\\
-67.8168869	0.00313656661\\
-67.1375122	0.00313057564\\
-66.4580383	0.0031247267\\
-65.7786026	0.00311901676\\
-65.0993576	0.00311344303\\
-64.4204254	0.00310800201\\
-63.741951	0.00310268928\\
-63.0640717	0.00309750182\\
-62.3869247	0.00309243705\\
-61.71064	0.0030874915\\
-61.0353584	0.00308266166\\
-60.3612175	0.00307794265\\
-59.688343	0.0030733312\\
-59.0168724	0.00306882523\\
-58.3469391	0.00306442217\\
-57.678669	0.00306011969\\
-57.0121956	0.00305591547\\
-56.3476448	0.00305180787\\
-55.6851387	0.00304779573\\
-55.0248108	0.00304387836\\
-54.3667717	0.00304005388\\
-53.7111549	0.00303631951\\
-53.0580711	0.00303267338\\
-52.4076424	0.00302911317\\
-51.7599831	0.00302563724\\
-51.1152	0.0030222442\\
-50.4734154	0.00301893149\\
-49.8347282	0.00301569654\\
-49.1992493	0.00301253749\\
-48.5670853	0.00300945179\\
-47.9383316	0.00300643826\\
-47.3130913	0.00300349738\\
-46.6914597	0.00300062844\\
-46.0735283	0.00299783028\\
-45.4593964	0.00299510267\\
-44.849144	0.00299244514\\
-44.2428665	0.00298985722\\
-43.6406364	0.002987338\\
-43.0425415	0.00298488676\\
-42.448658	0.00298250164\\
-41.8590622	0.00298017892\\
-41.2738228	0.0029779172\\
-40.6930161	0.00297571463\\
-40.1166992	0.00297357049\\
-39.5449448	0.00297148456\\
-38.9778061	0.00296945591\\
-38.4153481	0.00296748523\\
-37.8576202	0.00296557252\\
-37.3046799	0.00296371733\\
-36.7565765	0.00296191871\\
-36.2133522	0.00296017667\\
-35.6750526	0.00295849098\\
-35.1417198	0.0029568607\\
-34.6133919	0.0029552842\\
-34.090107	0.00295375939\\
-33.5718956	0.00295228465\\
-33.0587883	0.00295085786\\
-32.5508118	0.00294947741\\
-32.0479965	0.00294814096\\
-31.5503578	0.00294684642\\
-31.0579185	0.00294559239\\
-30.5706959	0.00294437679\\
-30.088707	0.00294319773\\
-29.6119614	0.0029420529\\
-29.1404705	0.00294093997\\
-28.674242	0.00293985684\\
-28.2132816	0.00293880119\\
-27.7575932	0.00293777138\\
-27.3071766	0.00293676602\\
-26.8620319	0.0029357837\\
-26.4221554	0.00293482305\\
-25.9875412	0.00293388264\\
-25.5581837	0.0029329625\\
-25.1340733	0.00293206167\\
-24.7151985	0.00293117925\\
-24.3015461	0.00293031475\\
-23.8931046	0.00292946748\\
-23.4898529	0.0029286372\\
-23.0917759	0.00292782346\\
-22.6988544	0.00292702671\\
-22.3110657	0.00292624743\\
-21.9283867	0.00292548537\\
-21.5507946	0.00292474148\\
-21.1782646	0.00292401621\\
-20.8107681	0.0029233091\\
-20.4482784	0.00292261993\\
-20.090765	0.00292194868\\
-19.7381973	0.00292129512\\
-19.390543	0.00292065926\\
-19.0477715	0.00292004109\\
-18.7098465	0.00291944086\\
-18.3767338	0.00291885855\\
-18.0483971	0.00291829486\\
-17.724802	0.00291774957\\
-17.4059086	0.00291722314\\
-17.0916805	0.00291671581\\
-16.7820759	0.00291622779\\
-16.4770565	0.00291575934\\
-16.1765823	0.00291531067\\
-15.8806124	0.00291488227\\
-15.5891027	0.00291447458\\
-15.3020134	0.00291408808\\
-15.0193014	0.00291372347\\
-14.7409229	0.00291338144\\
-14.4668341	0.00291306223\\
-14.196991	0.0029127663\\
-13.9313507	0.00291249412\\
-13.6698666	0.00291224592\\
-13.4124956	0.00291202264\\
-13.1591911	0.0029118245\\
-12.9099083	0.00291165197\\
-12.6646013	0.00291150599\\
-12.4232244	0.00291138724\\
-12.1857328	0.00291129644\\
-11.9520798	0.00291123404\\
-11.7222185	0.00291120075\\
-11.4961042	0.00291119749\\
-11.2736912	0.00291122473\\
-11.0549326	0.00291128294\\
-10.8397818	0.00291137281\\
-10.6281948	0.00291149481\\
-10.4201241	0.00291165011\\
-10.2155237	0.00291183894\\
-10.0143499	0.00291206199\\
-9.81655598	0.00291231996\\
-9.62209702	0.00291261333\\
-9.43092823	0.00291294302\\
-9.24300289	0.00291330973\\
-9.05827904	0.00291371392\\
-8.87670994	0.00291415653\\
-8.69825268	0.00291463826\\
-8.52286243	0.0029151598\\
-8.35049629	0.00291572185\\
-8.18111134	0.00291632535\\
-8.01466274	0.00291697076\\
-7.85110855	0.00291765877\\
-7.6904068	0.00291839032\\
-7.53251457	0.00291916542\\
-7.37739086	0.00291998521\\
-7.22499323	0.00292085018\\
-7.07528162	0.00292176055\\
-6.92821455	0.00292271702\\
-6.78375196	0.00292372005\\
-6.64185381	0.00292477012\\
-6.50248051	0.00292586791\\
-6.36559296	0.00292701321\\
-6.23115158	0.00292820693\\
-6.09911919	0.00292944931\\
-5.96945667	0.00293074059\\
-5.84212732	0.00293208123\\
-5.71709299	0.00293347123\\
-5.59431791	0.00293491129\\
-5.4737649	0.00293640164\\
-5.35539818	0.00293794251\\
-5.23918247	0.00293953391\\
-5.12508249	0.0029411763\\
-5.01306343	0.00294286991\\
-4.90309095	0.00294461497\\
-4.79513168	0.00294641219\\
-4.68915176	0.00294826343\\
-4.58511829	0.00295017194\\
-4.48299885	0.00295213982\\
-4.382761	0.002954175\\
-4.28437376	0.00295626209\\
-4.1878047	0.00295845093\\
-4.09302378	0.00296068029\\
-3.9079206	0.00296474434\\
-3.8178525	0.00296697603\\
-3.72974896	0.0029693353\\
-3.64356518	0.00297174184\\
-3.55925632	0.00297418959\\
-3.47677946	0.00297667691\\
-3.3960917	0.0029792022\\
-3.31715178	0.00298176566\\
-3.23991895	0.00298436778\\
-3.16435361	0.00298700784\\
-3.09041691	0.00298968609\\
-3.01807046	0.0029924023\\
-2.94727731	0.00299515552\\
-2.87800097	0.00299794599\\
-2.8102057	0.00300077302\\
-2.74385667	0.00300363614\\
-2.67891979	0.00300653558\\
-2.61536169	0.00300947018\\
-2.55314946	0.0030124404\\
-2.4922514	0.00301544531\\
-2.43263578	0.00301848515\\
-2.37427211	0.00302155921\\
-2.31713033	0.00302466773\\
-2.26118088	0.00302781072\\
-2.20639539	0.00303098792\\
-2.15274525	0.00303419912\\
-2.10020328	0.00303744478\\
-2.04874182	0.00304072467\\
-1.99833488	0.00304403901\\
-1.94895637	0.00304738805\\
-1.90058088	0.00305077201\\
-1.85318351	0.00305419113\\
-1.80673993	0.0030576461\\
-1.76122606	0.00306113739\\
-1.71661866	0.00306466524\\
-1.67289472	0.00306823081\\
-1.6300317	0.00307183457\\
-1.58800757	0.00307547743\\
-1.54680061	0.00307916035\\
-1.50638986	0.00308288448\\
-1.4667542	0.00308665098\\
-1.42787349	0.00309046102\\
-1.38972759	0.00309431623\\
-1.35229695	0.00309821824\\
-1.31556225	0.00310216891\\
-1.27950454	0.00310616987\\
-1.24410546	0.00311022368\\
-1.20934653	0.00311433244\\
-1.17521	0.00311849872\\
-1.14167821	0.00312272529\\
-1.10873413	0.00312701496\\
-1.07636058	0.00313137146\\
-1.044541	0.00313579803\\
-1.01325893	0.00314029888\\
-0.982498348	0.0031448782\\
-0.952243388	0.00314954109\\
-0.922478497	0.00315429224\\
-0.893188357	0.00315913791\\
-0.864357948	0.00316408416\\
-0.835972309	0.00316913798\\
-0.808016896	0.00317430706\\
-0.780477345	0.00317959976\\
-0.753339469	0.00318502542\\
-0.726589322	0.00319059473\\
-0.700213075	0.0031963191\\
-0.674197137	0.00320221158\\
-0.648528218	0.00320828636\\
-0.623193085	0.00321456022\\
-0.598178685	0.0032210513\\
-0.573472083	0.00322778057\\
-0.549060643	0.00323477224\\
-0.524931729	0.00324205426\\
-0.501073003	0.00324965874\\
-0.477472156	0.00325762248\\
-0.45411703	0.00326598808\\
-0.430995613	0.00327480235\\
-0.408095986	0.00328411651\\
-0.385406405	0.0032939876\\
-0.362915158	0.0033044836\\
-0.340610713	0.00331569812\\
-0.318481535	0.0033277825\\
-0.296516269	0.00334099052\\
-0.274703592	0.00335572776\\
-0.253032297	0.00337256398\\
-0.231491223	0.00339211733\\
-0.210069284	0.00341471471\\
-0.188755453	0.00343966973\\
-0.167538762	0.00346435932\\
-0.146408305	0.0034830512\\
-0.125353187	0.00348763168\\
-0.104362577	0.00346867344\\
-0.0834256858	0.00342689571\\
-0.0625317246	0.00336531037\\
-0.0416699462	0.00340168481\\
-0.0208296124	0.00578542007\\
0.020833334	-4.52145832e-06\\
0.0416666679	-3.42311e-06\\
0.0625	-2.00077451e-07\\
0.0833333358	3.064677e-06\\
0.104166664	6.23515825e-06\\
0.125	9.51022503e-06\\
0.145833328	1.28798602e-05\\
0.166666672	1.63185905e-05\\
0.1875	1.97527006e-05\\
0.208333328	2.31022077e-05\\
0.229166672	2.62824415e-05\\
0.25	2.92147524e-05\\
0.270833343	3.183015e-05\\
0.291666657	3.40725564e-05\\
0.3125	3.59014593e-05\\
0.333333343	3.72919931e-05\\
0.354166657	3.82353537e-05\\
0.375	3.87377113e-05\\
0.395833343	3.88185836e-05\\
0.416666657	3.85091626e-05\\
0.4375	3.78495461e-05\\
0.458333343	3.68872243e-05\\
0.479166657	3.56738892e-05\\
0.5	3.42642124e-05\\
0.520833313	3.27134367e-05\\
0.541666687	3.10765463e-05\\
0.5625	2.93949488e-05\\
0.583333313	2.7688493e-05\\
0.604166687	2.59886419e-05\\
0.625	2.43557133e-05\\
0.645833313	2.28589433e-05\\
0.666666687	2.15681812e-05\\
0.6875	2.05509914e-05\\
0.708333313	1.98652106e-05\\
0.729166687	1.95579523e-05\\
0.75	1.96612782e-05\\
0.770833313	2.01880248e-05\\
0.791666687	2.11330134e-05\\
0.8125	2.24741361e-05\\
0.833333313	2.41739781e-05\\
0.854166687	2.61832392e-05\\
0.875	2.84460075e-05\\
0.895833313	3.09032985e-05\\
0.916666687	3.34949291e-05\\
0.9375	3.61627681e-05\\
0.958333313	3.88550434e-05\\
0.979166687	4.15276518e-05\\
1	4.41418997e-05\\
1.02083337	4.66661986e-05\\
1.04166663	4.90807761e-05\\
1.0625	5.13741543e-05\\
1.08333337	5.35380932e-05\\
1.10416663	5.55709266e-05\\
1.125	5.74828264e-05\\
1.14583337	5.92902179e-05\\
1.16666663	6.09998388e-05\\
1.1875	6.26017063e-05\\
1.20833337	6.4070824e-05\\
1.22916663	6.53640236e-05\\
1.25	6.64146573e-05\\
1.27083337	6.71293965e-05\\
1.29166663	6.7386005e-05\\
1.3125	6.70321242e-05\\
1.33333337	6.58876452e-05\\
1.35416663	6.37520934e-05\\
1.375	6.04171473e-05\\
1.39583337	5.56824234e-05\\
1.41666663	4.9373255e-05\\
1.4375	4.13576272e-05\\
1.45833337	3.15600082e-05\\
1.47916663	1.99701608e-05\\
1.5	6.6458897e-06\\
1.52083337	-8.28979046e-06\\
1.54166663	-2.46563213e-05\\
1.5625	-4.22275407e-05\\
1.58333337	-6.07469265e-05\\
1.60416663	-7.99442714e-05\\
1.625	-9.95522205e-05\\
1.64583337	-0.000119321041\\
1.66666663	-0.000139030541\\
1.6875	-0.000158498267\\
1.70833337	-0.000177583803\\
1.72916663	-0.000196188877\\
1.75	-0.000214254105\\
1.77083337	-0.000231753453\\
1.79166663	-0.000248687546\\
1.8125	-0.000265076727\\
1.83333337	-0.000280954264\\
1.85416663	-0.000296360726\\
1.875	-0.000311339449\\
1.89583337	-0.000325933564\\
1.91666663	-0.000340183906\\
1.9375	-0.000354128017\\
1.95833337	-0.000367799483\\
1.97916663	-0.000381228165\\
2	-0.000394440343\\
2.02083325	-0.000407459156\\
2.04166675	-0.000420304859\\
2.0625	-0.000432995177\\
2.08333325	-0.000445545651\\
2.10416675	-0.000457969873\\
2.125	-0.000470279716\\
2.14583325	-0.000482485397\\
2.16666675	-0.000494595675\\
2.1875	-0.00050661806\\
2.20833325	-0.000518559013\\
2.22916675	-0.000530423771\\
2.25	-0.000542216934\\
2.27083325	-0.000553941994\\
2.29166675	-0.000565602095\\
2.3125	-0.000577199622\\
2.33333325	-0.000588736671\\
2.35416675	-0.00060021464\\
2.375	-0.000611634809\\
2.39583325	-0.000622998108\\
2.41666675	-0.000634305296\\
2.4375	-0.000645556778\\
2.45833325	-0.000656752964\\
2.47916675	-0.00066789391\\
2.5	-0.000678979675\\
2.52083325	-0.000690010027\\
2.54166675	-0.000700984732\\
2.5625	-0.000711903209\\
2.58333325	-0.000722764933\\
2.60416675	-0.000733568973\\
2.625	-0.00074431434\\
2.64583325	-0.00075499987\\
2.66666675	-0.000765624165\\
2.6875	-0.000776185654\\
2.70833325	-0.000786682649\\
2.72916675	-0.000797113054\\
2.75	-0.000807474891\\
2.77083325	-0.000817765831\\
2.79166675	-0.000827983546\\
2.8125	-0.000838125416\\
2.83333325	-0.000848188938\\
2.85416675	-0.000858171319\\
2.875	-0.000868069939\\
2.89583325	-0.000877882005\\
2.91666675	-0.00088760478\\
2.9375	-0.000897235645\\
2.95833325	-0.000906771864\\
2.97916675	-0.000916210993\\
3	-0.000925550528\\
3.02083325	-0.000934788026\\
3.04166675	-0.000943921274\\
3.0625	-0.000952948001\\
3.08333325	-0.000961866172\\
3.10416675	-0.000970673631\\
3.125	-0.000979368459\\
3.14583325	-0.000987948733\\
3.16666675	-0.000996412593\\
3.1875	-0.00100475841\\
3.20833325	-0.0010129842\\
3.22916675	-0.00102108845\\
3.25	-0.0010290693\\
3.27083325	-0.00103692524\\
3.29166675	-0.00104465452\\
3.3125	-0.00105225539\\
3.33333325	-0.00105972611\\
3.35416675	-0.00106706505\\
3.375	-0.00107427035\\
3.39583325	-0.00108134025\\
3.41666675	-0.00108827266\\
3.4375	-0.00109506585\\
3.45833325	-0.00110171793\\
3.47916675	-0.00110822672\\
3.5	-0.00111459021\\
3.52083325	-0.00112080644\\
3.54166675	-0.00112687331\\
3.5625	-0.00113278872\\
3.58333325	-0.00113855058\\
3.60416675	-0.00114415679\\
3.625	-0.00114960549\\
3.64583325	-0.00115489447\\
3.66666675	-0.00116002187\\
3.6875	-0.0011649857\\
3.70833325	-0.00116978423\\
3.72916675	-0.00117441558\\
3.75	-0.00117887824\\
3.77083325	-0.00118317036\\
3.79166675	-0.00118729053\\
3.8125	-0.00119123724\\
3.83333325	-0.00119500922\\
3.85416675	-0.00119860494\\
3.875	-0.00120202336\\
3.89583325	-0.00120526319\\
3.91666675	-0.00120832329\\
3.9375	-0.0012112027\\
3.95833325	-0.0012139004\\
3.97916675	-0.0012164152\\
4	-0.0012187463\\
4.02083349	-0.00122089253\\
4.04166651	-0.0012228532\\
4.0625	-0.00122462725\\
4.08333349	-0.00122621353\\
4.10416651	-0.00122761144\\
4.125	-0.00122881972\\
4.14583349	-0.00122983754\\
4.16666651	-0.00123066397\\
4.1875	-0.00123129797\\
4.20833349	-0.0012317386\\
4.22916651	-0.00123198493\\
4.25	-0.00123203604\\
4.27083349	-0.00123189087\\
4.29166651	-0.00123154873\\
4.3125	-0.00123100856\\
4.33333349	-0.00123026955\\
4.35416651	-0.00122933101\\
4.375	-0.00122819201\\
4.39583349	-0.00122685195\\
4.41666651	-0.00122531026\\
4.4375	-0.00122356624\\
4.45833349	-0.00122161943\\
4.47916651	-0.00121946924\\
4.5	-0.00121711555\\
4.52083349	-0.00121455779\\
4.54166651	-0.00121179584\\
4.5625	-0.00120882946\\
4.58333349	-0.00120565866\\
4.60416651	-0.00120228331\\
4.625	-0.00119870342\\
4.64583349	-0.00119491923\\
4.66666651	-0.00119093072\\
4.6875	-0.00118673826\\
4.70833349	-0.00118234218\\
4.72916651	-0.00117774261\\
4.75	-0.00117294013\\
4.77083349	-0.0011679352\\
4.79166651	-0.00116272818\\
4.8125	-0.00115731987\\
4.83333349	-0.00115171087\\
4.85416651	-0.00114590186\\
4.875	-0.00113989355\\
4.89583349	-0.00113368686\\
4.91666651	-0.00112728262\\
4.9375	-0.00112068199\\
4.95833349	-0.0011138859\\
4.97916651	-0.00110689562\\
5	-0.00109971222\\
5.02083349	-0.00109233696\\
5.04166651	-0.00108477136\\
5.0625	-0.0010770167\\
5.08333349	-0.00106907473\\
5.10416651	-0.00106094684\\
5.125	-0.00105263491\\
5.14583349	-0.00104414055\\
5.16666651	-0.00103546574\\
5.1875	-0.00102661236\\
5.20833349	-0.00101758237\\
5.22916651	-0.00100837788\\
5.25	-0.000999001088\\
5.27083349	-0.000989454216\\
5.29166651	-0.000979739358\\
5.3125	-0.000969859073\\
5.33333349	-0.000959815574\\
5.35416651	-0.000949611305\\
5.375	-0.000939248886\\
5.39583349	-0.000928730704\\
5.41666651	-0.000918059377\\
5.4375	-0.000907237467\\
5.45833349	-0.000896267651\\
5.47916651	-0.000885152607\\
5.5	-0.000873894955\\
5.52083349	-0.000862497487\\
5.54166651	-0.000850962882\\
5.5625	-0.000839293993\\
5.58333349	-0.000827493554\\
5.60416651	-0.000815564359\\
5.625	-0.000803509203\\
5.64583349	-0.000791331055\\
5.66666651	-0.000779032591\\
5.6875	-0.000766616839\\
5.70833349	-0.00075408665\\
5.72916651	-0.000741444877\\
5.75	-0.000728694431\\
5.77083349	-0.00071583828\\
5.79166651	-0.000702879392\\
5.8125	-0.000689820619\\
5.83333349	-0.000676664989\\
5.85416651	-0.000663415354\\
5.875	-0.000650074857\\
5.89583349	-0.000636646291\\
5.91666651	-0.000623132684\\
5.9375	-0.000609537063\\
5.95833349	-0.000595862279\\
5.97916651	-0.000582111359\\
6	-0.000568287272\\
6.02083349	-0.000554392929\\
6.04166651	-0.000540431298\\
6.0625	-0.000526405289\\
6.08333349	-0.000512317813\\
6.10416651	-0.000498171838\\
6.125	-0.000483970158\\
6.14583349	-0.000469715713\\
6.16666651	-0.000455411355\\
6.1875	-0.000441059936\\
6.20833349	-0.000426664308\\
6.22916651	-0.000412227295\\
6.25	-0.00039775169\\
6.27083349	-0.000383240316\\
6.29166651	-0.00036869591\\
6.3125	-0.000354121235\\
6.33333349	-0.000339519058\\
6.35416651	-0.000324892113\\
6.375	-0.000310243078\\
6.39583349	-0.00029557466\\
6.41666651	-0.000280889537\\
6.4375	-0.000266190385\\
6.45833349	-0.000251479767\\
6.47916651	-0.000236760374\\
6.5	-0.000222034723\\
6.52083349	-0.00020730542\\
6.54166651	-0.000192574982\\
6.5625	-0.000177845926\\
6.58333349	-0.000163120698\\
6.60416651	-0.000148401756\\
6.625	-0.00013369153\\
6.64583349	-0.000118992386\\
6.66666651	-0.000104306666\\
6.6875	-8.9636691e-05\\
6.70833349	-7.49847459e-05\\
6.72916651	-6.03530607e-05\\
6.75	-4.57438473e-05\\
6.77083349	-3.11592739e-05\\
6.79166651	-1.66014652e-05\\
6.8125	-2.07251423e-06\\
6.83333349	1.24255303e-05\\
6.85416651	2.68906606e-05\\
6.875	4.13209091e-05\\
6.89583349	5.57143467e-05\\
6.91666651	7.00690798e-05\\
6.9375	8.4383275e-05\\
6.95833349	9.86551095e-05\\
6.97916651	0.00011288283\\
7	0.000127064704\\
7.02083349	0.000141199023\\
7.04166651	0.000155284171\\
7.0625	0.000169318519\\
7.08333349	0.000183300508\\
7.10416651	0.000197228583\\
7.125	0.000211101244\\
7.14583349	0.00022491705\\
7.16666651	0.000238674562\\
7.1875	0.000252372382\\
7.20833349	0.000266009156\\
7.22916651	0.000279583532\\
7.25	0.000293094228\\
7.27083349	0.000306539936\\
7.29166651	0.000319919433\\
7.3125	0.000333231408\\
7.33333349	0.00034647467\\
7.35416651	0.000359647966\\
7.375	0.000372750073\\
7.39583349	0.000385779713\\
7.41666651	0.000398735632\\
7.4375	0.000411616522\\
7.45833349	0.000424421014\\
7.47916651	0.000437147653\\
7.5	0.000449794927\\
7.52083349	0.000462361146\\
7.54166651	0.000474844506\\
7.5625	0.000487243\\
7.58333349	0.000499554328\\
7.60416651	0.000511775957\\
7.625	0.000523904979\\
7.64583349	0.000535938016\\
7.66666651	0.000547871226\\
7.6875	0.000559699663\\
7.70833349	0.000571417506\\
7.72916651	0.000583016896\\
7.75	0.000594487821\\
7.77083349	0.00060581672\\
7.79166651	0.000616987527\\
7.8125	0.000627978356\\
7.83333349	0.000638775527\\
7.85416651	0.000649344816\\
7.875	0.000659732206\\
7.89583349	0.000669804227\\
7.91666651	0.000679981546\\
7.9375	0.000689163571\\
7.95833349	0.000700129895\\
7.97916651	0.000703911763\\
8.02083778	0.000720021199\\
8.04170799	0.000736109505\\
8.06264305	0.000746689155\\
8.08367729	0.000759218005\\
8.10484314	0.00077105721\\
8.12617397	0.000783165509\\
8.14770222	0.000795124448\\
8.16946316	0.000807045843\\
8.19149113	0.000818875851\\
8.2138195	0.000830641133\\
8.23648357	0.000842353154\\
8.25951958	0.000854035665\\
8.28296375	0.000865710492\\
8.30685234	0.000877400045\\
8.33122349	0.000889124814\\
8.35611534	0.000900904008\\
8.38156605	0.000912755087\\
8.40761757	0.000924694061\\
8.43430901	0.000936735712\\
8.46168327	0.000948893605\\
8.48978329	0.000961180194\\
8.51865292	0.000973607064\\
8.54833794	0.000986184808\\
8.57888508	0.000998923322\\
8.61034203	0.00101183145\\
8.64275837	0.00102491723\\
8.67618465	0.00103818811\\
8.71067333	0.00105165015\\
8.74627972	0.00106530928\\
8.78305912	0.0010791698\\
8.82106876	0.00109323568\\
8.86036873	0.0011075096\\
8.90102196	0.00112199329\\
8.94309044	0.00113668793\\
8.98664188	0.00115159305\\
9.031744	0.00116670749\\
9.07846737	0.00118202891\\
9.12688637	0.00119755394\\
9.17707634	0.00121327781\\
9.22911549	0.00122919504\\
9.28308773	0.00124529866\\
9.33907604	0.00126158085\\
9.39716911	0.00127803243\\
9.4574585	0.0012946428\\
9.5200386	0.00131140009\\
9.58500767	0.00132829195\\
9.65246773	0.00134530477\\
9.7225256	0.00136242353\\
9.79529095	0.00137963158\\
9.87087727	0.00139691227\\
9.94940472	0.00141424872\\
10.0309963	0.00143162115\\
10.1157799	0.00144901045\\
10.2038889	0.00146639836\\
10.2954617	0.00148376392\\
10.3906431	0.00150108722\\
10.4895811	0.0015183494\\
10.592433	0.00153552997\\
10.6993589	0.00155261008\\
10.8105278	0.00156957074\\
10.9261141	0.00158639357\\
11.0462999	0.0016030618\\
11.1712732	0.00161955832\\
11.3012314	0.0016358674\\
11.4363775	0.00165197439\\
11.5769253	0.0016678652\\
11.7230959	0.00168352807\\
11.8751173	0.0016989531\\
12.0332298	0.00171413017\\
12.1976814	0.00172905205\\
12.3687305	0.00174371328\\
12.5466471	0.00175810675\\
12.7317095	0.00177222933\\
12.9242086	0.00178607856\\
13.1244478	0.00179964968\\
13.3327408	0.00181294058\\
13.5494156	0.00182595244\\
13.7748137	0.00183868664\\
14.0092878	0.00185114262\\
14.2532063	0.00186331989\\
14.5069542	0.00187521882\\
14.7709293	0.00188683881\\
15.0455465	0.00189817743\\
15.3312378	0.00190922979\\
15.6284523	0.00191998936\\
15.9376564	0.00193044927\\
16.2593365	0.00194060185\\
16.593998	0.00195043813\\
16.9421673	0.0019599488\\
17.3043919	0.00196912582\\
17.6812401	0.00197796384\\
18.073307	0.00198646192\\
18.481205	0.00199462124\\
18.9055767	0.00200244761\\
19.3470898	0.00200995617\\
19.8064365	0.00201716716\\
20.2843418	0.00202410598\\
20.7815533	0.0020308001\\
21.2988548	0.00203728001\\
21.8370609	0.00204357714\\
22.3970127	0.00204972457\\
22.9795952	0.00205575\\
23.5857201	0.00206167158\\
24.2163429	0.00206749956\\
24.8724537	0.0020732386\\
25.5550842	0.0020788915\\
26.2653065	0.00208445569\\
27.0042381	0.00208992814\\
27.7730389	0.00209530792\\
28.5729198	0.0021005962\\
29.4051342	0.00210579601\\
30.2709942	0.00211091153\\
31.1718578	0.00211594626\\
32.1091423	0.00212090067\\
33.0843201	0.00212577032\\
34.0989227	0.00213054521\\
35.1545486	0.00213520718\\
36.2528534	0.00213972898\\
37.395565	0.00214405521\\
38.5844803	0.00214811601\\
39.8214684	0.00215172442\\
41.1084671	0.00215476262\\
42.4475098	0.00215645414\\
43.8406906	0.00215760223\\
45.2902069	0.0021544348\\
46.798336	0.00215237797\\
48.3674431	0.00214179186\\
};
\addlegendentry{Ref data}

\addplot [color=mycolor2, draw=none, mark=x, mark options={solid, mycolor2}, only marks]
  table[row sep=crcr]{%
-3.956	0.00288\\
-1.804	0.00285\\
-0.804	0.00311\\
0.484	9e-05\\
1.804	-0.00011\\
2.804	-0.00052\\
3.804	-0.00102\\
4.804	-0.00082\\
5.882	-0.00022\\
7.09	0.00045\\
7.978	0.00102\\
9.196	0.00118\\
11.196	0.00154\\
13.196	0.00173\\
16.196	0.00191\\
20.196	0.00202\\
24.136	0.00204\\
28.13	0.00205\\
32.196	0.00212\\
35.994	0.0023\\
};
\addlegendentry{Experiment}

\addplot [color=mycolor3]
  table[row sep=crcr]{%
0	0\\
0.05	-2.73642133427827e-06\\
0.1	8.60199744470485e-07\\
0.15	1.63031249997315e-05\\
0.2	3.14838415922525e-05\\
0.25	4.07366468242058e-05\\
0.3	4.37922492100848e-05\\
0.35	4.26059031041435e-05\\
0.4	3.95874241105936e-05\\
0.45	3.66285027896292e-05\\
0.5	3.52610799930467e-05\\
0.55	3.60409383067227e-05\\
0.6	3.88150438903365e-05\\
0.65	4.42444067871579e-05\\
0.7	5.08983074813096e-05\\
0.75	5.82407269195218e-05\\
0.8	6.54942104610032e-05\\
0.85	7.23842052473773e-05\\
0.9	7.7101279746057e-05\\
0.95	8.11437233884541e-05\\
1	8.05628357746715e-05\\
1.05	7.99819481608889e-05\\
1.1	6.96490925794473e-05\\
1.15	5.8358773965633e-05\\
1.2	3.96306902225167e-05\\
1.25	1.25891566107557e-05\\
1.3	-1.44524304159583e-05\\
1.35	-5.06798403235744e-05\\
1.4	-8.93519992146586e-05\\
1.45	-0.000128024158105743\\
1.5	-0.000167843937261849\\
1.55	-0.000207883518949516\\
1.6	-0.000247922833562418\\
1.65	-0.000285088957853907\\
1.7	-0.000320286741127658\\
1.75	-0.00035548452440141\\
1.8	-0.000390562124030931\\
1.85	-0.000420570644621558\\
1.9	-0.000450579165212185\\
1.95	-0.000480587685802812\\
2	-0.000510254350694294\\
2.05	-0.000536708106163337\\
2.1	-0.000563161861632379\\
2.15	-0.000589612946353772\\
2.2	-0.000616066701822814\\
2.25	-0.000640690995151898\\
2.3	-0.000664789151194015\\
2.35	-0.000688884636488483\\
2.4	-0.0007129827925306\\
2.45	-0.000737032875115025\\
2.5	-0.000759234800325944\\
2.55	-0.000781436725536863\\
2.6	-0.000803638650747781\\
2.65	-0.00082584324670635\\
2.7	-0.000848045171917268\\
2.75	-0.000868358878539938\\
2.8	-0.000888194521333333\\
2.85	-0.000908030164126728\\
2.9	-0.000927865806920123\\
2.95	-0.000947701449713518\\
3	-0.000967371506152639\\
3.05	-0.000983711140272741\\
3.1	-0.00100004810364519\\
3.15	-0.00101638773776529\\
3.2	-0.00103272737188539\\
3.25	-0.0010490670060055\\
3.3	-0.0010654066401256\\
3.35	-0.0010779324466021\\
3.4	-0.00108899735411429\\
3.45	-0.00110006226162648\\
3.5	-0.00111112983988632\\
3.55	-0.00112219474739852\\
3.6	-0.00113325965491071\\
3.65	-0.00114432723317054\\
3.7	-0.0011496153135167\\
3.75	-0.00115313802966649\\
3.8	-0.00115666074581628\\
3.85	-0.00116018346196607\\
3.9	-0.00116370884886351\\
3.95	-0.0011672315650133\\
4	-0.00117075428116309\\
4.05	-0.00117324074722485\\
4.1	-0.00116759745744129\\
4.15	-0.00116195416765774\\
4.2	-0.00115631087787419\\
4.25	-0.00115066758809064\\
4.3	-0.00114502429830708\\
4.35	-0.00113938100852353\\
4.4	-0.00113373771873998\\
4.45	-0.00112809442895643\\
4.5	-0.00111578762378718\\
4.55	-0.00110055902068931\\
4.6	-0.00108533041759143\\
4.65	-0.00107010181449355\\
4.7	-0.00105487054064802\\
4.75	-0.00103964193755014\\
4.8	-0.00102441333445227\\
4.85	-0.00100918473135439\\
4.9	-0.000993956128256511\\
4.95	-0.000975049905645168\\
5	-0.000951371056984034\\
5.05	-0.0009276922083229\\
5.1	-0.000904010688914117\\
5.15	-0.000880331840252983\\
5.2	-0.000856652991591849\\
5.25	-0.000832971472183066\\
5.3	-0.000809292623521932\\
5.35	-0.000785613774860798\\
5.4	-0.000761934926199664\\
5.45	-0.000736902008480196\\
5.5	-0.000707790859099826\\
5.55	-0.000678679709719456\\
5.6	-0.000649568560339086\\
5.65	-0.000620457410958716\\
5.7	-0.000591346261578346\\
5.75	-0.000562235112197977\\
5.8	-0.000533123962817607\\
5.85	-0.000504012813437237\\
5.9	-0.000474901664056867\\
5.95	-0.000445790514676497\\
6	-0.000416612596604888\\
6.05	-0.000385987133307209\\
6.1	-0.00035535899926188\\
6.15	-0.000324733535964201\\
6.2	-0.000294105401918873\\
6.25	-0.000263480205695958\\
6.3	-0.000232853407024455\\
6.35	-0.000202227142502481\\
6.4	-0.000171600610905742\\
6.45	-0.000140974346383768\\
6.5	-0.000110347814787029\\
6.55	-7.97210161155254e-05\\
6.6	-4.90947515935515e-05\\
6.65	-1.98156121860233e-05\\
6.7	8.93728310463253e-06\\
6.75	3.76903920551003e-05\\
6.8	6.64436612504271e-05\\
6.85	9.5196396296224e-05\\
6.9	0.000123949665491551\\
6.95	0.000152702667612113\\
7	0.000181455669732674\\
7.05	0.000210208938928001\\
7.1	0.000238961941048563\\
7.15	0.00026771574439342\\
7.2	0.000296469013588747\\
7.25	0.000325222282784073\\
7.3	0.000353273145347562\\
7.35	0.00037767843736938\\
7.4	0.000402086400138848\\
7.45	0.000426494362908316\\
7.5	0.000450902325677784\\
7.55	0.000475310288447252\\
7.6	0.00049971825121672\\
7.65	0.000524126213986188\\
7.7	0.000548534176755656\\
7.75	0.000572942139525124\\
7.8	0.000597350102294591\\
7.85	0.000621758065064059\\
7.9	0.000646166027833527\\
7.95	0.000670573990602995\\
8	0.000694981953372463\\
8.05	0.0007180144811024\\
8.1	0.000736885983994298\\
8.15	0.000755757486886197\\
8.2	0.000774628989778095\\
8.25	0.000793503163417643\\
8.3	0.000812374666309542\\
8.35	0.00083124616920144\\
8.4	0.000850117672093339\\
8.45	0.000868989174985237\\
8.5	0.000887863348624785\\
8.55	0.000906734851516684\\
8.6	0.000925606354408582\\
8.65	0.000944477857300481\\
8.7	0.000963349360192379\\
8.75	0.000982223533831927\\
8.8	0.00100109503672383\\
8.85	0.00101996653961572\\
8.9	0.00103434851570869\\
8.95	0.00104789988928263\\
9	0.00106145126285658\\
9.05	0.00107500263643052\\
9.1	0.00108855401000447\\
9.15	0.00110210538357841\\
9.2	0.00111565675715235\\
9.25	0.0011292081307263\\
9.3	0.00114275950430024\\
9.35	0.00115631087787419\\
9.4	0.00116986225144813\\
9.45	0.00118341362502208\\
9.5	0.00119696499859602\\
9.55	0.00121051637216997\\
9.6	0.00122406774574391\\
9.65	0.00123761911931786\\
9.7	0.0012511704928918\\
9.75	0.00126472186646575\\
9.8	0.00127482530482409\\
9.85	0.00128405807944867\\
9.9	0.0012932935248209\\
9.95	0.00130252629944548\\
10	0.00131176174481771\\
10.05	0.0013209945194423\\
10.1	0.00133022996481453\\
10.15	0.00133946273943911\\
10.2	0.00134869818481134\\
10.25	0.00135793095943592\\
10.3	0.00136716640480815\\
10.35	0.00137639917943273\\
10.4	0.00138563462480496\\
10.45	0.00139486739942955\\
10.5	0.00140410284480178\\
10.55	0.00141333561942636\\
10.6	0.00142257106479859\\
10.65	0.00143180383942317\\
10.7	0.0014410392847954\\
10.75	0.00145027205941998\\
10.8	0.00145654297490119\\
10.85	0.00146265364552341\\
10.9	0.00146876164539799\\
10.95	0.00147487231602022\\
11	0.0014809803158948\\
11.05	0.00148709098651703\\
11.1	0.00149319898639161\\
11.15	0.00149930965701383\\
11.2	0.00150541765688841\\
11.25	0.00151152565676299\\
11.3	0.00151763632738522\\
11.35	0.0015237443272598\\
11.4	0.00152985499788203\\
11.45	0.0015359629977566\\
11.5	0.00154207366837883\\
11.55	0.00154818166825341\\
11.6	0.00155429233887564\\
11.65	0.00156040033875022\\
11.7	0.0015665083386248\\
11.75	0.00157261900924702\\
11.8	0.0015787270091216\\
11.85	0.00158459731245537\\
11.9	0.00158861678766798\\
11.95	0.00159263359213294\\
12	0.0015966503965979\\
12.05	0.00160066720106286\\
12.1	0.00160468400552782\\
12.15	0.00160870080999279\\
12.2	0.00161271761445775\\
12.25	0.00161673708967036\\
12.3	0.00162075389413532\\
12.35	0.00162477069860028\\
12.4	0.00162878750306524\\
12.45	0.0016328043075302\\
12.5	0.00163682111199516\\
12.55	0.00164083791646012\\
12.6	0.00164485739167274\\
12.65	0.0016488741961377\\
12.7	0.00165289100060266\\
12.75	0.00165690780506762\\
12.8	0.00166092460953258\\
12.85	0.00166494141399754\\
12.9	0.0016689582184625\\
12.95	0.00167297769367511\\
13	0.00167699449814007\\
13.05	0.00168088310671786\\
13.1	0.00168355919586273\\
13.15	0.00168623795575525\\
13.2	0.00168891404490013\\
13.25	0.001691590134045\\
13.3	0.00169426889393753\\
13.35	0.0016969449830824\\
13.4	0.00169962374297493\\
13.45	0.0017022998321198\\
13.5	0.00170497859201232\\
13.55	0.0017076546811572\\
13.6	0.00171033077030207\\
13.65	0.0017130095301946\\
13.7	0.00171568561933947\\
13.75	0.00171836437923199\\
13.8	0.00172104046837687\\
13.85	0.00172371655752174\\
13.9	0.00172639531741427\\
13.95	0.00172907140655914\\
14	0.00173175016645166\\
14.05	0.00173442625559654\\
14.1	0.00173710501548906\\
14.15	0.00173978110463394\\
14.2	0.00174245719377881\\
14.25	0.00174513595367134\\
14.3	0.00174781204281621\\
14.35	0.00175049080270873\\
14.4	0.0017525900103613\\
14.45	0.00175441146025831\\
14.5	0.00175623291015532\\
14.55	0.00175805703079998\\
14.6	0.00175987848069699\\
14.65	0.001761699930594\\
14.7	0.00176352138049101\\
14.75	0.00176534550113567\\
14.8	0.00176716695103268\\
14.85	0.00176898840092969\\
14.9	0.00177081252157435\\
14.95	0.00177263397147136\\
15	0.00177445542136837\\
15.05	0.00177627687126538\\
15.1	0.00177810099191004\\
15.15	0.00177992244180705\\
15.2	0.00178174389170406\\
15.25	0.00178356801234872\\
15.3	0.00178538946224573\\
15.35	0.00178721091214274\\
15.4	0.00178903236203975\\
15.45	0.00179085648268441\\
15.5	0.00179267793258142\\
15.55	0.00179449938247843\\
15.6	0.00179632350312309\\
15.65	0.0017981449530201\\
15.7	0.00179996640291712\\
15.75	0.00180178785281413\\
15.8	0.00180361197345879\\
15.85	0.0018050942384043\\
15.9	0.0018063655142855\\
15.95	0.0018076367901667\\
16	0.00180890806604789\\
16.05	0.00181017934192909\\
16.1	0.00181145061781029\\
16.15	0.00181272189369149\\
16.2	0.00181399316957268\\
16.25	0.00181526444545388\\
16.3	0.00181653572133508\\
16.35	0.00181780699721628\\
16.4	0.00181907827309748\\
16.45	0.00182034954897867\\
16.5	0.00182162082485987\\
16.55	0.00182289210074107\\
16.6	0.00182416337662227\\
16.65	0.00182543465250346\\
16.7	0.00182670592838466\\
16.75	0.00182797720426586\\
16.8	0.00182924848014706\\
16.85	0.00183051975602826\\
16.9	0.00183179103190945\\
16.95	0.00183306230779065\\
17	0.00183433358367185\\
17.05	0.00183560485955305\\
17.1	0.00183687613543425\\
17.15	0.00183814741131544\\
17.2	0.00183941868719664\\
17.25	0.00184068996307784\\
17.3	0.00184196123895904\\
17.35	0.00184323251484023\\
17.4	0.00184450379072143\\
17.45	0.00184554538230477\\
17.5	0.00184646411949622\\
17.55	0.00184738018594002\\
17.6	0.00184829625238383\\
17.65	0.00184921231882763\\
17.7	0.00185013105601909\\
17.75	0.00185104712246289\\
17.8	0.0018519631889067\\
17.85	0.00185288192609815\\
17.9	0.00185379799254195\\
17.95	0.00185471405898576\\
18	0.00185563012542956\\
18.05	0.00185654886262102\\
18.1	0.00185746492906482\\
18.15	0.00185838099550862\\
18.2	0.00185929973270008\\
18.25	0.00186021579914388\\
18.3	0.00186113186558769\\
18.35	0.00186204793203149\\
18.4	0.00186296666922295\\
18.45	0.00186388273566675\\
18.5	0.00186479880211055\\
18.55	0.00186571753930201\\
18.6	0.00186663360574581\\
18.65	0.00186754967218962\\
18.7	0.00186846840938107\\
18.75	0.00186938447582487\\
18.8	0.00187030054226868\\
18.85	0.00187121660871248\\
18.9	0.00187213534590394\\
18.95	0.00187305141234774\\
19	0.00187396747879155\\
19.05	0.001874886215983\\
19.1	0.0018758022824268\\
19.15	0.00187671834887061\\
19.2	0.00187752758540843\\
19.25	0.00187821663830202\\
19.3	0.00187890569119561\\
19.35	0.00187959741483685\\
19.4	0.00188028646773044\\
19.45	0.00188097552062403\\
19.5	0.00188166724426527\\
19.55	0.00188235629715886\\
19.6	0.0018830480208001\\
19.65	0.00188373707369369\\
19.7	0.00188442612658728\\
19.75	0.00188511785022852\\
19.8	0.00188580690312211\\
19.85	0.00188649862676335\\
19.9	0.00188718767965694\\
19.95	0.00188787673255053\\
20	0.00188856845619177\\
20.05	0.00188925750908536\\
20.1	0.0018899492327266\\
20.15	0.00189063828562019\\
20.2	0.00189132733851378\\
20.25	0.00189201906215502\\
20.3	0.00189270811504861\\
20.35	0.00189339716794221\\
20.4	0.00189408889158345\\
20.45	0.00189477794447704\\
20.5	0.00189546966811828\\
20.55	0.00189615872101187\\
20.6	0.00189684777390546\\
20.65	0.0018975394975467\\
20.7	0.00189822855044029\\
20.75	0.00189892027408153\\
20.8	0.00189960932697512\\
20.85	0.00190029837986871\\
20.9	0.00190099010350995\\
20.95	0.00190167915640354\\
21	0.00190237088004478\\
21.05	0.00190305993293837\\
21.1	0.00190374898583196\\
21.15	0.00190432052582897\\
21.2	0.00190486535834948\\
21.25	0.00190540752012234\\
21.3	0.00190594968189521\\
21.35	0.00190649184366807\\
21.4	0.00190703400544094\\
21.45	0.0019075761672138\\
21.5	0.00190811832898666\\
21.55	0.00190866049075953\\
21.6	0.00190920265253239\\
21.65	0.00190974481430526\\
21.7	0.00191028697607812\\
21.75	0.00191082913785098\\
21.8	0.00191137129962385\\
21.85	0.00191191346139671\\
21.9	0.00191245562316957\\
21.95	0.00191300045569009\\
22	0.00191354261746295\\
22.05	0.00191408477923581\\
22.1	0.00191462694100868\\
22.15	0.00191516910278154\\
22.2	0.00191571126455441\\
22.25	0.00191625342632727\\
22.3	0.00191679558810013\\
22.35	0.001917337749873\\
22.4	0.00191787991164586\\
22.45	0.00191842207341872\\
22.5	0.00191896423519159\\
22.55	0.00191950639696445\\
22.6	0.00192004855873732\\
22.65	0.00192059072051018\\
22.7	0.00192113555303069\\
22.75	0.00192167771480356\\
22.8	0.00192221987657642\\
22.85	0.00192276203834928\\
22.9	0.00192330420012215\\
22.95	0.00192384636189501\\
23	0.00192438852366788\\
23.05	0.00192493068544074\\
23.1	0.0019254728472136\\
23.15	0.00192601500898647\\
23.2	0.00192655717075933\\
23.25	0.00192707529580335\\
23.3	0.00192751863991318\\
23.35	0.00192795931327536\\
23.4	0.00192840265738519\\
23.45	0.00192884600149502\\
23.5	0.00192928934560485\\
23.55	0.00192973001896703\\
23.6	0.00193017336307686\\
23.65	0.00193061670718669\\
23.7	0.00193105738054887\\
23.75	0.00193150072465869\\
23.8	0.00193194406876852\\
23.85	0.00193238741287835\\
23.9	0.00193282808624053\\
23.95	0.00193327143035036\\
24	0.00193371477446019\\
24.05	0.00193415544782237\\
24.1	0.0019345987919322\\
24.15	0.00193504213604203\\
24.2	0.00193548548015186\\
24.25	0.00193592615351404\\
24.3	0.00193636949762387\\
24.35	0.0019368128417337\\
24.4	0.00193725618584353\\
24.45	0.00193769685920571\\
24.5	0.00193814020331554\\
24.55	0.00193858354742537\\
24.6	0.00193902422078755\\
24.65	0.00193946756489738\\
24.7	0.00193991090900721\\
24.75	0.00194035425311704\\
24.8	0.00194079492647922\\
24.85	0.00194123827058905\\
24.9	0.00194168161469888\\
24.95	0.00194212228806105\\
25	0.00194256563217089\\
25.05	0.00194300897628071\\
25.1	0.00194345232039054\\
25.15	0.00194389299375272\\
25.2	0.00194433633786255\\
25.25	0.00194477968197238\\
25.3	0.00194522302608221\\
25.35	0.00194566369944439\\
25.4	0.00194610704355422\\
25.45	0.00194655038766405\\
25.5	0.00194699106102623\\
25.55	0.00194743440513606\\
25.6	0.00194784570027409\\
25.65	0.00194821693419739\\
25.7	0.00194859083886833\\
25.75	0.00194896207279162\\
25.8	0.00194933330671491\\
25.85	0.00194970721138585\\
25.9	0.00195007844530914\\
25.95	0.00195044967923243\\
26	0.00195082358390337\\
26.05	0.00195119481782666\\
26.1	0.00195156605174995\\
26.15	0.00195193728567324\\
26.2	0.00195231119034418\\
26.25	0.00195268242426748\\
26.3	0.00195305365819077\\
26.35	0.00195342756286171\\
26.4	0.001953798796785\\
26.45	0.00195417003070829\\
26.5	0.00195454393537923\\
26.55	0.00195491516930252\\
26.6	0.00195528640322581\\
26.65	0.00195566030789675\\
26.7	0.00195603154182004\\
26.75	0.00195640277574333\\
26.8	0.00195677400966663\\
26.85	0.00195714791433757\\
26.9	0.00195751914826086\\
26.95	0.00195789038218415\\
27	0.00195826428685509\\
27.05	0.00195863552077838\\
27.1	0.00195900675470167\\
27.15	0.00195938065937261\\
27.2	0.0019597518932959\\
27.25	0.00196012312721919\\
27.3	0.00196049436114248\\
27.35	0.00196086826581342\\
27.4	0.00196123949973672\\
27.45	0.00196161073366001\\
27.5	0.00196198463833095\\
27.55	0.00196235587225424\\
27.6	0.00196272710617753\\
27.65	0.00196310101084847\\
27.7	0.00196347224477176\\
27.75	0.00196384347869505\\
27.8	0.00196421738336599\\
27.85	0.00196458861728928\\
27.9	0.00196495985121257\\
27.95	0.00196533108513586\\
28	0.0019657049898068\\
28.05	0.0019660762237301\\
28.1	0.00196644745765339\\
28.15	0.00196682136232433\\
28.2	0.00196713918129463\\
28.25	0.00196745700026493\\
28.3	0.00196777481923523\\
28.35	0.00196809263820553\\
28.4	0.00196841045717582\\
28.45	0.00196872827614612\\
28.5	0.00196904609511642\\
28.55	0.00196936391408672\\
28.6	0.00196968173305702\\
28.65	0.00196999955202732\\
28.7	0.00197031737099762\\
28.75	0.00197063251922027\\
28.8	0.00197095033819057\\
28.85	0.00197126815716087\\
28.9	0.00197158597613117\\
28.95	0.00197190379510147\\
29	0.00197222161407177\\
29.05	0.00197253943304207\\
29.1	0.00197285725201237\\
29.15	0.00197317507098267\\
29.2	0.00197349288995297\\
29.25	0.00197381070892327\\
29.3	0.00197412585714592\\
29.35	0.00197444367611621\\
29.4	0.00197476149508651\\
29.45	0.00197507931405681\\
29.5	0.00197539713302711\\
29.55	0.00197571495199741\\
29.6	0.00197603277096771\\
29.65	0.00197635058993801\\
29.7	0.00197666840890831\\
29.75	0.00197698622787861\\
29.8	0.00197730404684891\\
29.85	0.00197762186581921\\
29.9	0.00197793701404186\\
29.95	0.00197825483301216\\
30	0.00197857265198246\\
30.05	0.00197889047095276\\
30.1	0.00197920828992306\\
30.15	0.00197952610889336\\
30.2	0.00197984392786366\\
30.25	0.00198016174683396\\
30.3	0.00198047956580425\\
30.35	0.00198079738477455\\
30.4	0.00198111520374485\\
30.45	0.0019814303519675\\
30.5	0.0019817481709378\\
30.55	0.0019820659899081\\
30.6	0.0019823838088784\\
30.65	0.0019827016278487\\
30.7	0.001983019446819\\
30.75	0.0019833372657893\\
30.8	0.0019836550847596\\
30.85	0.0019839729037299\\
30.9	0.0019842907227002\\
30.95	0.0019846085416705\\
31	0.0019849103361549\\
31.05	0.00198518275241516\\
31.1	0.00198545249792776\\
31.15	0.00198572491418802\\
31.2	0.00198599733044828\\
31.25	0.00198626707596088\\
31.3	0.00198653949222114\\
31.35	0.0019868119084814\\
31.4	0.001987081653994\\
31.45	0.00198735407025426\\
31.5	0.00198762648651452\\
31.55	0.00198789623202713\\
31.6	0.00198816864828738\\
31.65	0.00198844106454764\\
31.7	0.00198871081006025\\
31.75	0.0019889832263205\\
31.8	0.00198925564258076\\
31.85	0.00198952538809337\\
31.9	0.00198979780435362\\
31.95	0.00199007022061388\\
32	0.00199034263687414\\
32.05	0.00199061238238674\\
32.1	0.001990884798647\\
32.15	0.00199115721490726\\
32.2	0.00199142696041986\\
32.25	0.00199169937668012\\
32.3	0.00199197179294038\\
32.35	0.00199224153845298\\
32.4	0.00199251395471324\\
32.45	0.0019927863709735\\
32.5	0.0019930561164861\\
32.55	0.00199332853274636\\
32.6	0.00199360094900662\\
32.65	0.00199387069451923\\
32.7	0.00199414311077948\\
32.75	0.00199441552703974\\
32.8	0.00199468527255235\\
32.85	0.0019949576888126\\
32.9	0.00199523010507286\\
32.95	0.00199550252133312\\
33	0.00199577226684572\\
33.05	0.00199604468310598\\
33.1	0.00199631709936624\\
33.15	0.00199658684487884\\
33.2	0.0019968592611391\\
33.25	0.00199713167739936\\
33.3	0.00199740142291196\\
33.35	0.00199767383917222\\
33.4	0.00199794625543248\\
33.45	0.00199821600094508\\
33.5	0.00199848841720534\\
33.55	0.0019987608334656\\
33.6	0.0019990305789782\\
33.65	0.00199930299523846\\
33.7	0.00199957541149872\\
33.75	0.00199984515701133\\
33.8	0.00200011757327158\\
33.85	0.00200038998953184\\
33.9	0.00200065973504445\\
33.95	0.0020009321513047\\
34	0.00200120456756496\\
34.05	0.00200147698382522\\
34.1	0.00200174672933782\\
34.15	0.00200197641363569\\
34.2	0.0020022087686812\\
34.25	0.00200243845297906\\
34.3	0.00200267080802457\\
34.35	0.00200290316307009\\
34.4	0.00200313284736795\\
34.45	0.00200336520241346\\
34.5	0.00200359488671133\\
34.55	0.00200382724175684\\
34.6	0.00200405959680235\\
34.65	0.00200428928110022\\
34.7	0.00200452163614573\\
34.75	0.00200475132044359\\
34.8	0.00200498367548911\\
34.85	0.00200521335978697\\
34.9	0.00200544571483248\\
34.95	0.002005678069878\\
35	0.00200590775417586\\
35.05	0.00200614010922137\\
35.1	0.00200636979351924\\
35.15	0.00200660214856475\\
35.2	0.00200683183286261\\
35.25	0.00200706418790813\\
35.3	0.00200729654295364\\
35.35	0.0020075262272515\\
35.4	0.00200775858229702\\
35.45	0.00200798826659488\\
35.5	0.00200822062164039\\
35.55	0.00200845297668591\\
35.6	0.00200868266098377\\
35.65	0.00200891501602928\\
35.7	0.00200914470032715\\
35.75	0.00200937705537266\\
35.8	0.00200960673967052\\
35.85	0.00200983909471604\\
35.9	0.00201007144976155\\
35.95	0.00201030113405941\\
36	0.00201053348910493\\
36.05	0.00201076317340279\\
36.1	0.0020109955284483\\
36.15	0.00201122521274616\\
36.2	0.00201145756779168\\
36.25	0.00201168992283719\\
36.3	0.00201191960713505\\
36.35	0.00201215196218057\\
36.4	0.00201238164647843\\
36.45	0.00201261400152394\\
36.5	0.00201284635656946\\
36.55	0.00201307604086732\\
36.6	0.00201330839591283\\
36.65	0.0020135380802107\\
36.7	0.00201377043525621\\
36.75	0.00201400011955407\\
36.8	0.00201423247459959\\
36.85	0.0020144648296451\\
36.9	0.00201469451394296\\
36.95	0.00201492686898848\\
37	0.00201515655328634\\
37.05	0.00201538890833185\\
37.1	0.00201561859262972\\
37.15	0.00201585094767523\\
37.2	0.00201608330272074\\
37.25	0.00201631298701861\\
37.3	0.00201654534206412\\
37.35	0.00201677502636198\\
37.4	0.00201700738140749\\
37.45	0.00201723973645301\\
37.5	0.00201746942075087\\
37.55	0.00201767773906754\\
37.6	0.00201786736215066\\
37.65	0.00201805431448613\\
37.7	0.0020182412668216\\
37.75	0.00201843088990472\\
37.8	0.00201861784224019\\
37.85	0.00201880479457566\\
37.9	0.00201899441765878\\
37.95	0.00201918136999425\\
38	0.00201936832232972\\
38.05	0.00201955527466519\\
38.1	0.00201974489774831\\
38.15	0.00201993185008378\\
38.2	0.00202011880241925\\
38.25	0.00202030842550237\\
38.3	0.00202049537783784\\
38.35	0.00202068233017331\\
38.4	0.00202086928250878\\
38.45	0.0020210589055919\\
38.5	0.00202124585792737\\
38.55	0.00202143281026284\\
38.6	0.00202162243334596\\
38.65	0.00202180938568143\\
38.7	0.0020219963380169\\
38.75	0.00202218329035237\\
38.8	0.00202237291343549\\
38.85	0.00202255986577096\\
38.9	0.00202274681810643\\
38.95	0.00202293644118955\\
39	0.00202312339352502\\
39.05	0.00202331034586049\\
39.1	0.00202349996894361\\
39.15	0.00202368692127908\\
39.2	0.00202387387361455\\
39.25	0.00202406082595002\\
39.3	0.00202425044903314\\
39.35	0.00202443740136861\\
39.4	0.00202462435370408\\
39.45	0.0020248139767872\\
39.5	0.00202500092912267\\
39.55	0.00202518788145814\\
39.6	0.00202537483379361\\
39.65	0.00202556445687673\\
39.7	0.0020257514092122\\
39.75	0.00202593836154767\\
39.8	0.00202612798463079\\
39.85	0.00202631493696626\\
39.9	0.00202650188930174\\
39.95	0.00202668884163721\\
40	0.00202687846472032\\
40.05	0.0020270654170558\\
40.1	0.00202725236939127\\
40.15	0.00202744199247438\\
40.2	0.00202762894480986\\
40.25	0.00202781589714533\\
40.3	0.00202800552022845\\
40.35	0.00202819247256392\\
40.4	0.00202837942489939\\
40.45	0.00202856637723486\\
40.5	0.00202875600031798\\
40.55	0.00202894295265345\\
40.6	0.00202912990498892\\
40.65	0.00202931952807204\\
40.7	0.00202950648040751\\
40.75	0.00202969343274298\\
40.8	0.00202988038507845\\
40.85	0.00203007000816157\\
40.9	0.00203025696049704\\
40.95	0.00203044391283251\\
41	0.00203063353591563\\
41.05	0.0020308204882511\\
41.1	0.00203100744058657\\
41.15	0.00203119439292204\\
41.2	0.00203138401600516\\
41.25	0.00203157096834063\\
41.3	0.00203175524992845\\
41.35	0.00203190481179683\\
41.4	0.00203205704441285\\
41.45	0.00203220660628123\\
41.5	0.00203235883889725\\
41.55	0.00203251107151328\\
41.6	0.00203266063338166\\
41.65	0.00203281286599768\\
41.7	0.00203296242786606\\
41.75	0.00203311466048208\\
41.8	0.00203326689309811\\
41.85	0.00203341645496648\\
41.9	0.00203356868758251\\
41.95	0.00203371824945089\\
42	0.00203387048206691\\
42.05	0.00203402271468294\\
42.1	0.00203417227655131\\
42.15	0.00203432450916734\\
42.2	0.00203447407103572\\
42.25	0.00203462630365174\\
42.3	0.00203477853626777\\
42.35	0.00203492809813614\\
42.4	0.00203508033075217\\
42.45	0.00203522989262055\\
42.5	0.00203538212523657\\
42.55	0.0020355343578526\\
42.6	0.00203568391972097\\
42.65	0.002035836152337\\
42.7	0.00203598571420538\\
42.75	0.0020361379468214\\
42.8	0.00203629017943743\\
42.85	0.0020364397413058\\
42.9	0.00203659197392183\\
42.95	0.00203674153579021\\
43	0.00203689376840623\\
43.05	0.00203704600102226\\
43.1	0.00203719556289063\\
43.15	0.00203734779550666\\
43.2	0.00203749735737504\\
43.25	0.00203764958999106\\
43.3	0.00203780182260709\\
43.35	0.00203795138447546\\
43.4	0.00203810361709149\\
43.45	0.00203825317895987\\
43.5	0.00203840541157589\\
43.55	0.00203855764419192\\
43.6	0.00203870720606029\\
43.65	0.00203885943867632\\
43.7	0.0020390090005447\\
43.75	0.00203916123316072\\
43.8	0.00203931346577675\\
43.85	0.00203946302764512\\
43.9	0.00203961526026115\\
43.95	0.00203976482212953\\
44	0.00203991705474555\\
44.05	0.00204006928736158\\
44.1	0.00204021884922995\\
44.15	0.00204037108184598\\
44.2	0.00204052064371435\\
44.25	0.00204067287633038\\
44.3	0.00204082510894641\\
44.35	0.00204097467081478\\
44.4	0.00204112690343081\\
44.45	0.00204127646529918\\
44.5	0.00204142869791521\\
44.55	0.00204158093053124\\
44.6	0.00204173049239961\\
44.65	0.00204188272501564\\
44.7	0.00204203228688401\\
44.75	0.00204218451950004\\
44.8	0.00204233675211607\\
44.85	0.00204248631398444\\
44.9	0.00204263854660047\\
44.95	0.00204278810846884\\
45	0.00204294034108487\\
45.05	0.00204308990295325\\
45.1	0.00204324213556927\\
45.15	0.0020433943681853\\
45.2	0.00204354393005367\\
45.25	0.0020436961626697\\
45.3	0.00204384572453808\\
45.35	0.0020439979571541\\
45.4	0.00204415018977013\\
45.45	0.00204428372715261\\
45.5	0.0020443478250962\\
45.55	0.00204441459378744\\
45.6	0.00204448136247868\\
45.65	0.00204454813116991\\
45.7	0.00204461489986115\\
45.75	0.00204467899780474\\
45.8	0.00204474576649598\\
45.85	0.00204481253518722\\
45.9	0.00204487930387846\\
45.95	0.00204494340182205\\
46	0.00204501017051329\\
46.05	0.00204507693920453\\
46.1	0.00204514370789577\\
46.15	0.00204520780583936\\
46.2	0.0020452745745306\\
46.25	0.00204534134322184\\
46.3	0.00204540811191308\\
46.35	0.00204547488060432\\
46.4	0.00204553897854791\\
46.45	0.00204560574723915\\
46.5	0.00204567251593039\\
46.55	0.00204573928462162\\
46.6	0.00204580338256521\\
46.65	0.00204587015125645\\
46.7	0.00204593691994769\\
46.75	0.00204600368863893\\
46.8	0.00204607045733017\\
46.85	0.00204613455527376\\
46.9	0.002046201323965\\
46.95	0.00204626809265624\\
47	0.00204633486134748\\
47.05	0.00204639895929107\\
47.1	0.00204646572798231\\
47.15	0.00204653249667355\\
47.2	0.00204659926536479\\
47.25	0.00204666336330838\\
47.3	0.00204673013199962\\
47.35	0.00204679690069086\\
47.4	0.0020468636693821\\
47.45	0.00204693043807334\\
47.5	0.00204699453601693\\
47.55	0.00204706130470816\\
47.6	0.0020471280733994\\
47.65	0.00204719484209064\\
47.7	0.00204725894003423\\
47.75	0.00204732570872547\\
47.8	0.00204739247741671\\
47.85	0.00204745924610795\\
47.9	0.00204752601479919\\
47.95	0.00204759011274278\\
48	0.00204765688143402\\
48.05	0.00204772365012526\\
48.1	0.0020477904188165\\
48.15	0.00204785451676009\\
48.2	0.00204792128545133\\
48.25	0.00204798805414257\\
48.3	0.00204805482283381\\
48.35	0.00204812159152505\\
48.4	0.00204818568946864\\
48.45	0.00204825245815988\\
48.5	0.00204831922685111\\
48.55	0.00204838599554235\\
48.6	0.00204845009348594\\
48.65	0.00204851686217718\\
48.7	0.00204858363086842\\
48.75	0.00204865039955966\\
48.8	0.00204871449750325\\
48.85	0.00204878126619449\\
48.9	0.00204884803488573\\
48.95	0.00204891480357697\\
49	0.00204898157226821\\
49.05	0.0020490456702118\\
49.1	0.00204911243890304\\
49.15	0.00204917920759428\\
49.2	0.00204924597628552\\
49.25	0.00204931007422911\\
49.3	0.00204937684292035\\
49.35	0.00204944361161159\\
49.4	0.00204951038030283\\
49.45	0.00204957447824641\\
49.5	0.00204964124693765\\
49.55	0.00204970801562889\\
49.6	0.00204977478432013\\
49.65	0.00204984155301137\\
49.7	0.00204990565095496\\
49.75	0.0020499724196462\\
49.8	0.00205003918833744\\
49.85	0.00205010595702868\\
49.9	0.00205017005497227\\
49.95	0.00205023682366351\\
50	0.00205030359235475\\
};
\addlegendentry{Van Leer}

\addplot [color=mycolor4]
  table[row sep=crcr]{%
0	0\\
0.05	8.03165909257777e-06\\
0.1	3.12571927642706e-05\\
0.15	4.39560957162116e-05\\
0.2	4.750739320054e-05\\
0.25	4.8223641484852e-05\\
0.3	4.95695145207708e-05\\
0.35	5.27705748821247e-05\\
0.4	5.74944983369617e-05\\
0.45	6.23863217255604e-05\\
0.5	6.66883564933334e-05\\
0.55	7.00701499044055e-05\\
0.6	7.24172733414995e-05\\
0.65	7.00701499044055e-05\\
0.7	6.33426562208885e-05\\
0.75	5.2199768578651e-05\\
0.8	3.36144757514967e-05\\
0.85	1.31974962425022e-05\\
0.9	-1.285391629843e-05\\
0.95	-3.93923188542462e-05\\
1	-6.78075111473573e-05\\
1.05	-9.66034192138392e-05\\
1.1	-0.000125373928404991\\
1.15	-0.00015411984079056\\
1.2	-0.000182866020532711\\
1.25	-0.000210986051158826\\
1.3	-0.000239065176227832\\
1.35	-0.000267121041274166\\
1.4	-0.000294913025373648\\
1.45	-0.000322704742116548\\
1.5	-0.000350496458859448\\
1.55	-0.000378774764582359\\
1.6	-0.000407101194490105\\
1.65	-0.000435427624397852\\
1.7	-0.000464267378943851\\
1.75	-0.000493914550368769\\
1.8	-0.000523561721793686\\
1.85	-0.00055320621965278\\
1.9	-0.000583676849351561\\
1.95	-0.000615032429338163\\
2	-0.000646385335758941\\
2.05	-0.000677740915745544\\
2.1	-0.000709291666037315\\
2.15	-0.000741997396765155\\
2.2	-0.000774705801058819\\
2.25	-0.000807414205352483\\
2.3	-0.000840119936080323\\
2.35	-0.000872790910452448\\
2.4	-0.000905392372113143\\
2.45	-0.000937993833773838\\
2.5	-0.000970595295434533\\
2.55	-0.0010031940835294\\
2.6	-0.0010357955451901\\
2.65	-0.00106546945227326\\
2.7	-0.0010951086030007\\
2.75	-0.00112474775372815\\
2.8	-0.00115438423088977\\
2.85	-0.00118402338161721\\
2.9	-0.00121326417103685\\
2.95	-0.00123574618605282\\
3	-0.00125823087463462\\
3.05	-0.00128071556321642\\
3.1	-0.00130319757823239\\
3.15	-0.00132568226681419\\
3.2	-0.00134816695539599\\
3.25	-0.00136515746620899\\
3.3	-0.00137598808136295\\
3.35	-0.00138681869651692\\
3.4	-0.00139764663810506\\
3.45	-0.00140847725325903\\
3.5	-0.001419307868413\\
3.55	-0.00143013848356696\\
3.6	-0.0014385388273867\\
3.65	-0.00143445896593892\\
3.7	-0.00143037910449115\\
3.75	-0.00142629924304337\\
3.8	-0.00142222205516141\\
3.85	-0.00141814219371363\\
3.9	-0.00141406233226585\\
3.95	-0.00140998247081807\\
4	-0.00140511390745215\\
4.05	-0.00138544715724909\\
4.1	-0.00136578040704603\\
4.15	-0.00134611365684298\\
4.2	-0.00132644690663992\\
4.25	-0.00130677748287104\\
4.3	-0.00128711073266798\\
4.35	-0.00126744398246493\\
4.4	-0.00124777723226187\\
4.45	-0.00122750090905089\\
4.5	-0.00119459198732041\\
4.55	-0.00116168306558992\\
4.6	-0.00112877414385944\\
4.65	-0.00109586522212896\\
4.7	-0.0010629589739643\\
4.75	-0.00103005005223382\\
4.8	-0.00099714113050334\\
4.85	-0.000964232208772859\\
4.9	-0.000931323287042377\\
4.95	-0.000897965206253425\\
5	-0.000856212128776381\\
5.05	-0.000814459051299337\\
5.1	-0.000772703300256469\\
5.15	-0.000730950222779425\\
5.2	-0.000689197145302381\\
5.25	-0.000647441394259513\\
5.3	-0.00060568831678247\\
5.35	-0.000563935239305426\\
5.4	-0.000522179488262558\\
5.45	-0.000480426410785514\\
5.5	-0.000438670659742646\\
5.55	-0.00039697907427956\\
5.6	-0.000355290162382298\\
5.65	-0.000313601250485036\\
5.7	-0.000271912338587773\\
5.75	-0.000230222089907599\\
5.8	-0.000188532643297172\\
5.85	-0.000146843464043328\\
5.9	-0.000105154017432901\\
5.95	-6.34648381790559e-05\\
6	-2.1775284625996e-05\\
6.05	1.99142421914058e-05\\
6.1	6.16035016522251e-05\\
6.15	9.73958641241418e-05\\
6.2	0.000129882629810977\\
6.25	0.000162369395497812\\
6.3	0.00019485642854123\\
6.35	0.000227342926871483\\
6.4	0.0002598299599149\\
6.45	0.000292316992958318\\
6.5	0.000324803491288571\\
6.55	0.000357289989618824\\
6.6	0.000389776487949077\\
6.65	0.00042226298627933\\
6.7	0.000454749484609582\\
6.75	0.000487238656505659\\
6.8	0.000519725154835912\\
6.85	0.000542319459616504\\
6.9	0.000564616998590605\\
6.95	0.000586917211130531\\
7	0.000609217423670457\\
7.05	0.000631514962644559\\
7.1	0.000653815175184484\\
7.15	0.00067611538772441\\
7.2	0.000698412926698512\\
7.25	0.000720713139238438\\
7.3	0.000743013351778364\\
7.35	0.000765310890752465\\
7.4	0.000787611103292391\\
7.45	0.000809908642266493\\
7.5	0.000832208854806418\\
7.55	0.000854509067346344\\
7.6	0.000871058439798345\\
7.65	0.000886145371744491\\
7.7	0.000901234977256461\\
7.75	0.000916321909202607\\
7.8	0.000931411514714577\\
7.85	0.000946498446660722\\
7.9	0.000961588052172692\\
7.95	0.000976677657684662\\
8	0.000991764589630808\\
8.05	0.00100685419514278\\
8.1	0.00102194112708892\\
8.15	0.00103703073260089\\
8.2	0.00105211766454704\\
8.25	0.00106720727005901\\
8.3	0.00108229420200516\\
8.35	0.00109738380751713\\
8.4	0.00111247073946327\\
8.45	0.00112270782300426\\
8.5	0.00113271497988436\\
8.55	0.00114271946319864\\
8.6	0.00115272662007874\\
8.65	0.00116273377695885\\
8.7	0.00117274093383895\\
8.75	0.00118274541715323\\
8.8	0.00119275257403333\\
8.85	0.00120275973091344\\
8.9	0.00121276688779354\\
8.95	0.00122277137110782\\
9	0.00123277852798792\\
9.05	0.00124278568486803\\
9.1	0.00125279284174813\\
9.15	0.00126279999862823\\
9.2	0.00127280448194251\\
9.25	0.00128281163882262\\
9.3	0.00129281879570272\\
9.35	0.0013019516965583\\
9.4	0.00130859818119734\\
9.45	0.00131524466583638\\
9.5	0.00132189115047543\\
9.55	0.00132853763511447\\
9.6	0.00133518411975351\\
9.65	0.00134183060439255\\
9.7	0.00134847976259742\\
9.75	0.00135512624723646\\
9.8	0.00136177273187551\\
9.85	0.00136841921651455\\
9.9	0.00137506570115359\\
9.95	0.00138171218579264\\
10	0.00138835867043168\\
10.05	0.00139500515507072\\
10.1	0.00140165163970976\\
10.15	0.00140829812434881\\
10.2	0.00141494460898785\\
10.25	0.00142159109362689\\
10.3	0.00142823757826593\\
10.35	0.00143488406290498\\
10.4	0.00144045042695103\\
10.45	0.00144497410032563\\
10.5	0.00144949777370023\\
10.55	0.00145402144707484\\
10.6	0.00145854244688361\\
10.65	0.00146306612025822\\
10.7	0.00146758979363282\\
10.75	0.0014721107934416\\
10.8	0.0014766344668162\\
10.85	0.0014811581401908\\
10.9	0.0014856818135654\\
10.95	0.00149020281337418\\
11	0.00149472648674878\\
11.05	0.00149925016012338\\
11.1	0.00150377115993216\\
11.15	0.00150829483330676\\
11.2	0.00151281850668136\\
11.25	0.00151734218005597\\
11.3	0.00152186317986474\\
11.35	0.00152638685323935\\
11.4	0.00153091052661395\\
11.45	0.00153543152642273\\
11.5	0.00153995519979733\\
11.55	0.00154384523807159\\
11.6	0.00154703212853407\\
11.65	0.00155021634543073\\
11.7	0.00155340323589322\\
11.75	0.0015565901263557\\
11.8	0.00155977701681819\\
11.85	0.00156296123371485\\
11.9	0.00156614812417734\\
11.95	0.00156933501463982\\
12	0.00157251923153648\\
12.05	0.00157570612199897\\
12.1	0.00157889301246145\\
12.15	0.00158207990292394\\
12.2	0.0015852641198206\\
12.25	0.00158845101028309\\
12.3	0.00159163790074557\\
12.35	0.00159482479120806\\
12.4	0.00159800900810472\\
12.45	0.0016011958985672\\
12.5	0.00160438278902969\\
12.55	0.00160756700592635\\
12.6	0.00161075389638884\\
12.65	0.00161394078685132\\
12.7	0.00161712767731381\\
12.75	0.00162031189421047\\
12.8	0.00162349878467295\\
12.85	0.00162590232034894\\
12.9	0.00162822297548437\\
12.95	0.00163054630418563\\
13	0.00163286695932107\\
13.05	0.00163519028802232\\
13.1	0.00163751094315776\\
13.15	0.00163983427185902\\
13.2	0.00164215492699445\\
13.25	0.00164447558212988\\
13.3	0.00164679891083114\\
13.35	0.00164911956596658\\
13.4	0.00165144289466783\\
13.45	0.00165376354980327\\
13.5	0.00165608687850453\\
13.55	0.00165840753363996\\
13.6	0.00166073086234122\\
13.65	0.00166305151747665\\
13.7	0.00166537484617791\\
13.75	0.00166769550131335\\
13.8	0.00167001615644878\\
13.85	0.00167233948515004\\
13.9	0.00167466014028547\\
13.95	0.00167698346898673\\
14	0.00167930412412217\\
14.05	0.00168162745282342\\
14.1	0.00168394810795886\\
14.15	0.00168627143666012\\
14.2	0.00168859209179555\\
14.25	0.00169060796042702\\
14.3	0.0016923484517786\\
14.35	0.00169408894313017\\
14.4	0.00169582943448175\\
14.45	0.0016975672522675\\
14.5	0.00169930774361907\\
14.55	0.00170104823497065\\
14.6	0.0017027860527564\\
14.65	0.00170452654410798\\
14.7	0.00170626703545955\\
14.75	0.0017080048532453\\
14.8	0.00170974534459688\\
14.85	0.00171148583594846\\
14.9	0.00171322365373421\\
14.95	0.00171496414508578\\
15	0.00171670463643736\\
15.05	0.00171844245422311\\
15.1	0.00172018294557469\\
15.15	0.00172192343692626\\
15.2	0.00172366392827784\\
15.25	0.00172540174606359\\
15.3	0.00172714223741516\\
15.35	0.00172888272876674\\
15.4	0.00173062054655249\\
15.45	0.00173236103790407\\
15.5	0.00173410152925564\\
15.55	0.00173583934704139\\
15.6	0.00173757983839297\\
15.65	0.00173932032974454\\
15.7	0.00174105814753029\\
15.75	0.00174279863888187\\
15.8	0.00174452843597015\\
15.85	0.00174585987175062\\
15.9	0.00174719130753108\\
15.95	0.00174852274331155\\
16	0.00174985417909202\\
16.05	0.00175118294130666\\
16.1	0.00175251437708713\\
16.15	0.0017538458128676\\
16.2	0.00175517724864807\\
16.25	0.00175650601086271\\
16.3	0.00175783744664318\\
16.35	0.00175916888242365\\
16.4	0.00176050031820412\\
16.45	0.00176183175398458\\
16.5	0.00176316051619923\\
16.55	0.0017644919519797\\
16.6	0.00176582338776016\\
16.65	0.00176715482354063\\
16.7	0.0017684862593211\\
16.75	0.00176981502153574\\
16.8	0.00177114645731621\\
16.85	0.00177247789309668\\
16.9	0.00177380932887715\\
16.95	0.00177513809109179\\
17	0.00177646952687226\\
17.05	0.00177780096265273\\
17.1	0.00177913239843319\\
17.15	0.00178046383421366\\
17.2	0.00178179259642831\\
17.25	0.00178312403220877\\
17.3	0.00178445546798924\\
17.35	0.00178578690376971\\
17.4	0.00178711833955018\\
17.45	0.00178844710176482\\
17.5	0.00178977853754529\\
17.55	0.00179109393193081\\
17.6	0.00179212860190479\\
17.65	0.00179316327187877\\
17.7	0.00179419794185275\\
17.75	0.0017952299382609\\
17.8	0.00179626460823488\\
17.85	0.00179729927820886\\
17.9	0.00179833394818283\\
17.95	0.00179936861815681\\
18	0.00180040328813079\\
18.05	0.00180143795810477\\
18.1	0.00180247262807875\\
18.15	0.00180350729805272\\
18.2	0.0018045419680267\\
18.25	0.00180557396443486\\
18.3	0.00180660863440883\\
18.35	0.00180764330438281\\
18.4	0.00180867797435679\\
18.45	0.00180971264433077\\
18.5	0.00181074731430475\\
18.55	0.00181178198427872\\
18.6	0.0018128166542527\\
18.65	0.00181385132422668\\
18.7	0.00181488599420066\\
18.75	0.00181591799060881\\
18.8	0.00181695266058279\\
18.85	0.00181798733055677\\
18.9	0.00181902200053075\\
18.95	0.00182005667050473\\
19	0.0018210913404787\\
19.05	0.00182212601045268\\
19.1	0.00182316068042666\\
19.15	0.00182419535040064\\
19.2	0.00182522734680879\\
19.25	0.00182626201678277\\
19.3	0.00182729668675675\\
19.35	0.00182833135673073\\
19.4	0.0018293660267047\\
19.45	0.00183040069667868\\
19.5	0.00183138189533618\\
19.55	0.00183219465934674\\
19.6	0.00183300742335731\\
19.65	0.00183382018736788\\
19.7	0.00183463295137844\\
19.75	0.00183544571538901\\
19.8	0.00183625847939958\\
19.85	0.00183707124341014\\
19.9	0.00183788668098653\\
19.95	0.0018386994449971\\
20	0.00183951220900767\\
20.05	0.00184032497301823\\
20.1	0.0018411377370288\\
20.15	0.00184195050103937\\
20.2	0.00184276326504993\\
20.25	0.0018435760290605\\
20.3	0.00184438879307107\\
20.35	0.00184520155708164\\
20.4	0.0018460143210922\\
20.45	0.00184682708510277\\
20.5	0.00184763984911334\\
20.55	0.0018484526131239\\
20.6	0.00184926537713447\\
20.65	0.00185008081471086\\
20.7	0.00185089357872143\\
20.75	0.00185170634273199\\
20.8	0.00185251910674256\\
20.85	0.00185333187075313\\
20.9	0.00185414463476369\\
20.95	0.00185495739877426\\
21	0.00185577016278483\\
21.05	0.00185658292679539\\
21.1	0.00185739569080596\\
21.15	0.00185820845481653\\
21.2	0.00185902121882709\\
21.25	0.00185983398283766\\
21.3	0.00186064674684823\\
21.35	0.00186146218442462\\
21.4	0.00186227494843519\\
21.45	0.00186308771244575\\
21.5	0.00186390047645632\\
21.55	0.00186471324046689\\
21.6	0.00186552600447745\\
21.65	0.00186631203282978\\
21.7	0.00186695636219342\\
21.75	0.00186759801799123\\
21.8	0.00186823967378905\\
21.85	0.00186888400315269\\
21.9	0.0018695256589505\\
21.95	0.00187016998831415\\
22	0.00187081164411196\\
22.05	0.0018714559734756\\
22.1	0.00187209762927342\\
22.15	0.00187274195863706\\
22.2	0.00187338361443487\\
22.25	0.00187402794379851\\
22.3	0.00187466959959633\\
22.35	0.00187531392895997\\
22.4	0.00187595558475779\\
22.45	0.00187659991412143\\
22.5	0.00187724156991924\\
22.55	0.00187788589928288\\
22.6	0.0018785275550807\\
22.65	0.00187917188444434\\
22.7	0.00187981354024215\\
22.75	0.00188045519603997\\
22.8	0.00188109952540361\\
22.85	0.00188174118120142\\
22.9	0.00188238551056507\\
22.95	0.00188302716636288\\
23	0.00188367149572652\\
23.05	0.00188431315152434\\
23.1	0.00188495748088798\\
23.15	0.00188559913668579\\
23.2	0.00188624346604943\\
23.25	0.00188688512184725\\
23.3	0.00188752945121089\\
23.35	0.0018881711070087\\
23.4	0.00188881543637234\\
23.45	0.00188945709217016\\
23.5	0.0018901014215338\\
23.55	0.00189074307733162\\
23.6	0.00189138740669526\\
23.65	0.00189202906249307\\
23.7	0.00189267339185671\\
23.75	0.00189331504765453\\
23.8	0.00189395670345234\\
23.85	0.00189460103281598\\
23.9	0.0018952426886138\\
23.95	0.00189588701797744\\
24	0.00189652867377526\\
24.05	0.00189712755251988\\
24.1	0.00189763820359231\\
24.15	0.00189814885466474\\
24.2	0.00189865683217135\\
24.25	0.00189916748324377\\
24.3	0.0018996781343162\\
24.35	0.00190018611182281\\
24.4	0.00190069676289524\\
24.45	0.00190120741396766\\
24.5	0.00190171539147427\\
24.55	0.0019022260425467\\
24.6	0.00190273669361913\\
24.65	0.00190324467112573\\
24.7	0.00190375532219816\\
24.75	0.00190426597327059\\
24.8	0.00190477395077719\\
24.85	0.00190528460184962\\
24.9	0.00190579525292205\\
24.95	0.00190630590399448\\
25	0.00190681388150108\\
25.05	0.00190732453257351\\
25.1	0.00190783518364594\\
25.15	0.00190834316115254\\
25.2	0.00190885381222497\\
25.25	0.0019093644632974\\
25.3	0.001909872440804\\
25.35	0.00191038309187643\\
25.4	0.00191089374294886\\
25.45	0.00191140172045546\\
25.5	0.00191191237152789\\
25.55	0.00191242302260032\\
25.6	0.00191293100010692\\
25.65	0.00191344165117935\\
25.7	0.00191395230225178\\
25.75	0.00191446027975839\\
25.8	0.00191497093083081\\
25.85	0.00191548158190324\\
25.9	0.00191598955940985\\
25.95	0.00191650021048228\\
26	0.0019170108615547\\
26.05	0.00191751883906131\\
26.1	0.00191802949013374\\
26.15	0.00191854014120617\\
26.2	0.00191905079227859\\
26.25	0.0019195587697852\\
26.3	0.00192006942085763\\
26.35	0.00192058007193005\\
26.4	0.00192108804943666\\
26.45	0.00192159870050909\\
26.5	0.00192210935158152\\
26.55	0.00192261732908812\\
26.6	0.00192312798016055\\
26.65	0.00192363863123298\\
26.7	0.00192412254664716\\
26.75	0.00192452625508662\\
26.8	0.00192492996352608\\
26.85	0.00192533367196554\\
26.9	0.001925737380405\\
26.95	0.00192614376241028\\
27	0.00192654747084974\\
27.05	0.0019269511792892\\
27.1	0.00192735488772866\\
27.15	0.00192776126973394\\
27.2	0.0019281649781734\\
27.25	0.00192856868661286\\
27.3	0.00192897239505232\\
27.35	0.00192937877705761\\
27.4	0.00192978248549706\\
27.45	0.00193018619393652\\
27.5	0.00193059257594181\\
27.55	0.00193099628438127\\
27.6	0.00193139999282073\\
27.65	0.00193180370126018\\
27.7	0.00193221008326547\\
27.75	0.00193261379170493\\
27.8	0.00193301750014439\\
27.85	0.00193342120858385\\
27.9	0.00193382759058913\\
27.95	0.00193423129902859\\
28	0.00193463500746805\\
28.05	0.00193503871590751\\
28.1	0.00193544242434697\\
28.15	0.00193584880635225\\
28.2	0.00193625251479171\\
28.25	0.00193665622323117\\
28.3	0.00193705993167063\\
28.35	0.00193746631367591\\
28.4	0.00193787002211537\\
28.45	0.00193827373055483\\
28.5	0.00193868011256011\\
28.55	0.00193908382099957\\
28.6	0.00193948752943903\\
28.65	0.00193989123787849\\
28.7	0.00194029761988377\\
28.75	0.00194070132832323\\
28.8	0.00194110503676269\\
28.85	0.00194150874520215\\
28.9	0.00194191512720743\\
28.95	0.00194231883564689\\
29	0.00194272254408635\\
29.05	0.00194312625252581\\
29.1	0.00194353263453109\\
29.15	0.00194393634297055\\
29.2	0.00194434005141001\\
29.25	0.00194474375984947\\
29.3	0.00194515014185475\\
29.35	0.00194555385029421\\
29.4	0.00194595755873367\\
29.45	0.00194636126717313\\
29.5	0.00194676764917842\\
29.55	0.00194717135761787\\
29.6	0.00194757506605733\\
29.65	0.00194795203883855\\
29.7	0.00194827286673746\\
29.75	0.00194859369463637\\
29.8	0.00194891184896945\\
29.85	0.00194923267686836\\
29.9	0.00194955350476727\\
29.95	0.00194987433266617\\
30	0.00195019248699926\\
30.05	0.00195051331489817\\
30.1	0.00195083414279707\\
30.15	0.00195115497069598\\
30.2	0.00195147579859489\\
30.25	0.00195179395292797\\
30.3	0.00195211478082688\\
30.35	0.00195243560872579\\
30.4	0.0019527564366247\\
30.45	0.00195307459095778\\
30.5	0.00195339541885669\\
30.55	0.0019537162467556\\
30.6	0.0019540370746545\\
30.65	0.00195435790255341\\
30.7	0.0019546760568865\\
30.75	0.0019549968847854\\
30.8	0.00195531771268431\\
30.85	0.00195563854058322\\
30.9	0.00195595936848213\\
30.95	0.00195627752281521\\
31	0.00195659835071412\\
31.05	0.00195691917861303\\
31.1	0.00195724000651194\\
31.15	0.00195755816084502\\
31.2	0.00195787898874393\\
31.25	0.00195819981664283\\
31.3	0.00195852064454174\\
31.35	0.00195883879887483\\
31.4	0.00195915962677373\\
31.45	0.00195948045467264\\
31.5	0.00195980128257155\\
31.55	0.00196012211047046\\
31.6	0.00196044026480354\\
31.65	0.00196076109270245\\
31.7	0.00196108192060136\\
31.75	0.00196140274850027\\
31.8	0.00196172357639917\\
31.85	0.00196204173073226\\
31.9	0.00196236255863117\\
31.95	0.00196268338653007\\
32	0.00196300421442898\\
32.05	0.00196332236876206\\
32.1	0.00196364319666097\\
32.15	0.00196396402455988\\
32.2	0.00196428485245879\\
32.25	0.0019646056803577\\
32.3	0.00196492383469078\\
32.35	0.00196524466258969\\
32.4	0.0019655654904886\\
32.45	0.0019658863183875\\
32.5	0.00196620447272059\\
32.55	0.0019665253006195\\
32.6	0.0019668461285184\\
32.65	0.00196716695641731\\
32.7	0.00196748778431622\\
32.75	0.0019678059386493\\
32.8	0.00196812676654821\\
32.85	0.00196844759444712\\
32.9	0.00196876842234603\\
32.95	0.00196902508466515\\
33	0.00196927907341845\\
33.05	0.00196953038860593\\
33.1	0.00196978170379341\\
33.15	0.00197003569254671\\
33.2	0.00197028700773419\\
33.25	0.00197054099648749\\
33.3	0.00197079231167497\\
33.35	0.00197104630042827\\
33.4	0.00197129761561575\\
33.45	0.00197155160436905\\
33.5	0.00197180291955653\\
33.55	0.00197205423474401\\
33.6	0.00197230822349731\\
33.65	0.00197255953868479\\
33.7	0.00197281352743809\\
33.75	0.00197306484262557\\
33.8	0.00197331883137887\\
33.85	0.00197357014656635\\
33.9	0.00197382413531965\\
33.95	0.00197407545050713\\
34	0.00197432943926043\\
34.05	0.00197458075444791\\
34.1	0.00197483206963539\\
34.15	0.00197508605838869\\
34.2	0.00197533737357617\\
34.25	0.00197559136232947\\
34.3	0.00197584267751695\\
34.35	0.00197609666627025\\
34.4	0.00197634798145773\\
34.45	0.00197660197021103\\
34.5	0.00197685328539851\\
34.55	0.00197710460058598\\
34.6	0.00197735858933929\\
34.65	0.00197760990452677\\
34.7	0.00197786389328007\\
34.75	0.00197811520846754\\
34.8	0.00197836919722085\\
34.85	0.00197862051240832\\
34.9	0.00197887450116163\\
34.95	0.00197912581634911\\
35	0.00197937713153658\\
35.05	0.00197963112028988\\
35.1	0.00197988243547736\\
35.15	0.00198013642423066\\
35.2	0.00198038773941814\\
35.25	0.00198064172817144\\
35.3	0.00198089304335892\\
35.35	0.00198114703211222\\
35.4	0.0019813983472997\\
35.45	0.001981652336053\\
35.5	0.00198190365124048\\
35.55	0.00198215496642796\\
35.6	0.00198240895518126\\
35.65	0.00198266027036874\\
35.7	0.00198291425912204\\
35.75	0.00198316557430952\\
35.8	0.00198341956306282\\
35.85	0.0019836708782503\\
35.9	0.0019839248670036\\
35.95	0.00198417618219108\\
36	0.00198442749737856\\
36.05	0.00198468148613186\\
36.1	0.00198493280131934\\
36.15	0.00198518679007264\\
36.2	0.00198543810526012\\
36.25	0.00198569209401342\\
36.3	0.0019859434092009\\
36.35	0.0019861973979542\\
36.4	0.00198644871314168\\
36.45	0.00198670002832916\\
36.5	0.00198695401708246\\
36.55	0.00198718661730917\\
36.6	0.00198737911404851\\
36.65	0.00198757161078786\\
36.7	0.0019877641075272\\
36.75	0.00198795660426655\\
36.8	0.00198814910100589\\
36.85	0.00198834159774524\\
36.9	0.00198853409448458\\
36.95	0.00198872659122392\\
37	0.00198891641439745\\
37.05	0.00198910891113679\\
37.1	0.00198930140787613\\
37.15	0.00198949390461548\\
37.2	0.00198968640135482\\
37.25	0.00198987889809417\\
37.3	0.00199007139483351\\
37.35	0.00199026389157286\\
37.4	0.0019904563883122\\
37.45	0.00199064888505155\\
37.5	0.00199084138179089\\
37.55	0.00199103387853024\\
37.6	0.00199122637526958\\
37.65	0.0019914161984431\\
37.7	0.00199160869518245\\
37.75	0.00199180119192179\\
37.8	0.00199199368866114\\
37.85	0.00199218618540048\\
37.9	0.00199237868213983\\
37.95	0.00199257117887917\\
38	0.00199276367561852\\
38.05	0.00199295617235786\\
38.1	0.00199314866909721\\
38.15	0.00199334116583655\\
38.2	0.0019935336625759\\
38.25	0.00199372615931524\\
38.3	0.00199391865605459\\
38.35	0.00199410847922811\\
38.4	0.00199430097596745\\
38.45	0.00199449347270679\\
38.5	0.00199468596944614\\
38.55	0.00199487846618549\\
38.6	0.00199507096292483\\
38.65	0.00199526345966417\\
38.7	0.00199545595640352\\
38.75	0.00199564845314286\\
38.8	0.00199584094988221\\
38.85	0.00199603344662155\\
38.9	0.0019962259433609\\
38.95	0.00199641844010024\\
39	0.00199660826327376\\
39.05	0.00199680076001311\\
39.1	0.00199699325675245\\
39.15	0.0019971857534918\\
39.2	0.00199737825023114\\
39.25	0.00199757074697049\\
39.3	0.00199776324370983\\
39.35	0.00199795574044918\\
39.4	0.00199814823718852\\
39.45	0.00199834073392787\\
39.5	0.00199853323066721\\
39.55	0.00199872572740656\\
39.6	0.0019989182241459\\
39.65	0.00199910804731942\\
39.7	0.00199930054405877\\
39.75	0.00199949304079811\\
39.8	0.00199968553753746\\
39.85	0.0019998780342768\\
39.9	0.00200007053101614\\
39.95	0.00200026302775549\\
40	0.00200045552449483\\
40.05	0.00200064802123418\\
40.1	0.00200084051797352\\
40.15	0.00200103301471287\\
40.2	0.00200122551145221\\
40.25	0.00200141800819156\\
40.3	0.0020016105049309\\
40.35	0.00200180032810442\\
40.4	0.00200199282484377\\
40.45	0.00200218532158311\\
40.5	0.00200237781832246\\
40.55	0.0020025703150618\\
40.6	0.00200271736118214\\
40.65	0.00200284836590752\\
40.7	0.00200297937063291\\
40.75	0.0020031103753583\\
40.8	0.00200324138008368\\
40.85	0.0020033750583749\\
40.9	0.00200350606310028\\
40.95	0.00200363706782567\\
41	0.00200376807255106\\
41.05	0.00200389907727645\\
41.1	0.00200403008200183\\
41.15	0.00200416108672722\\
41.2	0.00200429209145261\\
41.25	0.002004423096178\\
41.3	0.00200455410090338\\
41.35	0.00200468510562877\\
41.4	0.00200481611035416\\
41.45	0.00200494711507955\\
41.5	0.00200508079337076\\
41.55	0.00200521179809614\\
41.6	0.00200534280282153\\
41.65	0.00200547380754692\\
41.7	0.00200560481227231\\
41.75	0.00200573581699769\\
41.8	0.00200586682172308\\
41.85	0.00200599782644847\\
41.9	0.00200612883117386\\
41.95	0.00200625983589924\\
42	0.00200639084062463\\
42.05	0.00200652184535002\\
42.1	0.00200665285007541\\
42.15	0.00200678385480079\\
42.2	0.002006917533092\\
42.25	0.00200704853781739\\
42.3	0.00200717954254278\\
42.35	0.00200731054726817\\
42.4	0.00200744155199355\\
42.45	0.00200757255671894\\
42.5	0.00200770356144433\\
42.55	0.00200783456616972\\
42.6	0.0020079655708951\\
42.65	0.00200809657562049\\
42.7	0.00200822758034588\\
42.75	0.00200835858507127\\
42.8	0.00200848958979665\\
42.85	0.00200862059452204\\
42.9	0.00200875427281325\\
42.95	0.00200888527753864\\
43	0.00200901628226403\\
43.05	0.00200914728698942\\
43.1	0.0020092782917148\\
43.15	0.00200940929644019\\
43.2	0.00200954030116558\\
43.25	0.00200967130589096\\
43.3	0.00200980231061635\\
43.35	0.00200993331534174\\
43.4	0.00201006432006713\\
43.45	0.00201019532479251\\
43.5	0.0020103263295179\\
43.55	0.00201045733424329\\
43.6	0.0020105910125345\\
43.65	0.00201072201725989\\
43.7	0.00201085302198528\\
43.75	0.00201098402671066\\
43.8	0.00201111503143605\\
43.85	0.00201124603616144\\
43.9	0.00201137704088683\\
43.95	0.00201150804561221\\
44	0.0020116390503376\\
44.05	0.00201177005506299\\
44.1	0.00201190105978837\\
44.15	0.00201203206451376\\
44.2	0.00201216306923915\\
44.25	0.00201229407396454\\
44.3	0.00201242775225575\\
44.35	0.00201255875698114\\
44.4	0.00201268976170652\\
44.45	0.00201282076643191\\
44.5	0.0020129517711573\\
44.55	0.00201308277588269\\
44.6	0.00201321378060807\\
44.65	0.00201334478533346\\
44.7	0.00201347579005885\\
44.75	0.00201360679478424\\
44.8	0.00201373779950962\\
44.85	0.00201386880423501\\
44.9	0.0020139998089604\\
44.95	0.00201413081368579\\
45	0.002014264491977\\
45.05	0.00201437143460997\\
45.1	0.0020144195587948\\
45.15	0.00201446768297964\\
45.2	0.00201451580716447\\
45.25	0.00201456393134931\\
45.3	0.00201461205553415\\
45.35	0.00201466017971898\\
45.4	0.00201470830390382\\
45.45	0.00201475642808866\\
45.5	0.00201480455227349\\
45.55	0.00201485267645833\\
45.6	0.00201490080064316\\
45.65	0.002014948924828\\
45.7	0.00201499704901284\\
45.75	0.00201504517319767\\
45.8	0.00201509329738251\\
45.85	0.00201514142156735\\
45.9	0.00201518954575218\\
45.95	0.00201523766993702\\
46	0.00201528579412185\\
46.05	0.00201533391830669\\
46.1	0.00201538204249153\\
46.15	0.00201543016667636\\
46.2	0.0020154782908612\\
46.25	0.00201552641504603\\
46.3	0.00201557453923087\\
46.35	0.00201562266341571\\
46.4	0.00201567078760054\\
46.45	0.00201571891178538\\
46.5	0.00201576703597022\\
46.55	0.00201581516015505\\
46.6	0.00201586328433989\\
46.65	0.00201591140852472\\
46.7	0.00201595953270956\\
46.75	0.0020160076568944\\
46.8	0.00201605578107923\\
46.85	0.00201610390526407\\
46.9	0.0020161520294489\\
46.95	0.00201620015363374\\
47	0.00201624827781858\\
47.05	0.00201629640200341\\
47.1	0.00201634452618825\\
47.15	0.00201639265037309\\
47.2	0.00201644077455792\\
47.25	0.00201648889874276\\
47.3	0.00201653702292759\\
47.35	0.00201658514711243\\
47.4	0.00201663327129727\\
47.45	0.0020166813954821\\
47.5	0.00201672951966694\\
47.55	0.00201677764385178\\
47.6	0.00201682576803661\\
47.65	0.00201687389222145\\
47.7	0.00201692201640628\\
47.75	0.00201697014059112\\
47.8	0.00201701826477596\\
47.85	0.00201706638896079\\
47.9	0.00201711451314563\\
47.95	0.00201716263733046\\
48	0.0020172107615153\\
48.05	0.00201725888570014\\
48.1	0.00201730700988497\\
48.15	0.00201735780763563\\
48.2	0.00201740325825465\\
48.25	0.00201745405600531\\
48.3	0.00201750218019014\\
48.35	0.00201755030437498\\
48.4	0.00201759842855982\\
48.45	0.00201764655274465\\
48.5	0.00201769467692949\\
48.55	0.00201774280111432\\
48.6	0.00201779092529916\\
48.65	0.002017839049484\\
48.7	0.00201788717366883\\
48.75	0.00201793529785367\\
48.8	0.0020179834220385\\
48.85	0.00201803154622334\\
48.9	0.00201807967040818\\
48.95	0.00201812779459301\\
49	0.00201817591877785\\
49.05	0.00201822404296269\\
49.1	0.00201827216714752\\
49.15	0.00201832029133236\\
49.2	0.00201836841551719\\
49.25	0.00201841653970203\\
49.3	0.00201846466388687\\
49.35	0.0020185127880717\\
49.4	0.00201856091225654\\
49.45	0.00201860903644137\\
49.5	0.00201865716062621\\
49.55	0.00201870528481105\\
49.6	0.00201875340899588\\
49.65	0.00201880153318072\\
49.7	0.00201884965736556\\
49.75	0.00201889778155039\\
49.8	0.00201894590573523\\
49.85	0.00201899402992006\\
49.9	0.0020190421541049\\
49.95	0.00201909027828974\\
50	0.00201913840247457\\
};
\addlegendentry{Upwind}
\addplot [color=black, dashed]
  table[row sep=crcr]
	\caption{Comparison between the solution from \cite{sitonasa}, the 
	experimental values from \cite{driver} and the solutions computed using Van 
	Leer and Upwind.}
	\label{fig:bfs_fin}
\end{figure}
\FloatBarrier
\begin{thebibliography}{99}\label{sec:bib}
	
	\bibitem{sitonasa} Langley Research Center, \emph{Turbulence Modeling 
	Resource}.\\ 
	\hyperlink{https://turbmodels.larc.nasa.gov/backstep_val.html}{https://turbmodels.larc.nasa.gov/backstep\_val.html}
	
	\bibitem{driver} D.M. Driver, H.L. Seegmiller. Features of Reattaching 
	Turbulent Shear Layer in Divergent Channel Flow. \emph{AIAA Journal}, 
	23(2):163-171, 1985.
	
\end{thebibliography}

\end{document}