\documentclass[11pt, a4paper]{article}
\usepackage[british]{babel}
\usepackage[utf8]{inputenc}
\usepackage{amsmath}
\usepackage{amssymb}
\usepackage{amsthm} % per i teoremi e le definizioni
\usepackage{booktabs} % per le righe di separazione nelle tabelle
\usepackage{subfig} % per mettere tante figure affiancate
\usepackage{bbold}
\usepackage[hidelinks]{hyperref} % per i collegamenti ipertestuali
\usepackage{graphicx} % per le immagini
\usepackage{xspace}
\usepackage{pgfplots} % per i plot
\pgfplotsset{compat=newest}
%\usetikzlibrary{plotmarks}
%\usetikzlibrary{arrows.meta}
%\usepgfplotslibrary{patchplots}
%\usepackage{grffile}
\usepackage[section]{placeins}
\usepackage{tikz} % per i plot
\usepackage{pgfgantt}
\usepackage{pdflscape}
\usepackage[separate-uncertainty=true]{siunitx} % per le unitàdi misura
\pgfplotsset{plot coordinates/math parser=false}

\newlength{\figwidth}
\setlength{\figwidth}{0.9\textwidth}

\newcommand{\DUMUX}{DuMu$^\mathrm{x}$\xspace} %for convenience

\graphicspath{{img/}}

\theoremstyle{definition}
\newtheorem{defn}{Definition}

\title{\textbf{Cavities}}
\author{Andrea Vescovini}
\date{\today}

\begin{document}
	\maketitle
	
\section{Introduction}
Our goal is to study the behaviour of a coupled turbulent free-flow and 
porous-medium 
system, that are in contact by means of a rough interface or that presents 
obstacles. A first 
approximation of this kind of interface can be obtained with a series of little 
cavities as we used when we studied the \emph{rough channel} with the 
Navier-Stokes equations. Now we want to use the RANS equations and as 
turbulence model we 
choose the $k\text{-}\omega$. Since the domain is not flat we have decided not 
to employ any wall law for the boundary layer, because their application in 
such a situation could become complex. So we have decided to integrate the RANS 
equations up to the boundary, therefore we need very small cells next to it in 
order to resolve the viscous sub-layer. Employing structured grid it can become 
quite expensive to have small cells near a wall that has different orientations 
through the domain, so we decide to start from a simple 
configuration with only two cavities. We start our analysis from the free-flow, 
then we take into account the effect of a porous-medium.

\section{The single-domain test}
In this first test we simulate only the free-flow in the domain depicted in 
Figure~\ref{fig:singledomain}. It consists of a channel with inflow boundary 
condition on the left boundary $\Gamma_{in}$, outflow boundary conditions on 
the 
right boundary $\Gamma_{out}$, symmetry boundary conditions on the upper part 
$\Gamma_{sym}$ and no-slip conditions on the lower boundary $\Gamma_w$, where 
we have also the two cavities.\\
The first portion of the lower wall is flat, in order to le the flow develop, 
then there are the two cavities of height $h=\SI{0.1}{m}$ and length 
$\SI{0.5}{m}$, spaced by a tunable distance $d$. The height of the channel has 
been chosen such that the flow is uniform in the upper part of the domain. 
\begin{figure}[h]
	\centering
	\includegraphics[width=\textwidth]{cavities_domain}
	\caption{Domain for the single-domain test.}
	\label{fig:singledomain}
\end{figure}
\begin{table}[h]
	\centering
	\[
	\begin{array}{ccc}
	\toprule
	\text{Density} & \varrho & \SI{1}{kg/m^3}\\
	\midrule
	\text{Kinematic viscosity} & \nu & \SI{e-5}{m^2/s}\\
	\midrule
	\text{Inlet velocity} & \mathbf{v}_{in} & [\SI{1}{m/s}, 0]^\mathrm{T}\\
	\midrule
	\text{Outlet pressure} & p_{ext} & \SI{1.1e5}{Pa}\\
	\midrule
	\text{Reynolds number} & Re & \num{6e5}\\
	\bottomrule
	\end{array}
	\]
	\caption{Properties of the fluid used for the test.}
	\label{tab:fluidprop}
\end{table}

We want to see how the flow field in the second cavity is affected by the 
presence of the first cavity, depending on the distance $d$ between them that 
has been varied from $\SI{0.05}{m}$ to $\SI{2}{m}$.

As previously said we employ an unsteady $k\text{-}\omega$ model and we 
simulate the time interval $[0, T_{end}]$, starting from the initial 
conditions of uniform parallel flow for $y>h$ and no flow for $y\leq h$. The 
simulation time has been chosen equal to $\SI{30}{s}$, time after which 
the solution has reached an equilibrium.
The fluid properties are set as in Table~\ref{tab:fluidprop}. After having 
performed a sensitivity analysis we have chosen a mesh with $\num{377x50}{}$ 
elements with gradings in the inlet and outlet portions of the channel in order 
to have small uniform cells around the two cavities with sizes of 
$\SI{5x3.3}{mm}$.
% eulero all'indietro o bdf2 per la discretizzazione temporatle è uguale e poi 
%usiamo sempre i nostr TVD con Van Leer

\subsection{Results}
At a first glance the behaviour of the fluid in the two cavities is very 
similar, there is not any strong influence of the first one on the second one 
and the pattern of the velocity field is the same (see Figure~\ref{fig:veld1}). 
%aggiungere cenno a quali recirculations ci sono e mettere le freccette 
%nell'immagine?
\begin{figure}
	\centering
	\includegraphics[width=\textwidth, trim={0 0 0 6cm}, 
	clip]{cavities_dist1_vel}
	\caption{Magnitude of the velocity field [$\si{m/s}$] around the cavities 
		with $d=\SI{1}{m}$ at $t=\SI{30}{s}$. $Re=\num{6e5}$.}
	\label{fig:veld1}
\end{figure}

In order to better analyse the results of the simulations we compare the 
profiles of the velocity in the first cells above $y=h$, always at 
$t=\SI{30}{s}$. The velocity in these points drives the flow in the cavities, 
since we start from zero velocity inside them.
\begin{figure*}
	\centering
	% This file was created by matlab2tikz.
%
\definecolor{mycolor1}{rgb}{0.00000,0.44700,0.74100}%
%
\begin{tikzpicture}

\begin{axis}[%
width=0.951\figwidth,
height=0.75\cavheight,
at={(0\figwidth,0\cavheight)},
scale only axis,
xmin=3.5,
xmax=7,
xlabel style={font=\color{white!15!black}},
xlabel={$x$ [m]},
ymin=0.15,
ymax=0.4,
ylabel style={font=\color{white!15!black}},
ylabel={$u$ [m/s]},
axis background/.style={fill=white},
xmajorgrids,
ymajorgrids
]
\addplot [color=mycolor1, line width=1.0pt, forget plot]
  table[row sep=crcr]{%
3.505	0.262676\\
3.51	0.262616\\
3.515	0.262616\\
3.52	0.262616\\
3.525	0.262616\\
3.53	0.262616\\
3.535	0.262616\\
3.54	0.262616\\
3.545	0.262616\\
3.55	0.262616\\
3.555	0.262616\\
3.56	0.262558\\
3.565	0.262558\\
3.57	0.262558\\
3.575	0.262558\\
3.58	0.262558\\
3.585	0.262558\\
3.59	0.262558\\
3.595	0.262558\\
3.6	0.262558\\
3.605	0.262503\\
3.61	0.262503\\
3.615	0.262503\\
3.62	0.262503\\
3.625	0.262503\\
3.63	0.262503\\
3.635	0.262503\\
3.64	0.262503\\
3.645	0.26245\\
3.65	0.26245\\
3.655	0.26245\\
3.66	0.26245\\
3.665	0.26245\\
3.67	0.26245\\
3.675	0.26245\\
3.68	0.26245\\
3.685	0.262399\\
3.69	0.262399\\
3.695	0.262399\\
3.7	0.262399\\
3.705	0.262399\\
3.71	0.262399\\
3.715	0.262399\\
3.72	0.262351\\
3.725	0.262351\\
3.73	0.262351\\
3.735	0.262351\\
3.74	0.262351\\
3.745	0.262351\\
3.75	0.262306\\
3.755	0.262306\\
3.76	0.262306\\
3.765	0.262306\\
3.77	0.262306\\
3.775	0.262265\\
3.78	0.262265\\
3.785	0.262265\\
3.79	0.262265\\
3.795	0.262265\\
3.8	0.262227\\
3.805	0.262227\\
3.81	0.262227\\
3.815	0.262227\\
3.82	0.262227\\
3.825	0.262195\\
3.83	0.262195\\
3.835	0.262195\\
3.84	0.262195\\
3.845	0.262168\\
3.85	0.262168\\
3.855	0.262168\\
3.86	0.262168\\
3.865	0.262148\\
3.87	0.262148\\
3.875	0.262148\\
3.88	0.262148\\
3.885	0.262137\\
3.89	0.262137\\
3.895	0.262137\\
3.9	0.262135\\
3.905	0.262135\\
3.91	0.262135\\
3.915	0.262145\\
3.92	0.262145\\
3.925	0.262168\\
3.93	0.262168\\
3.935	0.262168\\
3.94	0.262209\\
3.945	0.262209\\
3.95	0.262272\\
3.955	0.262272\\
3.96	0.262368\\
3.965	0.262368\\
3.97	0.262512\\
3.975	0.262728\\
3.98	0.262728\\
3.985	0.263067\\
3.99	0.263662\\
3.995	0.26509\\
4	0.268628\\
4.005	0.268628\\
4.01	0.27414\\
4.015	0.280032\\
4.02	0.285537\\
4.025	0.290514\\
4.03	0.294956\\
4.035	0.298941\\
4.04	0.302551\\
4.045	0.305844\\
4.05	0.308863\\
4.055	0.311641\\
4.06	0.314202\\
4.065	0.316568\\
4.07	0.318763\\
4.075	0.320816\\
4.08	0.322754\\
4.085	0.324599\\
4.09	0.326371\\
4.095	0.328094\\
4.1	0.329791\\
4.105	0.331472\\
4.11	0.333147\\
4.115	0.334828\\
4.12	0.33652\\
4.125	0.33823\\
4.13	0.339963\\
4.135	0.341723\\
4.14	0.343509\\
4.145	0.345319\\
4.15	0.347147\\
4.155	0.348985\\
4.16	0.350827\\
4.165	0.352663\\
4.17	0.354484\\
4.175	0.356285\\
4.18	0.358057\\
4.185	0.359792\\
4.19	0.361487\\
4.195	0.363137\\
4.2	0.364737\\
4.205	0.366281\\
4.21	0.367768\\
4.215	0.369193\\
4.22	0.370556\\
4.225	0.371852\\
4.23	0.373082\\
4.235	0.374244\\
4.24	0.375337\\
4.245	0.376361\\
4.25	0.377316\\
4.255	0.378202\\
4.26	0.37902\\
4.265	0.379771\\
4.27	0.380456\\
4.275	0.381076\\
4.28	0.381633\\
4.285	0.38213\\
4.29	0.382567\\
4.295	0.382947\\
4.3	0.383274\\
4.305	0.383549\\
4.31	0.383777\\
4.315	0.38396\\
4.32	0.384103\\
4.325	0.384211\\
4.33	0.384287\\
4.335	0.384334\\
4.34	0.384355\\
4.345	0.384354\\
4.35	0.384333\\
4.355	0.384293\\
4.36	0.384235\\
4.365	0.38416\\
4.37	0.384064\\
4.375	0.383945\\
4.38	0.383796\\
4.385	0.383606\\
4.39	0.383366\\
4.395	0.383059\\
4.4	0.382666\\
4.405	0.382164\\
4.41	0.381524\\
4.415	0.380707\\
4.42	0.379663\\
4.425	0.378339\\
4.43	0.376682\\
4.435	0.374643\\
4.44	0.372186\\
4.445	0.369265\\
4.45	0.365813\\
4.455	0.361743\\
4.46	0.356928\\
4.465	0.351207\\
4.47	0.344355\\
4.475	0.336065\\
4.48	0.325962\\
4.485	0.313661\\
4.49	0.298724\\
4.495	0.282999\\
4.5	0.295616\\
4.505	0.264581\\
4.51	0.193707\\
4.515	0.16835\\
4.52	0.163515\\
4.525	0.167251\\
4.53	0.173801\\
4.535	0.181008\\
4.54	0.188012\\
4.545	0.194581\\
4.55	0.194581\\
4.555	0.200482\\
4.56	0.205777\\
4.565	0.210593\\
4.57	0.210593\\
4.575	0.214974\\
4.58	0.218972\\
4.585	0.222644\\
4.59	0.222644\\
4.595	0.226024\\
4.6	0.229138\\
4.605	0.229138\\
4.61	0.232011\\
4.615	0.234665\\
4.62	0.234665\\
4.625	0.237118\\
4.63	0.237118\\
4.635	0.239386\\
4.64	0.241484\\
4.645	0.241484\\
4.65	0.243426\\
4.655	0.243426\\
4.66	0.245225\\
4.665	0.245225\\
4.67	0.24689\\
4.675	0.24689\\
4.68	0.248431\\
4.685	0.249852\\
4.69	0.249852\\
4.695	0.251162\\
4.7	0.251162\\
4.705	0.252364\\
4.71	0.252364\\
4.715	0.253465\\
4.72	0.253465\\
4.725	0.253465\\
4.73	0.25447\\
4.735	0.25447\\
4.74	0.255385\\
4.745	0.255385\\
4.75	0.256216\\
4.755	0.256216\\
4.76	0.256971\\
4.765	0.256971\\
4.77	0.257656\\
4.775	0.257656\\
4.78	0.257656\\
4.785	0.258274\\
4.79	0.258274\\
4.795	0.258831\\
4.8	0.258831\\
4.805	0.258831\\
4.81	0.25933\\
4.815	0.25933\\
4.82	0.259773\\
4.825	0.259773\\
4.83	0.259773\\
4.835	0.260165\\
4.84	0.260165\\
4.845	0.260165\\
4.85	0.260507\\
4.855	0.260507\\
4.86	0.260804\\
4.865	0.260804\\
4.87	0.260804\\
4.875	0.261058\\
4.88	0.261058\\
4.885	0.261058\\
4.89	0.261272\\
4.895	0.261272\\
4.9	0.261272\\
4.905	0.26145\\
4.91	0.26145\\
4.915	0.26145\\
4.92	0.261597\\
4.925	0.261597\\
4.93	0.261597\\
4.935	0.261716\\
4.94	0.261716\\
4.945	0.261716\\
4.95	0.26181\\
4.955	0.26181\\
4.96	0.26181\\
4.965	0.26181\\
4.97	0.261883\\
4.975	0.261883\\
4.98	0.261883\\
4.985	0.26194\\
4.99	0.26194\\
4.995	0.26194\\
5	0.26194\\
5.005	0.261979\\
5.01	0.261979\\
5.015	0.261979\\
5.02	0.261996\\
5.025	0.261996\\
5.03	0.261996\\
5.035	0.261995\\
5.04	0.261995\\
5.045	0.261995\\
5.05	0.261995\\
5.055	0.261983\\
5.06	0.261983\\
5.065	0.261983\\
5.07	0.261961\\
5.075	0.261961\\
5.08	0.261961\\
5.085	0.26193\\
5.09	0.26193\\
5.095	0.26193\\
5.1	0.261892\\
5.105	0.261892\\
5.11	0.261892\\
5.115	0.26185\\
5.12	0.26185\\
5.125	0.26185\\
5.13	0.261804\\
5.135	0.261804\\
5.14	0.261804\\
5.145	0.261757\\
5.15	0.261757\\
5.155	0.261709\\
5.16	0.261709\\
5.165	0.261709\\
5.17	0.261661\\
5.175	0.261661\\
5.18	0.261661\\
5.185	0.261614\\
5.19	0.261614\\
5.195	0.261568\\
5.2	0.261568\\
5.205	0.261568\\
5.21	0.261522\\
5.215	0.261522\\
5.22	0.261476\\
5.225	0.261476\\
5.23	0.261476\\
5.235	0.261432\\
5.24	0.261432\\
5.245	0.261388\\
5.25	0.261388\\
5.255	0.261344\\
5.26	0.261344\\
5.265	0.261302\\
5.27	0.261302\\
5.275	0.26126\\
5.28	0.26126\\
5.285	0.26126\\
5.29	0.261219\\
5.295	0.261219\\
5.3	0.261179\\
5.305	0.261179\\
5.31	0.261141\\
5.315	0.261141\\
5.32	0.261103\\
5.325	0.261067\\
5.33	0.261067\\
5.335	0.261032\\
5.34	0.261032\\
5.345	0.260999\\
5.35	0.260999\\
5.355	0.260967\\
5.36	0.260967\\
5.365	0.260937\\
5.37	0.26091\\
5.375	0.26091\\
5.38	0.260884\\
5.385	0.260884\\
5.39	0.260861\\
5.395	0.260841\\
5.4	0.260841\\
5.405	0.260825\\
5.41	0.260813\\
5.415	0.260813\\
5.42	0.260805\\
5.425	0.260803\\
5.43	0.260807\\
5.435	0.260807\\
5.44	0.260818\\
5.445	0.260838\\
5.45	0.260869\\
5.455	0.260869\\
5.46	0.260914\\
5.465	0.260979\\
5.47	0.261071\\
5.475	0.261202\\
5.48	0.261396\\
5.485	0.261698\\
5.49	0.262222\\
5.495	0.26347\\
5.5	0.266629\\
5.505	0.266629\\
5.51	0.271595\\
5.515	0.276913\\
5.52	0.281899\\
5.525	0.28642\\
5.53	0.290467\\
5.535	0.294107\\
5.54	0.297413\\
5.545	0.300435\\
5.55	0.303213\\
5.555	0.305775\\
5.56	0.30814\\
5.565	0.310327\\
5.57	0.312362\\
5.575	0.314269\\
5.58	0.316073\\
5.585	0.317791\\
5.59	0.319442\\
5.595	0.321042\\
5.6	0.322609\\
5.605	0.324163\\
5.61	0.325711\\
5.615	0.327261\\
5.62	0.328822\\
5.625	0.330398\\
5.63	0.331996\\
5.635	0.333617\\
5.64	0.335264\\
5.645	0.336936\\
5.65	0.338634\\
5.655	0.340354\\
5.66	0.342088\\
5.665	0.34383\\
5.67	0.345572\\
5.675	0.347305\\
5.68	0.349021\\
5.685	0.350713\\
5.69	0.352373\\
5.695	0.353994\\
5.7	0.355573\\
5.705	0.357103\\
5.71	0.358579\\
5.715	0.36\\
5.72	0.361363\\
5.725	0.362667\\
5.73	0.363909\\
5.735	0.365089\\
5.74	0.366205\\
5.745	0.367258\\
5.75	0.368245\\
5.755	0.369167\\
5.76	0.370024\\
5.765	0.370815\\
5.77	0.371541\\
5.775	0.372204\\
5.78	0.372803\\
5.785	0.373342\\
5.79	0.37382\\
5.795	0.374241\\
5.8	0.374607\\
5.805	0.374921\\
5.81	0.375186\\
5.815	0.375404\\
5.82	0.375579\\
5.825	0.375716\\
5.83	0.375816\\
5.835	0.375883\\
5.84	0.375918\\
5.845	0.375925\\
5.85	0.375907\\
5.855	0.375866\\
5.86	0.3758\\
5.865	0.375709\\
5.87	0.37559\\
5.875	0.375439\\
5.88	0.375247\\
5.885	0.375009\\
5.89	0.374712\\
5.895	0.374345\\
5.9	0.373886\\
5.905	0.37331\\
5.91	0.372592\\
5.915	0.371699\\
5.92	0.370589\\
5.925	0.369218\\
5.93	0.367541\\
5.935	0.365513\\
5.94	0.363089\\
5.945	0.36021\\
5.95	0.356812\\
5.955	0.352805\\
5.96	0.348069\\
5.965	0.342443\\
5.97	0.335705\\
5.975	0.327557\\
5.98	0.317633\\
5.985	0.305563\\
5.99	0.290921\\
5.995	0.275548\\
6	0.288264\\
6.005	0.189663\\
6.01	0.189663\\
6.015	0.164357\\
6.02	0.160921\\
6.025	0.167072\\
6.03	0.167072\\
6.035	0.176188\\
6.04	0.185964\\
6.045	0.185964\\
6.05	0.195421\\
6.055	0.195421\\
6.06	0.203955\\
6.065	0.203955\\
6.07	0.211669\\
6.075	0.211669\\
6.08	0.21859\\
6.085	0.21859\\
6.09	0.21859\\
6.095	0.224818\\
6.1	0.224818\\
6.105	0.224818\\
6.11	0.230405\\
6.115	0.230405\\
6.12	0.230405\\
6.125	0.23541\\
6.13	0.23541\\
6.135	0.23541\\
6.14	0.239845\\
6.145	0.239845\\
6.15	0.239845\\
6.155	0.239845\\
6.16	0.24374\\
6.165	0.24374\\
6.17	0.24374\\
6.175	0.24374\\
6.18	0.24713\\
6.185	0.24713\\
6.19	0.24713\\
6.195	0.24713\\
6.2	0.24713\\
6.205	0.250021\\
6.21	0.250021\\
6.215	0.250021\\
6.22	0.250021\\
6.225	0.250021\\
6.23	0.252439\\
6.235	0.252439\\
6.24	0.252439\\
6.245	0.252439\\
6.25	0.252439\\
6.255	0.254431\\
6.26	0.254431\\
6.265	0.254431\\
6.27	0.254431\\
6.275	0.254431\\
6.28	0.254431\\
6.285	0.256028\\
6.29	0.256028\\
6.295	0.256028\\
6.3	0.256028\\
6.305	0.256028\\
6.31	0.256028\\
6.315	0.256028\\
6.32	0.257261\\
6.325	0.257261\\
6.33	0.257261\\
6.335	0.257261\\
6.34	0.257261\\
6.345	0.257261\\
6.35	0.257261\\
6.355	0.257261\\
6.36	0.258168\\
6.365	0.258168\\
6.37	0.258168\\
6.375	0.258168\\
6.38	0.258168\\
6.385	0.258168\\
6.39	0.258168\\
6.395	0.258168\\
6.4	0.258804\\
6.405	0.258804\\
6.41	0.258804\\
6.415	0.258804\\
6.42	0.258804\\
6.425	0.258804\\
6.43	0.258804\\
6.435	0.258804\\
6.44	0.258804\\
6.445	0.259215\\
6.45	0.259215\\
6.455	0.259215\\
6.46	0.259215\\
6.465	0.259215\\
6.47	0.259215\\
6.475	0.259215\\
6.48	0.259215\\
6.485	0.259215\\
6.49	0.259449\\
6.495	0.259449\\
6.5	0.259449\\
6.505	0.259449\\
6.51	0.259449\\
6.515	0.259449\\
6.52	0.259449\\
6.525	0.259449\\
6.53	0.259449\\
6.535	0.259449\\
6.54	0.259449\\
6.545	0.25952\\
6.55	0.25952\\
6.555	0.25952\\
6.56	0.25952\\
6.565	0.25952\\
6.57	0.25952\\
6.575	0.25952\\
6.58	0.25952\\
6.585	0.25952\\
6.59	0.25952\\
6.595	0.25952\\
6.6	0.25952\\
6.605	0.259476\\
6.61	0.259476\\
6.615	0.259476\\
6.62	0.259476\\
6.625	0.259476\\
6.63	0.259476\\
6.635	0.259476\\
6.64	0.259476\\
6.645	0.259476\\
6.65	0.259476\\
6.655	0.259476\\
6.66	0.259476\\
6.665	0.259476\\
6.67	0.259376\\
6.675	0.259376\\
6.68	0.259376\\
6.685	0.259376\\
6.69	0.259376\\
6.695	0.259376\\
6.7	0.259376\\
6.705	0.259376\\
6.71	0.259376\\
6.715	0.259376\\
6.72	0.259376\\
6.725	0.259376\\
6.73	0.259376\\
6.735	0.259376\\
6.74	0.259219\\
6.745	0.259219\\
6.75	0.259219\\
6.755	0.259219\\
6.76	0.259219\\
6.765	0.259219\\
6.77	0.259219\\
6.775	0.259219\\
6.78	0.259219\\
6.785	0.259219\\
6.79	0.259219\\
6.795	0.259219\\
6.8	0.259219\\
6.805	0.259219\\
6.81	0.259219\\
6.815	0.259219\\
6.82	0.25916\\
6.825	0.25916\\
6.83	0.25916\\
6.835	0.25916\\
6.84	0.25916\\
6.845	0.25916\\
6.85	0.25916\\
6.855	0.25916\\
6.86	0.25916\\
6.865	0.25916\\
6.87	0.25916\\
6.875	0.25916\\
6.88	0.25916\\
6.885	0.25916\\
6.89	0.25916\\
6.895	0.25916\\
6.9	0.25916\\
6.905	0.259157\\
6.91	0.259157\\
6.915	0.259157\\
6.92	0.259157\\
6.925	0.259157\\
6.93	0.259157\\
6.935	0.259157\\
6.94	0.259157\\
6.945	0.259157\\
6.95	0.259157\\
6.955	0.259157\\
6.96	0.259157\\
6.965	0.259157\\
6.97	0.259157\\
6.975	0.259157\\
6.98	0.259157\\
6.985	0.259157\\
6.99	0.259157\\
6.995	0.259157\\
7	0.259157\\
};
\end{axis}
\end{tikzpicture}%
	\caption{Velocity profile at $y=h$, with a distance $d=\SI{1}{m}$ between 
		cavities, at $t=\SI{30}{s}$. $Re = \num{6e5}$.}
	\label{fig:velprofile1}
\end{figure*}

Looking at the profiles in Figure~\ref{fig:velprofile1}, there is a positive 
peak of the velocity near the centre of each cavity, then, when the wall starts 
again, there is a negative peak, followed by a space that we call recovery 
distance $d_{rec}$ of approximately $\SI{0.5}{m}$, in which there is a recovery 
of the value that the velocity had before the cavity. The values in the second 
cavity are slightly lower than the correspondent ones in the first cavity, but 
the shape is the same and this confirms the fact that the behaviour of the flow 
in the two cavities is the analogous.

The peaks above the cavities are due to the fact that those sections are not 
touching the wall, so the wall friction is not highly affecting the velocity as 
in the other sections and this allows the velocity to reach higher values.
When cavity a is over and the wall begins there is a sudden decrease in the 
velocity because before the corner of the cavity the flow is split into a 
portion that goes down and generates the recirculation and another portion that 
instead goes up (see Figure~\ref{fig:vel_split}).
\begin{figure}
	\centering
	\includegraphics[width=0.7\textwidth]{cavities_vel_split.png}
	\caption{$y$-component of the velocity near end of a cavity. It can be 
		observed that the flow splits.}
	\label{fig:vel_split}
\end{figure}
This change in the flow 
direction implies a reduction of the $x$ component of the velocity. After the 
distance $d_{rec}$ the effect of the wall has restored the boundary layer that 
there was before the cavity. If $d < d_{rec}$, i.e. the distance between the 
cavities is smaller than the recovery distance, then the flow in the second 
cavity will be slower because it will be driven by lower velocities.

We report in Figure~\ref{fig:velpeaks} the maximum values of the $x$-component 
of the velocity in the peak in correspondence of the second cavity, depending 
of the distance between the cavities.
\begin{figure}
	\centering
	% This file was created by matlab2tikz.
%
\definecolor{mycolor1}{rgb}{0.00000,0.44700,0.74100}%
%
\begin{tikzpicture}

\begin{axis}[%
width=0.951\figwidth,
height=0.75\cavheight,
at={(0\figwidth,0\cavheight)},
scale only axis,
xmin=0,
xmax=2,
xlabel style={font=\color{white!15!black}},
xlabel={$d$ [m]},
ymin=0.35,
ymax=0.38,
ylabel style={font=\color{white!15!black}},
ylabel={velocity [m/s]},
axis background/.style={fill=white},
xmajorgrids,
ymajorgrids,
yticklabel style={/pgf/number format/.cd,fixed,precision=3}
]
\addplot [color=mycolor1, line width=1.0pt, mark size=2.5pt, mark=o, mark options={solid, mycolor1}, forget plot]
  table[row sep=crcr]{%
0.05	0.352715\\
0.2	0.371891\\
0.3	0.375066\\
0.5	0.376806\\
1	0.375925\\
1.5	0.374456\\
2	0.373188\\
};
\end{axis}
\end{tikzpicture}%
	\caption{Relation between the maximum velocity above the second cavity and 
	the distance between the cavities $d$.}
	\label{fig:velpeaks}
\end{figure}
We see that if $d < d_{rec}$ there velocities are affected by the presence of 
the first cavity and so the value is lower. For $d>d_{rec}$ the velocities are 
slightly decreasing as $d$ increases, this fact is due to the turbulence that 
is generated near the end of the first cavity. As it can be seen in 
Figure~\ref{fig:kd1} the turbulent kinetic energy that is generated from the 
first cavity is transported towards the second one, leading to an increased 
thickness of the boundary layer as we move in the $x$ direction.
\begin{figure}
	\centering
	\includegraphics[width=\textwidth, trim={0 0 0 7cm}, 
	clip]{cavities_dist1_k}
	\caption{Turbulent kinetic energy [$\si{m^2/s^2}$] around the cavities with 
		$d=\SI{1}{m}$, at $t=\SI{30}{s}$. $Re=\num{6e5}$.}
	\label{fig:kd1}
\end{figure}

\FloatBarrier
\section{The multi-domain test}
In this second test we simulate the same situation of the single-domain test 
with a distance between the cavities $d=\SI{1}{m}$, but now we couple the 
free-flow with the simulation of a porous-medium that fills the space between 
the cavities, as it is shown in Figure~\ref{fig:multidomain}. We have so two 
domains, $\Omega_{ff}$ and $\Omega_{pm}$, that are in contact through the 
interface $\Gamma_{int}$.
\begin{figure}[h]
	\centering
	\includegraphics[width=\textwidth]{cavities_multidomain.png}
	\caption{The domain for the multi-domain test.}
	\label{fig:multidomain}
\end{figure}

In the domain $\Omega_{ff}$ we solve as before the RANS equations with the 
$k\text{-}\omega$ model, while in the $\Omega_{pm}$ we solve the Forchheimer 
equation. We use the same boundary conditions we used before, but this time we 
have in addition to impose suitable coupling conditions at the interface 
$\Gamma_{int}$ and no flow conditions on the boundary $\Gamma_{base}$.
We employ the same mesh we used before in $\Omega_{ff}$ and we extend it in 
$\Omega_{pm}$ in a conforming way.

We want to analyse the differences that the presence of the porous-medium 
induces in the flow field and how it changes varying the permeability.
% riempiamo il buco con un mezzo poroso
% usiamo forcheimer, l'equazione è fatta così e così viene dal volume averaging 
%oppure empirica, il coefficiente scelto così e così (prima va spiegata anche 
%darcy), perchè usiamo forcheimer, quando vale
% coupled model con delle coupling conditions che vanno spiegate
% mesh come prima, variamo la permeabilità (cos'è la permeabilità)
\subsection{Results}
We compare the profile of the velocity in the $x$ direction at the position 
$y=h$ (see Figure~\ref{fig:md_wall_velocities}).
\begin{figure}[h]
	\centering
	\input{img/cavities_coupled_velx}
	\caption{Velocity profile at $y=h$, at $t=\SI{30}{s}$ for different values 
		of permeability.}
	\label{fig:md_wall_velocities}
\end{figure}

For values of permeability lower than $\SI{e-8}{m^2}$ the porous-medium results 
to be almost impermeable, or at least the flow field in $\Omega_{ff}$ is the 
same that we had in the single domain test and the maximum velocities in the 
corner of the porous-medium are of some orders of magnitude lower than those in 
the cavities.

Starting from $K=\SI{3.1e-8}{m^2}$ we observe some changes in the velocity 
profile and increasing the permeability those changes are enhanced. In 
correspondence of the first cavity the maximum velocity increases because the 
porous-medium reduces its wall effect and so the flow is less forced to reduce 
its velocity because a more relevant quantity of fluid enters into the 
porous-medium (see Figure~\ref{fig:coupled_arrows}). Then this flow comes out 
from the top of $\Omega_{pm}$ within the first $\SI{20}{cm}$; this spreads and 
moves a bit ahead the zone with a peak in the $y$-component of the velocity, as 
it can be seen in Figure~\ref{fig:coupled_vel_split}, where we can compare the 
velocities for different values of permeability.
\begin{figure}
	\centering
	\includegraphics[width=\textwidth]{coupled_first_cavity.png}
	\caption{Velocity field near the end of the first cavity where some flow 
		enters into the porous-medium. $K=\SI{3.1e-7}{m^2}$. The length of the 
		arrows is not scaled. $Re=\num{6e5}$.}
	\label{fig:coupled_arrows}
\end{figure}
\begin{figure}
	\centering
	\includegraphics[width=\textwidth]{split_comparison.png}
	\caption{$y$-component of the velocity near the end of the first cavity in 
	the free-flow domain $\Omega_{ff}$. $K = \SI{3.1e-7}{m^2}$ on the left and 
	$K = \SI{3.1e-9}{m^2}$ on the right. $Re=\num{6e5}$.}
	\label{fig:coupled_vel_split}
\end{figure}

After a certain distance $d_{rec}$ the value of the velocity that there was 
before the cavity is recovered and the boundary layer is restored. With high 
values of permeability this equilibrium value is higher than that one with low 
permeabilities; this is due to the Beavers-Joseph-Saffman condition that allows 
more slip as the permeability increases. Eventually the behaviour around the 
second cavity is the same with all the permeabilities; there is some flow that 
goes back inside the porous-medium, but since it comes from the corner eddy the 
values involved are lower compared to what happens in the first cavity (see 
Figure~\ref{fig:coupledsecondcav}).
% risultati: il fluido non attraversa mai il mezzo poroso

\begin{figure}
	\centering
	\includegraphics[width=\textwidth]{coupled_second_cavity.png}
	\caption{Velocity field near the beginning of the second cavity where some 
flow enters into the porous medium. $K=\SI{3.1e-7}{m^2}$. The length of the 
arrows is not scaled. $Re=\num{6e5}$.}
	\label{fig:coupledsecondcav}
\end{figure}

\begin{figure}
	\centering
	\includegraphics[width=\textwidth]{coupled_vel_3.png}
	\caption{Magnitude of the velocity field for $K=\SI{3.1e-9}{m^2}$, at 
	$t=\SI{30}{s}$. $Re=\num{6e5}$.}
	\label{fig:coupled_vel_3}
\end{figure}

\end{document}