% !TEX encoding = UTF-8 Unicode
% !TEX TS-program = pdflatex
 \documentclass[%
 	12pt,
 	a4paper,
	twoside, openright
%	oneside, openany
]{book}
%%%%%%%%%%%%%%%%%%%%%%%%%%%%%%%%%%%%%%%%%%%%%%%%%%%%
\usepackage[utf8]{inputenc}
\usepackage[T1]{fontenc}
\usepackage[british, italian]{babel}
\usepackage{lmodern}
\usepackage{amsmath}
\usepackage{amssymb}
\usepackage{amsthm}
\usepackage{xspace}% per i colori
\usepackage{tocbibind}%per aggiungere toc, listoffig, listoftab alla toc
\usepackage{graphicx}
\graphicspath{{../../img/}}
\usepackage{tikz}
\usepackage{pgfplots}
\pgfplotsset{compat=newest}
\pgfplotsset{plot coordinates/math parser=false}
%\usepackage{pgfgantt} %gantbar (mai usata..)
%\usepackage{pdflscape} % forse inutile
\usepackage{subfig} % per posizionare tante figure
\usepackage{bbold} % per \mathbb
\usepackage{booktabs} % per i filetti delle tabelle
%\usepackage[font=small]{caption} % font diverso nelle didascalie
\usepackage[section]{placeins} % ridefinisce section inserendo un \FloatBarrier
\usepackage{emptypage} % pagina tutta bianca a fine capitolo
\usepackage{pdfpages} % per inserire pdf
\usepackage[separate-uncertainty=true]{siunitx} % per le unitàdi misura
\usepackage[autostyle]{csquotes} % per la bibliografia
\usepackage[style=alphabetic,maxbibnames=99,maxcitenames=2,backend=biber]{biblatex}%styles:
% numeric,trad-plain
%\usepackage[bindingoffset=4mm]{geometry} % aggiunge spazio per la rilegatura
\usepackage{fancyhdr} % per le testatine
\pagestyle{fancy}
\renewcommand{\chaptermark}[1]{\markboth{\chaptername\ \thechapter\ -\ #1}{}}
\renewcommand{\sectionmark}[1]{\markright{\thesection\ -\ #1}}
\fancyhf{}
\fancyhead[RO]{\slshape\nouppercase{\rightmark}}%twoside
\fancyhead[LE]{\slshape\nouppercase{\leftmark}}%twhoside	
%\fancyhead[C]{\slshape\nouppercase{\leftmark}}%oneside
\fancyfoot[C]{\thepage}
\setlength{\headheight}{14.5pt}
\PassOptionsToPackage{hyphens}{url}
\usepackage{hyperref} % va caricato per ultimo
\hypersetup{%
	pdfpagemode={UseOutlines},
	bookmarksopen,
	pdfstartview={FitH},
	hidelinks
}
\pdfminorversion=7 % per togliere il warning delle immagini
\pdfsuppresswarningpagegroup=1 % warning per due immagini pdf in una pagina
\newcommand{\DUMUX}{DuMu$^\mathrm{x}$\xspace} %for convenience
\renewcommand*{\pagenumbering}[1]{} % per mettere un'unica numerazione

\newlength{\figwidth}\setlength{\figwidth}{0.39\textwidth}
\newlength{\widthsette}\setlength{\widthsette}{0.7\textwidth}
\newlength{\widthsei}\setlength{\widthsei}{0.6\textwidth}
\newlength{\roughwidth}\setlength{\roughwidth}{0.9\textwidth}
\newlength{\roughheight}\setlength{\roughheight}{\roughwidth}
\newlength{\bfshalfwidth}\setlength{\bfshalfwidth}{0.41\textwidth}
\newlength{\bfsheight}\setlength{\bfsheight}{0.4\textwidth}

\makeatletter
\newcommand{\addemptysup}{\@ifnextchar^{}{^{}}}
\makeatother

\newcommand{\RUD}{R_{U,D}\addemptysup}
\newcommand{\RFUU}{R_{FU,U}\addemptysup}
\newcommand{\xU}{x_{U}\addemptysup}
\newcommand{\xD}{x_{D}\addemptysup}
\newcommand{\xFU}{x_{FU}\addemptysup}

\emergencystretch=1em % per gli overfull hbox nella bibliografia.
\addbibresource{../bibliotesi.bib}

\DeclareMathOperator{\esssup}{ess\,sup}

\begin{document}
\selectlanguage{british}
\frontmatter
%\pagenumbering{Roman}
\includepdf[pages=-]{Frontespizio-sss.pdf}
\chapter{Abstract}
Lorem ipsum dolor sit amet, consectetur adipiscing elit, sed do eiusmod tempor incididunt ut labore et dolore magna aliqua. Ut enim ad minim veniam, quis nostrud exercitation ullamco laboris nisi ut aliquip ex ea commodo consequat. Duis aute irure dolor in reprehenderit in voluptate velit esse cillum dolore eu fugiat nulla pariatur. Excepteur sint occaecat cupidatat non proident, sunt in culpa qui officia deserunt mollit anim id est laborum \cite{tesi:fetzer}.
Lorem ipsum dolor sit amet, consectetur adipiscing elit, sed do eiusmod tempor incididunt ut labore et dolore magna aliqua. Ut enim ad minim veniam, quis nostrud exercitation ullamco laboris nisi ut aliquip ex ea commodo consequat. Duis aute irure dolor in reprehenderit in voluptate velit esse cillum dolore eu fugiat nulla pariatur. Excepteur sint occaecat cupidatat non proident, sunt in culpa qui officia deserunt mollit anim id est laborum \cite{tesi:fetzer}.
Lorem ipsum dolor sit amet, consectetur adipiscing elit, sed do eiusmod tempor incididunt ut labore et dolore magna aliqua. Ut enim ad minim veniam, quis nostrud exercitation ullamco laboris nisi ut aliquip ex ea commodo consequat. Duis aute irure dolor in reprehenderit in voluptate velit esse cillum dolore eu fugiat nulla pariatur. Excepteur sint occaecat cupidatat non proident, sunt in culpa qui officia deserunt mollit anim id est laborum \cite{tesi:fetzer}.
Lorem ipsum dolor sit amet, consectetur adipiscing elit, sed do eiusmod tempor incididunt ut labore et dolore magna aliqua. Ut enim ad minim veniam, quis nostrud exercitation ullamco laboris nisi ut aliquip ex ea commodo consequat. Duis aute irure dolor in reprehenderit in voluptate velit esse cillum dolore eu fugiat nulla pariatur. Excepteur sint occaecat cupidatat non proident, sunt in culpa qui officia deserunt mollit anim id est laborum \cite{tesi:fetzer}.
Lorem ipsum dolor sit amet, consectetur adipiscing elit, sed do eiusmod tempor incididunt ut labore et dolore magna aliqua. Ut enim ad minim veniam, quis nostrud exercitation ullamco laboris nisi ut aliquip ex ea commodo consequat. Duis aute irure dolor in reprehenderit in voluptate velit esse cillum dolore eu fugiat nulla pariatur. Excepteur sint occaecat cupidatat non proident, sunt in culpa qui officia deserunt mollit anim id est laborum \cite{tesi:fetzer}.
Lorem ipsum dolor sit amet, consectetur adipiscing elit, sed do eiusmod tempor incididunt ut labore et dolore magna aliqua. Ut enim ad minim veniam, quis nostrud exercitation ullamco laboris nisi ut aliquip ex ea commodo consequat. Duis aute irure dolor in reprehenderit in voluptate velit esse cillum dolore eu fugiat nulla pariatur. Excepteur sint occaecat cupidatat non proident, sunt in culpa qui officia deserunt mollit anim id est laborum \cite{tesi:fetzer}.
Lorem ipsum dolor sit amet, consectetur adipiscing elit, sed do eiusmod tempor incididunt ut labore et dolore magna aliqua. Ut enim ad minim veniam, quis nostrud exercitation ullamco laboris nisi ut aliquip ex ea commodo consequat. Duis aute irure dolor in reprehenderit in voluptate velit esse cillum dolore eu fugiat nulla pariatur. Excepteur sint occaecat cupidatat non proident, sunt in culpa qui officia deserunt mollit anim id est laborum \cite{tesi:fetzer}.
Lorem ipsum dolor sit amet, consectetur adipiscing elit, sed do eiusmod tempor incididunt ut labore et dolore magna aliqua. Ut enim ad minim veniam, quis nostrud exercitation ullamco laboris nisi ut aliquip ex ea commodo consequat. Duis aute irure dolor in reprehenderit in voluptate velit esse cillum dolore eu fugiat nulla pariatur. Excepteur sint occaecat cupidatat non proident, sunt in culpa qui officia deserunt mollit anim id est laborum \cite{tesi:fetzer}.
\tableofcontents
\listoffigures
\mainmatter
\chapter{Intro}
Lorem ipsum dolor sit amet, consectetur adipiscing elit, sed do eiusmod tempor incididunt ut labore et dolore magna aliqua. Ut enim ad minim veniam, quis nostrud exercitation ullamco laboris nisi ut aliquip ex ea commodo consequat. Duis aute irure dolor in reprehenderit in voluptate velit esse cillum dolore eu fugiat nulla pariatur. Excepteur sint occaecat cupidatat non proident, sunt in culpa qui officia deserunt mollit anim id est laborum \cite{tesi:fetzer}.
Lorem ipsum dolor sit amet, consectetur adipiscing elit, sed do eiusmod tempor incididunt ut labore et dolore magna aliqua. Ut enim ad minim veniam, quis nostrud exercitation ullamco laboris nisi ut aliquip ex ea commodo consequat. Duis aute irure dolor in reprehenderit in voluptate velit esse cillum dolore eu fugiat nulla pariatur. Excepteur sint occaecat cupidatat non proident, sunt in culpa qui officia deserunt mollit anim id est laborum \cite{tesi:fetzer}.
Lorem ipsum dolor sit amet, consectetur adipiscing elit, sed do eiusmod tempor incididunt ut labore et dolore magna aliqua. Ut enim ad minim veniam, quis nostrud exercitation ullamco laboris nisi ut aliquip ex ea commodo consequat. Duis aute irure dolor in reprehenderit in voluptate velit esse cillum dolore eu fugiat nulla pariatur. Excepteur sint occaecat cupidatat non proident, sunt in culpa qui officia deserunt mollit anim id est laborum \cite{tesi:fetzer}.
Lorem ipsum dolor sit amet, consectetur adipiscing elit, sed do eiusmod tempor incididunt ut labore et dolore magna aliqua. Ut enim ad minim veniam, quis nostrud exercitation ullamco laboris nisi ut aliquip ex ea commodo consequat. Duis aute irure dolor in reprehenderit in voluptate velit esse cillum dolore eu fugiat nulla pariatur. Excepteur sint occaecat cupidatat non proident, sunt in culpa qui officia deserunt mollit anim id est laborum \cite{tesi:fetzer}.
Lorem ipsum dolor sit amet, consectetur adipiscing elit, sed do eiusmod tempor incididunt ut labore et dolore magna aliqua. Ut enim ad minim veniam, quis nostrud exercitation ullamco laboris nisi ut aliquip ex ea commodo consequat. Duis aute irure dolor in reprehenderit in voluptate velit esse cillum dolore eu fugiat nulla pariatur. Excepteur sint occaecat cupidatat non proident, sunt in culpa qui officia deserunt mollit anim id est laborum \cite{tesi:fetzer}.
Lorem ipsum dolor sit amet, consectetur adipiscing elit, sed do eiusmod tempor incididunt ut labore et dolore magna aliqua. Ut enim ad minim veniam, quis nostrud exercitation ullamco laboris nisi ut aliquip ex ea commodo consequat. Duis aute irure dolor in reprehenderit in voluptate velit esse cillum dolore eu fugiat nulla pariatur. Excepteur sint occaecat cupidatat non proident, sunt in culpa qui officia deserunt mollit anim id est laborum \cite{tesi:fetzer}.
Lorem ipsum dolor sit amet, consectetur adipiscing elit, sed do eiusmod tempor incididunt ut labore et dolore magna aliqua. Ut enim ad minim veniam, quis nostrud exercitation ullamco laboris nisi ut aliquip ex ea commodo consequat. Duis aute irure dolor in reprehenderit in voluptate velit esse cillum dolore eu fugiat nulla pariatur. Excepteur sint occaecat cupidatat non proident, sunt in culpa qui officia deserunt mollit anim id est laborum \cite{tesi:fetzer}.
Lorem ipsum dolor sit amet, consectetur adipiscing elit, sed do eiusmod tempor incididunt ut labore et dolore magna aliqua. Ut enim ad minim veniam, quis nostrud exercitation ullamco laboris nisi ut aliquip ex ea commodo consequat. Duis aute irure dolor in reprehenderit in voluptate velit esse cillum dolore eu fugiat nulla pariatur. Excepteur sint occaecat cupidatat non proident, sunt in culpa qui officia deserunt mollit anim id est laborum \cite{tesi:fetzer}.
\begin{figure}
\centering
% This file was created by matlab2tikz.
%
\begin{tikzpicture}

\begin{axis}[%
width=1.25\widthsette,
height=0.75\widthsette,
at={(0\widthsette,0\widthsette)},
scale only axis,
xmin=0,
xmax=4,
xlabel style={font=\color{white!15!black}},
xlabel={$r$},
ymin=0,
ymax=2.5,
ylabel style={font=\color{white!15!black}, rotate=-90},
ylabel={$\psi$},
axis background/.style={fill=white},
xmajorgrids,
ymajorgrids
]

\addplot[area legend, line width=1.0pt, draw=black, fill=white!80!black, fill opacity=0.5, forget plot]
table[row sep=crcr] {%
x	y\\
0	0\\
1	1\\
0.444444444444444	1\\
}--cycle;

\addplot[area legend, line width=1.0pt, draw=black, fill=white!80!black, fill opacity=0.5, forget plot]
table[row sep=crcr] {%
x	y\\
1	1\\
5.5	1\\
5.5	1.8\\
1.8	1.8\\
}--cycle;
\addplot [color=black, dashed, line width=1.0pt, forget plot]
  table[row sep=crcr]{%
0	1.8\\
1.8	1.8\\
};
\addplot [color=black, dashed, line width=1.0pt, forget plot]
  table[row sep=crcr]{%
0.444444444444444	1\\
1.11111111111111	2.5\\
};
\end{axis}

\begin{axis}[%
width=1.290\widthsette,
height=1.01\widthsette,
at={(-0.16\widthsette,-0.135\widthsette)},
scale only axis,
xmin=0,
xmax=1,
ymin=0,
ymax=1,
axis line style={draw=none},
ticks=none,
axis x line*=bottom,
axis y line*=left,
legend style={legend cell align=left, align=left, draw=white!15!black}
]
\node[below right, align=left, draw=none,]
at (rel axis cs:0.035,0.7) {\small$\RUD$};
\node[below right, align=left, draw=none]
at (rel axis cs:0.35,1) {\small $\dfrac{\RFUU}{\RFUU-1}r$};
\end{axis}
\end{tikzpicture}%
\caption[short text]{text}
\end{figure}
\appendix
\renewcommand{\chaptername}{Appendix}
\chapter{App}
Lorem ipsum dolor sit amet, consectetur adipiscing elit, sed do eiusmod tempor incididunt ut labore et dolore magna aliqua. Ut enim ad minim veniam, quis nostrud exercitation ullamco laboris nisi ut aliquip ex ea commodo consequat. Duis aute irure dolor in reprehenderit in voluptate velit esse cillum dolore eu fugiat nulla pariatur. Excepteur sint occaecat cupidatat non proident, sunt in culpa qui officia deserunt mollit anim id est laborum \cite{tesi:fetzer}.
Lorem ipsum dolor sit amet, consectetur adipiscing elit, sed do eiusmod tempor incididunt ut labore et dolore magna aliqua. Ut enim ad minim veniam, quis nostrud exercitation ullamco laboris nisi ut aliquip ex ea commodo consequat. Duis aute irure dolor in reprehenderit in voluptate velit esse cillum dolore eu fugiat nulla pariatur. Excepteur sint occaecat cupidatat non proident, sunt in culpa qui officia deserunt mollit anim id est laborum \cite{tesi:fetzer}.
Lorem ipsum dolor sit amet, consectetur adipiscing elit, sed do eiusmod tempor incididunt ut labore et dolore magna aliqua. Ut enim ad minim veniam, quis nostrud exercitation ullamco laboris nisi ut aliquip ex ea commodo consequat. Duis aute irure dolor in reprehenderit in voluptate velit esse cillum dolore eu fugiat nulla pariatur. Excepteur sint occaecat cupidatat non proident, sunt in culpa qui officia deserunt mollit anim id est laborum \cite{tesi:fetzer}.
Lorem ipsum dolor sit amet, consectetur adipiscing elit, sed do eiusmod tempor incididunt ut labore et dolore magna aliqua. Ut enim ad minim veniam, quis nostrud exercitation ullamco laboris nisi ut aliquip ex ea commodo consequat. Duis aute irure dolor in reprehenderit in voluptate velit esse cillum dolore eu fugiat nulla pariatur. Excepteur sint occaecat cupidatat non proident, sunt in culpa qui officia deserunt mollit anim id est laborum \cite{tesi:fetzer}.
Lorem ipsum dolor sit amet, consectetur adipiscing elit, sed do eiusmod tempor incididunt ut labore et dolore magna aliqua. Ut enim ad minim veniam, quis nostrud exercitation ullamco laboris nisi ut aliquip ex ea commodo consequat. Duis aute irure dolor in reprehenderit in voluptate velit esse cillum dolore eu fugiat nulla pariatur. Excepteur sint occaecat cupidatat non proident, sunt in culpa qui officia deserunt mollit anim id est laborum \cite{tesi:fetzer}.
Lorem ipsum dolor sit amet, consectetur adipiscing elit, sed do eiusmod tempor incididunt ut labore et dolore magna aliqua. Ut enim ad minim veniam, quis nostrud exercitation ullamco laboris nisi ut aliquip ex ea commodo consequat. Duis aute irure dolor in reprehenderit in voluptate velit esse cillum dolore eu fugiat nulla pariatur. Excepteur sint occaecat cupidatat non proident, sunt in culpa qui officia deserunt mollit anim id est laborum \cite{tesi:fetzer}.
Lorem ipsum dolor sit amet, consectetur adipiscing elit, sed do eiusmod tempor incididunt ut labore et dolore magna aliqua. Ut enim ad minim veniam, quis nostrud exercitation ullamco laboris nisi ut aliquip ex ea commodo consequat. Duis aute irure dolor in reprehenderit in voluptate velit esse cillum dolore eu fugiat nulla pariatur. Excepteur sint occaecat cupidatat non proident, sunt in culpa qui officia deserunt mollit anim id est laborum \cite{tesi:fetzer}.
Lorem ipsum dolor sit amet, consectetur adipiscing elit, sed do eiusmod tempor incididunt ut labore et dolore magna aliqua. Ut enim ad minim veniam, quis nostrud exercitation ullamco laboris nisi ut aliquip ex ea commodo consequat. Duis aute irure dolor in reprehenderit in voluptate velit esse cillum dolore eu fugiat nulla pariatur. Excepteur sint occaecat cupidatat non proident, sunt in culpa qui officia deserunt mollit anim id est laborum \cite{tesi:fetzer}.
\backmatter
\printbibliography[heading=bibintoc]
\end{document}
