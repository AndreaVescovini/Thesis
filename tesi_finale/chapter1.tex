\chapter{Introduction}
Flow and transport processes between turbulent free-flows and porous-media 
flows are common in a wide range of environmental, industrial, civil and medical 
applications.
For example \textcite{tesi:mosthaf}, \textcite{intro:davarzani} and \textcite  {tesi:fetzer} have studied the effect that a turbulent air flow has on the 
drying rate of an adjacent wet porous-medium, like a soil. This kind of system, despite its simplicity, involves many different physical phenomena that act at different scales, thus, because of this complexity, a reliable prediction of the evaporation rates is still a challenge. In Figure~\ref{fig:intro} we can see a schematic representation of the mechanisms that play a role in such a situation, where both transport and thermal effects can be taken into account. Moreover, considering natural phenomena, an additional difficulty is given by the intrinsic uncertainty and heterogeneity of material properties, such as the soil porosity, and atmospheric conditions, such as the air humidity or the solar radiation.
\begin{figure}[ht]
	\centering
	\includegraphics[width=\textwidth]{intropicture2.png}
	\caption[Exchange processes between free and porous-medium 
	flows]{Example of physical phenomena affecting the exchange processes 
		between the free-flow and the porous-medium flow. Figure source: 
		\cite{tesi:fetzer}.}
	\label{fig:intro}
\end{figure}

These studies can be exploited, for example, to better understand the process of soil salinization, that is one of the most serious agricultural problems in many arid and coastal areas in the world. It consists in the excessive accumulation of salt in the soil pores, with the consequence of a partial or complete loss of productivity. A limited amount of salt precipitation in the soil, due to evaporation of irrigation water, is inevitable, but a wrong water management could lead salinity problems in the long term, especially in arid areas where irrigation is necessary to increase the production for food requirements (see \cite{web:fao}). According to \cite{soil:munns}, more than 6\% of wolrd's total land area is salt affected.
The region of the soil near the surface is the most involved in this problem, hence the importance of studying the interactions between the free-flow and the porous-medium flow. \textcite{intro:salinization} have investigated the application of kinetic approaches to describe the salt precipitation in a coupled system.
\begin{figure}
	\centering
	\subfloat[]{\includegraphics[height=0.22\textheight]{salinity.jpg}}
	\subfloat[]{\includegraphics[height=0.22\textheight]{pem.png}}
	\caption{(a): Salt-affected soil. Figure source: \cite{web:fao}. (b): Principle of operation of PEM fuel cells. Figure source: \cite{intro:pemfig}.}
	\label{fig:intro2}
\end{figure}

PEM (Proton Exchange Membrane) fuel cells are a possible alternative power source, mainly for vehicles. In their design, transport phenomena through gas channels play an important role in the electrochemical reactions that determine the cell performances and efficiency (see \cite{wu:fuelcell}). As we can see in Figure~\ref{fig:intro2}, reactant gases are supplied at the anode and cathode through gas channels, then they are transported and they diffuse into porous layers called gas diffusers, that should deliver them uniformly and efficiently to the catalyst, where reactions take place (see \cite{tesi:pem}).

In order to study these processes we focus on the case of the evaporation from 
a porous material and we consider a system that involves two subdomains: the 
upper 
one with a free-flow and the lower one occupied by a porous-medium. At the 
interface between the two subdomains there is exchange of mass, momentum and 
energy, but in the surrounding environment there is a large variety of 
phenomena that have an influence on these exchanges, as we can see for example 
in Figure~\ref{fig:intro}.
%\begin{figure}[ht]
%	\centering
%	\includegraphics[width=\textwidth]{intropicture.png}
%	\caption[Exchange processes between free and porous-medium 
%	flows]{Example of physical phenomena affecting the exchange processes 
%	between the free-flow and the porous-medium flow. Figure source: 
%	\cite{tesi:fetzer}.}
%
%\end{figure}
Consequently a numerical study of this system generally involves a coupled 
model where the free-flow is described by the RANS equations and the 
porous-medium flow is described with a REV-scale approach, such as the Darcy's 
law. 
Simple models for single-phase systems can be generalized to take 
into account multicomponent non-isothermal flows, with one phase in the 
free-flow region and two phases in the porous-medium. Finally the problem 
has to be closed through suitable coupling conditions imposed at the 
interface; they are usually based on phenomenological arguments in order to 
keep the description simple and they should be as close as possible to the 
imposition of thermodynamic the equilibrium~\cite{paper:mosthaf}.

In this thesis the focus is on the improvement of the free-flow model. When the 
flow is in a turbulent regime, turbulent eddies develop near the interface and 
they cascade through consecutively smaller scales until the kinetic energy 
dissipates into internal thermal energy. Because of their location, they have 
a strong influence on the exchange processes between the two subdomains, so an 
accurate evaluation of their behaviour is of crucial importance. Improvements 
can be obtained of course with a refinement of the grid, but also employing 
high order methods; in 
particular, in the discretization of the Navier-Stokes equations using finite 
volumes, a key role is played by the approximation used for the non linear term 
$\nabla \cdot (\mathbf{v} \mathbf{v}^\mathrm{T})$. A common and easy 
choice is to employ a first order upwind approximation for the 
\emph{transported} velocity, but this option can produce solutions with 
excessive numerical diffusion. Other possibilities are given by high order 
methods 
like the Linear Upwind Differencing (LUD) scheme, the Central Differencing (CD) 
scheme or the Quadratic Upstream Interpolation for Convective Kinetics (QUICK) 
scheme; sometimes they can produce 
accurate solution but they have also shown to be unstable in certain 
situations and to produce overshoots or undershoots that 
may lead to unphysical values of quantities that for example have to be 
positive (see \cite{main:vermal}). With this in mind our interest will be in 
the Total Variation Diminishing (TVD) methods, a family of methods that has 
been derived with the purpose of providing a solution with a second order 
accuracy, but without any risk of numerical oscillations. They are called also 
high resolution methods.

In Chapter~\ref{chap:equations} the equations employed in the models will be 
presented, with particular attention to the free-flow equations. In 
Chapter~\ref{chap:discretization} the finite volume method will be described 
and in Subsection~\ref{subsec:tvd} the TVD methods will be introduced. At last, 
in Chapter~\ref{chap:results}, the numerical results will be shown; in 
particular we will see the application of the TVD methods to the Navier-Stokes 
equations in comparison to the first order upwind scheme, then two tests 
involving turbulent flows and finally a more complex scenario involving 
obstacles at the interface between a free-flow region and a porous-medium.

The high order methods mentioned above have been implemented in the framework 
of the open-source simulator \DUMUX: DUNE for multi-$\{$phase, component, 
scale, physics, \dots$\}$ flow in porous-media, see \cite{dumux:tutti} and 
\cite{dumux:flemisch}. \DUMUX is an 
additional module of DUNE (Distributed and Unified Numerics Environment, 
\cite{web:dune}) and, through the use of an object-oriented design in 
conjunction with template programming, it provides a C++ environment that 
allows an efficient implementation of numerical models related to porous-media 
flows.

All the source code used for the simulations performed can be 
found at \url{https://git.iws.uni-stuttgart.de/dumux-pub/vescovini2019a}, 
together with the instructions to install the required software.