\chapter{Numerical results} \label{chap:results}
%We have made a lot of computations.
\section{Navier-Stokes tests}
\subsection{Convergence of the TVD methods}
%We have a better convergence.
\subsection{Convergence with respect to time}
\subsection{Rough channel test}
%Here we see some differences.
\section{RANS tests}
\subsection{Backward facing step}
%The NASA LaRC CFD code CFL3D uses:
%\begin{itemize}
%	\item free-stream turbulence intensity = 0.061\%
%	\item free-stream turbulent viscosity (relative to laminar) = 0.009
%	\item in this case the simulations were \emph{quasi-steady}, i.e. the 
%	solution does not converge readily to a steady state result when a refined 
%	grid is used. However, when run-time accurately, the solution settles down 
%	and becomes reasonably steady (quasi-steady).
%	\item The code is for compressible flows at ``essentially incompressible" 
%	conditions of M=0.128. There may be a very small influence of 
%	compressibility.
%	\item non-dimensional CFD code
%	\item first order turbulence advection in the turbulence transport equations
%	\item Neglects two terms that are zero for incompressible flows
%	\item implicit time advancement (also second order)
%	\item Central differencing in Space? Non lo dice cosa usa, inoltre secondo 
%	me non usa una staggered grid.
%\end{itemize}
\subsection{Cavities reciprocal influence}
%An easy result.
\section{Coupled test}
\subsection{Cavities coupled with the porous-medium}
%A more complex result.