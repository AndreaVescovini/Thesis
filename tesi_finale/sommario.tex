\selectlanguage{italian}
\chapter*{Sommario}
Processi di scambio tra flussi liberi e flussi in mezzi porosi sono comuni in 
molte applicazioni industriali o ambientali. In di regimi turbolenti e 
interfacce che presentano eterogeneità, è importante avere un'accurata 
descrizione del flusso libero, in quanto i vortici che si vengono a creare 
vicino all'interfaccia, a causa della turbolenza, hanno una grande influenza 
sugli scambi.

L'obbiettivo di questa tesi è quindi l'applicazione di schemi ad alta 
risoluzione (high resolution schemes) per la discretizzazione a volumi finiti 
del termine convettivo nelle equazioni di Navier-Stokes incomprimibili. In 
particolare l'attenzione è rivolta ai metodi Total Variation Diminishing (TVD). 
Essi sono stati implementati all'interno del codice \DUMUX, nell'ambito di una 
discretizzazione su griglia sfalsata (staggered grid). Sono state inoltre 
considerate due possibili generalizzazioni al caso di griglie cartesiane non 
uniformi.

I molteplici test di confronto con il metodo upwind di ordine 1 che sono stati 
effettuati hanno evidenziato una migliore accuratezza dei metodi TVD a parità 
di griglia. 
In seguito, per simulare flussi turbolenti, sono state utilizzate le equazioni 
RANS, scegliendo il modello di turbolenza $k\text{-}\omega$. E' stato 
utilizzato il test del backward facing step per validare i risultati, 
confrontandoli con quelli disponibili prodotti dal codice CFL3D della NASA. È 
stata ottenuta una buona previsione della distanza di riattacco, in accordo con 
i risultati di riferimento. Infine è stato studiato un problema accoppiato tra 
flusso libero e flusso in un mezzo poroso, ponendo attenzione all'effetto che 
un'interfaccia con ostacoli ha sul campo di velocità. Si è ottenuto 
che per valori alti di permeabilità in flusso nel mezzo poroso, descritto con 
la legge di Forchheimer, influenza il flusso libero.
\\[\baselineskip]
\textbf{Parole chiave:} metodi TVD, RANS, mezzi porosi, problema accoppiato, 
\DUMUX.
\selectlanguage{british}