\chapter*{Abstract}
Exchange processes between free-flows and porous-media flows are common in many 
industrial and environmental applications. In the case of turbulent flows and 
rough interfaces, an accurate description of the free-flow is important because 
the turbulent eddies near the interface strongly affect the exchanges.

The aim of this thesis is the investigation of the effects of rough interfaces in coupled free-flow and porous-media flow systems. In particular, this work exploits the application of high resolution schemes for 
the finite volumes discretization of the convective term in the momentum 
equation of the incompressible Navier-Stokes equations. The 
focus is on the Total Variation Diminishing (TVD) methods, which have been 
implemented in the code \DUMUX, within the framework of a staggered-grid 
approach. Two possible extension to the case of non-uniform grids have been 
considered.

Several comparison tests with the first order upwind method have been 
performed, showing more accurate solutions for the TVD methods on the same grid.
Afterwards the RANS equations have been used in order to simulate turbulent 
flows, employing the $k\text{-}\omega$ turbulence model. The backward facing 
step test has been used to validate the results against the 
ones from the NASA CFL3D code. A good prediction of the reattachment length has 
been obtained. At last, a coupled free and porous-medium flow configuration has 
been studied, with focus on the effect that a rough interface has on the flow 
field. With high values of permeability, the porous-medium flow, modelled using 
the Forchheimer's law, showed to influence the flow in the free-flow region.
\\[\baselineskip]
\textbf{Keywords}: TVD methods, RANS, porous-media, coupled problem, \DUMUX.