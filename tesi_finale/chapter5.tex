\chapter{Conclusions and outlook}
\section{Conclusions}
In this work we have seen the application of the TVD methods for the 
approximation of the convective term in the momentum equation of the 
Navier-Stokes equations and of the RANS equations, within a finite volumes 
framework. The aim was to use them for the simulation of coupled free-flow and 
porous-medium flow models, improving the results obtained employing the 
first order upwind scheme, using a method that introduces less numerical 
viscosity in the solution, but that, at the same time, guarantees the stability 
and avoids the creation of unphysical oscillations.

In all the tests that have been performed, we have experienced an increased 
accuracy of the solution with respect to those obtained when using the first order 
upwind method. The spatial convergence rates in the tests with analytical 
solution were between 1.5 and 2 in many cases, while with the upwind method we had always a first order convergence. 
The differences were more relevant at $Re\simeq\num{e3}$ rather than at 
$Re\simeq1$, because of the increased importance of the convective term with 
respect to the diffusive one. Also with turbulent flows in the backward facing 
step test, we have 
obtained, using the Van Leer flux limiter, a satisfying agreement with 
validated numerical results from the CFL3D code and experimental data.
Moreover, employing a BDF2 scheme for the 
temporal discretization, we have available an high order discretization both 
in space and in time, thus allowing a more accurate study of transient 
phenomena.

With this tool, we have studied a coupled model made of a free-flow and a 
porous-medium flow, focusing in particular on the effects that a rough
interface, with cavities or porous obstacles, has on the flow field.

At first, we have considered only the free-flow and we have studied how the 
velocity is affected by the presence of two cavities at the interface.
In particular, we have investigated which is the distance between the cavities such that the variation in the flow field, 
because of the first cavity, does not significantly influence anymore the 
behaviour around the 
second one. We have obtained that, for cavities $\SI{0.5}{m}$ long, after a 
distance equal to one cavity the flow field has recovered the configuration it 
had before the first cavity. However each cavity introduces an increase in the 
turbulent kinetic energy, that accelerates the growth of the turbulent 
boundary layer.

Afterwards we have chosen a distance such that the two cavities do not 
influence each other and we have placed a porous-medium between them, in order 
to analyse how its presence affects the flow field. We have observed that, 
for values of permeability greater than $\SI{e-8}{m^2}$, the flow field in the 
free-flow shows noticeable differences when compared to a case without a 
porous-medium, both in the case of the shallow cavities and deep cavities.
In particular, not only the fact that the flow can enter into 
the porous-medium has an effect, but also the Beaver-Joseph-Saffman slip 
condition that was imposed at the interface between the two subdomains affects the flow dynamics.
The presence of eddies in the cavities influences the flow in the 
porous-medium, because in correspondence of the corner eddies it can enter also 
from a direction opposite to the one of the main flow. However, when this 
happens, the values of the velocity are relatively small and thus this 
behaviour does not affect the free-flow as it happens when the flow enters the 
porous-medium at the end of a cavity. We have evaluated the mass flow 
rate from the free-flow region to the porous-medium and we have noticed a 
relevant 
increase for values of permeability greater than $\SI{e-8}{m^2}$. Comparing 
the cases of shallow and deep cavities, the mass flow rate has shown to be 
proportional to the measure of the interface.

At last, we have considered a flow in a channel with a porous obstacles on the 
lower boundary. We have observed that the porous-medium flow influences the 
free-flow only at the frontal face of the obstacle, where the flow is stopped 
by the porous medium, while at the backward face the behaviour remains the same 
independently of the permeability.
%
\section{Future developments}
Future developments to this thesis can be devoted to the investigation of more 
complex models, involving multiphase, multi-component and non-isothermal 
models. Exploiting the accuracy of high-resolution methods, a more reliable 
prediction of the evaporation rate from a wet soil could be performed. 
Moreover, it could be interesting to study other types of \emph{rough} 
interfaces, for example to see if a smooth interface described by a sinusoidal 
function could behave differently with respect to those we have analysed.

In this work the TVD methods have been adopted only in the momentum equation, 
but, in particular with non-isothermal models with non-constant density, from 
the numerical point of view it could be important to analyse their 
application to the continuity equation.

From the point of view of the coupling, the interface conditions could be adapted in order to employ other finite volumes discretization schemes in the porous-medium, including, for example, a Multi Point Flux Approximation (MPFA).
Moreover, iterative algorithms to decouple the two subdomains could be taken 
into account and compared to the monolithic approach we have used.

Within the \DUMUX framework a possibility to improve the feasibility of the 
simulations could be to employ an adaptive algorithm in order to refine the grid 
only where it is useful. This approach can lead to non-conforming meshes 
containing hanging nodes, thus a challenge would be the application of the TVD 
scheme to this case. Moreover a parallelization of the code would be 
beneficial for the computational time, but the extended stencil of high order methods 
requires attention whilst decomposing the domain.